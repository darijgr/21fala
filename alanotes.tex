\documentclass[numbers=enddot,12pt,final,onecolumn,notitlepage]{scrartcl}%
\usepackage[headsepline,footsepline,manualmark]{scrlayer-scrpage}
\usepackage[all,cmtip]{xy}
\usepackage{amssymb}
\usepackage{amsmath}
\usepackage{amsthm}
\usepackage{framed}
\usepackage{comment}
\usepackage{color}
\usepackage{hyperref}
\usepackage[sc]{mathpazo}
\usepackage[T1]{fontenc}
\usepackage{tikz}
\usepackage{needspace}
\usepackage{tabls}
\usepackage{wasysym}
\usepackage{easytable}
\usepackage{pythonhighlight}
%TCIDATA{OutputFilter=latex2.dll}
%TCIDATA{Version=5.50.0.2960}
%TCIDATA{LastRevised=Thursday, September 16, 2021 00:18:44}
%TCIDATA{SuppressPackageManagement}
%TCIDATA{<META NAME="GraphicsSave" CONTENT="32">}
%TCIDATA{<META NAME="SaveForMode" CONTENT="1">}
%TCIDATA{BibliographyScheme=Manual}
%TCIDATA{Language=American English}
%BeginMSIPreambleData
\providecommand{\U}[1]{\protect\rule{.1in}{.1in}}
%EndMSIPreambleData
\usetikzlibrary{arrows.meta}
\usetikzlibrary{chains}
\newcounter{exer}
\newcounter{exera}
\numberwithin{exer}{subsection}
\theoremstyle{definition}
\newtheorem{theo}{Theorem}[subsection]
\newenvironment{theorem}[1][]
{\begin{theo}[#1]\begin{leftbar}}
{\end{leftbar}\end{theo}}
\newtheorem{lem}[theo]{Lemma}
\newenvironment{lemma}[1][]
{\begin{lem}[#1]\begin{leftbar}}
{\end{leftbar}\end{lem}}
\newtheorem{prop}[theo]{Proposition}
\newenvironment{proposition}[1][]
{\begin{prop}[#1]\begin{leftbar}}
{\end{leftbar}\end{prop}}
\newtheorem{defi}[theo]{Definition}
\newenvironment{definition}[1][]
{\begin{defi}[#1]\begin{leftbar}}
{\end{leftbar}\end{defi}}
\newtheorem{remk}[theo]{Remark}
\newenvironment{remark}[1][]
{\begin{remk}[#1]\begin{leftbar}}
{\end{leftbar}\end{remk}}
\newtheorem{coro}[theo]{Corollary}
\newenvironment{corollary}[1][]
{\begin{coro}[#1]\begin{leftbar}}
{\end{leftbar}\end{coro}}
\newtheorem{conv}[theo]{Convention}
\newenvironment{convention}[1][]
{\begin{conv}[#1]\begin{leftbar}}
{\end{leftbar}\end{conv}}
\newtheorem{quest}[theo]{Question}
\newenvironment{question}[1][]
{\begin{quest}[#1]\begin{leftbar}}
{\end{leftbar}\end{quest}}
\newtheorem{warn}[theo]{Warning}
\newenvironment{warning}[1][]
{\begin{warn}[#1]\begin{leftbar}}
{\end{leftbar}\end{warn}}
\newtheorem{conj}[theo]{Conjecture}
\newenvironment{conjecture}[1][]
{\begin{conj}[#1]\begin{leftbar}}
{\end{leftbar}\end{conj}}
\newtheorem{exam}[theo]{Example}
\newenvironment{example}[1][]
{\begin{exam}[#1]\begin{leftbar}}
{\end{leftbar}\end{exam}}
\newtheorem{exmp}[exer]{Exercise}
\newenvironment{exercise}[1][]
{\begin{exmp}[#1]\begin{leftbar}}
{\end{leftbar}\end{exmp}}
\newenvironment{statement}{\begin{quote}}{\end{quote}}
\newenvironment{fineprint}{\medskip \begin{small}}{\end{small} \medskip}
\iffalse
\newenvironment{proof}[1][Proof]{\noindent\textbf{#1.} }{\ \rule{0.5em}{0.5em}}
\newenvironment{question}[1][Question]{\noindent\textbf{#1.} }{\ \rule{0.5em}{0.5em}}
\newenvironment{warning}[1][Warning]{\noindent\textbf{#1.} }{\ \rule{0.5em}{0.5em}}
\newenvironment{teachingnote}[1][Teaching note]{\noindent\textbf{#1.} }{\ \rule{0.5em}{0.5em}}
\fi
\let\sumnonlimits\sum
\let\prodnonlimits\prod
\let\cupnonlimits\bigcup
\let\capnonlimits\bigcap
\renewcommand{\sum}{\sumnonlimits\limits}
\renewcommand{\prod}{\prodnonlimits\limits}
\renewcommand{\bigcup}{\cupnonlimits\limits}
\renewcommand{\bigcap}{\capnonlimits\limits}
\setlength\tablinesep{3pt}
\setlength\arraylinesep{3pt}
\setlength\extrarulesep{3pt}
\voffset=0cm
\hoffset=-0.7cm
\setlength\textheight{22.5cm}
\setlength\textwidth{15.5cm}
\newcommand\arxiv[1]{\href{http://www.arxiv.org/abs/#1}{\texttt{arXiv:#1}}}
\newenvironment{verlong}{}{}
\newenvironment{vershort}{}{}
\newenvironment{noncompile}{}{}
\newenvironment{teachingnote}{}{}
\excludecomment{verlong}
\includecomment{vershort}
\excludecomment{noncompile}
\excludecomment{teachingnote}
\newcommand{\CC}{\mathbb{C}}
\newcommand{\RR}{\mathbb{R}}
\newcommand{\QQ}{\mathbb{Q}}
\newcommand{\NN}{\mathbb{N}}
\newcommand{\ZZ}{\mathbb{Z}}
\newcommand{\KK}{\mathbb{K}}
\newcommand{\id}{\operatorname{id}}
\newcommand{\lcm}{\operatorname{lcm}}
\newcommand{\rev}{\operatorname{rev}}
\newcommand{\powset}[2][]{\ifthenelse{\equal{#2}{}}{\mathcal{P}\left(#1\right)}{\mathcal{P}_{#1}\left(#2\right)}}
\newcommand{\set}[1]{\left\{ #1 \right\}}
\newcommand{\abs}[1]{\left| #1 \right|}
\newcommand{\tup}[1]{\left( #1 \right)}
\newcommand{\ive}[1]{\left[ #1 \right]}
\newcommand{\floor}[1]{\left\lfloor #1 \right\rfloor}
\newcommand{\lf}[2]{#1^{\underline{#2}}}
\newcommand{\underbrack}[2]{\underbrace{#1}_{\substack{#2}}}
\newcommand{\horrule}[1]{\rule{\linewidth}{#1}}
\newcommand{\are}{\ar@{-}}
\newcommand{\nnn}{\nonumber\\}
\newcommand{\sslash}{\mathbin{/\mkern-6mu/}}
\newcommand{\numboxed}[2]{\underbrace{\boxed{#1}}_{\text{box } #2}}
\newcommand{\ig}[2]{\includegraphics[scale=#1]{#2.png}}
\newcommand{\UNFINISHED}{\begin{center} \Huge{\textbf{Unfinished material begins here.}} \end{center} }
\iffalse
\NOEXPAND{\today}{\today}
\NOEXPAND{\sslash}{\sslash}
\NOEXPAND{\numboxed}[2]{\numboxed}
\NOEXPAND{\UNFINISHED}{\UNFINISHED}
\fi
\ihead{Math 504 notes}
\ohead{page \thepage}
\cfoot{\today}
\begin{document}

\title{Math 504: Advanced Linear Algebra}
\author{Hugo Woerdeman, with edits by Darij Grinberg\thanks{Drexel University, Korman
Center, 15 S 33rd Street, Philadelphia PA, 19104, USA}}
\date{\today\ (unfinished!)}
\maketitle
\tableofcontents

\section*{Preface}

These are lecture notes originally written by Hugo Woerdeman and edited by
myself for the Math 504 (Advanced Linear Algebra) class at Drexel University
in Fall 2021. The website of this class can be found at%
\[
\text{\url{http://www.cip.ifi.lmu.de/~grinberg/t/21fala}\ \ .}%
\]


\textbf{This document is a work in progress.}

\textbf{Please report any errors you find to
\texttt{\href{mailto:darijgrinberg@gmail.com}{darijgrinberg@gmail.com}} .}

\subsection*{What is this?}

This is a second course on linear algebra, meant for (mostly graduate)
students that are already familiar with matrices, determinants and vector
spaces. Much of the prerequisites (but also some of our material, and even
some content that goes beyond our course) is covered by textbooks like
\cite{Heffer20}, \cite{LaNaSc16}, \cite{Taylor20}, \cite{Treil15},
\cite{Strick20}, \cite[Part I]{GalQua20}, \cite{Loehr14}, \cite{Woerde16}%
\footnote{This list is nowhere near complete. (It is biased towards freely
available sources, but even in that category it is probably far from
comprehensive.)}. The text we will follow the closest is \cite{HorJoh13}.

We will freely use the basic theory of complex numbers, including the
Fundamental Theorem of Algebra. See \cite[Chapters 2--3]{LaNaSc16} or
\cite[Chapters 9--10]{Korner20} for an introduction to these matters.

\subsection*{Notations}

\begin{itemize}
\item We let $\mathbb{N}:=\left\{  0,1,2,\ldots\right\}  $.

\item For any $n\in\mathbb{N}$, we let $\left[  n\right]  $ denote the
$n$-element set $\left\{  1,2,\ldots,n\right\}  $.

\item If $\mathbb{F}$ is a field, and $n,m\in\mathbb{N}$, then $\mathbb{F}%
^{n\times m}$ denotes the set (actually, an $\mathbb{F}$-vector space) of all
$n\times m$-matrices over $\mathbb{F}$.

\item If $\mathbb{F}$ is a field, and $n\in\mathbb{N}$, then the space
$\mathbb{F}^{n\times1}$ of all $n\times1$-matrices over $\mathbb{F}$ (that is,
column vectors of size $n$) is also denoted by $\mathbb{F}^{n}$.

\item The $n\times n$ identity matrix is denoted by $I_{n}$ or by $I$ if the
$n$ is clear from the context.

\item The transpose of a matrix $A$ is denoted by $A^{T}$.

\item If $A$ is an $n\times m$-matrix, and if $i\in\left[  n\right]  $ and
$j\in\left[  m\right]  $, then:

\begin{itemize}
\item we let $A_{i,j}$ denote the $\left(  i,j\right)  $-th entry of $A$ (that
is, the entry of $A$ in the $i$-th row and the $j$-th column);

\item we let $A_{i,\bullet}$ denote the $i$-th row of $A$;

\item we let $A_{\bullet,j}$ denote the $j$-th column of $A$.
\end{itemize}

\item The letter $i$ usually denotes the complex number $\sqrt{-1}$. Sometimes
(e.g. in the bullet point just above) it also stands for something else
(usually an \textbf{i}ndex that is an \textbf{i}nteger). I'll do my best to
avoid the latter meaning when there is any realistic chance that it be
confused for the former.

\item We use the notation $\operatorname*{diag}\left(  \lambda_{1},\lambda
_{2},\ldots,\lambda_{n}\right)  $ for the diagonal matrix with diagonal
entries $\lambda_{1},\lambda_{2},\ldots,\lambda_{n}$.
\end{itemize}

\subsection{Remark on exercises}

Each exercise gives a number of \textquotedblleft experience
points\textquotedblright, which roughly corresponds to its difficulty (with
some adjustment for its relevance). This is the number in the square (like
\fbox{3} or \fbox{5}). The harder or more important the exercise, the larger
is the number in the square. A \fbox{1} is a warm-up question whose solution
you will probably see right after reading; a \fbox{3} typically requires some
thinking or work; a \fbox{5} requires both; higher values tend to involve some
creativity or research.

\newpage

\section{Unitary matrices (\cite[\S 2.1]{HorJoh13})}

In this chapter, $n$ will usually denote a nonnegative integer.%

\[%
\begin{tabular}
[c]{||l||}\hline\hline
\textbf{Old lecture 1 starts here.}\\\hline\hline
\end{tabular}
\
\]


\subsection{Inner products}

\begin{definition}
\label{def.unitary.innerprod.innerprod}For any two vectors $x=\left(
\begin{array}
[c]{c}%
x_{1}\\
x_{2}\\
\vdots\\
x_{n}%
\end{array}
\right)  \in\mathbb{C}^{n}$ and $y=\left(
\begin{array}
[c]{c}%
y_{1}\\
y_{2}\\
\vdots\\
y_{n}%
\end{array}
\right)  \in\mathbb{C}^{n}$, we define the scalar%
\begin{equation}
\left\langle x,y\right\rangle :=x_{1}\overline{y_{1}}+x_{2}\overline{y_{2}%
}+\cdots+x_{n}\overline{y_{n}}\in\mathbb{C}
\label{eq.def.unitary.innerprod.innerprod.def}%
\end{equation}
(where $\overline{z}$ denotes the complex conjugate of a $z\in\mathbb{C}$).
This scalar $\left\langle x,y\right\rangle $ is called the \emph{inner
product} (or \emph{dot product}) of $x$ and $y$.
\end{definition}

\begin{example}
If $x=\left(
\begin{array}
[c]{c}%
1+i\\
2+3i
\end{array}
\right)  \in\mathbb{C}^{2}$ and $y=\left(
\begin{array}
[c]{c}%
-i\\
4+i
\end{array}
\right)  \in\mathbb{C}^{2}$, then%
\begin{align*}
\left\langle x,y\right\rangle  &  =\left(  1+i\right)  \left(  \overline
{-i}\right)  +\left(  2+3i\right)  \left(  \overline{4+i}\right) \\
&  =\left(  1+i\right)  i+\left(  2+3i\right)  \left(  4-i\right) \\
&  =i-1+8-2i+12i+3=10+11i.
\end{align*}

\end{example}

Some warnings about the literature are in order:

\begin{itemize}
\item Some authors (e.g., Treil in \cite{Treil15}) write $\left(  x,y\right)
$ instead of $\left\langle x,y\right\rangle $ for the inner product of $x$ and
$y$. This can be rather confusing, since $\left(  x,y\right)  $ also means the
pair consisting of $x$ and $y$.

\item The notation $\left\langle x,y\right\rangle $, too, can mean something
different in certain texts (namely, the span of $x$ and $y$); however, it
won't have this second meaning in our course.

\item If I am not mistaken, Definition \ref{def.unitary.innerprod.innerprod}
is also not the only game in town. Some authors follow a competing standard,
which causes their $\left\langle x,y\right\rangle $ to be what we would denote
$\left\langle y,x\right\rangle $.

\item Finally, the word \textquotedblleft dot product\textquotedblright\ often
means the analogue of $\left\langle x,y\right\rangle $ that does not use
complex conjugation (i.e., that replaces
(\ref{eq.def.unitary.innerprod.innerprod.def}) by $\left\langle
x,y\right\rangle :=x_{1}y_{1}+x_{2}y_{2}+\cdots+x_{n}y_{n}$). This convention
is used mostly in abstract algebra, where complex conjugation is not
considered intrinsic to the number system. We will not use this convention.
For vectors with real entries, the distinction disappears, since
$\overline{\lambda}=\lambda$ for any $\lambda\in\mathbb{R}$.
\end{itemize}

\begin{definition}
\label{def.unitary.innerprod.ystar}For any column vector $y=\left(
\begin{array}
[c]{c}%
y_{1}\\
y_{2}\\
\vdots\\
y_{n}%
\end{array}
\right)  \in\mathbb{C}^{n}$, we define the row vector
\[
y^{\ast}:=\left(
\begin{array}
[c]{cccc}%
\overline{y_{1}} & \overline{y_{2}} & \cdots & \overline{y_{n}}%
\end{array}
\right)  \in\mathbb{C}^{1\times n}.
\]

\end{definition}

\begin{proposition}
\label{prop.unitary.innerprod.props}Let $x\in\mathbb{C}^{n}$ and
$y\in\mathbb{C}^{n}$. Then:

\textbf{(a)} We have $\left\langle x,y\right\rangle =y^{\ast}x$.

\textbf{(b)} We have $\left\langle x,y\right\rangle =\overline{\left\langle
y,x\right\rangle }$.

\textbf{(c)} We have $\left\langle x+x^{\prime},y\right\rangle =\left\langle
x,y\right\rangle +\left\langle x^{\prime},y\right\rangle $ for any $x^{\prime
}\in\mathbb{C}^{n}$.

\textbf{(d)} We have $\left\langle x,y+y^{\prime}\right\rangle =\left\langle
x,y\right\rangle +\left\langle x,y^{\prime}\right\rangle $ for any $y^{\prime
}\in\mathbb{C}^{n}$.

\textbf{(e)} We have $\left\langle \lambda x,y\right\rangle =\lambda
\left\langle x,y\right\rangle $ for any $\lambda\in\mathbb{C}$.

\textbf{(f)} We have $\left\langle x,\lambda y\right\rangle =\overline
{\lambda}\left\langle x,y\right\rangle $ for any $\lambda\in\mathbb{C}$.

\textbf{(g)} We have $\left\langle x-x^{\prime},y\right\rangle =\left\langle
x,y\right\rangle -\left\langle x^{\prime},y\right\rangle $ for any $x^{\prime
}\in\mathbb{C}^{n}$.

\textbf{(h)} We have $\left\langle x,y-y^{\prime}\right\rangle =\left\langle
x,y\right\rangle -\left\langle x,y^{\prime}\right\rangle $ for any $y^{\prime
}\in\mathbb{C}^{n}$.

\textbf{(i)} We have $\left\langle \sum_{i=1}^{k}\lambda_{i}x_{i}%
,y\right\rangle =\sum_{i=1}^{k}\lambda_{i}\left\langle x_{i},y\right\rangle $
for any $k\in\mathbb{N}$, any $x_{1},x_{2},\ldots,x_{k}\in\mathbb{C}^{n}$ and
any $\lambda_{1},\lambda_{2},\ldots,\lambda_{k}\in\mathbb{C}$.

\textbf{(j)} We have $\left\langle x,\sum_{i=1}^{k}\lambda_{i}y_{i}%
\right\rangle =\sum_{i=1}^{k}\overline{\lambda_{i}}\left\langle x,y_{i}%
\right\rangle $ for any $k\in\mathbb{N}$, any $y_{1},y_{2},\ldots,y_{k}%
\in\mathbb{C}^{n}$ and any $\lambda_{1},\lambda_{2},\ldots,\lambda_{k}%
\in\mathbb{C}$.
\end{proposition}

\begin{proof}
Parts \textbf{(a)} till \textbf{(h)} are straightforward computations using
Definition \ref{def.unitary.innerprod.innerprod}, since

\begin{itemize}
\item the multiplication in $\mathbb{C}$ is commutative;

\item we have $\overline{\overline{z}}=z$ for any $z\in\mathbb{C}$.
\end{itemize}

Parts \textbf{(i)} and \textbf{(j)} follow from parts \textbf{(c)},
\textbf{(d)}, \textbf{(e)} and \textbf{(f)} by induction on $k$.
\end{proof}

\begin{proposition}
\label{prop.unitary.innerprod.pos}Let $x\in\mathbb{C}^{n}$. Then:

\textbf{(a)} The number $\left\langle x,x\right\rangle $ is a nonnegative real.

\textbf{(b)} We have $\left\langle x,x\right\rangle >0$ whenever $x\neq0$.
\end{proposition}

\begin{proof}
Write $x$ as $x=\left(
\begin{array}
[c]{cccc}%
x_{1} & x_{2} & \cdots & x_{n}%
\end{array}
\right)  ^{T}$. Then, the definition of $\left\langle x,x\right\rangle $
yields%
\begin{align*}
\left\langle x,x\right\rangle  &  =x_{1}\overline{x_{1}}+x_{2}\overline{x_{2}%
}+\cdots+x_{n}\overline{x_{n}}\\
&  =\left\vert x_{1}\right\vert ^{2}+\left\vert x_{2}\right\vert ^{2}%
+\cdots+\left\vert x_{n}\right\vert ^{2},
\end{align*}
since any complex number $z$ satisfies $z\overline{z}=\left\vert z\right\vert
^{2}$. Since the absolute values $\left\vert x_{1}\right\vert ,\left\vert
x_{2}\right\vert ,\ldots,\left\vert x_{n}\right\vert $ are real, this yields
immediately that $\left\langle x,x\right\rangle $ is a nonnegative real. Thus,
Proposition \ref{prop.unitary.innerprod.pos} \textbf{(a)} is proved.
Furthermore, if $x\neq0$, then at least one $i\in\left[  n\right]  $ satisfies
$x_{i}\neq0$ and therefore $\left\vert x_{i}\right\vert ^{2}>0$; but this
entails $\left\langle x,x\right\rangle =\left\vert x_{1}\right\vert
^{2}+\left\vert x_{2}\right\vert ^{2}+\cdots+\left\vert x_{n}\right\vert
^{2}>0$. This proves Proposition \ref{prop.unitary.innerprod.pos} \textbf{(b)}.
\end{proof}

\begin{definition}
\label{def.unitary.innerprod.len}Let $x\in\mathbb{C}^{n}$. We define the
\emph{length} (aka \emph{norm}) of $x$ to be the nonnegative real number
\[
\left\vert \left\vert x\right\vert \right\vert :=\sqrt{\left\langle
x,x\right\rangle }.
\]
This is well-defined, since Proposition \ref{prop.unitary.innerprod.pos}
\textbf{(a)} says that $\left\langle x,x\right\rangle $ is a nonnegative real.
\end{definition}

\begin{example}
If $x=\left(
\begin{array}
[c]{c}%
1+i\\
3-2i
\end{array}
\right)  \in\mathbb{C}^{2}$, then%
\begin{align*}
\left\langle x,x\right\rangle  &  =\left(  1+i\right)  \left(  \overline
{1+i}\right)  +\left(  3-2i\right)  \left(  \overline{3+2i}\right)  =\left(
1+i\right)  \left(  1-i\right)  +\left(  3-2i\right)  \left(  3+2i\right) \\
&  =1+1+9+4=15
\end{align*}
and thus $\left\vert \left\vert x\right\vert \right\vert =\sqrt{\left\langle
x,x\right\rangle }=\sqrt{15}$.
\end{example}

\begin{proposition}
For any $\lambda\in\mathbb{C}$ and $x\in\mathbb{C}^{n}$, we have $\left\vert
\left\vert \lambda x\right\vert \right\vert =\left\vert \lambda\right\vert
\cdot\left\vert \left\vert x\right\vert \right\vert $.
\end{proposition}

\begin{proof}
Straightforward.
\end{proof}

\begin{exercise}
\fbox{3} Let $x\in\mathbb{C}^{n}$ and $y\in\mathbb{C}^{n}$. Prove that
\[
\left\vert \left\vert x+y\right\vert \right\vert ^{2}-\left\vert \left\vert
x\right\vert \right\vert ^{2}-\left\vert \left\vert y\right\vert \right\vert
^{2}=\left\langle x,y\right\rangle +\left\langle y,x\right\rangle
=2\cdot\operatorname*{Re}\left\langle x,y\right\rangle .
\]
Here, $\operatorname*{Re}z$ denotes the real part of any complex number $z$.
\end{exercise}

\subsection{Orthogonality and orthonormality}

We shall now define orthogonality first for two vectors, then for any tuple of vectors.

\begin{definition}
\label{def.unitary.innerprod.orth}Let $x\in\mathbb{C}^{n}$ and $y\in
\mathbb{C}^{n}$ be two vectors. We say that $x$ is \emph{orthogonal} to $y$ if
and only if $\left\langle x,y\right\rangle =0$. The shorthand notation for
this is \textquotedblleft$x\perp y$\textquotedblright.
\end{definition}

The relation $\perp$ is symmetric:

\begin{proposition}
\label{prop.unitary.innerprod.orth-symm}Let $x\in\mathbb{C}^{n}$ and
$y\in\mathbb{C}^{n}$ be two vectors. Then, $x\perp y$ holds if and only if
$y\perp x$.
\end{proposition}

\begin{proof}
Follows from Proposition \ref{prop.unitary.innerprod.props} \textbf{(b)}.
\end{proof}

\begin{definition}
Let $\left(  u_{1},u_{2},\ldots,u_{k}\right)  $ be a tuple of vectors in
$\mathbb{C}^{n}$. Then:

\textbf{(a)} We say that the tuple $\left(  u_{1},u_{2},\ldots,u_{k}\right)  $
is \emph{orthogonal} if we have
\[
u_{p}\perp u_{q}\ \ \ \ \ \ \ \ \ \ \text{whenever }p\neq q.
\]


\textbf{(b)} We say that the tuple $\left(  u_{1},u_{2},\ldots,u_{k}\right)  $
is \emph{orthonormal} if it is orthogonal \textbf{and} satisfies%
\[
\left\vert \left\vert u_{1}\right\vert \right\vert =\left\vert \left\vert
u_{2}\right\vert \right\vert =\cdots=\left\vert \left\vert u_{k}\right\vert
\right\vert =1.
\]


\textbf{(c)} We note that the orthogonality and the orthonormality of a tuple
are preserved when the entries of the tuple are permuted. Thus, we can extend
both notions (\textquotedblleft orthogonal\textquotedblright\ and
\textquotedblleft orthonormal\textquotedblright) to finite sets of vectors in
$\mathbb{C}^{n}$: A set $\left\{  u_{1},u_{2},\ldots,u_{k}\right\}  $ of
vectors in $\mathbb{C}^{n}$ (with $u_{1},u_{2},\ldots,u_{k}$ being distinct)
is said to be \emph{orthogonal} (or \emph{orthonormal}, respectively) if and
only if the tuple $\left(  u_{1},u_{2},\ldots,u_{k}\right)  $ is orthogonal
(resp., orthonormal).
\end{definition}

\begin{example}
\textbf{(a)} The tuple $\left(  \left(
\begin{array}
[c]{c}%
1\\
0\\
0
\end{array}
\right)  ,\left(
\begin{array}
[c]{c}%
0\\
1\\
0
\end{array}
\right)  ,\left(
\begin{array}
[c]{c}%
0\\
0\\
1
\end{array}
\right)  \right)  $ of vectors in $\mathbb{C}^{3}$ is orthonormal. It is also
a basis of $\mathbb{C}^{3}$, and known as the \emph{standard basis}.

\textbf{(b)} More generally: Let $n\in\mathbb{N}$. Let $e_{1},e_{2}%
,\ldots,e_{n}\in\mathbb{C}^{n}$ be the vectors defined by%
\[
e_{i}=\underbrace{\left(
\begin{array}
[c]{ccccccccc}%
0 & 0 & \cdots & 0 & 1 & 0 & 0 & \cdots & 0
\end{array}
\right)  ^{T}}_{\substack{\text{the }1\text{ is in the }i\text{-th
position;}\\\text{all other entries are }0}}.
\]
Then, $\left(  e_{1},e_{2},\ldots,e_{n}\right)  $ is an orthonormal basis of
$\mathbb{C}^{n}$, and is known as the \emph{standard basis} of $\mathbb{C}%
^{n}$.

\textbf{(c)} The pair $\left(  \left(
\begin{array}
[c]{c}%
1\\
-i\\
2
\end{array}
\right)  ,\left(
\begin{array}
[c]{c}%
0\\
2i\\
1
\end{array}
\right)  \right)  $ of vectors in $\mathbb{C}^{3}$ is orthogonal (but not
orthonormal). Indeed,%
\[
\left\langle \left(
\begin{array}
[c]{c}%
1\\
-i\\
2
\end{array}
\right)  ,\left(
\begin{array}
[c]{c}%
0\\
2i\\
1
\end{array}
\right)  \right\rangle =1\cdot\overline{0}+\left(  -i\right)  \cdot
\overline{2i}+2\cdot\overline{1}=0-2+2=0.
\]


\textbf{(d)} The pair $\left(  \dfrac{1}{\sqrt{6}}\left(
\begin{array}
[c]{c}%
1\\
-i\\
2
\end{array}
\right)  ,\dfrac{1}{\sqrt{5}}\left(
\begin{array}
[c]{c}%
0\\
2i\\
1
\end{array}
\right)  \right)  $ of vectors in $\mathbb{C}^{3}$ is orthonormal. (This is
just the previous pair, with each vector scaled so that its length becomes $1$.)
\end{example}

\begin{proposition}
\label{prop.unitary.innerprod.orth-norm}Let $\left(  u_{1},u_{2},\ldots
,u_{k}\right)  $ be an orthogonal tuple of nonzero vectors in $\mathbb{C}^{n}%
$. Then, the tuple%
\[
\left(  \dfrac{1}{\left\vert \left\vert u_{1}\right\vert \right\vert }%
u_{1},\ \ \dfrac{1}{\left\vert \left\vert u_{2}\right\vert \right\vert }%
u_{2},\ \ \ldots,\ \ \dfrac{1}{\left\vert \left\vert u_{k}\right\vert
\right\vert }u_{k}\right)
\]
is orthonormal.
\end{proposition}

\begin{proof}
Straightforward.
\end{proof}

\begin{proposition}
\label{prop.unitary.orthog.indep}Any orthogonal tuple of nonzero vectors in
$\mathbb{C}^{n}$ is linearly independent.
\end{proposition}

\begin{proof}
Let $\left(  u_{1},u_{2},\ldots,u_{k}\right)  $ be an orthogonal tuple of
nonzero vectors in $\mathbb{C}^{n}$. We must prove that it is linearly independent.

Indeed, for any $i\in\left[  k\right]  $ and any $\lambda_{1},\lambda
_{2},\ldots,\lambda_{k}\in\mathbb{C}$, we have
\begin{align}
&  \left\langle \lambda_{1}u_{1}+\lambda_{2}u_{2}+\cdots+\lambda_{k}%
u_{k},u_{i}\right\rangle \nonumber\\
&  =\lambda_{1}\left\langle u_{1},u_{i}\right\rangle +\lambda_{2}\left\langle
u_{2},u_{i}\right\rangle +\cdots+\lambda_{k}\left\langle u_{k},u_{i}%
\right\rangle \nonumber\\
&  \ \ \ \ \ \ \ \ \ \ \ \ \ \ \ \ \ \ \ \ \left(  \text{by parts \textbf{(c)}
and \textbf{(e)} of Proposition \ref{prop.unitary.innerprod.props}}\right)
\nonumber\\
&  =\lambda_{i}\left\langle u_{i},u_{i}\right\rangle +\sum_{\substack{j\in
\left[  k\right]  ;\\j\neq i}}\lambda_{j}\underbrace{\left\langle u_{j}%
,u_{i}\right\rangle }_{\substack{=0\\\text{(since }u_{j}\perp u_{i}%
\\\text{(because }\left(  u_{1},u_{2},\ldots,u_{k}\right)  \text{
is}\\\text{an orthogonal tuple))}}}\nonumber\\
&  =\lambda_{i}\left\langle u_{i},u_{i}\right\rangle .
\label{pf.prop.unitary.orthog.indep.1}%
\end{align}


For any $i\in\left[  k\right]  $, we have $u_{i}\neq0$ (since $\left(
u_{1},u_{2},\ldots,u_{k}\right)  $ is a tuple of nonzero vectors) and thus
\begin{equation}
\left\langle u_{i},u_{i}\right\rangle >0
\label{pf.prop.unitary.orthog.indep.pos}%
\end{equation}
(by Proposition \ref{prop.unitary.innerprod.pos} \textbf{(b)}, applied to
$x=u_{i}$).

Now, let $\lambda_{1},\lambda_{2},\ldots,\lambda_{k}\in\mathbb{C}$ be such
that $\lambda_{1}u_{1}+\lambda_{2}u_{2}+\cdots+\lambda_{k}u_{k}=0$. Then, for
each $i\in\left[  k\right]  $, we have%
\begin{align*}
\lambda_{i}\left\langle u_{i},u_{i}\right\rangle  &  =\left\langle
\underbrace{\lambda_{1}u_{1}+\lambda_{2}u_{2}+\cdots+\lambda_{k}u_{k}}%
_{=0},u_{i}\right\rangle \ \ \ \ \ \ \ \ \ \ \left(  \text{by
(\ref{pf.prop.unitary.orthog.indep.1})}\right) \\
&  =\left\langle 0,u_{i}\right\rangle =0
\end{align*}
and therefore $\lambda_{i}=0$ (indeed, we can divide by $\left\langle
u_{i},u_{i}\right\rangle $, because of (\ref{pf.prop.unitary.orthog.indep.pos})).

Forget that we fixed $\lambda_{1},\lambda_{2},\ldots,\lambda_{k}$. We thus
have shown that if $\lambda_{1},\lambda_{2},\ldots,\lambda_{k}\in\mathbb{C}$
are such that $\lambda_{1}u_{1}+\lambda_{2}u_{2}+\cdots+\lambda_{k}u_{k}=0$,
then we have $\lambda_{i}=0$ for each $i\in\left[  k\right]  $. In other
words, $\left(  u_{1},u_{2},\ldots,u_{k}\right)  $ is linearly independent.
This proves Proposition \ref{prop.unitary.orthog.indep}.
\end{proof}

The following simple lemma will be used further below:

\begin{lemma}
\label{lem.unitary.orthog.one-more}Let $k<n$. Let $a_{1},a_{2},\ldots,a_{k}$
be $k$ vectors in $\mathbb{C}^{n}$. Then, there exists a nonzero vector
$b\in\mathbb{C}^{n}$ that is orthogonal to each of $a_{1},a_{2},\ldots,a_{k}$.
\end{lemma}

\begin{proof}
Write each vector $a_{i}$ as $a_{i}=\left(
\begin{array}
[c]{cccc}%
a_{i,1} & a_{i,2} & \cdots & a_{i,n}%
\end{array}
\right)  ^{T}$. Now, consider an arbitrary vector $b=\left(
\begin{array}
[c]{cccc}%
b_{1} & b_{2} & \cdots & b_{n}%
\end{array}
\right)  ^{T}\in\mathbb{C}^{n}$, whose entries $b_{1},b_{2},\ldots,b_{n}$ are
so far undetermined. This new vector $b$ is orthogonal to each of $a_{1}%
,a_{2},\ldots,a_{k}$ if and only if it satisfies%
\[
\left\langle b,a_{i}\right\rangle =0\ \ \ \ \ \ \ \ \ \ \text{for all }%
i\in\left[  k\right]  .
\]
In other words, this new vector $b$ is orthogonal to each of $a_{1}%
,a_{2},\ldots,a_{k}$ if and only if it satisfies%
\[
b_{1}\overline{a_{i,1}}+b_{2}\overline{a_{i,2}}+\cdots+b_{n}\overline{a_{i,n}%
}=0\ \ \ \ \ \ \ \ \ \ \text{for all }i\in\left[  k\right]
\]
(since $\left\langle b,a_{i}\right\rangle =b_{1}\overline{a_{i,1}}%
+b_{2}\overline{a_{i,2}}+\cdots+b_{n}\overline{a_{i,n}}$ for each $i\in\left[
k\right]  $). In other words, this new vector $b$ is orthogonal to each of
$a_{1},a_{2},\ldots,a_{k}$ if and only if it satisfies the system of equations%
\[
\left\{
\begin{array}
[c]{c}%
b_{1}\overline{a_{1,1}}+b_{2}\overline{a_{1,2}}+\cdots+b_{n}\overline{a_{1,n}%
}=0;\\
b_{1}\overline{a_{2,1}}+b_{2}\overline{a_{2,2}}+\cdots+b_{n}\overline{a_{2,n}%
}=0;\\
\cdots;\\
b_{1}\overline{a_{k,1}}+b_{2}\overline{a_{k,2}}+\cdots+b_{n}\overline{a_{k,n}%
}=0.
\end{array}
\right.
\]
But this is a system of $k$ homogeneous linear equations in the $n$ unknowns
$b_{1},b_{2},\ldots,b_{n}$, and thus (by a classical fact in linear
algebra\footnote{The fact we are using here is the following: If $p$ and $q$
are two integers such that $0\leq p<q$, then any system of $p$ homogeneous
linear equations in $q$ unknowns has at least one nonzero solution. Rewritten
in terms of matrices, this is saying that if $p$ and $q$ are two integers such
that $0\leq p<q$, then any $p\times q$-matrix has a nonzero vector in its
kernel (= nullspace). For a proof, see, e.g., \cite[Remark 8.9]{Strick20} or
(rewritten in the language of linear maps) \cite[Corollary 6.5.3 item
1]{LaNaSc16}.}) has at least one nonzero solution (since $k<n$). In other
words, there exists at least one nonzero vector $b=\left(
\begin{array}
[c]{cccc}%
b_{1} & b_{2} & \cdots & b_{n}%
\end{array}
\right)  ^{T}\in\mathbb{C}^{n}$ that is orthogonal to each of $a_{1}%
,a_{2},\ldots,a_{k}$. This proves Lemma \ref{lem.unitary.orthog.one-more}.
\medskip

\begin{fineprint}
Here is a neater way to state the same argument: We define a map
$f:\mathbb{C}^{n}\rightarrow\mathbb{C}^{k}$ by setting%
\[
f\left(  w\right)  =\left(
\begin{array}
[c]{c}%
\left\langle w,a_{1}\right\rangle \\
\left\langle w,a_{2}\right\rangle \\
\vdots\\
\left\langle w,a_{k}\right\rangle
\end{array}
\right)  \ \ \ \ \ \ \ \ \ \ \text{for each }w\in\mathbb{C}^{n}.
\]
It is easy to see that this map $f$ is $\mathbb{C}$-linear. (Indeed,
Proposition \ref{prop.unitary.innerprod.props} \textbf{(c)} shows that every
two vectors $x,x^{\prime}\in\mathbb{C}^{n}$ and every $i\in\left[  k\right]  $
satisfy $\left\langle x+x^{\prime},a_{i}\right\rangle =\left\langle
x,a_{i}\right\rangle +\left\langle x^{\prime},a_{i}\right\rangle $; therefore,
every two vectors $x,x^{\prime}\in\mathbb{C}^{n}$ satisfy $f\left(
x+x^{\prime}\right)  =f\left(  x\right)  +f\left(  x^{\prime}\right)  $.
Similarly, Proposition \ref{prop.unitary.innerprod.props} \textbf{(e)} can be
used to show that $f\left(  \lambda x\right)  =\lambda f\left(  x\right)  $
for each $\lambda\in\mathbb{C}$ and $x\in\mathbb{C}^{n}$. Hence, $f$ is
$\mathbb{C}$-linear.)

Now, we know that $f$ is a $\mathbb{C}$-linear map from $\mathbb{C}^{n}$ to
$\mathbb{C}^{k}$. Hence, the rank-nullity theorem (see, e.g., \cite[Chapter 2,
Theorem 7.2]{Treil15} or \cite[Chapter II, Corollary 2.15]{Knapp1} or
\cite[Proposition 3.3.35]{Goodman}) yields that%
\[
n=\dim\left(  \operatorname*{Ker}f\right)  +\dim\left(  \operatorname{Im}%
f\right)  ,
\]
where $\operatorname*{Ker}f$ denotes the kernel of $f$ (that is, the subspace
of $\mathbb{C}^{n}$ that consists of all vectors $v\in\mathbb{C}^{n}$
satisfying $f\left(  v\right)  =0$), and where $\operatorname{Im}f$ denotes
the image\footnote{also known as \textquotedblleft range\textquotedblright} of
$f$ (that is, the subspace of $\mathbb{C}^{k}$ consisting of all vectors of
the form $f\left(  v\right)  $ with $v\in\mathbb{C}^{n}$). Therefore,%
\[
\dim\left(  \operatorname*{Ker}f\right)  =n-\dim\left(  \operatorname{Im}%
f\right)  .
\]
However, $\operatorname{Im}f$ is a vector subspace of $\mathbb{C}^{k}$, and
thus has dimension $\leq k$. Thus, $\dim\left(  \operatorname{Im}f\right)
\leq k<n$, so that%
\[
\dim\left(  \operatorname*{Ker}f\right)  =n-\underbrace{\dim\left(
\operatorname{Im}f\right)  }_{<n}>n-n=0.
\]
This shows that the vector space $\operatorname*{Ker}f$ contains at least one
nonzero vector $b$. Consider this $b$. Thus, $b\in\operatorname*{Ker}%
f\subseteq\mathbb{C}^{n}$.

However, $b\in\operatorname*{Ker}f$ shows that $f\left(  b\right)  =0$. But
the definition of $f$ yields $f\left(  b\right)  =\left(
\begin{array}
[c]{c}%
\left\langle b,a_{1}\right\rangle \\
\left\langle b,a_{2}\right\rangle \\
\vdots\\
\left\langle b,a_{k}\right\rangle
\end{array}
\right)  $. Thus, $\left(
\begin{array}
[c]{c}%
\left\langle b,a_{1}\right\rangle \\
\left\langle b,a_{2}\right\rangle \\
\vdots\\
\left\langle b,a_{k}\right\rangle
\end{array}
\right)  =f\left(  b\right)  =0$. In other words, each $i\in\left[  k\right]
$ satisfies $\left\langle b,a_{i}\right\rangle =0$. In other words, each
$i\in\left[  k\right]  $ satisfies $b\perp a_{i}$. In other words, $b$ is
orthogonal to each of $a_{1},a_{2},\ldots,a_{k}$. Thus, we have found a
nonzero vector $b\in\mathbb{C}^{n}$ that is orthogonal to each of $a_{1}%
,a_{2},\ldots,a_{k}$. This proves Lemma \ref{lem.unitary.orthog.one-more}.
\end{fineprint}
\end{proof}

\begin{corollary}
\label{cor.unitary.orthog-extend}Let $\left(  u_{1},u_{2},\ldots,u_{k}\right)
$ be an orthogonal $k$-tuple of nonzero vectors in $\mathbb{C}^{n}$. Then, we
have $k\leq n$, and we can find $n-k$ further nonzero vectors $u_{k+1}%
,u_{k+2},\ldots,u_{n}$ such that $\left(  u_{1},u_{2},\ldots,u_{n}\right)  $
is an orthogonal basis of $\mathbb{C}^{n}$.
\end{corollary}

\begin{exercise}
\fbox{2} Prove Corollary \ref{cor.unitary.orthog-extend}.
\end{exercise}

\subsection{Conjugate transposes}

The following definition generalizes Definition
\ref{def.unitary.innerprod.ystar}:

\begin{definition}
\label{def.unitary.innerprod.A*}Let $A=\left(
\begin{array}
[c]{ccc}%
a_{1,1} & \cdots & a_{1,m}\\
\vdots & \ddots & \vdots\\
a_{n,1} & \cdots & a_{n,m}%
\end{array}
\right)  \in\mathbb{C}^{n\times m}$ be any $n\times m$-matrix. Then, we define
the $m\times n$-matrix%
\[
A^{\ast}:=\left(
\begin{array}
[c]{ccc}%
\overline{a_{1,1}} & \cdots & \overline{a_{n,1}}\\
\vdots & \ddots & \vdots\\
\overline{a_{1,m}} & \cdots & \overline{a_{n,m}}%
\end{array}
\right)  \in\mathbb{C}^{m\times n}.
\]
This matrix $A^{\ast}$ is called the \emph{conjugate transpose} of $A$.
\end{definition}

This conjugate transpose $A^{\ast}$ can thus be obtained from the usual
transpose $A^{T}$ by conjugating all entries.

\begin{example}
$\left(
\begin{array}
[c]{ccc}%
1+i & 2-3i & 5i\\
6 & 2+4i & 10-i
\end{array}
\right)  ^{\ast}=\left(
\begin{array}
[c]{cc}%
1-i & 6\\
2+3i & 2-4i\\
-5i & 10+i
\end{array}
\right)  $.
\end{example}

In the olden days, the conjugate transpose of a matrix was also known as the
\textquotedblleft adjoint\textquotedblright\ of $A$. Unsurprisingly, this word
has at least one other meaning, which opens the door to a lot of unwanted
confusion; thus we will speak of the \textquotedblleft conjugate
transpose\textquotedblright\ instead.

Some authors use the alternative notation $A^{\dag}$ (read \textquotedblleft%
$A$ dagger\textquotedblright) for $A^{\ast}$. (The Wikipedia suggests calling
it the \textquotedblleft bedaggered matrix $A$\textquotedblright, although I
am not aware of anyone using this terminology outside of the Wikipedia.)

The following rules for conjugate transposes are straightforward to check:

\begin{proposition}
\label{prop.unitary.(AB)*}\textbf{(a)} If $A\in\mathbb{C}^{n\times m}$ and
$B\in\mathbb{C}^{n\times m}$ are two matrices, then $\left(  A+B\right)
^{\ast}=A^{\ast}+B^{\ast}$.

\textbf{(b)} If $A\in\mathbb{C}^{n\times m}$ and $\lambda\in\mathbb{C}$, then
$\left(  \lambda A\right)  ^{\ast}=\overline{\lambda}A^{\ast}$.

\textbf{(c)} If $A\in\mathbb{C}^{n\times m}$ and $B\in\mathbb{C}^{m\times k}$
are two matrices, then $\left(  AB\right)  ^{\ast}=B^{\ast}A^{\ast}$.

\textbf{(d)} If $A\in\mathbb{C}^{n\times m}$, then $\left(  A^{\ast}\right)
^{\ast}=A$.
\end{proposition}

\subsection{Isometries}

\begin{definition}
\label{def.unitary.innerprod.isometry}An $n\times k$-matrix $A$ is said to be
an \emph{isometry} if $A^{\ast}A=I_{k}$.
\end{definition}

\begin{proposition}
\label{prop.unitary.innerprod.isometry.2}An $n\times k$-matrix $A$ is an
isometry if and only if its columns form an orthonormal tuple of vectors.
\end{proposition}

\begin{proof}
Let $A$ be an $n\times k$-matrix with columns $a_{1},a_{2},\ldots,a_{k}$ from
left to right. Therefore,%
\[
A=\left(
\begin{array}
[c]{ccc}%
\mid &  & \mid\\
a_{1} & \cdots & a_{k}\\
\mid &  & \mid
\end{array}
\right)  \ \ \ \ \ \ \ \ \ \ \text{and thus}\ \ \ \ \ \ \ \ \ \ A^{\ast
}=\left(
\begin{array}
[c]{ccc}%
\text{---} & a_{1}^{\ast} & \text{---}\\
& \vdots & \\
\text{---} & a_{k}^{\ast} & \text{---}%
\end{array}
\right)  .
\]
Hence,%
\begin{align*}
A^{\ast}A  &  =\left(
\begin{array}
[c]{cccc}%
a_{1}^{\ast}a_{1} & a_{1}^{\ast}a_{2} & \cdots & a_{1}^{\ast}a_{k}\\
a_{2}^{\ast}a_{1} & a_{2}^{\ast}a_{2} & \cdots & a_{2}^{\ast}a_{k}\\
\vdots & \vdots & \ddots & \vdots\\
a_{k}^{\ast}a_{1} & a_{k}^{\ast}a_{2} & \cdots & a_{k}^{\ast}a_{k}%
\end{array}
\right) \\
&  =\left(
\begin{array}
[c]{cccc}%
\left\vert \left\vert a_{1}\right\vert \right\vert ^{2} & \left\langle
a_{1},a_{2}\right\rangle  & \cdots & \left\langle a_{1},a_{k}\right\rangle \\
\left\langle a_{2},a_{1}\right\rangle  & \left\vert \left\vert a_{2}%
\right\vert \right\vert ^{2} & \cdots & \left\langle a_{2},a_{k}\right\rangle
\\
\vdots & \vdots & \ddots & \vdots\\
\left\langle a_{k},a_{1}\right\rangle  & \left\langle a_{k},a_{2}\right\rangle
& \cdots & \left\vert \left\vert a_{k}\right\vert \right\vert ^{2}%
\end{array}
\right)  .
\end{align*}
On the other hand,%
\[
I_{k}=\left(
\begin{array}
[c]{cccc}%
1 & 0 & \cdots & 0\\
0 & 1 & \cdots & 0\\
\vdots & \vdots & \ddots & \vdots\\
0 & 0 & \cdots & 1
\end{array}
\right)  .
\]
Thus, $A^{\ast}A=I_{k}$ holds if and only if we have%
\[
\left\langle a_{p},a_{q}\right\rangle =0\ \ \ \ \ \ \ \ \ \ \text{for all
}p\neq q
\]
and%
\[
\left\vert \left\vert a_{p}\right\vert \right\vert ^{2}%
=1\ \ \ \ \ \ \ \ \ \ \text{for each }p.
\]
In other words, $A^{\ast}A=I_{k}$ holds if and only if we have%
\[
a_{p}\perp a_{q}\ \ \ \ \ \ \ \ \ \ \text{for all }p\neq q
\]
and%
\[
\left\vert \left\vert a_{1}\right\vert \right\vert =\left\vert \left\vert
a_{2}\right\vert \right\vert =\cdots=\left\vert \left\vert a_{k}\right\vert
\right\vert =1.
\]
In other words, $A$ is an isometry if and only if $\left(  a_{1},a_{2}%
,\ldots,a_{k}\right)  $ is orthonormal. This proves Proposition
\ref{prop.unitary.innerprod.isometry.2}.
\end{proof}

Isometries are called isometries because they preserve lengths:

\begin{proposition}
\label{prop.unitary.innerprod.isometry.len}Let $A\in\mathbb{C}^{n\times k}$ be
an isometry. Then, each $x\in\mathbb{C}^{k}$ satisfies $\left\vert \left\vert
Ax\right\vert \right\vert =\left\vert \left\vert x\right\vert \right\vert $.
\end{proposition}

\begin{proof}
We have $A^{\ast}A=I_{k}$ (since $A$ is an isometry). Let $x\in\mathbb{C}^{k}%
$. Then, the definition of $\left\vert \left\vert Ax\right\vert \right\vert $
yields $\left\vert \left\vert Ax\right\vert \right\vert =\sqrt{\left\langle
Ax,Ax\right\rangle }$. Hence,%
\begin{align}
\left\vert \left\vert Ax\right\vert \right\vert ^{2}  &  =\left\langle
Ax,Ax\right\rangle \nonumber\\
&  =\underbrace{\left(  Ax\right)  ^{\ast}}_{\substack{=x^{\ast}A^{\ast
}\\\text{(by Proposition \ref{prop.unitary.(AB)*} \textbf{(c)})}%
}}Ax\ \ \ \ \ \ \ \ \ \ \left(  \text{by Proposition
\ref{prop.unitary.innerprod.props} \textbf{(a)}}\right) \nonumber\\
&  =x^{\ast}A^{\ast}Ax\label{pf.prop.unitary.innerprod.isometry.len.1}\\
&  =x^{\ast}x\ \ \ \ \ \ \ \ \ \ \left(  \text{since }A^{\ast}A=I_{k}\right)
\nonumber\\
&  =\left\langle x,x\right\rangle \ \ \ \ \ \ \ \ \ \ \left(  \text{by
Proposition \ref{prop.unitary.innerprod.props} \textbf{(a)}}\right)
\nonumber\\
&  =\left\vert \left\vert x\right\vert \right\vert ^{2}\nonumber
\end{align}
(since the definition of $\left\vert \left\vert x\right\vert \right\vert $
yields $\left\vert \left\vert x\right\vert \right\vert =\sqrt{\left\langle
x,x\right\rangle }$). In other words, we have $\left\vert \left\vert
Ax\right\vert \right\vert =\left\vert \left\vert x\right\vert \right\vert $
(since $\left\vert \left\vert Ax\right\vert \right\vert $ and $\left\vert
\left\vert x\right\vert \right\vert $ are nonnegative reals). This proves
Proposition \ref{prop.unitary.innerprod.isometry.len}.
\end{proof}

\begin{remark}
Another warning on terminology: Some authors (e.g., Conrad in
\cite[\textquotedblleft Isometries\textquotedblright]{Conrad}) use the word
\textquotedblleft isometry\textquotedblright\ in a wider sense than we do.
Namely, they use it for arbitrary maps from $\mathbb{C}^{k}$ to $\mathbb{C}%
^{n}$ that preserve distances. Our isometries can be viewed as \textbf{linear}
isometries in this wider sense, because a matrix $A\in\mathbb{C}^{n\times k}$
corresponds to a linear map from $\mathbb{C}^{k}$ to $\mathbb{C}^{n}$.
However, not all isometries in this wider sense are linear.
\end{remark}

\subsection{Unitary matrices}

\subsubsection{Definition, examples, basic properties}

\begin{definition}
\label{def.unitary.unitary.unitary}A matrix $U\in\mathbb{C}^{n\times k}$ is
said to be \emph{unitary} if and only if both $U$ and $U^{\ast}$ are isometries.
\end{definition}

\begin{example}
\textbf{(a)} The matrix $A=\dfrac{1}{\sqrt{2}}\left(
\begin{array}
[c]{cc}%
1 & 1\\
1 & -1
\end{array}
\right)  $ is unitary. Indeed, it is easy to see that $A^{\ast}A=I_{2}$, so
that $A$ is an isometry. Thus, $A^{\ast}$ is an isometry as well, since
$A^{\ast}=A$. Hence, $A$ is unitary.

\textbf{(b)} A $1\times1$-matrix $\left(
\begin{array}
[c]{c}%
\lambda
\end{array}
\right)  \in\mathbb{C}^{1\times1}$ is unitary if and only if $\left\vert
\lambda\right\vert =1$.

\textbf{(c)} For any $n\in\mathbb{N}$, the identity matrix $I_{n}$ is unitary.

\textbf{(d)} Let $n\in\mathbb{N}$, and let $\sigma$ be a permutation of
$\left[  n\right]  $ (that is, a bijective map from $\left[  n\right]  $ to
$\left[  n\right]  $). Let $P_{\sigma}$ be the \emph{permutation matrix} of
$\sigma$; this is the $n\times n$-matrix whose $\left(  \sigma\left(
j\right)  ,j\right)  $-th entry is $1$ for each $j\in\left[  n\right]  $, and
whose all other entries are $0$. For instance, if $n=3$ and if $\sigma$ is the
cyclic permutation sending $1,2,3$ to $2,3,1$ (respectively), then%
\[
P_{\sigma}=\left(
\begin{array}
[c]{ccc}%
0 & 0 & 1\\
1 & 0 & 0\\
0 & 1 & 0
\end{array}
\right)  .
\]


The permutation matrix $P_{\sigma}$ is always unitary (for any $n$ and any
permutation $\sigma$). Indeed, its conjugate transpose $\left(  P_{\sigma
}\right)  ^{\ast}$ is easily seen to be the permutation matrix $P_{\sigma
^{-1}}$ of the inverse permutation $\sigma^{-1}$; but this latter permutation
matrix $P_{\sigma^{-1}}$ is also the inverse of $P_{\sigma}$.

\textbf{(e)} A diagonal matrix $\operatorname*{diag}\left(  \lambda
_{1},\lambda_{2},\ldots,\lambda_{n}\right)  \in\mathbb{C}^{n\times n}$ is
unitary if and only if its diagonal entries $\lambda_{1},\lambda_{2}%
,\ldots,\lambda_{n}$ lie on the unit circle (i.e., their absolute values
$\left\vert \lambda_{1}\right\vert ,\left\vert \lambda_{2}\right\vert
,\ldots,\left\vert \lambda_{n}\right\vert $ all equal $1$).
\end{example}

Unitary matrices can be characterized in many other ways:

\begin{theorem}
\label{thm.unitary.unitary.eqs}Let $U\in\mathbb{C}^{n\times k}$ be a matrix.
The following six statements are equivalent:

\begin{itemize}
\item $\mathcal{A}$: The matrix $U$ is unitary.

\item $\mathcal{B}$: The matrices $U$ and $U^{\ast}$ are isometries.

\item $\mathcal{C}$: We have $UU^{\ast}=I_{n}$ and $U^{\ast}U=I_{k}$.

\item $\mathcal{D}$: The matrix $U$ is square (that is, $n=k$) and invertible
and satisfies $U^{-1}=U^{\ast}$.

\item $\mathcal{E}$: The columns of $U$ form an orthonormal basis of
$\mathbb{C}^{n}$.

\item $\mathcal{F}$\textit{:} The matrix $U$ is square (that is, $n=k$) and is
an isometry.
\end{itemize}
\end{theorem}

\begin{proof}
The equivalence $\mathcal{A}\Longleftrightarrow\mathcal{B}$ follows
immediately from Definition \ref{def.unitary.unitary.unitary}. The equivalence
$\mathcal{B}\Longleftrightarrow\mathcal{C}$ follows immediately from the
definition of an isometry (since $\left(  U^{\ast}\right)  ^{\ast}=U$). The
implication $\mathcal{D}\Longrightarrow\mathcal{C}$ is obvious. The
implication $\mathcal{C}\Longrightarrow\mathcal{D}$ follows from the known
fact (see, e.g., \cite[Chapter 2, Corollary 3.7]{Treil15}) that every
invertible matrix is square. Let us now prove some of the other implications:

\begin{itemize}
\item $\mathcal{D}\Longrightarrow\mathcal{E}$\textit{:} Assume that statement
$\mathcal{D}$ holds. Then, $U^{\ast}U=I_{k}$ (since $U^{-1}=U^{\ast}$), and
therefore $U$ is an isometry. Hence, Proposition
\ref{prop.unitary.innerprod.isometry.2} shows that the tuple of columns of $U$
is orthonormal. However, the columns of $U$ form a basis of $\mathbb{C}^{n}$
(because $U$ is invertible), and this basis is orthonormal (since we have just
shown that the tuple of columns of $U$ is orthonormal). Thus, statement
$\mathcal{E}$ holds. We have thus proved the implication $\mathcal{D}%
\Longrightarrow\mathcal{E}$.

\item $\mathcal{E}\Longrightarrow\mathcal{D}$\textit{:} Assume that statement
$\mathcal{E}$ holds. Then, the columns of $U$ form an orthonormal basis, hence
an orthogonal tuple. Thus, Proposition \ref{prop.unitary.innerprod.isometry.2}
shows that $U$ is an isometry, so that $U^{\ast}U=I_{k}$. However, $U$ is
invertible because the columns of $U$ form a basis of $\mathbb{C}^{n}$.
Therefore, from $U^{\ast}U=I_{k}$, we obtain $U^{-1}=U^{\ast}$. Finally, the
matrix $U$ is square, since any invertible matrix is square. Thus, statement
$\mathcal{D}$ holds. We have thus proved the implication $\mathcal{E}%
\Longrightarrow\mathcal{D}$.

\item $\mathcal{D}\Longrightarrow\mathcal{F}$\textit{:} The implication
$\mathcal{D}\Longrightarrow\mathcal{F}$ is easy (since $U^{-1}=U^{\ast}$
entails $U^{\ast}U=I_{k}$, which shows that $U$ is an isometry).

\item $\mathcal{F}\Longrightarrow\mathcal{D}$\textit{:} Assume that statement
$\mathcal{F}$ holds. Thus, $U$ is an isometry; that is, we have $U^{\ast
}U=I_{k}=I_{n}$ (since $k=n$). However, it is known\footnote{One part of the
infamous \textquotedblleft inverse matrix theorem\textquotedblright\ that
lists many equivalent conditions for invertibility.} that a square matrix $A$
that has a left inverse (i.e., a further square matrix $B$ satisfying $BA=I$)
must be invertible. We can apply this to the square matrix $U$ (which has a
left inverse, since $U^{\ast}U=I_{n}$), and thus conclude that $U$ is
invertible. Hence, from $U^{\ast}U=I_{n}$, we obtain $U^{-1}=U^{\ast}$.
Therefore, statement $\mathcal{D}$ holds. We have thus proved the implication
$\mathcal{F}\Longrightarrow\mathcal{D}$.
\end{itemize}

Altogether, we have thus proved that all six statements $\mathcal{A}%
,\mathcal{B},\mathcal{C},\mathcal{D},\mathcal{E},\mathcal{F}$ are equivalent.
\end{proof}

Note that Theorem \ref{thm.unitary.unitary.eqs} (specifically, the implication
$\mathcal{A}\Longrightarrow\mathcal{D}$) shows that any unitary matrix is
square. In contrast, an isometry can be rectangular -- but only tall, not
wide, as the following exercise shows:

\begin{exercise}
\label{exe.unitary.isometry-tall}\fbox{1} Let $A\in\mathbb{C}^{n\times k}$ be
an isometry. Show that $n\geq k$.
\end{exercise}

\begin{exercise}
\label{exe.unitary.group}\fbox{3} \textbf{(a)} Prove that the product $AB$ of
two isometries $A\in\mathbb{C}^{n\times m}$ and $B\in\mathbb{C}^{m\times k}$
is always an isometry.

\textbf{(b)} Prove that the product $AB$ of two unitary matrices
$A\in\mathbb{C}^{n\times n}$ and $B\in\mathbb{C}^{n\times n}$ is always unitary.

\textbf{(c)} Prove that the inverse of a unitary matrix $A\in\mathbb{C}%
^{n\times n}$ is always unitary.
\end{exercise}

Exercise \ref{exe.unitary.group} shows that the set of all unitary $n\times
n$-matrices over $\mathbb{C}$ (for a given $n\in\mathbb{N}$) is a group under
multiplication. This group is known as the $n$\emph{-th unitary group}, and is
denoted by $\operatorname*{U}\nolimits_{n}\left(  \mathbb{C}\right)  $.

\begin{exercise}
\label{exe.unitary.det-eval}\fbox{2} Let $U\in\mathbb{C}^{n\times n}$ be a
unitary matrix.

\textbf{(a)} Prove that $\left\vert \det U\right\vert =1$.

\textbf{(b)} Prove that any eigenvalue $\lambda$ of $U$ satisfies $\left\vert
\lambda\right\vert =1$.
\end{exercise}

\subsubsection{Various constructions of unitary matrices}

The next two exercises show some ways to generate unitary matrices:

\begin{exercise}
\label{exe.unitary.house}\fbox{3} Let $w\in\mathbb{C}^{n}$ be a nonzero
vector. Then, $w^{\ast}w=\left\langle w,w\right\rangle >0$ (by Proposition
\ref{prop.unitary.innerprod.pos} \textbf{(b)}). Thus, we can define an
$n\times n$-matrix
\[
U_{w}:=I_{n}-2\left(  w^{\ast}w\right)  ^{-1}ww^{\ast}\in\mathbb{C}^{n\times
n}.
\]
This is called a \emph{Householder matrix}.

Show that this matrix $U_{w}$ is unitary and satisfies $U_{w}^{\ast}=U_{w}$.
\end{exercise}

The next exercise uses the notion of a skew-Hermitian matrix:

\begin{definition}
\label{def.unitary.skew-herm}A matrix $S\in\mathbb{C}^{n\times n}$ is said to
be \emph{skew-Hermitian} if and only if $S^{\ast}=-S$.
\end{definition}

For instance, the matrix $\left(
\begin{array}
[c]{cc}%
i & 1\\
-1 & 0
\end{array}
\right)  $ is skew-Hermitian.

\begin{exercise}
\fbox{5} \label{exe.unitary.skew-herm.1}Let $S\in\mathbb{C}^{n\times n}$ be a
skew-Hermitian matrix.

\textbf{(a)} Prove that the matrix $I_{n}-S$ is invertible.

[\textbf{Hint:} Show first that the matrix $I_{n}+S^{\ast}S$ is invertible,
since each nonzero vector $v\in\mathbb{C}^{n}$ satisfies $v^{\ast}\left(
I_{n}+S^{\ast}S\right)  v=\underbrace{\left\langle v,v\right\rangle }%
_{>0}+\underbrace{\left\langle Sv,Sv\right\rangle }_{\geq0}>0$. Then, expand
the product $\left(  I_{n}-S^{\ast}\right)  \left(  I_{n}-S\right)  $.]

\textbf{(b)} Prove that the matrices $I_{n}+S$ and $\left(  I_{n}-S\right)
^{-1}$ commute (i.e., satisfy $\left(  I_{n}+S\right)  \cdot\left(
I_{n}-S\right)  ^{-1}=\left(  I_{n}-S\right)  ^{-1}\cdot\left(  I_{n}%
+S\right)  $).

\textbf{(c)} Prove that the matrix $U:=\left(  I_{n}-S\right)  ^{-1}%
\cdot\left(  I_{n}+S\right)  $ is unitary.

\textbf{(d)} Prove that the matrix $U+I_{n}$ is invertible.

\textbf{(e)} Prove that $S=\left(  U-I_{n}\right)  \cdot\left(  U+I_{n}%
\right)  ^{-1}$.
\end{exercise}

Exercise \ref{exe.unitary.skew-herm.1} constructs a map\footnote{Recall that
$\operatorname*{U}\nolimits_{n}\left(  \mathbb{C}\right)  $ denotes the $n$-th
unitary group (i.e., the set of all unitary $n\times n$-matrices).}%
\begin{align*}
\left\{  \text{skew-Hermitian matrices in }\mathbb{C}^{n\times n}\right\}   &
\rightarrow\left\{  U\in\operatorname*{U}\nolimits_{n}\left(  \mathbb{C}%
\right)  \text{ }\mid\ U+I_{n}\text{ is invertible}\right\}  ,\\
S  &  \mapsto\left(  I_{n}-S\right)  ^{-1}\cdot\left(  I_{n}+S\right)  ;
\end{align*}
this map is known as the \emph{Cayley parametrization} of the unitary matrices
(and can be seen as an $n$-dimensional generalization of the stereographic
projection from the imaginary axis to the unit circle -- which is what it does
for $n=1$). Exercise \ref{exe.unitary.skew-herm.1} \textbf{(e)} shows that it
is injective. It is not hard to check that it is surjective, too.

How close is the set $\left\{  U\in\operatorname*{U}\nolimits_{n}\left(
\mathbb{C}\right)  \text{ }\mid\ U+I_{n}\text{ is invertible}\right\}  $ to
the whole unitary group $\operatorname*{U}\nolimits_{n}\left(  \mathbb{C}%
\right)  $ ? The answer is that it is almost the entire group
$\operatorname*{U}\nolimits_{n}\left(  \mathbb{C}\right)  $. Here is a
rigorous way to state this:

\begin{exercise}
\fbox{3} Let $A\in\mathbb{C}^{n\times n}$ be a matrix. Prove the following:

\textbf{(a)} If $A$ is unitary, then the matrix $\lambda A$ is unitary for
each $\lambda\in\mathbb{C}$ satisfying $\left\vert \lambda\right\vert =1$.

\textbf{(b)} If $A$ is invertible, then the matrix $\lambda A+I_{n}$ is
invertible for all but finitely many $\lambda\in\mathbb{C}$.

[\textbf{Hint:} The determinant $\det\left(  \lambda A+I_{n}\right)  $ is a
polynomial function in $\lambda$.]

\textbf{(c)} The set $\left\{  U\in\operatorname*{U}\nolimits_{n}\left(
\mathbb{C}\right)  \text{ }\mid\ U+I_{n}\text{ is invertible}\right\}  $ is
dense in $\operatorname*{U}\nolimits_{n}\left(  \mathbb{C}\right)  $. (That
is, each unitary matrix in $\operatorname*{U}\nolimits_{n}\left(
\mathbb{C}\right)  $ can be written as a limit $\lim\limits_{k\rightarrow
\infty}U_{k}$ of a sequence of unitary matrices $U_{k}$ such that $U_{k}%
+I_{n}$ is invertible for each $k$.)
\end{exercise}

Thus, if the Cayley parametrization does not hit a unitary matrix, then at
least it comes arbitrarily close.

\begin{remark}
A square matrix $A\in\mathbb{C}^{n\times n}$ satisfying $AA^{T}=A^{T}A=I_{n}$
is called \emph{orthogonal}. Thus, unitary matrices differ from orthogonal
matrices only in the use of the conjugate transpose $A^{\ast}$ instead of the
transpose $A^{T}$. In particular, a matrix $A\in\mathbb{R}^{n\times n}$ (with
real entries) is orthogonal if and only if it is unitary.
\end{remark}

\begin{exercise}
\fbox{5} A \emph{Pythagorean triple} is a triple $\left(  p,q,r\right)  $ of
positive integers satisfying $p^{2}+q^{2}=r^{2}$. (In other words, it is a
triple of positive integers that are the sides of a right-angled triangle.)
Two famous Pythagorean triples are $\left(  3,4,5\right)  $ and $\left(
5,12,13\right)  $.

\textbf{(a)} Prove that a triple $\left(  p,q,r\right)  $ of positive integers
is Pythagorean if and only if the matrix $\left(
\begin{array}
[c]{cc}%
p/r & -q/r\\
q/r & p/r
\end{array}
\right)  $ is unitary.

\textbf{(b)} Let $\left(
\begin{array}
[c]{cc}%
a & b\\
c & d
\end{array}
\right)  $ be any unitary matrix with rational entries. Assume that $a$ and
$c$ are positive, and write $a$ and $c$ as $p/r$ and $q/r$ for some positive
integers $p,q,r$. Show that $\left(  p,q,r\right)  $ is a Pythagorean triple.

\textbf{(c)} Find infinitely many Pythagorean triples that are pairwise
non-proportional (i.e., no two of them are obtained from one another just by
multiplying all three entries by the same number.)

[\textbf{Hint:} Use the $S\mapsto U$ construction from Exercise
\ref{exe.unitary.skew-herm.1}.]
\end{exercise}

For the next exercise, we recall the notion of similar matrices:

\begin{definition}
Let $A$ and $B$ be two matrices in $\mathbb{C}^{n\times n}$. We say that $A$
and $B$ are \emph{similar} if there exists an invertible matrix $W$ such that
$B=WAW^{-1}$.
\end{definition}

The relation \textquotedblleft similar\textquotedblright\ is easily seen to be
an equivalence relation. (Algebraists will recognize it as just being the
conjugacy relation in the ring $\mathbb{C}^{n\times n}$ of all $n\times n$-matrices.)

\begin{exercise}
\fbox{3} Let $A\in\mathbb{C}^{n\times n}$ be a matrix that is similar to some
unitary matrix. Prove that $A^{-1}$ is similar to $A^{\ast}$.
\end{exercise}

\subsubsection{Block matrices}

We record one more way to construct unitary matrices from smaller ones:

\begin{definition}
\label{def.blockmats.2x2}Let $\mathbb{F}$ be a field. Let $n,m,p,q\in
\mathbb{N}$. Let $A\in\mathbb{F}^{n\times p}$, $B\in\mathbb{F}^{n\times q}$,
$C\in\mathbb{F}^{m\times p}$ and $D\in\mathbb{F}^{m\times q}$ be four
matrices. Then, $\left(
\begin{array}
[c]{cc}%
A & B\\
C & D
\end{array}
\right)  $ shall denote the $\left(  n+m\right)  \times\left(  p+q\right)
$-matrix obtained by \textquotedblleft gluing\textquotedblright\ the four
matrices $A,B,C,D$ together in the manner suggested by the notation (i.e., we
glue $B$ to the right edge of $A$, we glue $C$ to the bottom edge of $A$, and
we glue $D$ to the right edge of $C$ and to the bottom edge of $B$). In other
words, we set%
\[
\left(
\begin{array}
[c]{cc}%
A & B\\
C & D
\end{array}
\right)  :=\left(
\begin{array}
[c]{cccccccc}%
A_{1,1} & A_{1,2} & \cdots & A_{1,p} & B_{1,1} & B_{1,2} & \cdots & B_{1,q}\\
A_{2,1} & A_{2,2} & \cdots & A_{2,p} & B_{2,1} & B_{2,2} & \cdots & B_{2,q}\\
\vdots & \vdots & \ddots & \vdots & \vdots & \vdots & \ddots & \vdots\\
A_{n,1} & A_{n,2} & \cdots & A_{n,p} & B_{n,1} & B_{n,2} & \cdots & B_{n,q}\\
C_{1,1} & C_{1,2} & \cdots & C_{1,p} & D_{1,1} & D_{1,2} & \cdots & D_{1,q}\\
C_{2,1} & C_{2,2} & \cdots & C_{2,p} & D_{2,1} & D_{2,2} & \cdots & D_{2,q}\\
\vdots & \vdots & \ddots & \vdots & \vdots & \vdots & \ddots & \vdots\\
C_{m,1} & C_{m,2} & \cdots & C_{m,p} & D_{m,1} & D_{m,2} & \cdots & D_{m,q}%
\end{array}
\right)
\]
(where, as we recall, the notation $M_{i,j}$ denotes the $\left(  i,j\right)
$-th entry of a matrix $M$).
\end{definition}

\begin{example}
If $A=\left(
\begin{array}
[c]{cc}%
a & a^{\prime}\\
a^{\prime\prime} & a^{\prime\prime\prime}%
\end{array}
\right)  $ and $B=\left(
\begin{array}
[c]{c}%
b\\
b^{\prime}%
\end{array}
\right)  $ and $C=\left(
\begin{array}
[c]{cc}%
c & c^{\prime}%
\end{array}
\right)  $ and $D=\left(
\begin{array}
[c]{c}%
d
\end{array}
\right)  $, then $\left(
\begin{array}
[c]{cc}%
A & B\\
C & D
\end{array}
\right)  =\left(
\begin{array}
[c]{ccc}%
a & a^{\prime} & b\\
a^{\prime\prime} & a^{\prime\prime\prime} & b^{\prime}\\
c & c^{\prime} & d
\end{array}
\right)  $.
\end{example}

The notation introduced in Definition \ref{def.blockmats.2x2} is called
\emph{block matrix notation}, and can be generalized to more than four matrices:

\begin{definition}
\label{def.blockmatrix.uxv}Let $\mathbb{F}$ be a field. Let $u,v\in\mathbb{N}%
$. Let $n_{1},n_{2},\ldots,n_{u}\in\mathbb{N}$ and $p_{1},p_{2},\ldots
,p_{v}\in\mathbb{N}$. For each $i\in\left[  u\right]  $ and $j\in\left[
v\right]  $, let $A\left(  i,j\right)  \in\mathbb{F}^{n_{i}\times p_{j}}$ be a
matrix. (We denote it by $A\left(  i,j\right)  $ instead of $A_{i,j}$ to avoid
mistaking it for a single entry.) Then,%
\begin{equation}
\left(
\begin{array}
[c]{cccc}%
A\left(  1,1\right)  & A\left(  1,2\right)  & \cdots & A\left(  1,v\right) \\
A\left(  2,1\right)  & A\left(  2,2\right)  & \cdots & A\left(  2,v\right) \\
\vdots & \vdots & \ddots & \vdots\\
A\left(  u,1\right)  & A\left(  u,2\right)  & \cdots & A\left(  u,v\right)
\end{array}
\right)  \label{eq.def.blockmatrix.uxv.blockmat}%
\end{equation}
shall denote the $\left(  n_{1}+n_{2}+\cdots+n_{u}\right)  \times\left(
p_{1}+p_{2}+\cdots+p_{v}\right)  $-matrix obtained by \textquotedblleft
gluing\textquotedblright\ the matrices $A\left(  i,j\right)  $ together in the
manner suggested by the notation. In other words,%
\[
\left(
\begin{array}
[c]{cccc}%
A\left(  1,1\right)  & A\left(  1,2\right)  & \cdots & A\left(  1,v\right) \\
A\left(  2,1\right)  & A\left(  2,2\right)  & \cdots & A\left(  2,v\right) \\
\vdots & \vdots & \ddots & \vdots\\
A\left(  u,1\right)  & A\left(  u,2\right)  & \cdots & A\left(  u,v\right)
\end{array}
\right)
\]
shall denote the $\left(  n_{1}+n_{2}+\cdots+n_{u}\right)  \times\left(
p_{1}+p_{2}+\cdots+p_{v}\right)  $-matrix whose $\left(  n_{1}+n_{2}%
+\cdots+n_{i-1}+k,\ \ p_{1}+p_{2}+\cdots+p_{j-1}+\ell\right)  $-th entry is
$\left(  A\left(  i,j\right)  \right)  _{k,\ell}$ for all $i\in\left[
u\right]  $ and $j\in\left[  v\right]  $ and $k\in\left[  n_{i}\right]  $ and
$\ell\in\left[  p_{j}\right]  $.
\end{definition}

Alternatively, this matrix can be defined abstractly using direct sums of
vector spaces; see \cite[Chapter II, \S 10, section 2]{Bourba74} for this definition.

\begin{example}
Let $0_{2\times2}$ denote the zero matrix of size $2\times2$. Then,%
\[
\left(
\begin{array}
[c]{ccc}%
0_{2\times2} & I_{2} & 0_{2\times2}\\
I_{2} & 0_{2\times2} & 0_{2\times2}\\
0_{2\times2} & -I_{2} & I_{2}%
\end{array}
\right)  =\left(
\begin{array}
[c]{cccccc}%
0 & 0 & 1 & 0 & 0 & 0\\
0 & 0 & 0 & 1 & 0 & 0\\
1 & 0 & 0 & 0 & 0 & 0\\
0 & 1 & 0 & 0 & 0 & 0\\
0 & 0 & -1 & 0 & 1 & 0\\
0 & 0 & 0 & -1 & 0 & 1
\end{array}
\right)  .
\]

\end{example}

In Definition \ref{def.blockmatrix.uxv}, the big matrix
(\ref{eq.def.blockmatrix.uxv.blockmat}) is called the \emph{block matrix}
formed out of the matrices $A\left(  i,j\right)  $; the single matrices
$A\left(  i,j\right)  $ are called its \emph{blocks}.

One of the most useful properties of block matrices is that they can be
multiplied block by block (i.e., by the same formula as for regular matrices)
if the products make sense. More precisely, the following holds:

\begin{proposition}
\label{prop.blockmatrix.mult-2x2}Let $\mathbb{F}$ be a field. Let $n$,
$n^{\prime}$, $m$, $m^{\prime}$, $\ell$ and $\ell^{\prime}$ be six nonnegative
integers. Let $A\in\mathbb{F}^{n\times m}$, $B\in\mathbb{F}^{n\times
m^{\prime}}$, $C\in\mathbb{F}^{n^{\prime}\times m}$, $D\in\mathbb{F}%
^{n^{\prime}\times m^{\prime}}$, $A^{\prime}\in\mathbb{F}^{m\times\ell}$,
$B^{\prime}\in\mathbb{F}^{m\times\ell^{\prime}}$, $C^{\prime}\in
\mathbb{F}^{m^{\prime}\times\ell}$ and $D^{\prime}\in\mathbb{F}^{m^{\prime
}\times\ell^{\prime}}$. Then,
\[
\left(
\begin{array}
[c]{cc}%
A & B\\
C & D
\end{array}
\right)  \left(
\begin{array}
[c]{cc}%
A^{\prime} & B^{\prime}\\
C^{\prime} & D^{\prime}%
\end{array}
\right)  =\left(
\begin{array}
[c]{cc}%
AA^{\prime}+BC^{\prime} & AB^{\prime}+BD^{\prime}\\
CA^{\prime}+DC^{\prime} & CB^{\prime}+DD^{\prime}%
\end{array}
\right)  .
\]

\end{proposition}

Note that the products on the right hand sides are products of matrices, so
they cannot be reordered (e.g., we cannot replace $AA^{\prime}$ by $A^{\prime
}A$), since matrix multiplication is not commutative.

\begin{proof}
[Proof of Proposition \ref{prop.blockmatrix.mult-2x2}.]This is a
straightforward computation that is made painful by the notational load and
the amount of cases to distinguish (depending on which block our entry lies
in). Do one of the cases to convince yourself that there is nothing difficult
here. (See \cite{detnotes} for all the gory details.)
\end{proof}

Unsurprisingly, Proposition \ref{prop.blockmatrix.mult-2x2} generalizes to the
multi-block case:

\begin{proposition}
\label{prop.blockmatrix.mult-uxv}Let $\mathbb{F}$ be a field. Let
$u,v,w\in\mathbb{N}$. Let $n_{1},n_{2},\ldots,n_{u}\in\mathbb{N}$ and
$p_{1},p_{2},\ldots,p_{v}\in\mathbb{N}$ and $q_{1},q_{2},\ldots,q_{w}%
\in\mathbb{N}$. For each $i\in\left[  u\right]  $ and $j\in\left[  v\right]
$, let $A\left(  i,j\right)  \in\mathbb{F}^{n_{i}\times p_{j}}$ be a matrix.
For each $j\in\left[  v\right]  $ and $k\in\left[  w\right]  $, let $B\left(
j,k\right)  \in\mathbb{F}^{p_{j}\times q_{k}}$ be a matrix. Then,%
\begin{align*}
&  \left(
\begin{array}
[c]{cccc}%
A\left(  1,1\right)  & A\left(  1,2\right)  & \cdots & A\left(  1,v\right) \\
A\left(  2,1\right)  & A\left(  2,2\right)  & \cdots & A\left(  2,v\right) \\
\vdots & \vdots & \ddots & \vdots\\
A\left(  u,1\right)  & A\left(  u,2\right)  & \cdots & A\left(  u,v\right)
\end{array}
\right)  \left(
\begin{array}
[c]{cccc}%
B\left(  1,1\right)  & B\left(  1,2\right)  & \cdots & B\left(  1,w\right) \\
B\left(  2,1\right)  & B\left(  2,2\right)  & \cdots & B\left(  2,w\right) \\
\vdots & \vdots & \ddots & \vdots\\
B\left(  v,1\right)  & B\left(  v,2\right)  & \cdots & B\left(  v,w\right)
\end{array}
\right) \\
&  =\left(
\begin{array}
[c]{cccc}%
\sum_{j=1}^{v}A\left(  1,j\right)  B\left(  j,1\right)  & \sum_{j=1}%
^{v}A\left(  1,j\right)  B\left(  j,2\right)  & \cdots & \sum_{j=1}%
^{v}A\left(  1,j\right)  B\left(  j,w\right) \\
\sum_{j=1}^{v}A\left(  2,j\right)  B\left(  j,1\right)  & \sum_{j=1}%
^{v}A\left(  2,j\right)  B\left(  j,2\right)  & \cdots & \sum_{j=1}%
^{v}A\left(  2,j\right)  B\left(  j,w\right) \\
\vdots & \vdots & \ddots & \vdots\\
\sum_{j=1}^{v}A\left(  u,j\right)  B\left(  j,1\right)  & \sum_{j=1}%
^{v}A\left(  u,j\right)  B\left(  j,2\right)  & \cdots & \sum_{j=1}%
^{v}A\left(  u,j\right)  B\left(  j,w\right)
\end{array}
\right)  .
\end{align*}

\end{proposition}

\begin{proof}
Just like Proposition \ref{prop.blockmatrix.mult-2x2}, but with more indices.
In short, fun!
\end{proof}

\emph{Block-diagonal matrices} are block matrices of the form
(\ref{eq.def.blockmatrix.uxv.blockmat}), where

\begin{itemize}
\item we have $u=v$,

\item all matrices $A\left(  i,i\right)  $ are square (i.e., we have
$n_{i}=p_{i}$ for all $i\in\left[  u\right]  $), and

\item all $A\left(  i,j\right)  $ with $i\neq j$ are zero matrices.
\end{itemize}

In other words, block-diagonal matrices are block matrices of the form%
\[
\left(
\begin{array}
[c]{cccc}%
A\left(  1,1\right)  & 0 & \cdots & 0\\
0 & A\left(  2,2\right)  & \cdots & 0\\
\vdots & \vdots & \ddots & \vdots\\
0 & 0 & \cdots & A\left(  u,u\right)
\end{array}
\right)  ,
\]
where $A\left(  1,1\right)  ,A\left(  2,2\right)  ,\ldots,A\left(  u,u\right)
$ are arbitrary square matrices, and where each \textquotedblleft%
$0$\textquotedblright\ means a zero matrix of appropriate dimensions. As an
easy consequence of Proposition \ref{prop.blockmatrix.mult-uxv}, we obtain a
multiplication rule for block-diagonal matrices that looks exactly like
multiplication of usual diagonal matrices:

\begin{corollary}
\label{cor.blockmatrix.mult-diag}Let $u\in\mathbb{N}$. Let $n_{1},n_{2}%
,\ldots,n_{u}\in\mathbb{N}$. For each $i\in\left[  u\right]  $, let $A\left(
i,i\right)  $ and $B\left(  i,i\right)  $ be two $n_{i}\times n_{i}$-matrices.
Then,%
\begin{align*}
&  \left(
\begin{array}
[c]{cccc}%
A\left(  1,1\right)  & 0 & \cdots & 0\\
0 & A\left(  2,2\right)  & \cdots & 0\\
\vdots & \vdots & \ddots & \vdots\\
0 & 0 & \cdots & A\left(  u,u\right)
\end{array}
\right)  \left(
\begin{array}
[c]{cccc}%
B\left(  1,1\right)  & 0 & \cdots & 0\\
0 & B\left(  2,2\right)  & \cdots & 0\\
\vdots & \vdots & \ddots & \vdots\\
0 & 0 & \cdots & B\left(  u,u\right)
\end{array}
\right) \\
&  =\left(
\begin{array}
[c]{cccc}%
A\left(  1,1\right)  B\left(  1,1\right)  & 0 & \cdots & 0\\
0 & A\left(  2,2\right)  B\left(  2,2\right)  & \cdots & 0\\
\vdots & \vdots & \ddots & \vdots\\
0 & 0 & \cdots & A\left(  u,u\right)  B\left(  u,u\right)
\end{array}
\right)  .
\end{align*}
(Here, each \textquotedblleft$0$\textquotedblright\ means a zero matrix of
appropriate dimensions.)
\end{corollary}

Now, we claim that a block-diagonal matrix is unitary if and only if its
diagonal blocks are unitary:

\begin{proposition}
\label{prop.blockmatrix.unitary-diag}Let $u\in\mathbb{N}$. Let $n_{1}%
,n_{2},\ldots,n_{u}\in\mathbb{N}$. For each $i\in\left[  u\right]  $, let
$A_{i}\in\mathbb{C}^{n_{i}\times n_{i}}$ be a matrix. Then, the block-diagonal
matrix $\left(
\begin{array}
[c]{cccc}%
A_{1} & 0 & \cdots & 0\\
0 & A_{2} & \cdots & 0\\
\vdots & \vdots & \ddots & \vdots\\
0 & 0 & \cdots & A_{u}%
\end{array}
\right)  $ is unitary if and only if all $u$ matrices $A_{1},A_{2}%
,\ldots,A_{u}$ are unitary.
\end{proposition}

\begin{proof}
Let $N=n_{1}+n_{2}+\cdots+n_{u}$. Let
\begin{equation}
A=\left(
\begin{array}
[c]{cccc}%
A_{1} & 0 & \cdots & 0\\
0 & A_{2} & \cdots & 0\\
\vdots & \vdots & \ddots & \vdots\\
0 & 0 & \cdots & A_{u}%
\end{array}
\right)  . \label{pf.prop.blockmatrix.unitary-diag.A=}%
\end{equation}
Thus, we must prove that $A$ is unitary if and only if all $u$ matrices
$A_{1},A_{2},\ldots,A_{u}$ are unitary.

It is easy to see that
\[
A^{\ast}=\left(
\begin{array}
[c]{cccc}%
A_{1}^{\ast} & 0 & \cdots & 0\\
0 & A_{2}^{\ast} & \cdots & 0\\
\vdots & \vdots & \ddots & \vdots\\
0 & 0 & \cdots & A_{u}^{\ast}%
\end{array}
\right)  .
\]
Multiplying this equality by (\ref{pf.prop.blockmatrix.unitary-diag.A=}), we
obtain%
\begin{align*}
A^{\ast}A  &  =\left(
\begin{array}
[c]{cccc}%
A_{1}^{\ast} & 0 & \cdots & 0\\
0 & A_{2}^{\ast} & \cdots & 0\\
\vdots & \vdots & \ddots & \vdots\\
0 & 0 & \cdots & A_{u}^{\ast}%
\end{array}
\right)  \left(
\begin{array}
[c]{cccc}%
A_{1} & 0 & \cdots & 0\\
0 & A_{2} & \cdots & 0\\
\vdots & \vdots & \ddots & \vdots\\
0 & 0 & \cdots & A_{u}%
\end{array}
\right) \\
&  =\left(
\begin{array}
[c]{cccc}%
A_{1}^{\ast}A_{1} & 0 & \cdots & 0\\
0 & A_{2}^{\ast}A_{2} & \cdots & 0\\
\vdots & \vdots & \ddots & \vdots\\
0 & 0 & \cdots & A_{u}^{\ast}A_{u}%
\end{array}
\right)  \ \ \ \ \ \ \ \ \ \ \left(  \text{by Corollary
\ref{cor.blockmatrix.mult-diag}}\right)  .
\end{align*}
On the other hand, it is again easy to see that
\[
I_{N}=\left(
\begin{array}
[c]{cccc}%
I_{n_{1}} & 0 & \cdots & 0\\
0 & I_{n_{2}} & \cdots & 0\\
\vdots & \vdots & \ddots & \vdots\\
0 & 0 & \cdots & I_{n_{u}}%
\end{array}
\right)
\]
(since $N=n_{1}+n_{2}+\cdots+n_{u}$). In light of these two equalities, we see
that $A^{\ast}A=I_{N}$ holds if and only if
\[
\left(
\begin{array}
[c]{cccc}%
A_{1}^{\ast}A_{1} & 0 & \cdots & 0\\
0 & A_{2}^{\ast}A_{2} & \cdots & 0\\
\vdots & \vdots & \ddots & \vdots\\
0 & 0 & \cdots & A_{u}^{\ast}A_{u}%
\end{array}
\right)  =\left(
\begin{array}
[c]{cccc}%
I_{n_{1}} & 0 & \cdots & 0\\
0 & I_{n_{2}} & \cdots & 0\\
\vdots & \vdots & \ddots & \vdots\\
0 & 0 & \cdots & I_{n_{u}}%
\end{array}
\right)
\]
holds, i.e., if and only if we have $A_{i}^{\ast}A_{i}=I_{n_{i}}$ for each
$i\in\left[  u\right]  $. Likewise, we can see that $AA^{\ast}=I_{N}$ holds if
and only if $A_{i}A_{i}^{\ast}=I_{n_{i}}$ for each $i\in\left[  u\right]  $.
Hence, we have the following chain of equivalences:%
\begin{align*}
&  \ \left(  A\text{ is unitary}\right) \\
&  \Longleftrightarrow\ \left(  AA^{\ast}=I_{N}\text{ and }A^{\ast}%
A=I_{N}\right) \\
&  \ \ \ \ \ \ \ \ \ \ \ \ \ \ \ \ \ \ \ \ \left(  \text{by the equivalence
}\mathcal{A}\Longleftrightarrow\mathcal{C}\text{ in Theorem
\ref{thm.unitary.unitary.eqs}}\right) \\
&  \Longleftrightarrow\ \left(  \text{we have }A_{i}A_{i}^{\ast}=I_{n_{i}%
}\text{ and }A_{i}^{\ast}A_{i}=I_{n_{i}}\text{ for each }i\in\left[  u\right]
\right) \\
&  \ \ \ \ \ \ \ \ \ \ \ \ \ \ \ \ \ \ \ \ \left(
\begin{array}
[c]{c}%
\text{since we have shown that }AA^{\ast}=I_{N}\text{ holds if and only if}\\
\text{we have }A_{i}A_{i}^{\ast}=I_{n_{i}}\text{ for each }i\in\left[
u\right]  \text{, and since we have}\\
\text{shown that }A^{\ast}A=I_{N}\text{ holds if and only if}\\
\text{we have }A_{i}^{\ast}A_{i}=I_{n_{i}}\text{ for each }i\in\left[
u\right]
\end{array}
\right) \\
&  \Longleftrightarrow\ \left(  \text{the matrix }A_{i}\text{ is unitary for
each }i\in\left[  u\right]  \right) \\
&  \ \ \ \ \ \ \ \ \ \ \ \ \ \ \ \ \ \ \ \ \left(  \text{by the equivalence
}\mathcal{C}\Longleftrightarrow\mathcal{A}\text{ in Theorem
\ref{thm.unitary.unitary.eqs}}\right) \\
&  \Longleftrightarrow\ \left(  \text{all }u\text{ matrices }A_{1}%
,A_{2},\ldots,A_{u}\text{ are unitary}\right)  .
\end{align*}
But this is precisely what we need to show. Thus, Proposition
\ref{prop.blockmatrix.unitary-diag} is proven.
\end{proof}

\subsection{The Gram--Schmidt process}

\begin{theorem}
[Gram--Schmidt process]\label{thm.unitary.gs}Let $\left(  v_{1},v_{2}%
,\ldots,v_{m}\right)  $ be a linearly independent tuple of vectors in
$\mathbb{C}^{n}$.

Then, there is an orthogonal tuple $\left(  z_{1},z_{2},\ldots,z_{m}\right)  $
of vectors in $\mathbb{C}^{n}$ that satisfies%
\[
\operatorname*{span}\left\{  v_{1},v_{2},\ldots,v_{j}\right\}
=\operatorname*{span}\left\{  z_{1},z_{2},\ldots,z_{j}\right\}
\ \ \ \ \ \ \ \ \ \ \text{for all }j\in\left[  m\right]  .
\]


Furthermore, such a tuple $\left(  z_{1},z_{2},\ldots,z_{m}\right)  $ can be
constructed by the following recursive process:

\begin{itemize}
\item For each $p\in\left[  m\right]  $, if the first $p-1$ entries
$z_{1},z_{2},\ldots,z_{p-1}$ of this tuple have already been constructed, then
we define the $p$-th entry $z_{p}$ by the equality%
\begin{equation}
z_{p}=v_{p}-\sum_{k=1}^{p-1}\dfrac{\left\langle v_{p},z_{k}\right\rangle
}{\left\langle z_{k},z_{k}\right\rangle }z_{k}. \label{eq.thm.unitary.gs.zp=}%
\end{equation}
(Note that the sum on the right hand side of (\ref{eq.thm.unitary.gs.zp=}) is
an empty sum when $p=1$; thus, (\ref{eq.thm.unitary.gs.zp=}) simplifies to
$z_{1}=v_{1}$ in this case.)
\end{itemize}
\end{theorem}

Roughly speaking, the claim of Theorem \ref{thm.unitary.gs} is that if we
start with any linearly independent tuple $\left(  v_{1},v_{2},\ldots
,v_{m}\right)  $ of vectors in $\mathbb{C}^{n}$, then we can make this tuple
orthogonal by tweaking it as follows:

\begin{itemize}
\item leave $v_{1}$ unchanged;

\item modify $v_{2}$ by subtracting some scalar multiple of $v_{1}$;

\item modify $v_{3}$ by subtracting some linear combination of $v_{1}$ and
$v_{2}$;

\item modify $v_{4}$ by subtracting some linear combination of $v_{1}%
,v_{2},v_{3}$;

\item and so on.
\end{itemize}

\noindent Specifically, the equation (\ref{eq.thm.unitary.gs.zp=}) tells us
(recursively) the precise multiples (and linear combinations) that we need to
subtract. This recursive tweaking process is known as \emph{Gram--Schmidt
orthogonalization} or the \emph{Gram--Schmidt process}.

\begin{example}
Here is how the equalities (\ref{eq.thm.unitary.gs.zp=}) in Theorem
\ref{thm.unitary.gs} look like for $p\in\left\{  1,2,3,4\right\}  $:%
\begin{align*}
z_{1}  &  =v_{1};\\
z_{2}  &  =v_{2}-\dfrac{\left\langle v_{2},z_{1}\right\rangle }{\left\langle
z_{1},z_{1}\right\rangle }z_{1};\\
z_{3}  &  =v_{3}-\dfrac{\left\langle v_{3},z_{1}\right\rangle }{\left\langle
z_{1},z_{1}\right\rangle }z_{1}-\dfrac{\left\langle v_{3},z_{2}\right\rangle
}{\left\langle z_{2},z_{2}\right\rangle }z_{2};\\
z_{4}  &  =v_{4}-\dfrac{\left\langle v_{4},z_{1}\right\rangle }{\left\langle
z_{1},z_{1}\right\rangle }z_{1}-\dfrac{\left\langle v_{4},z_{2}\right\rangle
}{\left\langle z_{2},z_{2}\right\rangle }z_{2}-\dfrac{\left\langle v_{4}%
,z_{3}\right\rangle }{\left\langle z_{3},z_{3}\right\rangle }z_{3}.
\end{align*}

\end{example}

\begin{example}
Let us try out the recursive construction of $\left(  z_{1},z_{2},\ldots
,z_{m}\right)  $ from Theorem \ref{thm.unitary.gs} on an example. Let $n=4$
and $m=3$ and
\[
v_{1}=\left(
\begin{array}
[c]{c}%
1\\
1\\
1\\
1
\end{array}
\right)  ,\ \ \ \ \ \ \ \ \ \ v_{2}=\left(
\begin{array}
[c]{c}%
0\\
-2\\
0\\
-2
\end{array}
\right)  ,\ \ \ \ \ \ \ \ \ \ v_{3}=\left(
\begin{array}
[c]{c}%
2\\
-2\\
0\\
0
\end{array}
\right)  .
\]
Then, (\ref{eq.thm.unitary.gs.zp=}) becomes%
\begin{align*}
z_{1}  &  =v_{1}=\left(
\begin{array}
[c]{c}%
1\\
1\\
1\\
1
\end{array}
\right)  ;\\
z_{2}  &  =v_{2}-\dfrac{\left\langle v_{2},z_{1}\right\rangle }{\left\langle
z_{1},z_{1}\right\rangle }z_{1}=\left(
\begin{array}
[c]{c}%
0\\
-2\\
0\\
-2
\end{array}
\right)  -\dfrac{-4}{4}\left(
\begin{array}
[c]{c}%
1\\
1\\
1\\
1
\end{array}
\right)  =\left(
\begin{array}
[c]{c}%
1\\
-1\\
1\\
-1
\end{array}
\right)  ;\\
z_{3}  &  =v_{3}-\dfrac{\left\langle v_{3},z_{1}\right\rangle }{\left\langle
z_{1},z_{1}\right\rangle }z_{1}-\dfrac{\left\langle v_{3},z_{2}\right\rangle
}{\left\langle z_{2},z_{2}\right\rangle }z_{2}\\
&  =\left(
\begin{array}
[c]{c}%
2\\
-2\\
0\\
0
\end{array}
\right)  -\dfrac{0}{4}\left(
\begin{array}
[c]{c}%
1\\
1\\
1\\
1
\end{array}
\right)  -\dfrac{4}{4}\left(
\begin{array}
[c]{c}%
1\\
-1\\
1\\
-1
\end{array}
\right)  =\left(
\begin{array}
[c]{c}%
1\\
-1\\
-1\\
1
\end{array}
\right)  .
\end{align*}
So
\[
\left(  z_{1},z_{2},z_{3}\right)  =\left(  \left(
\begin{array}
[c]{c}%
1\\
1\\
1\\
1
\end{array}
\right)  ,\ \ \left(
\begin{array}
[c]{c}%
1\\
-1\\
1\\
-1
\end{array}
\right)  ,\ \ \left(
\begin{array}
[c]{c}%
1\\
-1\\
-1\\
1
\end{array}
\right)  \right)
\]
is an orthogonal tuple of vectors.

According to Proposition \ref{prop.unitary.innerprod.orth-norm}, we thus
obtain an orthonormal tuple%
\[
\left(  \dfrac{1}{\left\vert \left\vert z_{1}\right\vert \right\vert }%
z_{1},\ \ \dfrac{1}{\left\vert \left\vert z_{2}\right\vert \right\vert }%
z_{2},\ \ \dfrac{1}{\left\vert \left\vert z_{3}\right\vert \right\vert }%
z_{3}\right)  =\left(  \left(
\begin{array}
[c]{c}%
1/2\\
1/2\\
1/2\\
1/2
\end{array}
\right)  ,\ \ \left(
\begin{array}
[c]{c}%
1/2\\
-1/2\\
1/2\\
-1/2
\end{array}
\right)  ,\ \ \left(
\begin{array}
[c]{c}%
1/2\\
-1/2\\
-1/2\\
1/2
\end{array}
\right)  \right)  .
\]
(We are in luck with this example; normally we would get square roots at this step.)
\end{example}

For more examples of the Gram--Schmidt process, see \cite[Week 3,
\S 4]{Bartle14}. (These examples all use vectors in $\mathbb{R}^{n}$ rather
than $\mathbb{C}^{n}$, which allows for visualization and saves one the
trouble of complex conjugates.)

\begin{proof}
[Proof of Theorem \ref{thm.unitary.gs}.]We define a tuple $\left(  z_{1}%
,z_{2},\ldots,z_{m}\right)  $ recursively by (\ref{eq.thm.unitary.gs.zp=}).
First, we need to show that this tuple is actually well-defined -- i.e., that
the denominators $\left\langle z_{k},z_{k}\right\rangle $ never become $0$ in
the process (which would render (\ref{eq.thm.unitary.gs.zp=}) meaningless).
Second, we need to show that the resulting tuple does indeed satisfy
\[
\operatorname*{span}\left\{  v_{1},v_{2},\ldots,v_{j}\right\}
=\operatorname*{span}\left\{  z_{1},z_{2},\ldots,z_{j}\right\}
\ \ \ \ \ \ \ \ \ \ \text{for all }j\in\left[  m\right]  .
\]
Finally, we need to show that the resulting tuple is orthogonal.

Let us prove the first two of these three claims in lockstep, by showing the
following claim:

\begin{statement}
\textit{Claim 1:} For each $p\in\left[  m\right]  $, the vector $z_{p}$ is
well-defined (i.e., the denominators $\left\langle z_{k},z_{k}\right\rangle $
on the right hand side of (\ref{eq.thm.unitary.gs.zp=}) are nonzero) and
satisfies $\operatorname*{span}\left\{  v_{1},v_{2},\ldots,v_{p}\right\}
=\operatorname*{span}\left\{  z_{1},z_{2},\ldots,z_{p}\right\}  $.
\end{statement}

[\textit{Proof of Claim 1:} Strong induction on $p$. Thus, we fix some
$p\in\left[  m\right]  $, and assume that the vectors $z_{1},z_{2}%
,\ldots,z_{p-1}$ are well-defined and satisfy
\begin{equation}
\operatorname*{span}\left\{  v_{1},v_{2},\ldots,v_{j}\right\}
=\operatorname*{span}\left\{  z_{1},z_{2},\ldots,z_{j}\right\}
\ \ \ \ \ \ \ \ \ \ \text{for all }j<p. \label{pf.thm.unitary.gs.4}%
\end{equation}
We now need to show that the vector $z_{p}$ is well-defined and satisfies
\begin{equation}
\operatorname*{span}\left\{  v_{1},v_{2},\ldots,v_{p}\right\}
=\operatorname*{span}\left\{  z_{1},z_{2},\ldots,z_{p}\right\}  .
\label{pf.thm.unitary.gs.5}%
\end{equation}


The tuple $\left(  v_{1},v_{2},\ldots,v_{p}\right)  $ is linearly independent
(since the tuple $\left(  v_{1},v_{2},\ldots,v_{m}\right)  $ is linearly
independent). Thus, the span $\operatorname*{span}\left\{  v_{1},v_{2}%
,\ldots,v_{p-1}\right\}  $ is $\left(  p-1\right)  $-dimensional and we have
$v_{p}\notin\operatorname*{span}\left\{  v_{1},v_{2},\ldots,v_{p-1}\right\}
$. However, we have
\begin{equation}
\operatorname*{span}\left\{  v_{1},v_{2},\ldots,v_{p-1}\right\}
=\operatorname*{span}\left\{  z_{1},z_{2},\ldots,z_{p-1}\right\}  .
\label{pf.thm.unitary.gs.6}%
\end{equation}
(Indeed, this is obvious when $p=1$, and otherwise follows from
(\ref{pf.thm.unitary.gs.4}) (applied to $j=p-1$).) Hence,%
\[
v_{p}\notin\operatorname*{span}\left\{  v_{1},v_{2},\ldots,v_{p-1}\right\}
=\operatorname*{span}\left\{  z_{1},z_{2},\ldots,z_{p-1}\right\}  .
\]


Now, recall that the span $\operatorname*{span}\left\{  v_{1},v_{2}%
,\ldots,v_{p-1}\right\}  $ is $\left(  p-1\right)  $-dimensional. In view of
(\ref{pf.thm.unitary.gs.6}), we can rewrite this as follows: The span
$\operatorname*{span}\left\{  z_{1},z_{2},\ldots,z_{p-1}\right\}  $ is
$\left(  p-1\right)  $-dimensional. In other words, the tuple $\left(
z_{1},z_{2},\ldots,z_{p-1}\right)  $ is linearly independent. Hence, for each
$k\in\left[  p-1\right]  $, we have $z_{k}\neq0$ and therefore $\left\langle
z_{k},z_{k}\right\rangle >0$ (by Proposition \ref{prop.unitary.innerprod.pos}
\textbf{(b)}), so that $\left\langle z_{k},z_{k}\right\rangle \neq0$. Thus,
the denominators on the right hand side of (\ref{eq.thm.unitary.gs.zp=}) are
nonzero, so that $v_{p}$ is well-defined. It remains to prove that
\[
\operatorname*{span}\left\{  v_{1},v_{2},\ldots,v_{p}\right\}
=\operatorname*{span}\left\{  z_{1},z_{2},\ldots,z_{p}\right\}  .
\]
But this is easy: From (\ref{eq.thm.unitary.gs.zp=}), we see that $z_{p}$
equals $v_{p}$ minus some linear combination of $v_{1},v_{2},\ldots,v_{p-1}$.
Thus,%
\begin{align*}
\operatorname*{span}\left\{  v_{1},v_{2},\ldots,v_{p}\right\}   &
=\operatorname*{span}\left\{  v_{1},v_{2},\ldots,v_{p-1},z_{p}\right\} \\
&  =\underbrace{\operatorname*{span}\left\{  v_{1},v_{2},\ldots,v_{p-1}%
\right\}  }_{=\operatorname*{span}\left\{  z_{1},z_{2},\ldots,z_{p-1}\right\}
}+\operatorname*{span}\left\{  z_{p}\right\} \\
&  =\operatorname*{span}\left\{  z_{1},z_{2},\ldots,z_{p-1}\right\}
+\operatorname*{span}\left\{  z_{p}\right\} \\
&  =\operatorname*{span}\left\{  z_{1},z_{2},\ldots,z_{p}\right\}  .
\end{align*}
Thus, the induction step is complete, so that Claim 1 is proved by induction.]

Claim 1 shows that the tuple $\left(  z_{1},z_{2},\ldots,z_{m}\right)  $ is
well-defined and satisfies%
\[
\operatorname*{span}\left\{  v_{1},v_{2},\ldots,v_{j}\right\}
=\operatorname*{span}\left\{  z_{1},z_{2},\ldots,z_{j}\right\}
\ \ \ \ \ \ \ \ \ \ \text{for all }j\in\left[  m\right]  .
\]
It now remains to show that this tuple is orthogonal. We shall achieve this by
showing the following claim:

\begin{statement}
\textit{Claim 2:} For any $j\in\left\{  0,1,\ldots,m\right\}  $, the tuple
$\left(  z_{1},z_{2},\ldots,z_{j}\right)  $ is orthogonal.
\end{statement}

[\textit{Proof of Claim 2:} We proceed by induction on $j$:

\textit{Induction base:} Claim 2 clearly holds for $j=0$, since the (empty)
$0$-tuple is vacuously orthogonal.

\textit{Induction step:} Let $p\in\left[  m\right]  $. Assume (as the
induction hypothesis) that Claim 2 holds for $j=p-1$. We must show that Claim
2 holds for $j=p$.

Our induction hypothesis says that Claim 2 holds for $j=p-1$. In other words,
the tuple $\left(  z_{1},z_{2},\ldots,z_{p-1}\right)  $ is orthogonal. In
other words, we have%
\begin{equation}
z_{a}\perp z_{b}\ \ \ \ \ \ \ \ \ \ \text{whenever }a,b\in\left[  p-1\right]
\text{ satisfy }a\neq b. \label{pf.thm.unitary.gs.12}%
\end{equation}
In other words, we have%
\begin{equation}
\left\langle z_{a},z_{b}\right\rangle =0\ \ \ \ \ \ \ \ \ \ \text{whenever
}a,b\in\left[  p-1\right]  \text{ satisfy }a\neq b.
\label{pf.thm.unitary.gs.12b}%
\end{equation}


We must show that Claim 2 holds for $j=p$. In other words, we must show that
the tuple $\left(  z_{1},z_{2},\ldots,z_{p}\right)  $ is orthogonal. In other
words, we must show that%
\begin{equation}
z_{a}\perp z_{b}\ \ \ \ \ \ \ \ \ \ \text{whenever }a,b\in\left[  p\right]
\text{ satisfy }a\neq b. \label{pf.thm.unitary.gs.13}%
\end{equation}
It will clearly suffice to prove (\ref{pf.thm.unitary.gs.13}) in the case when
one of $a$ and $b$ equals $p$ (because in all other cases, we have
$a,b\in\left[  p-1\right]  $, and thus $z_{a}\perp z_{b}$ follows from
(\ref{pf.thm.unitary.gs.12})).

Thus, let $a,b\in\left[  p\right]  $ satisfy $a\neq b$, and assume that one of
$a$ and $b$ equals $p$. We must prove that $z_{a}\perp z_{b}$. Proposition
\ref{prop.unitary.innerprod.orth-symm} shows that $z_{a}\perp z_{b}$ is
equivalent to $z_{b}\perp z_{a}$. Hence, in our proof of $z_{a}\perp z_{b}$,
we can WLOG assume that $a\leq b$ (since otherwise, we can swap $a$ with $b$).
Assume this. Hence, $a<b$ (since $a\neq b$). Thus, $a<b\leq p$, so that $a\neq
p$. However, we assumed that one of $a$ and $b$ equals $p$; hence, $b=p$
(since $a\neq p$). Also, we have $a\in\left[  p-1\right]  $ (since $a<b=p$).

Now, (\ref{eq.thm.unitary.gs.zp=}) yields%
\begin{align*}
\left\langle z_{p},z_{a}\right\rangle  &  =\left\langle v_{p}-\sum_{k=1}%
^{p-1}\dfrac{\left\langle v_{p},z_{k}\right\rangle }{\left\langle z_{k}%
,z_{k}\right\rangle }z_{k},v_{a}\right\rangle \\
&  =\left\langle v_{p},z_{a}\right\rangle -\sum_{k=1}^{p-1}\dfrac{\left\langle
v_{p},z_{k}\right\rangle }{\left\langle z_{k},z_{k}\right\rangle }\left\langle
z_{k},z_{a}\right\rangle \\
&  =\left\langle v_{p},z_{a}\right\rangle -\left(  \sum_{\substack{k\in\left[
p-1\right]  ;\\k\neq a}}\dfrac{\left\langle v_{p},z_{k}\right\rangle
}{\left\langle z_{k},z_{k}\right\rangle }\left\langle z_{k},z_{a}\right\rangle
+\dfrac{\left\langle v_{p},z_{a}\right\rangle }{\left\langle z_{a}%
,z_{a}\right\rangle }\left\langle z_{a},z_{a}\right\rangle \right) \\
&  \ \ \ \ \ \ \ \ \ \ \ \ \ \ \ \ \ \ \ \ \left(
\begin{array}
[c]{c}%
\text{here, we have split off the addend for }k=a\\
\text{from the sum, since }a\in\left[  p-1\right]
\end{array}
\right) \\
&  =\left\langle v_{p},z_{a}\right\rangle -\sum_{\substack{k\in\left[
p-1\right]  ;\\k\neq a}}\dfrac{\left\langle v_{p},z_{k}\right\rangle
}{\left\langle z_{k},z_{k}\right\rangle }\underbrace{\left\langle z_{k}%
,z_{a}\right\rangle }_{\substack{=0\\\text{(by (\ref{pf.thm.unitary.gs.13}),
applied to }k\text{ and }a\\\text{instead of }a\text{ and }b\text{)}%
}}-\underbrace{\dfrac{\left\langle v_{p},z_{a}\right\rangle }{\left\langle
z_{a},z_{a}\right\rangle }\left\langle z_{a},z_{a}\right\rangle }%
_{=\left\langle v_{p},z_{a}\right\rangle }\\
&  =\left\langle v_{p},z_{a}\right\rangle -\underbrace{\sum_{\substack{k\in
\left[  p-1\right]  ;\\k\neq a}}\dfrac{\left\langle v_{p},z_{k}\right\rangle
}{\left\langle z_{k},z_{k}\right\rangle }0}_{=0}-\left\langle v_{p}%
,z_{a}\right\rangle =0.
\end{align*}
In view of $b=p$, this rewrites as $\left\langle z_{b},z_{a}\right\rangle =0$.
Thus, $z_{b}\perp z_{a}$, so that $z_{a}\perp z_{b}$ (by Proposition
\ref{prop.unitary.innerprod.orth-symm}).

As explained above, this completes our proof of the fact that Claim 2 holds
for $j=p$. Thus, the induction step is complete, and Claim 2 is proven.]

Now, applying Claim 2 to $j=m$, we obtain that the tuple $\left(  z_{1}%
,z_{2},\ldots,z_{m}\right)  $ is orthogonal. Thus, the proof of Theorem
\ref{thm.unitary.gs} is complete.
\end{proof}

One might wonder how the Gram--Schmidt process could be adapted to a tuple
$\left(  v_{1},v_{2},\ldots,v_{m}\right)  $ of vectors that is \textbf{not}
linearly independent. The equality (\ref{eq.thm.unitary.gs.zp=}) requires the
vectors $z_{k}$ to be nonzero, since the denominators in which they appear
would be $0$ otherwise. In Theorem \ref{thm.unitary.gs}, this requirement is
indeed satisfied (as we have shown in the proof above). However, if we do not
assume $\left(  v_{1},v_{2},\ldots,v_{m}\right)  $ to be linearly independent,
then some of the $z_{k}$ can be zero, and so the construction of the following
$z_{p}$ will fail. There are several ways to adapt the process to this
complication. We will take the most stupid-sounding one: In the cases where
the equality (\ref{eq.thm.unitary.gs.zp=}) would produce a zero vector $z_{p}%
$, we opt to instead pick some nonzero vector orthogonal to $z_{1}%
,z_{2},\ldots,z_{p-1}$ (using Lemma \ref{lem.unitary.orthog.one-more}) and
declare it to be $z_{p}$. This works well as long as $m\leq n$; here is the result:

\begin{theorem}
[Gram--Schmidt process, take 2]\label{thm.unitary.gs-2}Let $\left(
v_{1},v_{2},\ldots,v_{m}\right)  $ be any tuple of vectors in $\mathbb{C}^{n}$
with $m\leq n$.

Then, there is an orthogonal tuple $\left(  z_{1},z_{2},\ldots,z_{m}\right)  $
of nonzero vectors in $\mathbb{C}^{n}$ that satisfies%
\[
\operatorname*{span}\left\{  v_{1},v_{2},\ldots,v_{j}\right\}  \subseteq
\operatorname*{span}\left\{  z_{1},z_{2},\ldots,z_{j}\right\}
\ \ \ \ \ \ \ \ \ \ \text{for all }j\in\left[  m\right]  .
\]


Furthermore, such a tuple $\left(  z_{1},z_{2},\ldots,z_{m}\right)  $ can be
constructed by the following recursive process:

\begin{itemize}
\item For each $p\in\left[  m\right]  $, if the first $p-1$ entries
$z_{1},z_{2},\ldots,z_{p-1}$ of this tuple have already been constructed, then
we define the $p$-th entry $z_{p}$ as follows:

\begin{itemize}
\item \textit{If }$v_{p}-\sum_{k=1}^{p-1}\dfrac{\left\langle v_{p}%
,z_{k}\right\rangle }{\left\langle z_{k},z_{k}\right\rangle }z_{k}\neq0$, then
we define $z_{p}$ by the equality%
\begin{equation}
z_{p}=v_{p}-\sum_{k=1}^{p-1}\dfrac{\left\langle v_{p},z_{k}\right\rangle
}{\left\langle z_{k},z_{k}\right\rangle }z_{k}.
\label{eq.thm.unitary.gs-2.zp=sum}%
\end{equation}


\item \textit{If }$v_{p}-\sum_{k=1}^{p-1}\dfrac{\left\langle v_{p}%
,z_{k}\right\rangle }{\left\langle z_{k},z_{k}\right\rangle }z_{k}=0$, then we
pick an arbitrary nonzero vector $b\in\mathbb{C}^{n}$ that is orthogonal to
each of $z_{1},z_{2},\ldots,z_{p-1}$ (indeed, such a vector $b$ exists by
Lemma \ref{lem.unitary.orthog.one-more}, because $p-1<p\leq m\leq n$), and we
set%
\begin{equation}
z_{p}=b. \label{eq.thm.unitary.gs-2.zp=b}%
\end{equation}

\end{itemize}
\end{itemize}
\end{theorem}

\begin{proof}
[Proof of Theorem \ref{thm.unitary.gs-2}.]We define a tuple $\left(
z_{1},z_{2},\ldots,z_{m}\right)  $ by the recursive process described in
Theorem \ref{thm.unitary.gs-2}. It is clear that this tuple is actually
well-defined (indeed, the vectors $z_{p}$ are nonzero by their construction,
and thus the denominators $\left\langle z_{k},z_{k}\right\rangle $ in
(\ref{eq.thm.unitary.gs-2.zp=sum}) never become $0$, because Proposition
\ref{prop.unitary.innerprod.pos} \textbf{(b)} shows that any nonzero vector
$z$ satisfies $\left\langle z,z\right\rangle \neq0$). We do, however, need to
show that the resulting tuple does indeed satisfy
\[
\operatorname*{span}\left\{  v_{1},v_{2},\ldots,v_{j}\right\}  \subseteq
\operatorname*{span}\left\{  z_{1},z_{2},\ldots,z_{j}\right\}
\ \ \ \ \ \ \ \ \ \ \text{for all }j\in\left[  m\right]  ,
\]
and that this tuple is orthogonal.

Let us prove the first of these two claims:

\begin{statement}
\textit{Claim 1:} For each $p\in\left[  m\right]  $, we have
$\operatorname*{span}\left\{  v_{1},v_{2},\ldots,v_{p}\right\}  \subseteq
\operatorname*{span}\left\{  z_{1},z_{2},\ldots,z_{p}\right\}  $.
\end{statement}

[\textit{Proof of Claim 1:} Strong induction on $p$. Thus, we fix some
$p\in\left[  m\right]  $, and assume that%
\begin{equation}
\operatorname*{span}\left\{  v_{1},v_{2},\ldots,v_{j}\right\}  \subseteq
\operatorname*{span}\left\{  z_{1},z_{2},\ldots,z_{j}\right\}
\ \ \ \ \ \ \ \ \ \ \text{for all }j<p. \label{pf.thm.unitary.gs-2.4}%
\end{equation}
We now need to show that
\begin{equation}
\operatorname*{span}\left\{  v_{1},v_{2},\ldots,v_{p}\right\}  \subseteq
\operatorname*{span}\left\{  z_{1},z_{2},\ldots,z_{p}\right\}  .
\label{pf.thm.unitary.gs-2.5}%
\end{equation}


We shall first show that
\begin{equation}
v_{p}\in\operatorname*{span}\left\{  z_{1},z_{2},\ldots,z_{p}\right\}  .
\label{pf.thm.unitary.gs-2.5a}%
\end{equation}


Indeed, we recall our definition of $z_{p}$. This definition distinguishes
between two cases, depending on whether the difference $v_{p}-\sum_{k=1}%
^{p-1}\dfrac{\left\langle v_{p},z_{k}\right\rangle }{\left\langle z_{k}%
,z_{k}\right\rangle }z_{k}$ is $\neq0$ or $=0$. Let us analyze these two cases separately:

\begin{itemize}
\item \textit{Case 1:} We have $v_{p}-\sum_{k=1}^{p-1}\dfrac{\left\langle
v_{p},z_{k}\right\rangle }{\left\langle z_{k},z_{k}\right\rangle }z_{k}\neq0$.
In this case, $z_{p}$ is defined by the equality
(\ref{eq.thm.unitary.gs-2.zp=sum}). Solving this equality for $v_{p}$, we
obtain%
\[
v_{p}=z_{p}+\sum_{k=1}^{p-1}\dfrac{\left\langle v_{p},z_{k}\right\rangle
}{\left\langle z_{k},z_{k}\right\rangle }z_{k}\in\operatorname*{span}\left\{
z_{1},z_{2},\ldots,z_{p}\right\}  .
\]
Thus, (\ref{pf.thm.unitary.gs-2.5a}) is proved in Case 1.

\item \textit{Case 2:} We have $v_{p}-\sum_{k=1}^{p-1}\dfrac{\left\langle
v_{p},z_{k}\right\rangle }{\left\langle z_{k},z_{k}\right\rangle }z_{k}=0$. In
this case, we have%
\[
v_{p}=\sum_{k=1}^{p-1}\dfrac{\left\langle v_{p},z_{k}\right\rangle
}{\left\langle z_{k},z_{k}\right\rangle }z_{k}\in\operatorname*{span}\left\{
z_{1},z_{2},\ldots,z_{p-1}\right\}  \subseteq\operatorname*{span}\left\{
z_{1},z_{2},\ldots,z_{p}\right\}  .
\]
Hence, (\ref{pf.thm.unitary.gs-2.5a}) is proved in Case 2.
\end{itemize}

We have now proved (\ref{pf.thm.unitary.gs-2.5a}) in both cases. However, we
have
\[
\operatorname*{span}\left\{  v_{1},v_{2},\ldots,v_{p-1}\right\}
\subseteq\operatorname*{span}\left\{  z_{1},z_{2},\ldots,z_{p-1}\right\}  .
\]
(Indeed, this is obvious when $p=1$, and otherwise follows from
(\ref{pf.thm.unitary.gs-2.4}) (applied to $j=p-1$).) Thus, for each
$i\in\left[  p-1\right]  $, we have%
\begin{align*}
v_{i}  &  \in\left\{  v_{1},v_{2},\ldots,v_{p-1}\right\}  \subseteq
\operatorname*{span}\left\{  v_{1},v_{2},\ldots,v_{p-1}\right\} \\
&  \subseteq\operatorname*{span}\left\{  z_{1},z_{2},\ldots,z_{p-1}\right\}
\subseteq\operatorname*{span}\left\{  z_{1},z_{2},\ldots,z_{p}\right\}  .
\end{align*}
In other words, all $p-1$ vectors $v_{1},v_{2},\ldots,v_{p-1}$ belong to
$\operatorname*{span}\left\{  z_{1},z_{2},\ldots,z_{p}\right\}  $. Since the
vector $v_{p}$ also belongs to $\operatorname*{span}\left\{  z_{1}%
,z_{2},\ldots,z_{p}\right\}  $ (by (\ref{pf.thm.unitary.gs-2.5a})), we thus
conclude that all $p$ vectors $v_{1},v_{2},\ldots,v_{p}$ belong to
$\operatorname*{span}\left\{  z_{1},z_{2},\ldots,z_{p}\right\}  $. Therefore,
$\operatorname*{span}\left\{  v_{1},v_{2},\ldots,v_{p}\right\}  \subseteq
\operatorname*{span}\left\{  z_{1},z_{2},\ldots,z_{p}\right\}  $. Thus, the
induction step is complete, so that Claim 1 is proved by induction.]

It now remains to show that the tuple $\left(  z_{1},z_{2},\ldots
,z_{m}\right)  $ is orthogonal. We shall achieve this by showing the following claim:

\begin{statement}
\textit{Claim 2:} For any $j\in\left\{  0,1,\ldots,m\right\}  $, the tuple
$\left(  z_{1},z_{2},\ldots,z_{j}\right)  $ is orthogonal.
\end{statement}

[\textit{Proof of Claim 2:} We proceed by induction on $j$, similarly to the
proof of Claim 2 in the proof of Theorem \ref{thm.unitary.gs}. Only one minor
complication emerges in the induction step:

\textit{Induction step:} Let $p\in\left[  m\right]  $. Assume (as the
induction hypothesis) that Claim 2 holds for $j=p-1$. We must show that Claim
2 holds for $j=p$.

As in the proof of Theorem \ref{thm.unitary.gs}, we can convince ourselves
that it suffices to show that
\begin{equation}
z_{a}\perp z_{b}\ \ \ \ \ \ \ \ \ \ \text{whenever }a,b\in\left[  p\right]
\text{ satisfy }a\neq b. \label{pf.thm.unitary.gs-2.13}%
\end{equation}
Moreover, we only need to show this in the case when one of $a$ and $b$ equals
$p$ (because in all other cases, it follows from the induction hypothesis). In
other words, we only need to show that the vector $z_{p}$ is orthogonal to
each of $z_{1},z_{2},\ldots,z_{p-1}$.

Recall our definition of $z_{p}$. This definition distinguishes between two
cases, depending on whether the difference $v_{p}-\sum_{k=1}^{p-1}%
\dfrac{\left\langle v_{p},z_{k}\right\rangle }{\left\langle z_{k}%
,z_{k}\right\rangle }z_{k}$ is $\neq0$ or $=0$. In the first of these two
cases, the proof proceeds exactly as in the proof of Theorem
\ref{thm.unitary.gs}. Let us thus WLOG assume that we are in the second case.
That is, we assume that $v_{p}-\sum_{k=1}^{p-1}\dfrac{\left\langle v_{p}%
,z_{k}\right\rangle }{\left\langle z_{k},z_{k}\right\rangle }z_{k}=0$. Hence,
$z_{p}$ is defined by (\ref{eq.thm.unitary.gs-2.zp=b}), where $b$ is a nonzero
vector in $\mathbb{C}^{n}$ that is orthogonal to each of $z_{1},z_{2}%
,\ldots,z_{p-1}$. This shows that $z_{p}$ is orthogonal to each of
$z_{1},z_{2},\ldots,z_{p-1}$. But as we explained above, this is exactly what
we need to show. Thus, Claim 2 holds for $j=p$. The induction step is
complete, and Claim 2 is proved.]

Now, applying Claim 2 to $j=m$, we obtain that the tuple $\left(  z_{1}%
,z_{2},\ldots,z_{m}\right)  $ is orthogonal. Thus, the proof of Theorem
\ref{thm.unitary.gs-2} is complete.
\end{proof}

\begin{corollary}
\label{cor.unitary.gs-2nor}Let $\left(  v_{1},v_{2},\ldots,v_{m}\right)  $ be
any tuple of vectors in $\mathbb{C}^{n}$ with $m\leq n$.

Then, there is an orthonormal tuple $\left(  q_{1},q_{2},\ldots,q_{m}\right)
$ of vectors in $\mathbb{C}^{n}$ that satisfies%
\[
\operatorname*{span}\left\{  v_{1},v_{2},\ldots,v_{j}\right\}  \subseteq
\operatorname*{span}\left\{  q_{1},q_{2},\ldots,q_{j}\right\}
\ \ \ \ \ \ \ \ \ \ \text{for all }j\in\left[  m\right]  .
\]

\end{corollary}

\begin{proof}
[Proof of Corollary \ref{cor.unitary.gs-2nor}.]Theorem \ref{thm.unitary.gs-2}
shows that there is an orthogonal tuple $\left(  z_{1},z_{2},\ldots
,z_{m}\right)  $ of nonzero vectors in $\mathbb{C}^{n}$ that satisfies%
\[
\operatorname*{span}\left\{  v_{1},v_{2},\ldots,v_{j}\right\}  \subseteq
\operatorname*{span}\left\{  z_{1},z_{2},\ldots,z_{j}\right\}
\ \ \ \ \ \ \ \ \ \ \text{for all }j\in\left[  m\right]  .
\]
Consider this tuple $\left(  z_{1},z_{2},\ldots,z_{m}\right)  $. Proposition
\ref{prop.unitary.innerprod.orth-norm} (applied to $\left(  z_{1},z_{2}%
,\ldots,z_{m}\right)  $ instead of $\left(  u_{1},u_{2},\ldots,u_{k}\right)
$) then shows that the tuple%
\[
\left(  \dfrac{1}{\left\vert \left\vert z_{1}\right\vert \right\vert }%
z_{1},\ \ \dfrac{1}{\left\vert \left\vert z_{2}\right\vert \right\vert }%
z_{2},\ \ \ldots,\ \ \dfrac{1}{\left\vert \left\vert z_{m}\right\vert
\right\vert }z_{m}\right)
\]
is orthonormal. Moreover, we have
\[
\operatorname*{span}\left\{  v_{1},v_{2},\ldots,v_{j}\right\}  \subseteq
\operatorname*{span}\left\{  z_{1},z_{2},\ldots,z_{j}\right\}
=\operatorname*{span}\left\{  \dfrac{1}{\left\vert \left\vert z_{1}\right\vert
\right\vert }z_{1},\ \ \dfrac{1}{\left\vert \left\vert z_{2}\right\vert
\right\vert }z_{2},\ \ \ldots,\ \ \dfrac{1}{\left\vert \left\vert
z_{j}\right\vert \right\vert }z_{j}\right\}
\]
for all $j\in\left[  m\right]  $. Hence, Corollary \ref{cor.unitary.gs-2nor}
is proven (just take $q_{i}=\dfrac{1}{\left\vert \left\vert z_{i}\right\vert
\right\vert }z_{i}$).
\end{proof}

\subsection{QR factorization}

Recall that an isometry is a matrix whose columns form an orthonormal tuple.
(We saw this in Proposition \ref{prop.unitary.innerprod.isometry.2}.)

\begin{theorem}
[QR factorization, isometry version]\label{thm.unitary.QR1}Let $A\in
\mathbb{C}^{n\times m}$ satisfy $n\geq m$. Then, there exists an isometry
$Q\in\mathbb{C}^{n\times m}$ and an upper-triangular matrix $R\in
\mathbb{C}^{m\times m}$ such that $A=QR$.
\end{theorem}

The pair $\left(  Q,R\right)  $ in Theorem \ref{thm.unitary.QR1} is called a
\emph{QR factorization} of $A$. (We are using the indefinite article, since it
is usually not unique.)

\begin{example}
Let%
\[
A=\left(
\begin{array}
[c]{cccc}%
1 & 0 & 1 & 2\\
1 & -2 & 0 & 2\\
1 & 0 & 1 & 0\\
1 & -2 & 0 & 0
\end{array}
\right)  \in\mathbb{C}^{4\times4}.
\]
Then, one QR factorization of $A$ is given by%
\[
A=\underbrace{\left(
\begin{array}
[c]{cccc}%
1/2 & 1/2 & 1/2 & 1/2\\
1/2 & -1/2 & 1/2 & -1/2\\
1/2 & 1/2 & -1/2 & -1/2\\
1/2 & -1/2 & -1/2 & 1/2
\end{array}
\right)  }_{=Q}\underbrace{\left(
\begin{array}
[c]{cccc}%
2 & -2 & 1 & 0\\
0 & 2 & 1 & 2\\
0 & 0 & 0 & 0\\
0 & 0 & 0 & 2
\end{array}
\right)  }_{=R}.
\]
Another is given by%
\[
A=\underbrace{\left(
\begin{array}
[c]{cccc}%
1/2 & 1/2 & \sqrt{2}/2 & 0\\
1/2 & -1/2 & 0 & \sqrt{2}/2\\
1/2 & 1/2 & -\sqrt{2}/2 & 0\\
1/2 & -1/2 & 0 & -\sqrt{2}/2
\end{array}
\right)  }_{=Q}\underbrace{\left(
\begin{array}
[c]{cccc}%
2 & -2 & 1 & 2\\
0 & 2 & 1 & 0\\
0 & 0 & 0 & \sqrt{2}\\
0 & 0 & 0 & \sqrt{2}%
\end{array}
\right)  }_{=R}.
\]

\end{example}

\begin{proof}
[Proof of Theorem \ref{thm.unitary.QR1} (sketched).]Recall that $A_{\bullet
,1},A_{\bullet,2},\ldots,A_{\bullet,m}$ denote the $m$ columns of the matrix
$A$. Applying Corollary \ref{cor.unitary.gs-2nor} to $\left(  v_{1}%
,v_{2},\ldots,v_{m}\right)  =\left(  A_{\bullet,1},A_{\bullet,2}%
,\ldots,A_{\bullet,m}\right)  $, we conclude that there is an orthonormal
tuple $\left(  q_{1},q_{2},\ldots,q_{m}\right)  $ of vectors in $\mathbb{C}%
^{n}$ that satisfies%
\begin{align}
\operatorname*{span}\left\{  A_{\bullet,1},A_{\bullet,2},\ldots,A_{\bullet
,j}\right\}   &  \subseteq\operatorname*{span}\left\{  q_{1},q_{2}%
,\ldots,q_{j}\right\} \label{pf.thm.unitary.QR1.spansequal}\\
\ \ \ \ \ \ \ \ \ \ \text{for all }j  &  \in\left[  m\right]  .\nonumber
\end{align}
Consider this tuple $\left(  q_{1},q_{2},\ldots,q_{m}\right)  $. Let
$Q\in\mathbb{C}^{n\times m}$ be the matrix whose columns are $q_{1}%
,q_{2},\ldots,q_{m}$. Then, $Q$ is an isometry (by Proposition
\ref{prop.unitary.innerprod.isometry.2}, since its columns form an orthonormal
tuple). The definition of $Q$ shows that%
\begin{equation}
Q_{\bullet,i}=q_{i}\ \ \ \ \ \ \ \ \ \ \text{for each }i\in\left[  m\right]  .
\label{pf.thm.unitary.QR1.Qbull}%
\end{equation}


Now, let $j\in\left[  m\right]  $. Then,%
\[
A_{\bullet,j}\in\operatorname*{span}\left\{  A_{\bullet,1},A_{\bullet
,2},\ldots,A_{\bullet,j}\right\}  \subseteq\operatorname*{span}\left\{
q_{1},q_{2},\ldots,q_{j}\right\}
\]
(by (\ref{pf.thm.unitary.QR1.spansequal})). In other words, there exist
scalars $r_{1,j},r_{2,j},\ldots,r_{j,j}\in\mathbb{C}$ such that $A_{\bullet
,j}=\sum_{i=1}^{j}r_{i,j}q_{i}$. Consider these scalars $r_{1,j}%
,r_{2,j},\ldots,r_{j,j}$. Also, set
\begin{equation}
r_{i,j}=0\ \ \ \ \ \ \ \ \ \ \text{for each integer }i>j.
\label{pf.thm.unitary.QR1.triang}%
\end{equation}
Thus,%
\begin{equation}
A_{\bullet,j}=\sum_{i=1}^{j}r_{i,j}q_{i}=\sum_{i=1}^{m}r_{i,j}q_{i}
\label{pf.thm.unitary.QR1.col=}%
\end{equation}
(since $\sum_{i=1}^{m}r_{i,j}q_{i}=\sum_{i=1}^{j}r_{i,j}q_{i}+\sum_{i=j+1}%
^{m}\underbrace{r_{i,j}}_{\substack{=0\\\text{(by
(\ref{pf.thm.unitary.QR1.triang}))}}}q_{i}=\sum_{i=1}^{j}r_{i,j}q_{i}$).

Forget that we fixed $j$. Thus, for each $j\in\left[  m\right]  $, we have
defined scalars $r_{1,j},r_{2,j},r_{3,j},\ldots\in\mathbb{C}$ that satisfy
(\ref{pf.thm.unitary.QR1.triang}) and (\ref{pf.thm.unitary.QR1.col=}).

Now, let $R\in\mathbb{C}^{m\times m}$ be the $m\times m$-matrix whose $\left(
i,j\right)  $-th entry is $r_{i,j}$ for each $i,j\in\left[  n\right]  $. This
matrix $R$ is upper-triangular, because of (\ref{pf.thm.unitary.QR1.triang}).
The definition of $R$ yields%
\begin{equation}
R_{i,j}=r_{i,j}\ \ \ \ \ \ \ \ \ \ \text{for all }i,j\in\left[  m\right]  .
\label{pf.thm.unitary.QR1.Rij=}%
\end{equation}
Furthermore, for each $j\in\left[  m\right]  $, we have%
\begin{align*}
A_{\bullet,j}  &  =\sum_{i=1}^{m}\underbrace{r_{i,j}}_{\substack{=R_{i,j}%
\\\text{(by (\ref{pf.thm.unitary.QR1.Rij=}))}}}\underbrace{q_{i}%
}_{\substack{=Q_{\bullet,i}\\\text{(by (\ref{pf.thm.unitary.QR1.Qbull}))}%
}}\ \ \ \ \ \ \ \ \ \ \left(  \text{by (\ref{pf.thm.unitary.QR1.col=})}\right)
\\
&  =\sum_{i=1}^{m}R_{i,j}Q_{\bullet,i}=\left(  QR\right)  _{\bullet,j}%
\end{align*}
(by the definition of the product of two matrices). In other words, $A=QR$.

Thus, we have found an isometry $Q\in\mathbb{C}^{n\times m}$ and an
upper-triangular matrix $R\in\mathbb{C}^{m\times m}$ such that $A=QR$. This
proves Theorem \ref{thm.unitary.QR1}.
\end{proof}

Note that there are other variants of QR factorization, such as the following one:

\begin{theorem}
[QR factorization, unitary version]\label{thm.unitary.QR2}Let $A\in
\mathbb{C}^{n\times m}$. Then, there exists a unitary matrix $Q\in
\mathbb{C}^{n\times n}$ and an upper-triangular matrix $R\in\mathbb{C}%
^{n\times m}$ such that $A=QR$. Here, a rectangular matrix $R\in
\mathbb{C}^{n\times m}$ is said to be \emph{upper-triangular} if and only if
it satisfies%
\[
R_{i,j}=0\ \ \ \ \ \ \ \ \ \ \text{for all }i>j.
\]

\end{theorem}

\begin{exercise}
\fbox{5} Prove Theorem \ref{thm.unitary.QR2}.

[\textbf{Hint:} Reduce both cases $n>m$ and $n<m$ to the case $n=m$.]
\end{exercise}

\newpage

\begin{thebibliography}{99999999}                                                                                         %


\bibitem[AigZie14]{AigZie}%
\href{https://doi.org/10.1007/978-3-662-57265-8}{Martin Aigner, G\"{u}nter M.
Ziegler, \textit{Proofs from the Book}, 6th edition, Springer 2018.}

\bibitem[AndDos10]{AndDos}\href{https://bookstore.ams.org/xyz-13}{Titu
Andreescu, Gabriel Dospinescu, \textit{Problems from the Book}, 2nd edition,
XYZ Press 2010}.

\bibitem[AndDos12]{AndDosS}\href{https://bookstore.ams.org/xyz-6}{Titu
Andreescu, Gabriel Dospinescu, \textit{Straight from the Book}, XYZ Press
2012}.

\bibitem[Bartle14]{Bartle14}Padraic Bartlett, \textit{Math 108b: Advanced
Linear Algebra, Winter 2014}, 2014.\newline\url{http://web.math.ucsb.edu/~padraic/ucsb_2013_14/math108b_w2014/math108b_w2014.html}

\bibitem[Bourba74]{Bourba74}%
\href{http://libgen.rs/book/index.php?md5=3270565F6D0052635A1550883588204C}{Nicolas
Bourbaki, \textit{Algebra I: Chapters 1--3}, Addison-Wesley 1974}.

\bibitem[BoyDip12]{BoyDip12}William E. Boyce, Richard C. DiPrima,
\textit{Elementary Differential Equations}, 10th edition, Wiley 2012.

\bibitem[ChaSed97]{ChaSed97}%
\href{https://www.jstor.org/stable/10.4169/j.ctt19b9mbq}{Gengzhe Chang, Thomas
W. Sederberg, \textit{Over and Over Again}, Anneli Lax New Mathematical
Library \textbf{39}, The Mathematical Association of America 1997}.

\bibitem[Conrad]{Conrad}Keith Conrad, \textit{Expository notes
(\textquotedblleft blurbs\textquotedblright)}.\newline\url{https://kconrad.math.uconn.edu/blurbs/}

\bibitem[Edward05]{Edwards-Essays}%
\href{https://doi.org/10.1007/b138656}{Harold M. Edwards, \textit{Essays in
Constructive Mathematics}, Springer 2005}.\newline See
\url{https://www.math.nyu.edu/faculty/edwardsh/eserrata.pdf} for errata.

\bibitem[Edward95]{Edward95}%
\href{https://doi.org/10.1007/978-0-8176-4446-8}{Harold M. Edwards,
\textit{Linear Algebra}, Springer 1995}.

\bibitem[Elman20]{Elman20}Richard Elman, \textit{Lectures on Abstract
Algebra}, 28 September 2020.\newline\url{https://www.math.ucla.edu/~rse/algebra_book.pdf}

\bibitem[GalQua20]{GalQua20}Jean Gallier and Jocelyn Quaintance,
\textit{Algebra, Topology, Differential Calculus, and Optimization Theory For
Computer Science and Engineering}, 11 November 2020.\newline\url{https://www.cis.upenn.edu/~jean/gbooks/geomath.html}

\bibitem[Geck20]{Geck20}\href{https://doi.org/10.13001/ela.2020.5055}{Meinolf
Geck, \textit{On Jacob's construction of the rational canonical form of a
matrix}, Electronic Journal of Linear Algebra \textbf{36} (2020), pp.
177--182}.

\bibitem[GelAnd17]{GelAnd}%
\href{https://doi.org/10.1007/978-3-319-58988-6}{R\u{a}zvan Gelca, Titu
Andreescu, \textit{Putnam and Beyond}, 2nd edition, Springer 2017}.

\bibitem[Goodma15]{Goodman}Frederick M. Goodman, \textit{Algebra: Abstract and
Concrete}, edition 2.6, 1 May 2015.\newline%
\url{http://homepage.math.uiowa.edu/~goodman/algebrabook.dir/book.2.6.pdf} .

\bibitem[Grinbe15]{detnotes}Darij Grinberg, \textit{Notes on the combinatorial
fundamentals of algebra}, 10 January 2019.\newline%
\url{http://www.cip.ifi.lmu.de/~grinberg/primes2015/sols.pdf} \newline The
numbering of theorems and formulas in this link might shift when the project
gets updated; for a \textquotedblleft frozen\textquotedblright\ version whose
numbering is guaranteed to match that in the citations above, see
\url{https://github.com/darijgr/detnotes/releases/tag/2019-01-10} .

\bibitem[Grinbe19]{trach}Darij Grinberg, \textit{The trace Cayley-Hamilton
theorem}, 14 July 2019.\newline\url{https://www.cip.ifi.lmu.de/~grinberg/algebra/trach.pdf}

\bibitem[Heffer20]{Heffer20}Jim Hefferon, \textit{Linear Algebra}, 4th edition
2020.\newline\url{http://joshua.smcvt.edu/linearalgebra}

\bibitem[Ho14]{Ho-rear2}%
\href{https://www.math.hkust.edu.hk/excalibur/v19_n3.pdf}{Law Ka Ho,
\textit{Variations and Generalisations to the Rearrangement Inequality},
Mathematical Excalibur \textbf{19}, Number 3, pp. 1--2, 4}.

\bibitem[HorJoh13]{HorJoh13}%
\href{http://www.cse.zju.edu.cn/eclass/attachments/2015-10/01-1446086008-145421.pdf}{Roger
A. Horn, Charles R. Johnson, \textit{Matrix analysis}, Cambridge University
Press, 2nd edition 2013}.

\bibitem[Hung07]{Hung07}%
\href{http://refkol.ro/matek/mathbooks/!Books!/Secrets in Inequalities (volume 1) Pham Kim Hung.pdf}{Pham
Kim Hung, \textit{Secrets in Inequalities, volume 1}, GIL 2007}.

\bibitem[Ivanov08]{Ivanov08}Nikolai V. Ivanov, \textit{Linear Recurrences}, 17
January 2008.\newline\url{https://nikolaivivanov.files.wordpress.com/2014/02/ivanov2008arecurrence.pdf}

\bibitem[Knapp16]{Knapp1}Anthony W. Knapp, \textit{Basic Algebra}, digital
second edition 2016.\newline\url{http://www.math.stonybrook.edu/~aknapp/download.html}

\bibitem[Korner20]{Korner20}T. W. K\"{o}rner, \textit{Where Do Numbers Come
From?}, Cambridge University Press 2020.\newline See
\url{https://web.archive.org/web/20190813160507/https://www.dpmms.cam.ac.uk/~twk/Number.pdf}
for a preprint.\newline See \url{https://www.dpmms.cam.ac.uk/~twk/} for errata
and solutions.

\bibitem[LaNaSc16]{LaNaSc16}Isaiah Lankham, Bruno Nachtergaele, Anne
Schilling, \textit{Linear Algebra As an Introduction to Abstract Mathematics},
2016.\newline\url{https://www.math.ucdavis.edu/~anne/linear_algebra/mat67_course_notes.pdf}

\bibitem[Li99]{Li-rear1}%
\href{https://www.math.hkust.edu.hk/excalibur/v4_n3.pdf}{Kin-Yin Li,
\textit{Rearrangement Inequality}, Mathematical Excalibur \textbf{4}, Number
3, pp. 1--2, 4}.

\bibitem[Loehr14]{Loehr14}%
\href{https://elblogdecontar.files.wordpress.com/2017/01/ebookdaraz-advanced-linear-algebra.pdf}{Nicholas
Loehr, \textit{Advanced Linear Algebra}, CRC Press 2014}.

\bibitem[Markus83]{Markus83}%
\href{https://archive.org/details/recursion-sequences}{Aleksei Ivanovich
Markushevich, \textit{Recursion sequences}, Mir Publishers, Moscow, 2nd
printing 1983}.

\bibitem[Melian01]{Melian01}Mar\'{\i}a Victoria Meli\'{a}n, \textit{Linear
recurrence relations with constant coefficients}, 9 April 2001.\newline\url{http://matematicas.uam.es/~mavi.melian/CURSO_15_16/web_Discreta/recurrence.pdf}

\bibitem[Nathan21]{Nathan21}\href{https://arxiv.org/abs/2109.01746v1}{Melvyn
B. Nathanson, \textit{The Muirhead-Rado inequality, 1 Vector majorization and
the permutohedron}, arXiv:2109.01746v1}.

\bibitem[PolSze78]{PolSze78}%
\href{https://doi.org/10.1007/978-3-642-61983-0}{George P\'{o}lya, Gabor
Szeg\H{o}, \textit{Problems and Theorems in Analysis I}, Springer 1978
(reprinted 1998)}.

\bibitem[Prasol94]{Prasolov}Viktor V. Prasolov,
\textit{\href{http://www2.math.su.se/~mleites/books/prasolov-1994-problems.pdf}{\textit{Problems
and Theorems in Linear Algebra}}}, Translations of Mathematical Monographs,
vol. \#134, AMS 1994.

\bibitem[Steele04]{Steele04}%
\href{http://www.ma.huji.ac.il/~ehudf/courses/Ineq09/The Cauchy-Schwarz Master Class .pdf}{J.
Michael Steele, \textit{The Cauchy--Schwarz Master Class: An Introduction to
the Art of Mathematical Inequalities}, Cambridge University Press
2004}.\newline See
\url{http://www-stat.wharton.upenn.edu/~steele/Publications/Books/CSMC/CSMC_Errata.pdf}
for errata.

\bibitem[Steinb06]{Steinb06}Mark Steinberger, \textit{Algebra}, 31 August
2006.\newline\url{https://web.archive.org/web/20180821125315/https://www.albany.edu/~mark/algebra.pdf}

\bibitem[Strick20]{Strick20}Neil Strickland, \textit{Linear mathematics for
applications}, 11 February 2020.\newline\url{https://neilstrickland.github.io/linear_maths/notes/linear_maths.pdf}

\bibitem[Swanso20]{Swanso20}Irene Swanson, \textit{Introduction with Analysis
with Complex Numbers}, 2020.\newline\url{https://web.archive.org/web/20201012174324/https://people.reed.edu/~iswanson/analysisconstructR.pdf}

\bibitem[Taylor20]{Taylor20}%
\href{https://bookstore.ams.org/amstext-45/}{Michael Taylor, \textit{Linear
Algebra}, AMS 2020}.\newline See
\url{https://mtaylor.web.unc.edu/wp-content/uploads/sites/16915/2018/04/linalg.pdf}
for a preprint.

\bibitem[Treil15]{Treil15}Sergei Treil, \textit{Linear Algebra Done Wrong},
2017.\newline\url{https://www.math.brown.edu/~treil/papers/LADW/LADW.html}

\bibitem[Walker87]{Walker87}%
\href{https://web.archive.org/web/20170809055317/https://www.math.nmsu.edu/~elbert/AbsAlgeb.pdf}{Elbert
A. Walker, \textit{Introduction to Abstract Algebra}, Random House/Birkhauser,
New York, 1987.}

\bibitem[Woerde16]{Woerde16}Hugo J. Woerdeman, \textit{Advanced Linear
Algebra}, CRC Press 2016.

\bibitem[Zill17]{Zill17}Dennis G. Zill, \textit{A First Course in Differential
Equations with Modeling Applications}, Cengage 2017.
\end{thebibliography}


\end{document}