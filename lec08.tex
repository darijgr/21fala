\documentclass[numbers=enddot,12pt,final,onecolumn,notitlepage]{scrartcl}%
\usepackage[headsepline,footsepline,manualmark]{scrlayer-scrpage}
\usepackage[all,cmtip]{xy}
\usepackage{amssymb}
\usepackage{amsmath}
\usepackage{amsthm}
\usepackage{framed}
\usepackage{comment}
\usepackage{color}
\usepackage{hyperref}
\usepackage[sc]{mathpazo}
\usepackage[T1]{fontenc}
\usepackage{tikz}
\usepackage{needspace}
\usepackage{tabls}
\usepackage{wasysym}
\usepackage{easytable}
\usepackage{pythonhighlight}
%TCIDATA{OutputFilter=latex2.dll}
%TCIDATA{Version=5.50.0.2960}
%TCIDATA{LastRevised=Wednesday, October 06, 2021 11:50:22}
%TCIDATA{SuppressPackageManagement}
%TCIDATA{<META NAME="GraphicsSave" CONTENT="32">}
%TCIDATA{<META NAME="SaveForMode" CONTENT="1">}
%TCIDATA{BibliographyScheme=Manual}
%TCIDATA{Language=American English}
%BeginMSIPreambleData
\providecommand{\U}[1]{\protect\rule{.1in}{.1in}}
%EndMSIPreambleData
\usetikzlibrary{arrows.meta}
\usetikzlibrary{chains}
\newcounter{exer}
\newcounter{exera}
\numberwithin{exer}{subsection}
\theoremstyle{definition}
\newtheorem{theo}{Theorem}[subsection]
\newenvironment{theorem}[1][]
{\begin{theo}[#1]\begin{leftbar}}
{\end{leftbar}\end{theo}}
\newtheorem{lem}[theo]{Lemma}
\newenvironment{lemma}[1][]
{\begin{lem}[#1]\begin{leftbar}}
{\end{leftbar}\end{lem}}
\newtheorem{prop}[theo]{Proposition}
\newenvironment{proposition}[1][]
{\begin{prop}[#1]\begin{leftbar}}
{\end{leftbar}\end{prop}}
\newtheorem{defi}[theo]{Definition}
\newenvironment{definition}[1][]
{\begin{defi}[#1]\begin{leftbar}}
{\end{leftbar}\end{defi}}
\newtheorem{remk}[theo]{Remark}
\newenvironment{remark}[1][]
{\begin{remk}[#1]\begin{leftbar}}
{\end{leftbar}\end{remk}}
\newtheorem{coro}[theo]{Corollary}
\newenvironment{corollary}[1][]
{\begin{coro}[#1]\begin{leftbar}}
{\end{leftbar}\end{coro}}
\newtheorem{conv}[theo]{Convention}
\newenvironment{convention}[1][]
{\begin{conv}[#1]\begin{leftbar}}
{\end{leftbar}\end{conv}}
\newtheorem{quest}[theo]{Question}
\newenvironment{question}[1][]
{\begin{quest}[#1]\begin{leftbar}}
{\end{leftbar}\end{quest}}
\newtheorem{warn}[theo]{Warning}
\newenvironment{warning}[1][]
{\begin{warn}[#1]\begin{leftbar}}
{\end{leftbar}\end{warn}}
\newtheorem{conj}[theo]{Conjecture}
\newenvironment{conjecture}[1][]
{\begin{conj}[#1]\begin{leftbar}}
{\end{leftbar}\end{conj}}
\newtheorem{exam}[theo]{Example}
\newenvironment{example}[1][]
{\begin{exam}[#1]\begin{leftbar}}
{\end{leftbar}\end{exam}}
\newtheorem{exmp}[exer]{Exercise}
\newenvironment{exercise}[1][]
{\begin{exmp}[#1]\begin{leftbar}}
{\end{leftbar}\end{exmp}}
\newenvironment{statement}{\begin{quote}}{\end{quote}}
\newenvironment{fineprint}{\medskip \begin{small}}{\end{small} \medskip}
\iffalse
\newenvironment{proof}[1][Proof]{\noindent\textbf{#1.} }{\ \rule{0.5em}{0.5em}}
\newenvironment{question}[1][Question]{\noindent\textbf{#1.} }{\ \rule{0.5em}{0.5em}}
\newenvironment{warning}[1][Warning]{\noindent\textbf{#1.} }{\ \rule{0.5em}{0.5em}}
\newenvironment{teachingnote}[1][Teaching note]{\noindent\textbf{#1.} }{\ \rule{0.5em}{0.5em}}
\fi
\let\sumnonlimits\sum
\let\prodnonlimits\prod
\let\cupnonlimits\bigcup
\let\capnonlimits\bigcap
\renewcommand{\sum}{\sumnonlimits\limits}
\renewcommand{\prod}{\prodnonlimits\limits}
\renewcommand{\bigcup}{\cupnonlimits\limits}
\renewcommand{\bigcap}{\capnonlimits\limits}
\setlength\tablinesep{3pt}
\setlength\arraylinesep{3pt}
\setlength\extrarulesep{3pt}
\voffset=0cm
\hoffset=-0.7cm
\setlength\textheight{22.5cm}
\setlength\textwidth{15.5cm}
\newcommand\arxiv[1]{\href{http://www.arxiv.org/abs/#1}{\texttt{arXiv:#1}}}
\newenvironment{verlong}{}{}
\newenvironment{vershort}{}{}
\newenvironment{noncompile}{}{}
\newenvironment{teachingnote}{}{}
\excludecomment{verlong}
\includecomment{vershort}
\excludecomment{noncompile}
\excludecomment{teachingnote}
\newcommand{\CC}{\mathbb{C}}
\newcommand{\RR}{\mathbb{R}}
\newcommand{\QQ}{\mathbb{Q}}
\newcommand{\NN}{\mathbb{N}}
\newcommand{\ZZ}{\mathbb{Z}}
\newcommand{\KK}{\mathbb{K}}
\newcommand{\id}{\operatorname{id}}
\newcommand{\lcm}{\operatorname{lcm}}
\newcommand{\rev}{\operatorname{rev}}
\newcommand{\powset}[2][]{\ifthenelse{\equal{#2}{}}{\mathcal{P}\left(#1\right)}{\mathcal{P}_{#1}\left(#2\right)}}
\newcommand{\set}[1]{\left\{ #1 \right\}}
\newcommand{\abs}[1]{\left| #1 \right|}
\newcommand{\tup}[1]{\left( #1 \right)}
\newcommand{\ive}[1]{\left[ #1 \right]}
\newcommand{\floor}[1]{\left\lfloor #1 \right\rfloor}
\newcommand{\lf}[2]{#1^{\underline{#2}}}
\newcommand{\underbrack}[2]{\underbrace{#1}_{\substack{#2}}}
\newcommand{\horrule}[1]{\rule{\linewidth}{#1}}
\newcommand{\are}{\ar@{-}}
\newcommand{\nnn}{\nonumber\\}
\newcommand{\sslash}{\mathbin{/\mkern-6mu/}}
\newcommand{\numboxed}[2]{\underbrace{\boxed{#1}}_{\text{box } #2}}
\newcommand{\ig}[2]{\includegraphics[scale=#1]{#2.png}}
\newcommand{\UNFINISHED}{\begin{center} \Huge{\textbf{Unfinished material begins here.}} \end{center} }
\iffalse
\NOEXPAND{\today}{\today}
\NOEXPAND{\sslash}{\sslash}
\NOEXPAND{\numboxed}[2]{\numboxed}
\NOEXPAND{\UNFINISHED}{\UNFINISHED}
\fi
\ihead{Math 504 notes}
\ohead{page \thepage}
\cfoot{\today}
\begin{document}

\title{Math 504: Advanced Linear Algebra}
\author{Hugo Woerdeman, with edits by Darij Grinberg\thanks{Drexel University, Korman
Center, 15 S 33rd Street, Philadelphia PA, 19104, USA}}
\date{\today\ (unfinished!)}
\maketitle
\tableofcontents

\section*{Math 504 Lecture 8}

\section{Jordan canonical (aka normal) form (cont'd)}

\subsection{The Jordan canonical form (aka Jordan normal form) (cont'd)}

\textbf{Recall from last lecture:}

\begin{definition}
A \textbf{Jordan cell} is a $m\times m$-matrix of the form%
\[
\left(
\begin{array}
[c]{ccccc}%
\lambda & 1 & 0 & \cdots & 0\\
0 & \lambda & 1 & \cdots & 0\\
0 & 0 & \lambda & \cdots & 0\\
\vdots & \vdots & \vdots & \ddots & \vdots\\
0 & 0 & 0 & \cdots & \lambda
\end{array}
\right)  \ \ \ \ \ \ \ \ \ \ \text{for some }m>0\text{ and some }\lambda
\in\mathbb{C}.
\]
So its diagonal entries are $\lambda$; its entries directly above the diagonal
are $1$; all its other entries are $0$. In formal terms, it is the $m\times
m$-matrix $A$ whose entries are given by the rule%
\[
A_{i,j}=%
\begin{cases}
\lambda, & \text{if }i=j;\\
1, & \text{if }i=j-1;\\
0, & \text{otherwise.}%
\end{cases}
\]


Specifically, this matrix is called the \textbf{Jordan cell of size }$m$
\textbf{at eigenvalue }$\lambda$. It is denoted by $J_{m}\left(
\lambda\right)  $.
\end{definition}

\begin{definition}
A \textbf{Jordan matrix} is a block-diagonal matrix whose diagonal blocks are
Jordan cells. In other words, it is a matrix of the form%
\[
\left(
\begin{array}
[c]{cccc}%
J_{n_{1}}\left(  \lambda_{1}\right)  & 0 & \cdots & 0\\
0 & J_{n_{2}}\left(  \lambda_{2}\right)  & \cdots & 0\\
\vdots & \vdots & \ddots & \vdots\\
0 & 0 & \cdots & J_{n_{k}}\left(  \lambda_{k}\right)
\end{array}
\right)  ,
\]
where $n_{1},n_{2},\ldots,n_{k}$ are positive integers and $\lambda
_{1},\lambda_{2},\ldots,\lambda_{k}$ are scalars in $\mathbb{C}$ (not
necessarily distinct, but not necessarily equal either).
\end{definition}

We claimed:

\begin{theorem}
[Jordan canonical form theorem]Let $A$ be an $n\times n$-matrix over
$\mathbb{C}$. Then, there exists a Jordan matrix $J$ such that $A\sim J$.
Furthermore, this $J$ is unique up to the order of the diagonal blocks.
\end{theorem}

\begin{definition}
The matrix $J$ in this theorem is called a \textbf{Jordan canonical form} of
$A$ (or a \textbf{Jordan normal form} of $A$).
\end{definition}

\begin{proposition}
Let $A$ be an $n\times n$-matrix, and let $J=\left(
\begin{array}
[c]{cccc}%
J_{n_{1}}\left(  \lambda_{1}\right)  & 0 & \cdots & 0\\
0 & J_{n_{2}}\left(  \lambda_{2}\right)  & \cdots & 0\\
\vdots & \vdots & \ddots & \vdots\\
0 & 0 & \cdots & J_{n_{k}}\left(  \lambda_{k}\right)
\end{array}
\right)  $ be its Jordan canonical form. Then:

\textbf{(a)} We have $\sigma\left(  A\right)  =\left\{  \lambda_{1}%
,\lambda_{2},\ldots,\lambda_{k}\right\}  $.

\textbf{(b)} The geometric multiplicity of an eigenvalue $\lambda$ of $A$ is
the number of Jordan cells of $A$ at eigenvalue $\lambda$. In other words, it
is the number of $i\in\left[  k\right]  $ satisfying $\lambda_{i}=\lambda$.

\textbf{(c)} The algebraic multiplicity of an eigenvalue $\lambda$ of $A$ is
the \textbf{sum} of the sizes of all Jordan cells of $A$ at eigenvalue
$\lambda$. In other words, it is $\sum\limits_{\substack{i\in\left[  k\right]
;\\\lambda_{i}=\lambda}}n_{i}$.
\end{proposition}

Now, let us improve on this result somewhat:

\begin{proposition}
Let $A$ be an $n\times n$-matrix, and let $J=\left(
\begin{array}
[c]{cccc}%
J_{n_{1}}\left(  \lambda_{1}\right)   & 0 & \cdots & 0\\
0 & J_{n_{2}}\left(  \lambda_{2}\right)   & \cdots & 0\\
\vdots & \vdots & \ddots & \vdots\\
0 & 0 & \cdots & J_{n_{k}}\left(  \lambda_{k}\right)
\end{array}
\right)  $ be its Jordan canonical form. Let $\lambda\in\mathbb{C}$ and
$m\geq1$. Then,%
\begin{align*}
& \left(  \text{the number of }i\in\left[  k\right]  \text{ such that }%
\lambda_{i}=\lambda\text{ and }n_{i}\geq m\right)  \\
& =\dim\left(  \operatorname*{Ker}\left(  \left(  A-\lambda I_{n}\right)
^{m}\right)  \right)  -\dim\left(  \operatorname*{Ker}\left(  \left(
A-\lambda I_{n}\right)  ^{m-1}\right)  \right)  .
\end{align*}

\end{proposition}

This proposition gives us a way to tell how many Jordan blocks of the Jordan
form of $A$ have eigenvalue $\lambda$ and size $\geq m$. As a consequence,
this number is uniquely determined by $A$ and $\lambda$. Hence, the whole
structure of $J$ is determined uniquely by $A$, up to the order of the Jordan
blocks. This shows the \textquotedblleft uniqueness\textquotedblright\ part of
the Jordan canonical form theorem, as long as we can prove the proposition.

\begin{example}
Let $A$ be an $8\times8$-matrix. Assume that we know that $A$ has

\begin{itemize}
\item $1$ Jordan block of size $\geq1$ for eigenvalue $17$;

\item $0$ Jordan blocks of size $\geq2$ for eigenvalue $17$;

\item $3$ Jordan blocks of size $\geq1$ for eigenvalue $35$;

\item $1$ Jordan block of size $\geq2$ for eigenvalue $35$;

\item $0$ Jordan blocks of size $\geq3$ for eigenvalue $35$;

\item $1$ Jordan block of size $\geq1$ for eigenvalue $59$;

\item $1$ Jordan block of size $\geq2$ for eigenvalue $59$;

\item $1$ Jordan block of size $\geq3$ for eigenvalue $59$;

\item $0$ Jordan blocks of size $\geq4$ for eigenvalue $59$;

\item $0$ Jordan blocks of size $\geq1$ for eigenvalue $\lambda$ whenever
$\lambda\notin\left\{  17,35,59\right\}  $.
\end{itemize}

(This is the sort of information you can obtain from $A$ using the preceding proposition.)

How does the Jordan canonical form of $A$ look like? It is the block-diagonal
matrix%
\[
\left(
\begin{array}
[c]{ccccc}%
J_{1}\left(  17\right)   &  &  &  & \\
& J_{2}\left(  35\right)   &  &  & \\
&  & J_{1}\left(  35\right)   &  & \\
&  &  & J_{1}\left(  35\right)   & \\
&  &  &  & J_{3}\left(  59\right)
\end{array}
\right)
\]
(where all invisible entries are $0$s), or one that is obtained from it by
permuting the diagonal blocks.
\end{example}

Let us prove the proposition:

\begin{proof}
[Proof of Proposition.] We have $A\sim J$, so that $A-\lambda I_{n}\sim
J-\lambda I_{n}$, so that $\left(  A-\lambda I_{n}\right)  ^{m}\sim\left(
J-\lambda I_{n}\right)  ^{m}$ and therefore%
\[
\dim\left(  \operatorname*{Ker}\left(  \left(  A-\lambda I_{n}\right)
^{m}\right)  \right)  =\dim\left(  \operatorname*{Ker}\left(  \left(
J-\lambda I_{n}\right)  ^{m}\right)  \right)  .
\]
However%
\begin{align*}
J-\lambda I_{n}  & =\left(
\begin{array}
[c]{cccc}%
J_{n_{1}}\left(  \lambda_{1}\right)   & 0 & \cdots & 0\\
0 & J_{n_{2}}\left(  \lambda_{2}\right)   & \cdots & 0\\
\vdots & \vdots & \ddots & \vdots\\
0 & 0 & \cdots & J_{n_{k}}\left(  \lambda_{k}\right)
\end{array}
\right)  -\lambda I_{n}\\
& =\left(
\begin{array}
[c]{cccc}%
J_{n_{1}}\left(  \lambda_{1}-\lambda\right)   & 0 & \cdots & 0\\
0 & J_{n_{2}}\left(  \lambda_{2}-\lambda\right)   & \cdots & 0\\
\vdots & \vdots & \ddots & \vdots\\
0 & 0 & \cdots & J_{n_{k}}\left(  \lambda_{k}-\lambda\right)
\end{array}
\right)  ,
\end{align*}
so that%
\begin{align*}
\left(  J-\lambda I_{n}\right)  ^{m}  & =\left(
\begin{array}
[c]{cccc}%
J_{n_{1}}\left(  \lambda_{1}-\lambda\right)   & 0 & \cdots & 0\\
0 & J_{n_{2}}\left(  \lambda_{2}-\lambda\right)   & \cdots & 0\\
\vdots & \vdots & \ddots & \vdots\\
0 & 0 & \cdots & J_{n_{k}}\left(  \lambda_{k}-\lambda\right)
\end{array}
\right)  ^{m}\\
& =\left(
\begin{array}
[c]{cccc}%
\left(  J_{n_{1}}\left(  \lambda_{1}-\lambda\right)  \right)  ^{m} & 0 &
\cdots & 0\\
0 & \left(  J_{n_{2}}\left(  \lambda_{2}-\lambda\right)  \right)  ^{m} &
\cdots & 0\\
\vdots & \vdots & \ddots & \vdots\\
0 & 0 & \cdots & \left(  J_{n_{k}}\left(  \lambda_{k}-\lambda\right)  \right)
^{m}%
\end{array}
\right)  \\
& \ \ \ \ \ \ \ \ \ \ \ \ \ \ \ \ \ \ \ \ \left(
\begin{array}
[c]{c}%
\text{because multiplication of block-diagonal matrices}\\
\text{means that respective blocks get multiplied}%
\end{array}
\right)  .
\end{align*}
Thus,%
\begin{align*}
& \dim\left(  \operatorname*{Ker}\left(  \left(  J-\lambda I_{n}\right)
^{m}\right)  \right)  \\
& =\dim\left(  \operatorname*{Ker}\left(
\begin{array}
[c]{cccc}%
\left(  J_{n_{1}}\left(  \lambda_{1}-\lambda\right)  \right)  ^{m} & 0 &
\cdots & 0\\
0 & \left(  J_{n_{2}}\left(  \lambda_{2}-\lambda\right)  \right)  ^{m} &
\cdots & 0\\
\vdots & \vdots & \ddots & \vdots\\
0 & 0 & \cdots & \left(  J_{n_{k}}\left(  \lambda_{k}-\lambda\right)  \right)
^{m}%
\end{array}
\right)  \right)  \\
& =\dim\left(  \operatorname*{Ker}\left(  \left(  J_{n_{1}}\left(  \lambda
_{1}-\lambda\right)  \right)  ^{m}\right)  \right)  +\dim\left(
\operatorname*{Ker}\left(  \left(  J_{n_{2}}\left(  \lambda_{2}-\lambda
\right)  \right)  ^{m}\right)  \right)  +\cdots+\dim\left(
\operatorname*{Ker}\left(  \left(  J_{n_{k}}\left(  \lambda_{k}-\lambda
\right)  \right)  ^{m}\right)  \right)  \\
& =\sum_{i=1}^{k}\dim\left(  \operatorname*{Ker}\left(  \left(  J_{n_{i}%
}\left(  \lambda_{i}-\lambda\right)  \right)  ^{m}\right)  \right)  .
\end{align*}


Now, fix $i\in\left[  k\right]  $. If $\lambda_{i}\neq\lambda$, then
$J_{n_{i}}\left(  \lambda_{i}-\lambda\right)  $ is a triangular matrix with
nonzero entries on the diagonal (in fact, the diagonal entries are all
$\lambda_{i}-\lambda\neq0$), and thus has determinant $\neq0$ and therefore is
invertible. Hence, in this case, its power $\left(  J_{n_{i}}\left(
\lambda_{i}-\lambda\right)  \right)  ^{m}$ is also invertible, so it has
nullity $0$. Thus,%
\[
\text{if }\lambda_{i}\neq\lambda\text{, then }\dim\left(  \operatorname*{Ker}%
\left(  \left(  J_{n_{i}}\left(  \lambda_{i}-\lambda\right)  \right)
^{m}\right)  \right)  =0\text{.}%
\]


Now consider the case when $\lambda_{i}=\lambda$. Then, $J_{n_{i}}\left(
\lambda_{i}-\lambda\right)  =J_{n_{i}}\left(  0\right)  =\left(
\begin{array}
[c]{ccccc}%
0 & 1 & 0 & \cdots & 0\\
0 & 0 & 1 & \cdots & 0\\
0 & 0 & 0 & \cdots & 0\\
\vdots & \vdots & \vdots & \ddots & \vdots\\
0 & 0 & 0 & \cdots & 0
\end{array}
\right)  $. Let me call this matrix $B$. From last time, we know that
\[
\dim\left(  \operatorname*{Ker}\left(  B^{m}\right)  \right)  =%
\begin{cases}
m, & \text{if }m\leq n_{i};\\
n_{i}, & \text{if }m>n_{i}.
\end{cases}
\]
Thus,%
\[
\dim\left(  \operatorname*{Ker}\left(  \left(  J_{n_{i}}\left(  \lambda
_{i}-\lambda\right)  \right)  ^{m}\right)  \right)  =%
\begin{cases}
m, & \text{if }m\leq n_{i};\\
n_{i}, & \text{if }m>n_{i}%
\end{cases}
\ \ \ \ \ \ \ \ \ \ \text{in this case.}%
\]


Now let us see what this means for our sum:%
\begin{align*}
& \sum_{i=1}^{k}\dim\left(  \operatorname*{Ker}\left(  \left(  J_{n_{i}%
}\left(  \lambda_{i}-\lambda\right)  \right)  ^{m}\right)  \right)  \\
& =\sum_{\substack{i\in\left[  k\right]  ;\\\lambda_{i}=\lambda}}%
\begin{cases}
m, & \text{if }m\leq n_{i};\\
n_{i}, & \text{if }m>n_{i}.
\end{cases}
\end{align*}
So%
\begin{align*}
& \dim\left(  \operatorname*{Ker}\left(  \left(  A-\lambda I_{n}\right)
^{m}\right)  \right)  \\
& =\dim\left(  \operatorname*{Ker}\left(  \left(  J-\lambda I_{n}\right)
^{m}\right)  \right)  \\
& =\sum_{i=1}^{k}\dim\left(  \operatorname*{Ker}\left(  \left(  J_{n_{i}%
}\left(  \lambda_{i}-\lambda\right)  \right)  ^{m}\right)  \right)  \\
& =\sum_{\substack{i\in\left[  k\right]  ;\\\lambda_{i}=\lambda}}%
\begin{cases}
m, & \text{if }m\leq n_{i};\\
n_{i}, & \text{if }m>n_{i}.
\end{cases}
\end{align*}
The same argument, applied to $m-1$ instead of $m$, yields%
\begin{align*}
& \dim\left(  \operatorname*{Ker}\left(  \left(  A-\lambda I_{n}\right)
^{m-1}\right)  \right)  \\
& =\sum_{\substack{i\in\left[  k\right]  ;\\\lambda_{i}=\lambda}}%
\begin{cases}
m-1; & \text{if }m-1\leq n_{i};\\
n_{i}, & \text{if }m-1>n_{i}.
\end{cases}
\end{align*}
Subtracting these two equalities, we obtain%
\begin{align*}
& \dim\left(  \operatorname*{Ker}\left(  \left(  A-\lambda I_{n}\right)
^{m}\right)  \right)  -\dim\left(  \operatorname*{Ker}\left(  \left(
A-\lambda I_{n}\right)  ^{m-1}\right)  \right)  \\
& =\sum_{\substack{i\in\left[  k\right]  ;\\\lambda_{i}=\lambda}}%
\begin{cases}
m, & \text{if }m\leq n_{i};\\
n_{i}, & \text{if }m>n_{i}%
\end{cases}
-\sum_{\substack{i\in\left[  k\right]  ;\\\lambda_{i}=\lambda}}%
\begin{cases}
m-1; & \text{if }m-1\leq n_{i};\\
n_{i}, & \text{if }m-1>n_{i}%
\end{cases}
\\
& =\sum_{\substack{i\in\left[  k\right]  ;\\\lambda_{i}=\lambda}%
}\underbrace{\left(
\begin{cases}
m, & \text{if }m\leq n_{i};\\
n_{i}, & \text{if }m>n_{i}%
\end{cases}
-%
\begin{cases}
m-1; & \text{if }m-1\leq n_{i};\\
n_{i}, & \text{if }m-1>n_{i}%
\end{cases}
\right)  }_{\substack{=%
\begin{cases}
1, & \text{if }m\leq n_{i};\\
0, & \text{if }m>n_{i}%
\end{cases}
\\\text{(this can be directly checked in each}\\\text{of the three cases
}m\leq n_{i}\text{ and }m=n_{i}+1\text{ and }m>n_{i}+1\text{)}}}\\
& =\sum_{\substack{i\in\left[  k\right]  ;\\\lambda_{i}=\lambda}}%
\begin{cases}
1, & \text{if }m\leq n_{i};\\
0, & \text{if }m>n_{i}%
\end{cases}
\\
& =\left(  \text{the number of }i\in\left[  k\right]  \text{ such that
}\lambda_{i}=\lambda\text{ and }m\leq n_{i}\right)  \\
& =\left(  \text{the number of }i\in\left[  k\right]  \text{ such that
}\lambda_{i}=\lambda\text{ and }n_{i}\geq m\right)  .
\end{align*}
Thus, the proposition follows.
\end{proof}

So we are done proving the uniqueness part of the JCF theorem.
(\textquotedblleft JCF\textquotedblright\ is short for \textquotedblleft
Jordan Canonical Form\textquotedblright.)

Let us now approach the existence part.

\subsection{Proof of the existence part of the JCF}

\subsubsection{Step 1: Schur triangularization}

We have an $n\times n$-matrix $A\in\mathbb{C}^{n\times n}$, and we want to
make it into a Jordan matrix by conjugation.

First, we make it upper-triangular by conjugation. We know that this is
possible by the Schur triangularization theorem, but we are a bit pickier now.
To wit, we want the triangular matrix $T$ to have the property that equal
eigenvalues form contiguous blocks on the main diagonal. For instance, we
don't want%
\[
T=\left(
\begin{array}
[c]{cccc}%
1 & \ast & \ast & \ast\\
& 2 & \ast & \ast\\
&  & 1 & \ast\\
&  &  & 2
\end{array}
\right)  .
\]
Instead, we want%
\[
T=\left(
\begin{array}
[c]{cccc}%
1 & \ast & \ast & \ast\\
& 1 & \ast & \ast\\
&  & 2 & \ast\\
&  &  & 2
\end{array}
\right)  .
\]
To this purpose, we need a stronger version of Schur triangularization:

\begin{theorem}
[Schur triangularization with perscribed diagonal]Let $A\in\mathbb{C}^{n\times
n}$ be an $n\times n$-matrix. Let $\lambda_{1},\lambda_{2},\ldots,\lambda_{n}$
be its eigenvalues (listed with their algebraic multiplicities). Then, there
exists an upper-triangular matrix $T$ such that $A\overset{\operatorname*{us}%
}{\sim}T$ (this means \textquotedblleft$A$ is unitarily similar to
$T$\textquotedblright) and such that the diagonal entries of $T$ are
$\lambda_{1},\lambda_{2},\ldots,\lambda_{n}$ in this order.
\end{theorem}

\begin{proof}
We proceed as in the proof of the original Schur triangularization theorem,
but we pay some attention to the eigenvectors that we pick. That proof
constructed $T$ recursively, starting by picking an eigenvalue $\lambda$ of
$A$ and a corresponding $\lambda$-eigenvector $x\neq0$, and then finding a
unitary matrix $U$ such that%
\[
U^{\ast}AU=\left(
\begin{array}
[c]{cc}%
\lambda & p\\
0 & B
\end{array}
\right)  \ \ \ \ \ \ \ \ \ \ \text{for some }p\in\mathbb{C}^{1\times\left(
n-1\right)  }\text{ and }B\in\mathbb{C}^{\left(  n-1\right)  \times\left(
n-1\right)  }.
\]
Then, the algorithm was applied recursively to $B$.

Now, we perform this algorithm, but we make sure to pick $\lambda=\lambda_{1}%
$. Thus,%
\[
U^{\ast}AU=\left(
\begin{array}
[c]{cc}%
\lambda_{1} & p\\
0 & B
\end{array}
\right)  ,
\]
which is a good start. Now we want to apply the IH (= induction hypothesis) to
$B$. To that purpose, we need to know that the eigenvalues of $B$ are
$\lambda_{2},\lambda_{3},\ldots,\lambda_{n}$ (with algebraic multiplicities).
In other words, we need to know that%
\[
p_{B}\left(  t\right)  =\left(  t-\lambda_{2}\right)  \left(  t-\lambda
_{3}\right)  \cdots\left(  t-\lambda_{n}\right)  .
\]


Now, I need a lemma about determinants:

\begin{lemma}
Let $\mathbb{F}$ be a field. Let $X\in\mathbb{F}^{n\times n}$, $Y\in
\mathbb{F}^{n\times m}$ and $Z\in\mathbb{F}^{m\times m}$ be three matrices.
Then,%
\[
\det\left(
\begin{array}
[c]{cc}%
X & Y\\
0 & Z
\end{array}
\right)  =\det X\cdot\det Z.
\]

\end{lemma}

For example,%
\[
\det\left(
\begin{array}
[c]{cccc}%
a & b & c & d\\
a^{\prime} & b^{\prime} & c^{\prime} & d^{\prime}\\
0 & 0 & c^{\prime\prime} & d^{\prime\prime}\\
0 & 0 & c^{\prime\prime\prime} & d^{\prime\prime\prime}%
\end{array}
\right)  =\det\left(
\begin{array}
[c]{cc}%
a & b\\
a^{\prime} & b^{\prime}%
\end{array}
\right)  \cdot\det\left(
\begin{array}
[c]{cc}%
c^{\prime\prime} & d^{\prime\prime}\\
c^{\prime\prime\prime} & d^{\prime\prime\prime}%
\end{array}
\right)  .
\]


Now, since $U$ is unitary, we have $A\overset{\operatorname*{us}}{\sim}%
U^{\ast}AU$ and thus $A\sim U^{\ast}AU$. Thus,%
\begin{align*}
p_{A}  & =p_{U^{\ast}AU}=\det\left(  tI_{n}-U^{\ast}AU\right)  =\det\left(
tI_{n}-\left(
\begin{array}
[c]{cc}%
\lambda_{1} & p\\
0 & B
\end{array}
\right)  \right)  \\
& \ \ \ \ \ \ \ \ \ \ \left(  \text{since }U^{\ast}AU=\left(
\begin{array}
[c]{cc}%
\lambda_{1} & p\\
0 & B
\end{array}
\right)  \right)  \\
& =\det\left(
\begin{array}
[c]{cc}%
t-\lambda_{1} & -p\\
0 & tI_{n-1}-B
\end{array}
\right)  \\
& =\underbrace{\det\left(
\begin{array}
[c]{c}%
t-\lambda_{1}%
\end{array}
\right)  }_{t-\lambda_{1}}\cdot\underbrace{\det\left(  tI_{n-1}-B\right)
}_{=p_{B}}\ \ \ \ \ \ \ \ \ \ \left(
\begin{array}
[c]{c}%
\text{by the lemma, or by Laplace}\\
\text{expansion along the }1\text{st column}%
\end{array}
\right)  \\
& =\left(  t-\lambda_{1}\right)  \cdot p_{B},
\end{align*}
so that%
\[
\left(  t-\lambda_{1}\right)  \cdot p_{B}=p_{A}=\left(  t-\lambda_{1}\right)
\left(  t-\lambda_{2}\right)  \cdots\left(  t-\lambda_{n}\right)  .
\]
Now, we can cancel $t-\lambda_{1}$ (since this is a nonzero polynomial), and
this becomes%
\[
p_{B}=\left(  t-\lambda_{2}\right)  \left(  t-\lambda_{3}\right)
\cdots\left(  t-\lambda_{n}\right)  ,
\]
exactly as we wanted to show.
\end{proof}

\subsection{Step 2: Separating distinct eigenvalues}

Recall the corollary from last time:

\begin{corollary}
Let $A\in\mathbb{C}^{n\times n}$, $B\in\mathbb{C}^{m\times m}$ and
$C\in\mathbb{C}^{n\times m}$ be three matrices such that $\sigma\left(
A\right)  \cap\sigma\left(  B\right)  =\varnothing$. Then, the two $\left(
n+m\right)  \times\left(  n+m\right)  $-matrices%
\[
\left(
\begin{array}
[c]{cc}%
A & C\\
0 & B
\end{array}
\right)  \ \ \ \ \ \ \ \ \ \ \text{and}\ \ \ \ \ \ \ \ \ \ \left(
\begin{array}
[c]{cc}%
A & 0\\
0 & B
\end{array}
\right)
\]
(written in block-matrix notation) are similar.
\end{corollary}

Let us use this corollary to show that for any numbers $a,b,c,\ldots
,p\in\mathbb{C}$, we have%
\[
\left(
\begin{array}
[c]{cccccc}%
1 & a & b & c & d & e\\
& 1 & f & g & h & i\\
&  & 2 & j & k & \ell\\
&  &  & 2 & m & n\\
&  &  &  & 2 & p\\
&  &  &  &  & 3
\end{array}
\right)  \sim\left(
\begin{array}
[c]{cccccc}%
1 & a &  &  &  & \\
& 1 &  &  &  & \\
&  & 2 & j & k & \\
&  &  & 2 & m & \\
&  &  &  & 2 & \\
&  &  &  &  & 3
\end{array}
\right)
\]
(where invisible cells contain $0$s). Indeed, the triangular matrices $\left(
%
\begin{array}
[c]{cc}%
1 & a\\
& 1
\end{array}
\right)  $ and $\left(
\begin{array}
[c]{cccc}%
2 & j & k & \ell\\
& 2 & m & n\\
&  & 2 & p\\
&  &  & 3
\end{array}
\right)  $ have disjoint spectra (i.e., they have no eigenvalues in common),
because their diagonals have no entries in common. So, by the corollary,
\[
\left(
\begin{array}
[c]{cccccc}%
1 & a & b & c & d & e\\
& 1 & f & g & h & i\\
&  & 2 & j & k & \ell\\
&  &  & 2 & m & n\\
&  &  &  & 2 & p\\
&  &  &  &  & 3
\end{array}
\right)  \sim\left(
\begin{array}
[c]{cccccc}%
1 & a &  &  &  & \\
& 1 &  &  &  & \\
&  & 2 & j & k & \ell\\
&  &  & 2 & m & n\\
&  &  &  & 2 & p\\
&  &  &  &  & 3
\end{array}
\right)  .
\]
Now, the triangular matrices $\left(
\begin{array}
[c]{ccccc}%
1 & a &  &  & \\
& 1 &  &  & \\
&  & 2 & j & k\\
&  &  & 2 & m\\
&  &  &  & 2
\end{array}
\right)  $ and $\left(
\begin{array}
[c]{c}%
3
\end{array}
\right)  $ have disjoint spectra, so the corollary yields%
\[
\left(
\begin{array}
[c]{cccccc}%
1 & a &  &  &  & \\
& 1 &  &  &  & \\
&  & 2 & j & k & \ell\\
&  &  & 2 & m & n\\
&  &  &  & 2 & p\\
&  &  &  &  & 3
\end{array}
\right)  \sim\left(
\begin{array}
[c]{cccccc}%
1 & a &  &  &  & \\
& 1 &  &  &  & \\
&  & 2 & j & k & \\
&  &  & 2 & m & \\
&  &  &  & 2 & \\
&  &  &  &  & 3
\end{array}
\right)  .
\]
Since $\sim$ is an equivalence relation, we can combine the two similarities,
and my claim follows.

So when we have a triangular matrix where the diagonal has no interleaving
values (i.e., there is never a $\mu$ between two $\lambda$'s on the diagonal
when $\mu\neq\lambda$), we can \textquotedblleft clean out\textquotedblright%
\ all the above-diagonal entries that correspond to different diagonal entries
(i.e., that lie above a different diagonal entry than they stand to the right
of) by conjugating with an appropriate matrix.

Now, combining this with the Schur triangularization theorem (in its
above-stated stronger version), we obtain the following:

\begin{proposition}
Let $A\in\mathbb{C}^{n\times n}$ be an $n\times n$-matrix. Then, $A$ is
similar to a block-diagonal matrix of the form%
\[
\left(
\begin{array}
[c]{cccc}%
B_{1} &  &  & \\
& B_{2} &  & \\
&  & \ddots & \\
&  &  & B_{k}%
\end{array}
\right)  ,
\]
where each $B_{i}$ is an upper-triangular matrix with all entries on its
diagonal being equal.
\end{proposition}

This is not yet a Jordan canonical form, but it is already somewhat close. At
least, we have separated out all the distinct eigenvalues of $A$. We now only
need to care about the matrices $B_{1},B_{2},\ldots,B_{k}$, each of which has
only one distinct eigenvalue. Our next goal is to break up these matrices
$B_{1},B_{2},\ldots,B_{k}$ into Jordan cells (using conjugation).
\end{document}