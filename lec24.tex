\documentclass[numbers=enddot,12pt,final,onecolumn,notitlepage]{scrartcl}%
\usepackage[headsepline,footsepline,manualmark]{scrlayer-scrpage}
\usepackage[all,cmtip]{xy}
\usepackage{amssymb}
\usepackage{amsmath}
\usepackage{amsthm}
\usepackage{framed}
\usepackage{comment}
\usepackage{color}
\usepackage{hyperref}
\usepackage[sc]{mathpazo}
\usepackage[T1]{fontenc}
\usepackage{tikz}
\usepackage{needspace}
\usepackage{tabls}
\usepackage{wasysym}
\usepackage{easytable}
\usepackage{pythonhighlight}
%TCIDATA{OutputFilter=latex2.dll}
%TCIDATA{Version=5.50.0.2960}
%TCIDATA{LastRevised=Wednesday, December 01, 2021 11:49:12}
%TCIDATA{SuppressPackageManagement}
%TCIDATA{<META NAME="GraphicsSave" CONTENT="32">}
%TCIDATA{<META NAME="SaveForMode" CONTENT="1">}
%TCIDATA{BibliographyScheme=Manual}
%TCIDATA{Language=American English}
%BeginMSIPreambleData
\providecommand{\U}[1]{\protect\rule{.1in}{.1in}}
%EndMSIPreambleData
\usetikzlibrary{arrows.meta}
\usetikzlibrary{chains}
\newcounter{exer}
\newcounter{exera}
\numberwithin{exer}{subsection}
\theoremstyle{definition}
\newtheorem{theo}{Theorem}[subsection]
\newenvironment{theorem}[1][]
{\begin{theo}[#1]\begin{leftbar}}
{\end{leftbar}\end{theo}}
\newtheorem{lem}[theo]{Lemma}
\newenvironment{lemma}[1][]
{\begin{lem}[#1]\begin{leftbar}}
{\end{leftbar}\end{lem}}
\newtheorem{prop}[theo]{Proposition}
\newenvironment{proposition}[1][]
{\begin{prop}[#1]\begin{leftbar}}
{\end{leftbar}\end{prop}}
\newtheorem{defi}[theo]{Definition}
\newenvironment{definition}[1][]
{\begin{defi}[#1]\begin{leftbar}}
{\end{leftbar}\end{defi}}
\newtheorem{remk}[theo]{Remark}
\newenvironment{remark}[1][]
{\begin{remk}[#1]\begin{leftbar}}
{\end{leftbar}\end{remk}}
\newtheorem{coro}[theo]{Corollary}
\newenvironment{corollary}[1][]
{\begin{coro}[#1]\begin{leftbar}}
{\end{leftbar}\end{coro}}
\newtheorem{conv}[theo]{Convention}
\newenvironment{convention}[1][]
{\begin{conv}[#1]\begin{leftbar}}
{\end{leftbar}\end{conv}}
\newtheorem{quest}[theo]{Question}
\newenvironment{question}[1][]
{\begin{quest}[#1]\begin{leftbar}}
{\end{leftbar}\end{quest}}
\newtheorem{warn}[theo]{Warning}
\newenvironment{warning}[1][]
{\begin{warn}[#1]\begin{leftbar}}
{\end{leftbar}\end{warn}}
\newtheorem{conj}[theo]{Conjecture}
\newenvironment{conjecture}[1][]
{\begin{conj}[#1]\begin{leftbar}}
{\end{leftbar}\end{conj}}
\newtheorem{exam}[theo]{Example}
\newenvironment{example}[1][]
{\begin{exam}[#1]\begin{leftbar}}
{\end{leftbar}\end{exam}}
\newtheorem{exmp}[exer]{Exercise}
\newenvironment{exercise}[1][]
{\begin{exmp}[#1]\begin{leftbar}}
{\end{leftbar}\end{exmp}}
\newenvironment{statement}{\begin{quote}}{\end{quote}}
\newenvironment{fineprint}{\medskip \begin{small}}{\end{small} \medskip}
\iffalse
\newenvironment{proof}[1][Proof]{\noindent\textbf{#1.} }{\ \rule{0.5em}{0.5em}}
\newenvironment{question}[1][Question]{\noindent\textbf{#1.} }{\ \rule{0.5em}{0.5em}}
\newenvironment{warning}[1][Warning]{\noindent\textbf{#1.} }{\ \rule{0.5em}{0.5em}}
\newenvironment{teachingnote}[1][Teaching note]{\noindent\textbf{#1.} }{\ \rule{0.5em}{0.5em}}
\fi
\let\sumnonlimits\sum
\let\prodnonlimits\prod
\let\cupnonlimits\bigcup
\let\capnonlimits\bigcap
\renewcommand{\sum}{\sumnonlimits\limits}
\renewcommand{\prod}{\prodnonlimits\limits}
\renewcommand{\bigcup}{\cupnonlimits\limits}
\renewcommand{\bigcap}{\capnonlimits\limits}
\setlength\tablinesep{3pt}
\setlength\arraylinesep{3pt}
\setlength\extrarulesep{3pt}
\voffset=0cm
\hoffset=-0.7cm
\setlength\textheight{22.5cm}
\setlength\textwidth{15.5cm}
\newcommand\arxiv[1]{\href{http://www.arxiv.org/abs/#1}{\texttt{arXiv:#1}}}
\newenvironment{verlong}{}{}
\newenvironment{vershort}{}{}
\newenvironment{noncompile}{}{}
\newenvironment{teachingnote}{}{}
\excludecomment{verlong}
\includecomment{vershort}
\excludecomment{noncompile}
\excludecomment{teachingnote}
\newcommand{\CC}{\mathbb{C}}
\newcommand{\RR}{\mathbb{R}}
\newcommand{\QQ}{\mathbb{Q}}
\newcommand{\NN}{\mathbb{N}}
\newcommand{\ZZ}{\mathbb{Z}}
\newcommand{\KK}{\mathbb{K}}
\newcommand{\id}{\operatorname{id}}
\newcommand{\lcm}{\operatorname{lcm}}
\newcommand{\rev}{\operatorname{rev}}
\newcommand{\powset}[2][]{\ifthenelse{\equal{#2}{}}{\mathcal{P}\left(#1\right)}{\mathcal{P}_{#1}\left(#2\right)}}
\newcommand{\set}[1]{\left\{ #1 \right\}}
\newcommand{\abs}[1]{\left| #1 \right|}
\newcommand{\tup}[1]{\left( #1 \right)}
\newcommand{\ive}[1]{\left[ #1 \right]}
\newcommand{\floor}[1]{\left\lfloor #1 \right\rfloor}
\newcommand{\lf}[2]{#1^{\underline{#2}}}
\newcommand{\underbrack}[2]{\underbrace{#1}_{\substack{#2}}}
\newcommand{\horrule}[1]{\rule{\linewidth}{#1}}
\newcommand{\are}{\ar@{-}}
\newcommand{\nnn}{\nonumber\\}
\newcommand{\sslash}{\mathbin{/\mkern-6mu/}}
\newcommand{\numboxed}[2]{\underbrace{\boxed{#1}}_{\text{box } #2}}
\newcommand{\ig}[2]{\includegraphics[scale=#1]{#2.png}}
\newcommand{\UNFINISHED}{\begin{center} \Huge{\textbf{Unfinished material begins here.}} \end{center} }
\iffalse
\NOEXPAND{\today}{\today}
\NOEXPAND{\sslash}{\sslash}
\NOEXPAND{\numboxed}[2]{\numboxed}
\NOEXPAND{\UNFINISHED}{\UNFINISHED}
\fi
\ihead{Math 504 notes}
\ohead{page \thepage}
\cfoot{\today}
\begin{document}

\title{Math 504: Advanced Linear Algebra}
\author{Hugo Woerdeman, with edits by Darij Grinberg\thanks{Drexel University, Korman
Center, 15 S 33rd Street, Philadelphia PA, 19104, USA}}
\date{\today\ (unfinished!)}
\maketitle
\tableofcontents

\section*{Math 504 Lecture 24}

\section{Positive and nonnegative matrices ([HorJoh, Chapter 8]) (cont'd)}

\subsection{The spectral radius (cont'd)}

Recall from last lecture:

The \textbf{spectral radius} of a square matrix $A$ is%
\[
\rho\left(  A\right)  :=\max\left\{  \left\vert \lambda\right\vert
\ \mid\ \lambda\in\sigma\left(  A\right)  \right\}  .
\]


\begin{corollary}
Let $A\in\mathbb{R}^{n\times n}$ and $B\in\mathbb{R}^{n\times n}$ satisfy
$B\geq A\geq0$, then $\rho\left(  A\right)  \leq\rho\left(  B\right)  $.
\end{corollary}

(Inequalities between matrices are entrywise. Nonnegative or positive matrices
have real entries by definition.)

\bigskip

Let us now prove some bounds for $\rho\left(  A\right)  $ when $A\geq0$.

\begin{definition}
Let $\mathbb{F}$ be a field. Let $A\in\mathbb{F}^{n\times m}$.

\textbf{(a)} The \textbf{column sums} of $A$ are the $m$ sums
\[
\sum_{i=1}^{n}A_{i,j}=\left(  \text{the sum of all entries of the }j\text{-th
column of }A\right)  \ \ \ \ \ \ \ \ \ \ \text{for }j\in\left[  m\right]  .
\]


\textbf{(b)} The \textbf{row sums} of $A$ are%
\[
\sum_{j=1}^{n}A_{i,j}=\left(  \text{the sum of all entries of the }i\text{-th
row of }A\right)  \ \ \ \ \ \ \ \ \ \ \text{for }i\in\left[  n\right]  .
\]


\textbf{(c)} Now, assume that $\mathbb{F}=\mathbb{C}$ and $n>0$ and $m>0$.
Then, we set%
\[
\left\vert \left\vert A\right\vert \right\vert _{\infty}:=\left(
\text{largest row sum of }\left\vert A\right\vert \right)  =\max
\limits_{i\in\left[  n\right]  }\sum_{j=1}^{m}\left\vert A_{i,j}\right\vert
\]
and%
\[
\left\vert \left\vert A\right\vert \right\vert _{1}:=\left(  \text{largest
column sum of }\left\vert A\right\vert \right)  =\max\limits_{j\in\left[
m\right]  }\sum_{i=1}^{n}\left\vert A_{i,j}\right\vert .
\]
These two numbers are called the $\infty$\textbf{-norm} and the $1$%
\textbf{-norm} of $A$ (for reasons I will explain on zoom).
\end{definition}

\begin{example}
The column sums of $\left(
\begin{array}
[c]{cc}%
a & b\\
c & d
\end{array}
\right)  $ are $a+c$ and $b+d$.
\end{example}

\begin{warning}
The column sums of a matrix are \textbf{not} the entries of the sum of its
columns. Rather, the latter entries are the row sums, whereas the column sums
are the entries of the sum of the rows.
\end{warning}

\begin{remark}
Let $A\in\mathbb{F}^{n\times m}$.

\textbf{(a)} The row sums of $A$ are the column sums of $A^{T}$.

\textbf{(b)} If $\mathbb{F}=\mathbb{C}$, then $\left\vert \left\vert
A\right\vert \right\vert _{\infty}=\left\vert \left\vert A^{T}\right\vert
\right\vert _{1}$.
\end{remark}

\begin{lemma}
Let $A\in\mathbb{C}^{n\times n}$. Then:

\textbf{(a)} We have $\rho\left(  A\right)  \leq\left\vert \left\vert
A\right\vert \right\vert _{\infty}$.

\textbf{(b)} If $A\geq0$ and if all row sums of $A$ are equal, then
$\rho\left(  A\right)  =\left\vert \left\vert A\right\vert \right\vert
_{\infty}$.

\textbf{(c)} We have $\rho\left(  A\right)  \leq\left\vert \left\vert
A\right\vert \right\vert _{1}$.

\textbf{(d)} If $A\geq0$ and if all column sums of $A$ are equal, then
$\rho\left(  A\right)  =\left\vert \left\vert A\right\vert \right\vert _{1}$.
\end{lemma}

\begin{proof}
\textbf{(a)} We have $\rho\left(  A\right)  =\left\vert \lambda\right\vert $
for some eigenvalue $\lambda$ of $A$. Consider this $\lambda$, and let
$v=\left(  v_{1},v_{2},\ldots,v_{n}\right)  ^{T}\in\mathbb{C}^{n}$ be a
nonzero $\lambda$-eigenvector.

Choose an $i\in\left[  n\right]  $ such that $\left\vert v_{i}\right\vert
=\max\left\{  \left\vert v_{1}\right\vert ,\left\vert v_{2}\right\vert
,\ldots,\left\vert v_{n}\right\vert \right\}  $. Then, $\left\vert
v_{i}\right\vert >0$ (since $v$ is nonzero).

Now, the $i$-th entry of the column vector $Av$ is $\sum\limits_{j=1}%
^{n}A_{i,j}v_{j}$; however, the same entry is $\lambda v_{i}$ (since
$Av=\lambda v$). Comparing these facts, we obtain%
\[
\lambda v_{i}=\sum\limits_{j=1}^{n}A_{i,j}v_{j}.
\]
Taking absolute values, we obtain%
\begin{align*}
\left\vert \lambda v_{i}\right\vert  & =\left\vert \sum\limits_{j=1}%
^{n}A_{i,j}v_{j}\right\vert \leq\sum\limits_{j=1}^{n}\underbrace{\left\vert
A_{i,j}v_{j}\right\vert }_{\substack{=\left\vert A_{i,j}\right\vert
\cdot\left\vert v_{j}\right\vert \leq\left\vert A_{i,j}\right\vert
\cdot\left\vert v_{i}\right\vert \\\text{(since the choice of }i\\\text{yields
}\left\vert v_{j}\right\vert \leq\left\vert v_{i}\right\vert \text{)}%
}}\ \ \ \ \ \ \ \ \ \ \left(  \text{by the triangle inequality}\right)  \\
& \leq\sum\limits_{j=1}^{n}\left\vert A_{i,j}\right\vert \cdot\left\vert
v_{i}\right\vert .
\end{align*}
Since $\left\vert v_{i}\right\vert >0$, we can cancel $\left\vert
v_{i}\right\vert $ from this inequality (since the left hand side is
$\left\vert \lambda v_{i}\right\vert =\left\vert \lambda\right\vert
\cdot\left\vert v_{i}\right\vert $), and thus we obtain%
\begin{align*}
\left\vert \lambda\right\vert  & \leq\sum\limits_{j=1}^{n}\left\vert
A_{i,j}\right\vert =\left(  \text{the }i\text{-th row sum of }\left\vert
A\right\vert \right)  \\
& \leq\left(  \text{the largest row sum of }\left\vert A\right\vert \right)
=\left\vert \left\vert A\right\vert \right\vert _{\infty}.
\end{align*}
Since $\rho=\left\vert \lambda\right\vert $, this rewrites as $\rho
\leq\left\vert \left\vert A\right\vert \right\vert _{\infty}$, qed.

\textbf{(b)} Assume that $A\geq0$ and that all row sums of $A$ are equal. Let
$e=\left(  1,1,\ldots,1\right)  ^{T}\in\mathbb{R}^{n}$, and let $\kappa$ be
the common value of the row sums of $A$. Then, $Ae=\kappa e$ (since $Ae$ is
the vector whose entries are the row sums of $A$, but all these row sums are
equal to $\kappa$). Hence, $\kappa$ is an eigenvalue of $A$ (since $e\neq0$),
so that $\rho\left(  A\right)  \geq\left\vert \kappa\right\vert =\kappa$
(since $A\geq0$ entails $\kappa\geq0$).

On the other hand, part \textbf{(a)} of the lemma yields
\[
\rho\left(  A\right)  \leq\left\vert \left\vert A\right\vert \right\vert
_{\infty}=\left(  \text{the largest row sum of }\left\vert A\right\vert
\right)  =\left(  \text{the largest row sum of }A\right)  =\kappa
\]
(since all row sums of $A$ are $\kappa$). Combining this with $\rho\left(
A\right)  \geq\kappa$, we obtain $\rho\left(  A\right)  =\kappa=\left\vert
\left\vert A\right\vert \right\vert _{\infty}$. This proves part \textbf{(b)}.

\textbf{(c)} Apply part \textbf{(a)} to $A^{T}$ instead of $A$, and recall
that $\left\vert \left\vert A^{T}\right\vert \right\vert _{\infty}=\left\vert
\left\vert A\right\vert \right\vert _{1}$ and $\rho\left(  A^{T}\right)
=\rho\left(  A\right)  $.

\textbf{(d)} Similar to part \textbf{(c)}.
\end{proof}

Now, we can bound $\rho\left(  A\right)  $ from both sides when $A\geq0$:

\begin{theorem}
Let $A\in\mathbb{R}^{n\times n}$ satisfy $A\geq0$. Then,%
\[
\left(  \text{the smallest row sum of }A\right)  \leq\rho\left(  A\right)
\leq\left(  \text{the largest row sum of }A\right)  .
\]

\end{theorem}

\begin{proof}
The inequality $\rho\left(  A\right)  \leq\left(  \text{the largest row sum of
}A\right)  $ follows from part \textbf{(a)} of the lemma, because
\[
\left\vert \left\vert A\right\vert \right\vert _{\infty}=\left(  \text{the
largest row sum of }\underbrace{\left\vert A\right\vert }_{=A}\right)
=\left(  \text{the largest row sum of }A\right)  .
\]


It remains to prove that $\left(  \text{the smallest row sum of }A\right)
\leq\rho\left(  A\right)  $.

Let $r_{1},r_{2},\ldots,r_{n}$ be the row sums of $A$. Let $r_{i}$ be the
smallest among them. We must then prove that $r_{i}\leq\rho\left(  A\right)
$. If $r_{i}=0$, this is obvious. So WLOG assume that $r_{i}>0$. Hence, all
$r_{1},r_{2},\ldots,r_{n}$ are positive.

Let $B$ the $n\times n$-matrix whose $\left(  u,v\right)  $-th entry is
$\dfrac{r_{i}}{r_{u}}A_{u,v}$. So $B$ is obtained from $A$ by scaling each row
such that the row sums all become $r_{i}$. Hence, the matrix $B$ is $\geq0$
(since $A\geq0$), and its row sums are all equal to $r_{i}$. Hence, part
\textbf{(b)} of the above lemma (applied to $B$ instead of $A$) yields
\begin{align*}
\rho\left(  B\right)    & =\left\vert \left\vert B\right\vert \right\vert
_{\infty}=\left(  \text{the largest row sum of }\left\vert B\right\vert
\right)  \\
& =\left(  \text{the largest row sum of }B\right)  =r_{i}%
\end{align*}
(since all row sums of $B$ are $r_{i}$). However, for each $u\in\left[
n\right]  $, we have $\dfrac{r_{i}}{r_{u}}A_{u,v}\leq A_{u,v}$ (since
$r_{i}\leq r_{u}$). In other words, $B\leq A$. Hence, the Corollary from last
time (applied to $B$ and $A$ instead of $A$ and $B$) yields $\rho\left(
B\right)  \leq\rho\left(  A\right)  $. Hence, $r_{i}=\rho\left(  B\right)
\leq\rho\left(  A\right)  $, qed.
\end{proof}

\begin{corollary}
Let $A\in\mathbb{R}^{n\times n}$ satisfy $A\geq0$. Let $x_{1},x_{2}%
,\ldots,x_{n}$ be any $n$ positive reals. Then,%
\[
\min\limits_{i\in\left[  n\right]  }\sum_{j=1}^{n}\dfrac{x_{i}}{x_{j}}%
a_{i,j}\leq\rho\left(  A\right)  \leq\max\limits_{i\in\left[  n\right]  }%
\sum_{j=1}^{n}\dfrac{x_{i}}{x_{j}}a_{i,j}.
\]

\end{corollary}

\begin{proof}
Let $D=\operatorname*{diag}\left(  x_{1},x_{2},\ldots,x_{n}\right)  $. Then,
we can apply the previous theorem to $DAD^{-1}$ instead of $A$, and notice
that the row sums of $DAD^{-1}$ are exactly the sums $\sum_{j=1}^{n}%
\dfrac{x_{i}}{x_{j}}a_{i,j}$ for $i\in\left[  n\right]  $. (And, of course,
$\rho\left(  DAD^{-1}\right)  =\rho\left(  A\right)  $).
\end{proof}

\begin{remark}
If the $x_{1},x_{2},\ldots,x_{n}$ in the above corollary are chosen
appropriately, both of the inequalities can become equalities. (This follows
from the Perron--Frobenius theorems further below.)
\end{remark}

\subsection{Perron--Frobenius theorems}

We now come to the most important results about nonnegative matrices: the
Perron--Frobenius theorems.

\subsubsection{Motivation}

Recall a standard situation in probability theorem. Consider a system (e.g., a
slot machine) that can be in one of $n$ possible \textbf{states} $s_{1}%
,s_{2},\ldots,s_{n}$. Every minute, the system randomly changes states
according to the following rule: If the system is in state $s_{i}$, then it
changes to state $s_{j}$ with probability $P_{i,j}$, where $P$ is a (fixed,
pre-determined) nonnegative $n\times n$-matrix whose row sums all equal $1$
(such a matrix is called \textbf{row-stochastic}). This is commonly known as a
\textbf{Markov chain}.

Given such a Markov chain, one often wonders about its \textquotedblleft
steady state\textquotedblright: If you wait long enough, how likely is the
system to be in a given state?

\begin{example}
Let $P=\left(
\begin{array}
[c]{cc}%
0.9 & 0.1\\
0.5 & 0.5
\end{array}
\right)  $. We encode the two states $s_{1}$ and $s_{2}$ as the basis vectors
$e_{1}=\left(  1,0\right)  $ and $e_{2}=\left(  0,1\right)  $ of the vector
space $\mathbb{R}^{1\times2}$ (we work with row vectors here for convenience).
Thus, a probability distribution on the set of states (i.e., a distribution of
the form \textquotedblleft state $s_{1}$ with probability $a_{1}$ and state
$s_{2}$ with probability $a_{2}$\textquotedblright) corresponds to a row
vector $\left(  a_{1},a_{2}\right)  \in\mathbb{R}^{1\times2}$ satisfying
$a_{1}\geq0$ and $a_{2}\geq0$ and $a_{1}+a_{2}=1$.

If we start at state $s_{1}$ and let $k$ minutes pass, then the probability
distribution for the resulting state is $s_{1}P^{k}$. More generally, if we
start with a probability distribution $d\in\mathbb{R}^{1\times2}$ and let $k$
minutes pass, then the resulting state will be distributed according to
$dP^{k}$. So we wonder: What is $\lim\limits_{k\rightarrow\infty}dP^{k}$ as
$k\rightarrow\infty$ ? Does this limit even exist?

We can notice one thing right away: If $\lim\limits_{k\rightarrow\infty}%
dP^{k}$ exists, then this limit is a left $1$-eigenvector of $P$, in the sense
that it is a row vector $y$ such that $yP=y$ (since $y=\lim
\limits_{k\rightarrow\infty}dP^{k}=\lim\limits_{k\rightarrow\infty}%
dP^{k+1}=\left(  \lim\limits_{k\rightarrow\infty}dP^{k}\right)  P=yP$). Since
it is furthermore a vector whose coordinates add up to $1$ (because it is a
limit of such vectors), this often allows us to explicitly compute it. In the
above case, for example, we get
\[
\lim\limits_{k\rightarrow\infty}dP^{k}=\left(  \dfrac{5}{6},\dfrac{1}%
{6}\right)  .
\]


But does this limit actually exist? Yes, in this specific example, but this
isn't quite that obvious. Note that this limit (known as the \textbf{steady
state} of the Markov chain) actually does not depend on the starting
distribution $d$.
\end{example}

Does this generalize? Not always. Here are two bad examples:

\begin{itemize}
\item If $P=I_{2}=\left(
\begin{array}
[c]{cc}%
1 & 0\\
0 & 1
\end{array}
\right)  $, then $\lim\limits_{k\rightarrow\infty}dP^{k}=d$ for each $d$, so
the limits do depend on $d$.

\item If $P=\left(
\begin{array}
[c]{cc}%
0 & 1\\
1 & 0
\end{array}
\right)  $, then $\lim\limits_{k\rightarrow\infty}dP^{k}$ does not exist
unless $d=\left(  0.5,\ 0.5\right)  $, since in all other cases the sequence
$\left(  dP^{k}\right)  _{k\geq0}$ oscillates between $\left(  a_{1}%
,a_{2}\right)  $ and $\left(  a_{2},a_{1}\right)  $.
\end{itemize}

Perhaps surprisingly, such cases are an exception. For \textbf{most}
row-stochastic matrices $P$ (that is, nonnegative matrices whose row sums all
equal $1$), there is a \textbf{unique} steady state (i.e., left $1$%
-eigenvector), and it can be obtained as $\lim\limits_{k\rightarrow\infty
}dP^{k}$ for any starting distribution $d$. To be more precise, this holds
whenever $P$ is positive (i.e., all $P_{i,j}>0$). Some weaker assumptions also suffice.

More general versions of these facts hold even if we don't assume $P$ to be
row-stochastic, but merely require $P>0$ (or $P\geq0$ with some extra
conditions). These will be the Perron and Perron--Frobenius theorems.

\subsubsection{The theorems}

\begin{theorem}
[Perron theorem]Let $A\in\mathbb{R}^{n\times n}$ satisfy $A>0$. Then:

\textbf{(a)} We have $\rho\left(  A\right)  >0$.

\textbf{(b)} The number $\rho\left(  A\right)  $ is an eigenvalue of $A$ and
has algebraic multiplicity $1$ (and therefore geometric multiplicity $1$ as well).

\textbf{(c)} There is a unique $\rho\left(  A\right)  $-eigenvector $x=\left(
x_{1},x_{2},\ldots,x_{n}\right)  ^{T}\in\mathbb{C}^{n}$ of $A$ with
$x_{1}+x_{2}+\cdots+x_{n}=1$. This eigenvector $x$ is furthermore positive.
(It is called the \textbf{Perron vector} of $A$.)

\textbf{(d)} There is a unique vector $y=\left(  y_{1},y_{2},\ldots
,y_{n}\right)  ^{T}\in\mathbb{C}^{n}$ such that $y^{T}A=\rho\left(  A\right)
y^{T}$ and $x_{1}y_{1}+x_{2}y_{2}+\cdots+x_{n}y_{n}=1$. This vector $y$ is
also positive.

\textbf{(e)} We have%
\[
\left(  \dfrac{1}{\rho\left(  A\right)  }A\right)  ^{m}\rightarrow
xy^{T}\ \ \ \ \ \ \ \ \ \ \text{as }m\rightarrow\infty.
\]


\textbf{(f)} The only eigenvalue of $A$ that has absolute value $\rho\left(
A\right)  $ is $\rho\left(  A\right)  $ itself.
\end{theorem}

We will prove this next time.

\begin{theorem}
[Perron--Frobenius 1]Let $A\in\mathbb{R}^{n\times n}$ satisfy $A\geq0$. Then:

\textbf{(a)} The number $\rho\left(  A\right)  $ is an eigenvalue of $A$.

\textbf{(b)} The matrix $A$ has a nonzero nonnegative $\rho\left(  A\right)  $-eigenvector.
\end{theorem}

To get stronger statements without requiring $A>0$, we need two further
properties of $A$.

\begin{definition}
Let $A\in\mathbb{R}^{n\times n}$ be an $n\times n$-matrix.

\textbf{(a)} We say that $A$ is \textbf{reducible} if there exist two disjoint
nonempty subsets $I$ and $J$ of $\left[  n\right]  $ such that $I\cup
J=\left[  n\right]  $ and such that%
\[
A_{i,j}=0\ \ \ \ \ \ \ \ \ \ \text{for all }i\in I\text{ and }j\in J.
\]
Equivalently, $A$ is reducible if and only if there exists a permutation
matrix $P$ such that%
\[
P^{-1}AP=\left(
\begin{array}
[c]{cc}%
B & C\\
0_{\left(  n-r\right)  \times r} & D
\end{array}
\right)  \ \ \ \ \ \ \ \ \ \ \text{for some }0<r<n\text{ and some }B,C,D.
\]


\textbf{(b)} We say that $A$ is \textbf{irreducible} if $A$ is not reducible.

\textbf{(c)} We say that $A$ is \textbf{primitive} if there exists some $m>0$
such that $A^{m}>0$.
\end{definition}

\begin{theorem}
[Perron--Frobenius 2]Let $A\in\mathbb{R}^{n\times n}$ be nonnegative and
irreducible. Then:

\textbf{(a)} We have $\rho\left(  A\right)  >0$.

\textbf{(b)} The number $\rho\left(  A\right)  $ is an eigenvalue of $A$ and
has algebraic multiplicity $1$ (and therefore geometric multiplicity $1$ as well).

\textbf{(c)} There is a unique $\rho\left(  A\right)  $-eigenvector $x=\left(
x_{1},x_{2},\ldots,x_{n}\right)  ^{T}\in\mathbb{C}^{n}$ of $A$ with
$x_{1}+x_{2}+\cdots+x_{n}=1$. This eigenvector $x$ is furthermore positive.
(It is called the \textbf{Perron vector} of $A$.)

\textbf{(d)} There is a unique vector $y=\left(  y_{1},y_{2},\ldots
,y_{n}\right)  ^{T}\in\mathbb{C}^{n}$ such that $y^{T}A=\rho\left(  A\right)
y^{T}$ and $x_{1}y_{1}+x_{2}y_{2}+\cdots+x_{n}y_{n}=1$. This vector $y$ is
also positive.

\textbf{(e)} Assume furthermore that $A$ is primitive. We have%
\[
\left(  \dfrac{1}{\rho\left(  A\right)  }A\right)  ^{m}\rightarrow
xy^{T}\ \ \ \ \ \ \ \ \ \ \text{as }m\rightarrow\infty.
\]


\textbf{(f)} Assume again that $A$ is primitive. The only eigenvalue of $A$
that has absolute value $\rho\left(  A\right)  $ is $\rho\left(  A\right)  $ itself.
\end{theorem}

\begin{remark}
If $A$ is the row-stochastic matrix $P$ corresponding to a Markov chain, then:

\begin{itemize}
\item $A$ is irreducible if and only if there is no set of states from which
you cannot escape (except for the empty set and for the set of all states);

\item $A$ is primitive if and only if there is no \textquotedblleft
oscillation\textquotedblright.
\end{itemize}
\end{remark}


\end{document}