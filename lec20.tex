\documentclass[numbers=enddot,12pt,final,onecolumn,notitlepage]{scrartcl}%
\usepackage[headsepline,footsepline,manualmark]{scrlayer-scrpage}
\usepackage[all,cmtip]{xy}
\usepackage{amssymb}
\usepackage{amsmath}
\usepackage{amsthm}
\usepackage{framed}
\usepackage{comment}
\usepackage{color}
\usepackage{hyperref}
\usepackage[sc]{mathpazo}
\usepackage[T1]{fontenc}
\usepackage{tikz}
\usepackage{needspace}
\usepackage{tabls}
\usepackage{wasysym}
\usepackage{easytable}
\usepackage{pythonhighlight}
%TCIDATA{OutputFilter=latex2.dll}
%TCIDATA{Version=5.50.0.2960}
%TCIDATA{LastRevised=Wednesday, November 17, 2021 11:50:58}
%TCIDATA{SuppressPackageManagement}
%TCIDATA{<META NAME="GraphicsSave" CONTENT="32">}
%TCIDATA{<META NAME="SaveForMode" CONTENT="1">}
%TCIDATA{BibliographyScheme=Manual}
%TCIDATA{Language=American English}
%BeginMSIPreambleData
\providecommand{\U}[1]{\protect\rule{.1in}{.1in}}
%EndMSIPreambleData
\usetikzlibrary{arrows.meta}
\usetikzlibrary{chains}
\newcounter{exer}
\newcounter{exera}
\numberwithin{exer}{subsection}
\theoremstyle{definition}
\newtheorem{theo}{Theorem}[subsection]
\newenvironment{theorem}[1][]
{\begin{theo}[#1]\begin{leftbar}}
{\end{leftbar}\end{theo}}
\newtheorem{lem}[theo]{Lemma}
\newenvironment{lemma}[1][]
{\begin{lem}[#1]\begin{leftbar}}
{\end{leftbar}\end{lem}}
\newtheorem{prop}[theo]{Proposition}
\newenvironment{proposition}[1][]
{\begin{prop}[#1]\begin{leftbar}}
{\end{leftbar}\end{prop}}
\newtheorem{defi}[theo]{Definition}
\newenvironment{definition}[1][]
{\begin{defi}[#1]\begin{leftbar}}
{\end{leftbar}\end{defi}}
\newtheorem{remk}[theo]{Remark}
\newenvironment{remark}[1][]
{\begin{remk}[#1]\begin{leftbar}}
{\end{leftbar}\end{remk}}
\newtheorem{coro}[theo]{Corollary}
\newenvironment{corollary}[1][]
{\begin{coro}[#1]\begin{leftbar}}
{\end{leftbar}\end{coro}}
\newtheorem{conv}[theo]{Convention}
\newenvironment{convention}[1][]
{\begin{conv}[#1]\begin{leftbar}}
{\end{leftbar}\end{conv}}
\newtheorem{quest}[theo]{Question}
\newenvironment{question}[1][]
{\begin{quest}[#1]\begin{leftbar}}
{\end{leftbar}\end{quest}}
\newtheorem{warn}[theo]{Warning}
\newenvironment{warning}[1][]
{\begin{warn}[#1]\begin{leftbar}}
{\end{leftbar}\end{warn}}
\newtheorem{conj}[theo]{Conjecture}
\newenvironment{conjecture}[1][]
{\begin{conj}[#1]\begin{leftbar}}
{\end{leftbar}\end{conj}}
\newtheorem{exam}[theo]{Example}
\newenvironment{example}[1][]
{\begin{exam}[#1]\begin{leftbar}}
{\end{leftbar}\end{exam}}
\newtheorem{exmp}[exer]{Exercise}
\newenvironment{exercise}[1][]
{\begin{exmp}[#1]\begin{leftbar}}
{\end{leftbar}\end{exmp}}
\newenvironment{statement}{\begin{quote}}{\end{quote}}
\newenvironment{fineprint}{\medskip \begin{small}}{\end{small} \medskip}
\iffalse
\newenvironment{proof}[1][Proof]{\noindent\textbf{#1.} }{\ \rule{0.5em}{0.5em}}
\newenvironment{question}[1][Question]{\noindent\textbf{#1.} }{\ \rule{0.5em}{0.5em}}
\newenvironment{warning}[1][Warning]{\noindent\textbf{#1.} }{\ \rule{0.5em}{0.5em}}
\newenvironment{teachingnote}[1][Teaching note]{\noindent\textbf{#1.} }{\ \rule{0.5em}{0.5em}}
\fi
\let\sumnonlimits\sum
\let\prodnonlimits\prod
\let\cupnonlimits\bigcup
\let\capnonlimits\bigcap
\renewcommand{\sum}{\sumnonlimits\limits}
\renewcommand{\prod}{\prodnonlimits\limits}
\renewcommand{\bigcup}{\cupnonlimits\limits}
\renewcommand{\bigcap}{\capnonlimits\limits}
\setlength\tablinesep{3pt}
\setlength\arraylinesep{3pt}
\setlength\extrarulesep{3pt}
\voffset=0cm
\hoffset=-0.7cm
\setlength\textheight{22.5cm}
\setlength\textwidth{15.5cm}
\newcommand\arxiv[1]{\href{http://www.arxiv.org/abs/#1}{\texttt{arXiv:#1}}}
\newenvironment{verlong}{}{}
\newenvironment{vershort}{}{}
\newenvironment{noncompile}{}{}
\newenvironment{teachingnote}{}{}
\excludecomment{verlong}
\includecomment{vershort}
\excludecomment{noncompile}
\excludecomment{teachingnote}
\newcommand{\CC}{\mathbb{C}}
\newcommand{\RR}{\mathbb{R}}
\newcommand{\QQ}{\mathbb{Q}}
\newcommand{\NN}{\mathbb{N}}
\newcommand{\ZZ}{\mathbb{Z}}
\newcommand{\KK}{\mathbb{K}}
\newcommand{\id}{\operatorname{id}}
\newcommand{\lcm}{\operatorname{lcm}}
\newcommand{\rev}{\operatorname{rev}}
\newcommand{\powset}[2][]{\ifthenelse{\equal{#2}{}}{\mathcal{P}\left(#1\right)}{\mathcal{P}_{#1}\left(#2\right)}}
\newcommand{\set}[1]{\left\{ #1 \right\}}
\newcommand{\abs}[1]{\left| #1 \right|}
\newcommand{\tup}[1]{\left( #1 \right)}
\newcommand{\ive}[1]{\left[ #1 \right]}
\newcommand{\floor}[1]{\left\lfloor #1 \right\rfloor}
\newcommand{\lf}[2]{#1^{\underline{#2}}}
\newcommand{\underbrack}[2]{\underbrace{#1}_{\substack{#2}}}
\newcommand{\horrule}[1]{\rule{\linewidth}{#1}}
\newcommand{\are}{\ar@{-}}
\newcommand{\nnn}{\nonumber\\}
\newcommand{\sslash}{\mathbin{/\mkern-6mu/}}
\newcommand{\numboxed}[2]{\underbrace{\boxed{#1}}_{\text{box } #2}}
\newcommand{\ig}[2]{\includegraphics[scale=#1]{#2.png}}
\newcommand{\UNFINISHED}{\begin{center} \Huge{\textbf{Unfinished material begins here.}} \end{center} }
\iffalse
\NOEXPAND{\today}{\today}
\NOEXPAND{\sslash}{\sslash}
\NOEXPAND{\numboxed}[2]{\numboxed}
\NOEXPAND{\UNFINISHED}{\UNFINISHED}
\fi
\ihead{Math 504 notes}
\ohead{page \thepage}
\cfoot{\today}
\begin{document}

\title{Math 504: Advanced Linear Algebra}
\author{Hugo Woerdeman, with edits by Darij Grinberg\thanks{Drexel University, Korman
Center, 15 S 33rd Street, Philadelphia PA, 19104, USA}}
\date{\today\ (unfinished!)}
\maketitle
\tableofcontents

\section*{Math 504 Lecture 20}

\section{Hermitian matrices (cont'd)}

\subsection{Introduction to majorization theory (cont'd)}

Recall:

\begin{convention}
Let $x=\left(  x_{1},x_{2},\ldots,x_{n}\right)  ^{T}\in\mathbb{R}^{n}$ be a
column vector with real entries. Then, for each $i\in\left[  n\right]  $, we
let $x_{i}^{\downarrow}$ denote the $i$-th largest entry of $x$. So $\left(
x_{1}^{\downarrow},x_{2}^{\downarrow},\ldots,x_{n}^{\downarrow}\right)  $ is
the unique permutation of the tuple $\left(  x_{1},x_{2},\ldots,x_{n}\right)
$ that satisfies
\[
x_{1}^{\downarrow}\geq x_{2}^{\downarrow}\geq\cdots\geq x_{n}^{\downarrow}.
\]

\end{convention}

For example, if $x=\left(  3,5,2\right)  ^{T}$, then $x_{1}^{\downarrow}=5$
and $x_{2}^{\downarrow}=3$ and $x_{3}^{\downarrow}=2$.

Similarly, we define $x_{i}^{\uparrow}$ to be the $i$-th smallest entry of $x$.

\begin{definition}
Let $x\in\mathbb{R}^{n}$ and $y\in\mathbb{R}^{n}$ be two column vectors with
real entries. Then, we say that $x$ \textbf{majorizes} $y$ (and we write
$x\succcurlyeq y$) if and only if we have%
\[
\sum_{i=1}^{m}x_{i}^{\downarrow}\geq\sum_{i=1}^{m}y_{i}^{\downarrow
}\ \ \ \ \ \ \ \ \ \ \text{for each }m\in\left[  n\right]  ,
\]
with equality for $m=n$ (and possibly for other $m$'s). In other words, $x$
majorizes $y$ if and only if%
\begin{align*}
x_{1}^{\downarrow}  &  \geq y_{1}^{\downarrow};\\
x_{1}^{\downarrow}+x_{2}^{\downarrow}  &  \geq y_{1}^{\downarrow}%
+y_{2}^{\downarrow};\\
x_{1}^{\downarrow}+x_{2}^{\downarrow}+x_{3}^{\downarrow}  &  \geq
y_{1}^{\downarrow}+y_{2}^{\downarrow}+y_{3}^{\downarrow};\\
&  \ldots;\\
x_{1}^{\downarrow}+x_{2}^{\downarrow}+\cdots+x_{n-1}^{\downarrow}  &  \geq
y_{1}^{\downarrow}+y_{2}^{\downarrow}+\cdots+y_{n-1}^{\downarrow};\\
x_{1}^{\downarrow}+x_{2}^{\downarrow}+\cdots+x_{n}^{\downarrow}  &
=y_{1}^{\downarrow}+y_{2}^{\downarrow}+\cdots+y_{n}^{\downarrow}.
\end{align*}

\end{definition}

Last time, we gave an intuition for majorization: We said that $x$ majorizes
$y$ if and only if you can obtain $y$ from $x$ by \textquotedblleft having the
entries come closer together (while keeping the average
equal)\textquotedblright. Let us now turn this into an actual theorem. First,
some definitions:

\begin{definition}
\textbf{(a)} If $x\in\mathbb{R}^{n}$ is any column vector, then $x_{i}$ will
mean the $i$-th coordinate of $x$ (for any $i\in\left[  n\right]  $).

\textbf{(b)} If $x\in\mathbb{R}^{n}$ is any column vector, then $x^{\downarrow
}$ will mean the column vector obtained from $x$ by sorting the coordinates in
weakly decreasing order. Thus,%
\[
x^{\downarrow}=\left(  x_{1}^{\downarrow},x_{2}^{\downarrow},\ldots
,x_{n}^{\downarrow}\right)  ^{T}.
\]


\textbf{(c)} A vector $x\in\mathbb{R}^{n}$ is said to be \textbf{weakly
decreasing} if $x_{1}\geq x_{2}\geq\cdots\geq x_{n}$.
\end{definition}

\begin{lemma}
Let $x,y\in\mathbb{R}^{n}$. Then, $x\succcurlyeq y$ if and only if
$x^{\downarrow}\succcurlyeq y^{\downarrow}$.
\end{lemma}

\begin{proof}
The definition of $\succcurlyeq$ only involves $x^{\downarrow}$ and
$y^{\downarrow}$. In other words, whether or not we have $x\succcurlyeq y$
does not depend on the order of the coordinates of $x$ or of those of $y$.
Thus, replacing $x$ and $y$ by $x^{\downarrow}$ and $y^{\downarrow}$ doesn't
make any difference.
\end{proof}

\begin{definition}
Let $x\in\mathbb{R}^{n}$. Let $i,j\in\left[  n\right]  $ be such that
$x_{i}\leq x_{j}$. Let $t\in\left[  x_{i},x_{j}\right]  $ (that is,
$t\in\mathbb{R}$ and $x_{i}\leq t\leq x_{j}$). Let $y\in\mathbb{R}^{n}$ be the
column vector obtained from $x$ by%
\[
\text{replacing the coordinates }x_{i}\text{ and }x_{j}\text{ by }u\text{ and
}v
\]
for some $u,v\in\left[  x_{i},x_{j}\right]  $ satisfying $u+v=x_{i}+x_{j}$.

Then, we say that $y$ is obtained from $x$ by a \textbf{Robin Hood move}
(short: \textbf{RH move}), and we write
\[
x\overset{\text{RH}}{\longrightarrow}y.
\]


Moreover, if $x$ and $y$ are weakly decreasing, then this RH move is said to
be an \textbf{order-preserving RH move} (short \textbf{OPRH move}), and we
write
\[
x\overset{\text{OPRH}}{\longrightarrow}y.
\]

\end{definition}

\begin{example}
\textbf{(a)} Replacing two coordinates of a vector $x$ by their average is an
RH move.

\textbf{(b)} Swapping two coordinates of a vector $x$ is an RH move.

\textbf{(c)} If $x\in\mathbb{R}^{n}$ is weakly decreasing, then replacing two
adjacent entries of $x$ by their average is an OPRH move.

\textbf{(d)} More generally: If $x\in\mathbb{R}^{n}$ is weakly decreasing,
then replacing its coordinates $x_{i}$ and $x_{i+1}$ by $u$ and $x_{i}%
+x_{i+1}-u$ is an OPRH move if and only if $u\in\left[  \dfrac{x_{i}+x_{i+1}%
}{2},x_{i}\right]  $.
\end{example}

\begin{proposition}
If $x\overset{\text{RH}}{\longrightarrow}y$, then the sum of the entries of
$x$ equals the sum of the entries of $y$.
\end{proposition}

\begin{proof}
Clear.
\end{proof}

\begin{lemma}
Let $x,y\in\mathbb{R}^{n}$ be weakly decreasing column vectors such that $y$
is obtained from $x$ by a (finite) sequence of OPRH moves. Then,
$x\succcurlyeq y$.
\end{lemma}

\begin{proof}
Recall that the relation $\succcurlyeq$ is reflexive and transitive. Thus, if
$x_{\left[  0\right]  }\succcurlyeq x_{\left[  1\right]  }\succcurlyeq
\cdots\succcurlyeq x_{\left[  m\right]  }$, then $x_{\left[  0\right]
}\succcurlyeq x_{\left[  m\right]  }$. Therefore, it suffices to prove the
proposition in the case when $y$ is obtained from $x$ by a \textbf{single}
OPRH move.

So let us assume that $y$ is obtained from $x$ by a \textbf{single} RH move.
Let this move be replacing $x_{i}$ and $x_{j}$ by $u$ and $v$, where
$x_{i}\leq x_{j}$ and $u,v\in\left[  x_{i},x_{j}\right]  $ with $u+v=x_{i}%
+x_{j}$. WLOG we have $x_{i}<x_{j}$ (since otherwise, the OPRH move changes
nothing). Therefore, $i>j$ (since $x$ is weakly decreasing). Thus,%
\[
y=\left(  x_{1},x_{2},\ldots,x_{j-1},v,x_{j+1},x_{j+2},\ldots,x_{i-1}%
,u,x_{i+1},x_{i+2},\ldots,x_{n}\right)
\]
(since $y$ is obtained from $x$ by replacing $x_{i}$ and $x_{j}$ by $u$ and
$v$).

Now, we must prove that $x\succcurlyeq y$. In other words, we must prove that%
\[
x_{1}+x_{2}+\cdots+x_{m}\geq y_{1}+y_{2}+\cdots+y_{m}%
\]
for each $m\in\left[  n\right]  $ (since $x$ and $y$ are weakly decreasing),
and we must prove that%
\[
x_{1}+x_{2}+\cdots+x_{n}=y_{1}+y_{2}+\cdots+y_{n}.
\]


The latter equality follows from $u+v=x_{i}+x_{j}$. So we only need to prove
the former inequality. So let us fix an $m\in\left[  n\right]  $. We must show
that
\[
x_{1}+x_{2}+\cdots+x_{m}\geq y_{1}+y_{2}+\cdots+y_{m}%
\]
We are in one of the following cases:

\begin{enumerate}
\item We have $m<j$.

\item We have $j\leq m<i$.

\item We have $i\leq m$.
\end{enumerate}

In Case 1, we have $x_{1}+x_{2}+\cdots+x_{m}=y_{1}+y_{2}+\cdots+y_{m}$,
because $x_{p}=y_{p}$ for all $p\leq m$ in this case.

In Case 2, we have%
\begin{align*}
y_{1}+y_{2}+\cdots+y_{m}  & =x_{1}+x_{2}+\cdots+x_{j-1}+v+x_{j+1}%
+x_{j+2}+\cdots+x_{m}\\
& =\left(  x_{1}+x_{2}+\cdots+x_{m}\right)  +\underbrace{v-x_{j}%
}_{\substack{\leq0\\\text{(since }v\in\left[  x_{i},x_{j}\right]  \text{)}%
}}\\
& \leq x_{1}+x_{2}+\cdots+x_{m}.
\end{align*}


In Case 3, we have%
\begin{align*}
& y_{1}+y_{2}+\cdots+y_{m}\\
& =x_{1}+x_{2}+\cdots+x_{j-1}+v+x_{j+1}+x_{j+2}+\cdots+x_{i-1}+u+x_{i+1}%
+x_{i+2}+\cdots+x_{m}\\
& =\left(  x_{1}+x_{2}+\cdots+x_{m}\right)  +\underbrace{\left(
u-x_{i}\right)  +\left(  v-x_{j}\right)  }_{\substack{=0\\\text{(since
}u+v=x_{i}+x_{j}\text{)}}}\\
& =x_{1}+x_{2}+\cdots+x_{m}.
\end{align*}


So we have proved $x_{1}+x_{2}+\cdots+x_{m}\geq y_{1}+y_{2}+\cdots+y_{m}$ in
all cases, and we are done.
\end{proof}

\begin{theorem}
[RH criterion for majorization]Let $x,y\in\mathbb{R}^{n}$ be two weakly
decreasing column vectors. Then, $x\succcurlyeq y$ if and only if $y$ can be
obtained from $x$ by a (finite) sequence of OPRH moves.
\end{theorem}

\begin{example}
\textbf{(a)} We have $\left(  4,1,1\right)  \succcurlyeq\left(  2,2,2\right)
$, and indeed $\left(  2,2,2\right)  $ can be obtained from $\left(
4,1,1\right)  $ by OPRH moves as follows:%
\[
\left(  4,1,1\right)  \overset{\text{OPRH}}{\longrightarrow}\left(
3,2,1\right)  \overset{\text{OPRH}}{\longrightarrow}\left(  2,2,2\right)  .
\]


\textbf{(b)} We have $\left(  7,5,2,0\right)  \succcurlyeq\left(
4,4,3,3\right)  $, and indeed $\left(  4,4,3,3\right)  $ can be obtained from
$\left(  7,5,2,0\right)  $ by OPRH moves as follows:%
\[
\left(  7,5,2,0\right)  \overset{\text{OPRH}}{\longrightarrow}\left(
6,6,2,0\right)  \overset{\text{OPRH}}{\longrightarrow}\left(  6,5,3,0\right)
\overset{\text{OPRH}}{\longrightarrow}\left(  6,4,3,1\right)
\overset{\text{OPRH}}{\longrightarrow}\left(  4,4,3,3\right)  .
\]
Here is another way to do this:%
\[
\left(  7,5,2,0\right)  \overset{\text{OPRH}}{\longrightarrow}\left(
7,4,3,0\right)  \overset{\text{OPRH}}{\longrightarrow}\left(  4,4,3,3\right)
.
\]

\end{example}

\begin{proof}
[Proof of Theorem.] $\Longleftarrow:$ This follows from the lemma above.

$\Longrightarrow:$ Let $x\succcurlyeq y$. We must show that $y$ can be
obtained from $x$ by a finite sequence of OPRH moves.

If $x=y$, then this is clear (just take the empty sequence). So we WLOG assume
that $x\neq y$.. We claim now that there is a further weakly decreasing vector
$z\in\mathbb{R}^{n}$ such that

\begin{enumerate}
\item we have $x\overset{\text{OPRH}}{\longrightarrow}z$;

\item we have $z\succcurlyeq y$;

\item the vector $z$ has more entries in common with $y$ than $x$ does; in
other words, we have%
\[
\left\vert \left\{  i\in\left[  n\right]  \ \mid\ z_{i}=y_{i}\right\}
\right\vert >\left\vert \left\{  i\in\left[  n\right]  \ \mid\ x_{i}%
=y_{i}\right\}  \right\vert .
\]

\end{enumerate}

In other words, we claim that by making a strategic OPRH move starting at $x$,
we can reach a vector $z$ that still majorizes $y$ but has at least one more
entry in common with $y$ than $x$ does. If we can prove this claim, then we
will automatically obtain a recursive procedure to transform $x$ into $y$ by a
sequence of OPRH moves. (And in fact, this procedure will use at most $n$
moves, because each move makes the vector agree with $y$ in at least one more position.)

So let us prove our claim.

Since $x$ is weakly decreasing, we have $x=x^{\downarrow}$. Similarly,
$y=y^{\downarrow}$. Thus, from $x\succcurlyeq y$, we obtain%
\[
x_{1}+x_{2}+\cdots+x_{m}\geq y_{1}+y_{2}+\cdots+y_{m}%
\ \ \ \ \ \ \ \ \ \ \text{for all }m\in\left[  n\right]  ,
\]
as well as $x_{1}+x_{2}+\cdots+x_{n}=y_{1}+y_{2}+\cdots+y_{n}$.

Combining $x\neq y$ with $x_{1}+x_{2}+\cdots+x_{n}=y_{1}+y_{2}+\cdots+y_{n}$,
we see that there exists some $a\in\left[  n\right]  $ such that $x_{a}>y_{a}$
(why?). Moreover, there exists some pair $\left(  a,b\right)  \in\left[
n\right]  \times\left[  n\right]  $ such that
\[
x_{a}>y_{a}\ \ \ \ \ \ \ \ \ \ \text{and}\ \ \ \ \ \ \ \ \ \ x_{b}%
<y_{b}\ \ \ \ \ \ \ \ \ \ \text{and}\ \ \ \ \ \ \ \ \ \ a<b.
\]
(\textit{Proof:} Pick the smallest $a$ such that $x_{a}\neq y_{a}$, then the
inequality $x_{1}+x_{2}+\cdots+x_{a}\geq y_{1}+y_{2}+\cdots+y_{a}$ shows that
$x_{a}>y_{a}$. Now, pick the smallest $b>a$ such that $x_{1}+x_{2}%
+\cdots+x_{b}=y_{1}+y_{2}+\cdots+y_{b}$, then comparing this with $x_{1}%
+x_{2}+\cdots+x_{b-1}\geq y_{1}+y_{2}+\cdots+y_{b-1}$ yields $x_{b}<y_{b}$,
and thus we have found our pair $\left(  a,b\right)  $.)

So let us pick such a pair $\left(  a,b\right)  $ with smallest possible
$b-a$. Then,%
\begin{align*}
x_{a}  & >y_{a},\\
x_{j}  & =y_{j}\ \ \ \ \ \ \ \ \ \ \text{for all }a<j<b,\\
x_{b}  & <y_{b}%
\end{align*}
(here, the equalities $x_{j}=y_{j}$ come from the \textquotedblleft smallest
possible $b-a$\textquotedblright\ condition). Since $y$ is weakly decreasing,
we thus have%
\[
x_{a}>y_{a}\geq\left(  \text{all of the }x_{j}\text{ and }y_{j}\text{ with
}a<j<b\right)  \geq y_{b}>x_{b}.
\]
(If there are no $a<j<b$, then this is supposed to read $x_{a}>y_{a}\geq
y_{b}>x_{b}$.) This shows, in particular, that $y_{a},y_{b}$ lie in the open
interval $\left(  x_{a},x_{b}\right)  $.

Now, we make an RH move that \textquotedblleft squeezes $x_{a}$ and $x_{b}$
together\textquotedblright\ until either $x_{a}$ reaches $y_{a}$ or $x_{b}$
reaches $y_{b}$ (whatever happens first). In formal terms, this means that we
\begin{align*}
\text{replace }x_{a}\text{ and }x_{b}\text{ by }y_{a}\text{ and }x_{a}%
+x_{b}-y_{a}\text{ if }x_{a}-y_{a}  & \leq y_{b}-x_{b},\text{ and}\\
\text{replace }x_{a}\text{ and }x_{b}\text{ by }x_{a}+x_{b}-y_{b}\text{ and
}y_{b}\text{ if }x_{a}-y_{a}  & \geq y_{b}-x_{b}.
\end{align*}
Let $z\in\mathbb{R}^{n}$ be the resulting $n$-tuple. We claim that $z$ is
weakly decreasing and satisfies the three requirements 1, 2, 3 above:

\begin{enumerate}
\item we have $x\overset{\text{OPRH}}{\longrightarrow}z$;

\item we have $z\succcurlyeq y$;

\item the vector $z$ has more entries in common with $y$ than $x$ does; in
other words, we have%
\[
\left\vert \left\{  i\in\left[  n\right]  \ \mid\ z_{i}=y_{i}\right\}
\right\vert >\left\vert \left\{  i\in\left[  n\right]  \ \mid\ x_{i}%
=y_{i}\right\}  \right\vert .
\]

\end{enumerate}

Indeed, the chain of inequalities%
\[
x_{a}>y_{a}\geq\left(  \text{all of the }x_{j}\text{ and }y_{j}\text{ with
}a<j<b\right)  \geq y_{b}>x_{b}%
\]
reveals that $z$ is weakly decreasing; thus, our RH move is an OPRH move. So
requirement 1 holds.

Requirement 2 is not hard to check (distinguish between the cases $m<a$,
$a\leq m<b$ and $m\geq b$). Requirement 3 is easy: $x_{a}>y_{a}$ and
$x_{b}<y_{b}$ but one of $z_{a}$ and $z_{b}$ equals the corresponding $y_{a}$
or $y_{b}$.

This completes the proof, as explained above.
\end{proof}


\end{document}