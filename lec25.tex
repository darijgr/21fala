\documentclass[numbers=enddot,12pt,final,onecolumn,notitlepage]{scrartcl}%
\usepackage[headsepline,footsepline,manualmark]{scrlayer-scrpage}
\usepackage[all,cmtip]{xy}
\usepackage{amssymb}
\usepackage{amsmath}
\usepackage{amsthm}
\usepackage{framed}
\usepackage{comment}
\usepackage{color}
\usepackage{hyperref}
\usepackage[sc]{mathpazo}
\usepackage[T1]{fontenc}
\usepackage{tikz}
\usepackage{needspace}
\usepackage{tabls}
\usepackage{wasysym}
\usepackage{easytable}
\usepackage{pythonhighlight}
%TCIDATA{OutputFilter=latex2.dll}
%TCIDATA{Version=5.50.0.2960}
%TCIDATA{LastRevised=Friday, December 03, 2021 11:51:20}
%TCIDATA{SuppressPackageManagement}
%TCIDATA{<META NAME="GraphicsSave" CONTENT="32">}
%TCIDATA{<META NAME="SaveForMode" CONTENT="1">}
%TCIDATA{BibliographyScheme=Manual}
%TCIDATA{Language=American English}
%BeginMSIPreambleData
\providecommand{\U}[1]{\protect\rule{.1in}{.1in}}
%EndMSIPreambleData
\usetikzlibrary{arrows.meta}
\usetikzlibrary{chains}
\newcounter{exer}
\newcounter{exera}
\numberwithin{exer}{subsection}
\theoremstyle{definition}
\newtheorem{theo}{Theorem}[subsection]
\newenvironment{theorem}[1][]
{\begin{theo}[#1]\begin{leftbar}}
{\end{leftbar}\end{theo}}
\newtheorem{lem}[theo]{Lemma}
\newenvironment{lemma}[1][]
{\begin{lem}[#1]\begin{leftbar}}
{\end{leftbar}\end{lem}}
\newtheorem{prop}[theo]{Proposition}
\newenvironment{proposition}[1][]
{\begin{prop}[#1]\begin{leftbar}}
{\end{leftbar}\end{prop}}
\newtheorem{defi}[theo]{Definition}
\newenvironment{definition}[1][]
{\begin{defi}[#1]\begin{leftbar}}
{\end{leftbar}\end{defi}}
\newtheorem{remk}[theo]{Remark}
\newenvironment{remark}[1][]
{\begin{remk}[#1]\begin{leftbar}}
{\end{leftbar}\end{remk}}
\newtheorem{coro}[theo]{Corollary}
\newenvironment{corollary}[1][]
{\begin{coro}[#1]\begin{leftbar}}
{\end{leftbar}\end{coro}}
\newtheorem{conv}[theo]{Convention}
\newenvironment{convention}[1][]
{\begin{conv}[#1]\begin{leftbar}}
{\end{leftbar}\end{conv}}
\newtheorem{quest}[theo]{Question}
\newenvironment{question}[1][]
{\begin{quest}[#1]\begin{leftbar}}
{\end{leftbar}\end{quest}}
\newtheorem{warn}[theo]{Warning}
\newenvironment{warning}[1][]
{\begin{warn}[#1]\begin{leftbar}}
{\end{leftbar}\end{warn}}
\newtheorem{conj}[theo]{Conjecture}
\newenvironment{conjecture}[1][]
{\begin{conj}[#1]\begin{leftbar}}
{\end{leftbar}\end{conj}}
\newtheorem{exam}[theo]{Example}
\newenvironment{example}[1][]
{\begin{exam}[#1]\begin{leftbar}}
{\end{leftbar}\end{exam}}
\newtheorem{exmp}[exer]{Exercise}
\newenvironment{exercise}[1][]
{\begin{exmp}[#1]\begin{leftbar}}
{\end{leftbar}\end{exmp}}
\newenvironment{statement}{\begin{quote}}{\end{quote}}
\newenvironment{fineprint}{\medskip \begin{small}}{\end{small} \medskip}
\iffalse
\newenvironment{proof}[1][Proof]{\noindent\textbf{#1.} }{\ \rule{0.5em}{0.5em}}
\newenvironment{question}[1][Question]{\noindent\textbf{#1.} }{\ \rule{0.5em}{0.5em}}
\newenvironment{warning}[1][Warning]{\noindent\textbf{#1.} }{\ \rule{0.5em}{0.5em}}
\newenvironment{teachingnote}[1][Teaching note]{\noindent\textbf{#1.} }{\ \rule{0.5em}{0.5em}}
\fi
\let\sumnonlimits\sum
\let\prodnonlimits\prod
\let\cupnonlimits\bigcup
\let\capnonlimits\bigcap
\renewcommand{\sum}{\sumnonlimits\limits}
\renewcommand{\prod}{\prodnonlimits\limits}
\renewcommand{\bigcup}{\cupnonlimits\limits}
\renewcommand{\bigcap}{\capnonlimits\limits}
\setlength\tablinesep{3pt}
\setlength\arraylinesep{3pt}
\setlength\extrarulesep{3pt}
\voffset=0cm
\hoffset=-0.7cm
\setlength\textheight{22.5cm}
\setlength\textwidth{15.5cm}
\newcommand\arxiv[1]{\href{http://www.arxiv.org/abs/#1}{\texttt{arXiv:#1}}}
\newenvironment{verlong}{}{}
\newenvironment{vershort}{}{}
\newenvironment{noncompile}{}{}
\newenvironment{teachingnote}{}{}
\excludecomment{verlong}
\includecomment{vershort}
\excludecomment{noncompile}
\excludecomment{teachingnote}
\newcommand{\CC}{\mathbb{C}}
\newcommand{\RR}{\mathbb{R}}
\newcommand{\QQ}{\mathbb{Q}}
\newcommand{\NN}{\mathbb{N}}
\newcommand{\ZZ}{\mathbb{Z}}
\newcommand{\KK}{\mathbb{K}}
\newcommand{\id}{\operatorname{id}}
\newcommand{\lcm}{\operatorname{lcm}}
\newcommand{\rev}{\operatorname{rev}}
\newcommand{\powset}[2][]{\ifthenelse{\equal{#2}{}}{\mathcal{P}\left(#1\right)}{\mathcal{P}_{#1}\left(#2\right)}}
\newcommand{\set}[1]{\left\{ #1 \right\}}
\newcommand{\abs}[1]{\left| #1 \right|}
\newcommand{\tup}[1]{\left( #1 \right)}
\newcommand{\ive}[1]{\left[ #1 \right]}
\newcommand{\floor}[1]{\left\lfloor #1 \right\rfloor}
\newcommand{\lf}[2]{#1^{\underline{#2}}}
\newcommand{\underbrack}[2]{\underbrace{#1}_{\substack{#2}}}
\newcommand{\horrule}[1]{\rule{\linewidth}{#1}}
\newcommand{\are}{\ar@{-}}
\newcommand{\nnn}{\nonumber\\}
\newcommand{\sslash}{\mathbin{/\mkern-6mu/}}
\newcommand{\numboxed}[2]{\underbrace{\boxed{#1}}_{\text{box } #2}}
\newcommand{\ig}[2]{\includegraphics[scale=#1]{#2.png}}
\newcommand{\UNFINISHED}{\begin{center} \Huge{\textbf{Unfinished material begins here.}} \end{center} }
\iffalse
\NOEXPAND{\today}{\today}
\NOEXPAND{\sslash}{\sslash}
\NOEXPAND{\numboxed}[2]{\numboxed}
\NOEXPAND{\UNFINISHED}{\UNFINISHED}
\fi
\ihead{Math 504 notes}
\ohead{page \thepage}
\cfoot{\today}
\begin{document}

\title{Math 504: Advanced Linear Algebra}
\author{Hugo Woerdeman, with edits by Darij Grinberg\thanks{Drexel University, Korman
Center, 15 S 33rd Street, Philadelphia PA, 19104, USA}}
\date{\today\ (unfinished!)}
\maketitle
\tableofcontents

\section*{Math 504 Lecture 25}

\section{Positive and nonnegative matrices ([HorJoh, Chapter 8]) (cont'd)}

\subsection{The Perron theorems (cont'd)}

\subsubsection{Lemmas for the proofs}

Before we prove the Perron and Perron--Frobenius theorems, we need to state a
few lemmas. First, recall the corollary:

\begin{corollary}
\label{cor.posmat.rho-in-terms-of-rowsums-x}Let $A\in\mathbb{R}^{n\times n}$
satisfy $A\geq0$ and $n>0$. Let $x_{1},x_{2},\ldots,x_{n}$ be any $n$ positive
reals. Then,%
\[
\min\limits_{i\in\left[  n\right]  }\sum_{j=1}^{n}\dfrac{x_{i}}{x_{j}}%
A_{i,j}\leq\rho\left(  A\right)  \leq\max\limits_{i\in\left[  n\right]  }%
\sum_{j=1}^{n}\dfrac{x_{i}}{x_{j}}A_{i,j}.
\]

\end{corollary}

\begin{corollary}
Let $A\in\mathbb{R}^{n\times n}$ satisfy $A\geq0$ and $n>0$.

Let $x\in\mathbb{R}^{n}$ satisfy $x>0$.

Let $\alpha$ be a nonnegative real. Then:

\textbf{(a)} If $Ax\geq\alpha x$, then $\rho\left(  A\right)  \geq\alpha$.

\textbf{(b)} If $Ax>\alpha x$, then $\rho\left(  A\right)  >\alpha$.

\textbf{(c)} If $Ax\leq\alpha x$, then $\rho\left(  A\right)  \leq\alpha$.

\textbf{(d)} If $Ax<\alpha x$, then $\rho\left(  A\right)  <\alpha$.
\end{corollary}

\begin{proof}
Write $x$ as $x=\left(  \dfrac{1}{x_{1}},\dfrac{1}{x_{2}},\ldots,\dfrac
{1}{x_{n}}\right)  ^{T}$; then, $x_{1},x_{2},\ldots,x_{n}>0$.

\textbf{(a)} Assume that $Ax\geq\alpha x$. The old corollary yields
\[
\min\limits_{i\in\left[  n\right]  }\sum_{j=1}^{n}\dfrac{x_{i}}{x_{j}}%
A_{i,j}\leq\rho\left(  A\right)  .
\]
Thus,%
\[
\rho\left(  A\right)  \geq\min\limits_{i\in\left[  n\right]  }\sum_{j=1}%
^{n}\dfrac{x_{i}}{x_{j}}A_{i,j}=\min\limits_{i\in\left[  n\right]  }%
x_{i}\underbrace{\sum_{j=1}^{n}\dfrac{1}{x_{j}}A_{i,j}}_{\substack{=\left(
\text{the }i\text{-th entry of }Ax\right)  \geq\dfrac{\alpha}{x_{i}%
}\\\text{(since }Ax\geq\alpha x\text{)}}}\geq\min\limits_{i\in\left[
n\right]  }x_{i}\cdot\dfrac{\alpha}{x_{i}}=\alpha.
\]


\textbf{(b)} Use the same argument as before, but with $>$ sign.

\textbf{(c)} Analogous, but now use the other half of the old corollary.

\textbf{(d)} Analogous.
\end{proof}

\begin{corollary}
Let $A\in\mathbb{R}^{n\times n}$ satisfy $A>0$ and $n>0$ and $\rho\left(
A\right)  =1$. Let $w\in\mathbb{R}^{n}$ satisfy $w\geq0$ and $w\neq0$.

\textbf{(a)} We always have $Aw>0$.

\textbf{(b)} If $Aw\geq w$, then $Aw=w>0$.
\end{corollary}

\begin{proof}
\textbf{(a)} For each $i$, the $i$-th entry of $Aw$ is $\sum_{j=1}^{n}%
A_{i,j}w_{j}$ (where $w_{j}$ is the $j$-th entry of $w$). This is a sum of
nonnegative addends, and at least one of these addends is actually positive
(since $w\neq0$ entails that $w_{j}>0$ for some $j$, and then we also have
$A_{i,j}>0$ because $A>0$). So this sum is positive. Thus we have shown that
all entries of $Aw$ are positive. In other words, $Aw>0$.

\textbf{(b)} Assume that $Aw\geq w$. Let $z:=Aw-w$. Then, $z\geq0$. Hence, if
we had $z\neq0$, then part \textbf{(a)} (applied to $z$ instead of $w$) would
yield $Az>0$, so that $A\left(  Aw-w\right)  >0$.

Part \textbf{(a)} yields $Aw>0$. If we had $AAw>Aw$, then part \textbf{(b)} of
the preceding corollary (applied to $x=Aw$ and $\alpha=1$) would yield
$\rho\left(  A\right)  >1$, which would contradict $\rho\left(  A\right)  =1$.
So we cannot have $AAw>Aw$. In other words, we cannot have $A\left(
Aw-w\right)  >0$. Thus, we cannot have $z\neq0$ (by the previous paragraph).
So $z=0$ and therefore $Aw=w$ (since $z=Aw-w$). This also entails $w=Aw>0$ by
part \textbf{(a)}. So part \textbf{(b)} is proved.
\end{proof}

\begin{definition}
Fix $n>0$. Let $e=\left(  1,1,\ldots,1\right)  ^{T}$.
\end{definition}

\begin{remark}
\textbf{(a)} An $n\times n$-matrix $A$ satisfies $Ae=e$ if and only if all row
sums of $A$ equal $1$.

\textbf{(b)} An $n\times n$-matrix $A$ satisfies $e^{T}A=e^{T}$ if and only if
all column sums of $A$ equal $1$.
\end{remark}

\begin{lemma}
[crucifix lemma, special case]Let $A\in\mathbb{R}^{n\times n}$ satisfy $A>0$
and $Ae=e$. Let $y\in\mathbb{R}^{n}$ satisfy $y^{T}A=y^{T}$ and $y\geq0$ and
$y^{T}e=1$ (that is, the sum of all entries of $y$ is $1$). Then,%
\[
A^{m}\rightarrow ey^{T}\ \ \ \ \ \ \ \ \ \ \text{as }m\rightarrow\infty.
\]

\end{lemma}

\begin{proof}
The entries of the vector $y$ are nonnegative reals (since $y\geq0$) and their
sum is $1$ (since $y^{T}e=1$). Thus, all these entries lie in the interval
$\left[  0,1\right]  $.

From $Ae=e$, we conclude that all row sums of $A$ equal $1$. Since $A>0$, this
implies that all entries $A_{i,j}$ of $A$ satisfy%
\[
0<A_{i,j}\leq1.
\]


Let
\[
\mu:=1-\min\left\{  A_{i,j}\ \mid\ i,j\in\left[  n\right]  \right\}  .
\]
Then, $0\leq\mu<1$ (by the previous inequality).

\begin{statement}
\textit{Claim 1:} For each $i\in\left[  n\right]  $ and each \textbf{proper}
subset $K$ of $\left[  n\right]  $, we have%
\[
\sum_{k\in K}A_{i,k}\leq\mu.
\]

\end{statement}

[\textit{Proof of Claim 1:} Let $i\in\left[  n\right]  $. Let $K$ be a proper
subset of $\left[  n\right]  $. Then,
\[
\sum_{k\in K}A_{i,k}=1-\underbrace{\sum_{k\notin K}A_{i,k}}_{\substack{\geq
\min\left\{  A_{i,j}\ \mid\ i,j\in\left[  n\right]  \right\}  \\\text{(since
there exists at least one }k\notin K\text{)}}}\leq1-\min\left\{  A_{i,j}%
\ \mid\ i,j\in\left[  n\right]  \right\}  =\mu,
\]
so Claim 1 is proved.]

Now, we claim:

\begin{statement}
\textit{Claim 2:} For any $i,j\in\left[  n\right]  $ and any $m\in\mathbb{N}$,
we have%
\[
\left\vert \left(  A^{m}-ey^{T}\right)  _{i,j}\right\vert \leq\mu^{m}.
\]

\end{statement}

Once Claim 2 is proved, it will follow easily that $\left(  A^{m}%
-ey^{T}\right)  _{i,j}\rightarrow0$ as $m\rightarrow\infty$ (because $0\leq
\mu<1$), so that $A^{m}\rightarrow ey^{T}$, and the lemma will thus follow.

[\textit{Proof of Claim 2:} We induct on $m$:

\textit{Base case:} We need to show that $\left\vert \left(  I_{n}%
-ey^{T}\right)  _{i,j}\right\vert \leq1$ for all $i,j$. This follows from the
fact that the entries of $y$ lie in the interval $\left[  0,1\right]  $
(because $\left(  I_{n}-ey^{T}\right)  _{i,j}=\delta_{i,j}-\underbrace{y_{j}%
}_{\in\left[  0,1\right]  }\in\left[  -1,1\right]  $).

\textit{Induction step:} Let $p\in\mathbb{N}$. Assume (as the induction
hypothesis) that Claim 2 holds for $m=p$. We must now show that it also holds
for $m=p+1$.

Let $B:=A^{p}-ey^{T}$ and $C:=A^{p+1}-ey^{T}$. So our IH says that $\left\vert
B_{i,j}\right\vert \leq\mu^{p}$ for all $i,j$. Our goal is to show that
$\left\vert C_{i,j}\right\vert \leq\mu^{p+1}$ for all $i,j$.

Fix $i,j$. We have%
\[
AB=A\left(  A^{p}-ey^{T}\right)  =A^{p+1}-\underbrace{Ae}_{=e}y^{T}%
=A^{p+1}-ey^{T}=C.
\]
Hence, $C=AB$, so that%
\[
C_{i,j}=\sum_{k=1}^{n}\underbrace{A_{i,k}}_{>0}B_{k,j}=\sum_{k\in P}%
A_{i,k}\left\vert B_{k,j}\right\vert -\sum_{k\in N}A_{i,k}\left\vert
B_{k,j}\right\vert ,
\]
where%
\[
P:=\left\{  k\in\left[  n\right]  \ \mid\ B_{k,j}>0\right\}
\ \ \ \ \ \ \ \ \ \ \text{and}\ \ \ \ \ \ \ \ \ \ N:=\left\{  k\in\left[
n\right]  \ \mid\ B_{k,j}<0\right\}  .
\]


However, the entries of the $j$-th column of $B$ cannot all have the same sign
(i.e., both subsets $P$ and $N$ of $\left[  n\right]  $ are proper). The
reason for this is that%
\begin{align*}
y^{T}B  & =y^{T}\left(  A^{p}-ey^{T}\right)  =\underbrace{y^{T}A^{p}%
}_{\substack{=y^{T}\\\text{(since }y^{T}A=y^{T}\text{)}}}-\underbrace{y^{T}%
e}_{=1}y^{T}=y^{T}-y^{T}=0\\
& \Longrightarrow\ \ \ \ \ \ \ \ \ \ \left(  \text{look at the }j\text{-th
entry}\right)  \\
\sum_{k=1}^{n}\underbrace{y_{k}}_{\geq0}B_{k,j}  & =0
\end{align*}
and that $y^{T}$ is a nonzero nonnegative vector, so there is a nontrivial
linear combination of the entries of the $j$-th column of $B$ with nonnegative
coefficients that is $0$.

So both subsets $P$ and $N$ of $\left[  n\right]  $ are proper.

Now, from%
\[
C_{i,j}=\sum_{k=1}^{n}\underbrace{A_{i,k}}_{>0}B_{k,j}=\sum_{k\in P}%
A_{i,k}\left\vert B_{k,j}\right\vert -\sum_{k\in N}A_{i,k}\left\vert
B_{k,j}\right\vert ,
\]
we obtain%
\[
\left\vert C_{i,j}\right\vert =\left\vert \sum_{k\in P}A_{i,k}\left\vert
B_{k,j}\right\vert -\sum_{k\in N}A_{i,k}\left\vert B_{k,j}\right\vert
\right\vert \leq\max\left\{  \sum_{k\in P}A_{i,k}\left\vert B_{k,j}\right\vert
,\ \sum_{k\in N}A_{i,k}\left\vert B_{k,j}\right\vert \right\}  ,
\]
since any two nonnegative reals $x$ and $y$ satisfy $\left\vert x-y\right\vert
\leq\max\left\{  x,y\right\}  $. Thus,%
\[
\left\vert C_{i,j}\right\vert \leq\sum_{k\in K}A_{i,k}\left\vert
B_{k,j}\right\vert ,
\]
where $K$ is either $P$ or $N$. In either case, $K$ is a proper subset of
$\left[  n\right]  $. Therefore,%
\[
\left\vert C_{i,j}\right\vert \leq\sum_{k\in K}A_{i,k}\underbrace{\left\vert
B_{k,j}\right\vert }_{\substack{\leq\mu^{p}\\\text{(by IH)}}}\leq\mu
^{p}\underbrace{\sum_{k\in K}A_{i,k}}_{\substack{\leq\mu\\\text{(by Claim 1)}%
}}\leq\mu^{p}\mu=\mu^{p+1}.
\]
This completes the induction step. Thus, Claim 2 is proved.]
\end{proof}

\begin{lemma}
[crucifix lemma, general case]Let $A\in\mathbb{R}^{n\times n}$ satisfy $A>0$.
Let $x\in\mathbb{R}^{n}$ satisfy $Ax=x$ and $x>0$. Let $y\in\mathbb{R}^{n}$
satisfy $y^{T}A=y^{T}$ and $y\geq0$ and $y^{T}x=1$. Then,%
\[
A^{m}\rightarrow xy^{T}\ \ \ \ \ \ \ \ \ \ \text{as }m\rightarrow\infty.
\]

\end{lemma}

\begin{proof}
Let $x_{1},x_{2},\ldots,x_{n}$ be the entries of $x$; then $x_{i}>0$.

Let $D=\operatorname*{diag}\left(  x_{1},x_{2},\ldots,x_{n}\right)  $. Then,
$De=x$.

Now, apply the previous lemma to $D^{-1}AD$ and $Dy$ instead of $A$ and $y$.
Details are LTTR.
\end{proof}



\begin{proof}
[Proof of Perron's theorem.] \textbf{(a)} Recall the corollary we had a while
ago, which said that if $A\geq0$ satisfies $A_{i,i}>0$ for some $i\in\left[
n\right]  $, then $\rho\left(  A\right)  >0$. This clearly applies, since
$A>0$. So part \textbf{(a)} is proved.

Next, knowing that $\rho\left(  A\right)  >0$, we can replace $A$ by
$\dfrac{1}{\rho\left(  A\right)  }A$. This way, $\rho\left(  A\right)  $
becomes $1$, but nothing else significantly changes. So we have $\rho\left(
A\right)  =1$ now.

Next, we shall show that there is a positive $1$-eigenvector of $A$. Indeed,
from $\rho\left(  A\right)  =1$, we see that $A$ has an eigenvalue $\lambda$
with $\left\vert \lambda\right\vert =1$. Consider this $\lambda$. Pick any
nonzero $\lambda$-eigenvector $z=\left(  z_{1},z_{2},\ldots,z_{n}\right)
^{T}\in\mathbb{C}^{n}$ of $A$. Then, $Az=\lambda z$. Hence,%
\[
\underbrace{A}_{=\left\vert A\right\vert }\left\vert z\right\vert =\left\vert
A\right\vert \cdot\left\vert z\right\vert \geq\left\vert Az\right\vert
=\left\vert \lambda z\right\vert =\underbrace{\left\vert \lambda\right\vert
}_{=1}\cdot\left\vert z\right\vert =\left\vert z\right\vert .
\]
Hence, applying part \textbf{(b)} to the last corollary, we conclude that
$A\left\vert z\right\vert =\left\vert z\right\vert >0$. Thus, $\left\vert
z\right\vert $ is a positive $1$-eigenvector of $A$.

We have thus constructed a positive $1$-eigenvector of $A$. The same argument
(applied to $A^{T}$ instead of $A$) yields a positive $1$-eigenvector of
$A^{T}$, and thus (by transposing it) a positive left $1$-eigenvector of $A$.
Let these two eigenvectors be $x$ and $y$. Thus, $x\in\mathbb{R}^{n}$
satisfies $Ax=x$ and $x>0$, and $y\in\mathbb{R}^{n}$ satisfies $y^{T}A=y^{T}$
and $y>0$. Moreover, by scaling $y$ appropriately, we can achieve $y^{T}x=1$.
Thus, the crucifix lemma yields
\[
A^{m}\rightarrow xy^{T}\ \ \ \ \ \ \ \ \ \ \text{as }m\rightarrow\infty.
\]
This proves part \textbf{(e)}.

Remains to prove the uniqueness claims in parts \textbf{(c)} and \textbf{(d)},
and also parts \textbf{(b)} and \textbf{(f)}. We can do this in one fell swoop
if we can show the following: If $\lambda_{1},\lambda_{2},\ldots,\lambda_{n}$
are the eigenvalues of $A$, then only one of these eigenvalues has absolute
value $\geq1$.

This follows easily from $A^{m}\rightarrow xy^{T}$. Indeed, let $\left(
U,T\right)  $ be the Schur triangularization of $A$. Then,
\[
A=UTU^{\ast}=UTU^{-1},
\]
and the diagonal entries of $T$ are the eigenvalues $\lambda_{1},\lambda
_{2},\ldots,\lambda_{n}$ of $A$. Hence,%
\[
A^{m}=\left(  UTU^{-1}\right)  ^{m}=UT^{m}U^{-1}.
\]
Hence, taking the limit as $m\rightarrow\infty$, we get%
\[
xy^{T}=UT^{\infty}U^{-1},
\]
where $T^{\infty}=U^{-1}xy^{T}U$ is a triangular matrix with diagonal entries
$\lambda_{1}^{\infty},\lambda_{2}^{\infty},\ldots,\lambda_{n}^{\infty}$. This
shows that all $\lambda_{1},\lambda_{2},\ldots,\lambda_{n}$ have absolute
value $\leq1$, and the ones that have absolute value $1$ must equal $1$.
Moreover, if more than one of the $\lambda_{i}$s would equal $1$, then
$UT^{\infty}U^{-1}$ would have rank $>1$, but then it could not equal $xy^{T}$
(since $\operatorname*{rank}\left(  xy^{T}\right)  \leq1$). Qed.
\end{proof}

-----------

\begin{theorem}
[Perron theorem]\label{thm.posmat.perron}Let $A\in\mathbb{R}^{n\times n}$
satisfy $A>0$ and $n>0$. Then: \medskip

\textbf{(a)} We have $\rho\left(  A\right)  >0$. \medskip

\textbf{(b)} The number $\rho\left(  A\right)  $ is an eigenvalue of $A$ and
has algebraic multiplicity $1$ (and therefore geometric multiplicity $1$ as
well). \medskip

\textbf{(c)} There is a unique $\rho\left(  A\right)  $-eigenvector $x=\left(
x_{1},x_{2},\ldots,x_{n}\right)  ^{T}\in\mathbb{C}^{n}$ of $A$ with
$x_{1}+x_{2}+\cdots+x_{n}=1$. This eigenvector $x$ is furthermore positive.
(It is called the \emph{Perron vector} of $A$.) \medskip

\textbf{(d)} There is a unique vector $y=\left(  y_{1},y_{2},\ldots
,y_{n}\right)  ^{T}\in\mathbb{C}^{n}$ such that $y^{T}A=\rho\left(  A\right)
y^{T}$ and $x_{1}y_{1}+x_{2}y_{2}+\cdots+x_{n}y_{n}=1$. This vector $y$ is
also positive. \medskip

\textbf{(e)} We have%
\[
\left(  \dfrac{1}{\rho\left(  A\right)  }A\right)  ^{m}\rightarrow
xy^{T}\ \ \ \ \ \ \ \ \ \ \text{as }m\rightarrow\infty.
\]


\textbf{(f)} The only eigenvalue of $A$ that has absolute value $\rho\left(
A\right)  $ is $\rho\left(  A\right)  $ itself.
\end{theorem}


\end{document}