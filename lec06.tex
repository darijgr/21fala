\documentclass[numbers=enddot,12pt,final,onecolumn,notitlepage]{scrartcl}%
\usepackage[headsepline,footsepline,manualmark]{scrlayer-scrpage}
\usepackage[all,cmtip]{xy}
\usepackage{amssymb}
\usepackage{amsmath}
\usepackage{amsthm}
\usepackage{framed}
\usepackage{comment}
\usepackage{color}
\usepackage{hyperref}
\usepackage[sc]{mathpazo}
\usepackage[T1]{fontenc}
\usepackage{tikz}
\usepackage{needspace}
\usepackage{tabls}
\usepackage{wasysym}
\usepackage{easytable}
\usepackage{pythonhighlight}
%TCIDATA{OutputFilter=latex2.dll}
%TCIDATA{Version=5.50.0.2960}
%TCIDATA{LastRevised=Friday, October 01, 2021 11:51:08}
%TCIDATA{SuppressPackageManagement}
%TCIDATA{<META NAME="GraphicsSave" CONTENT="32">}
%TCIDATA{<META NAME="SaveForMode" CONTENT="1">}
%TCIDATA{BibliographyScheme=Manual}
%TCIDATA{Language=American English}
%BeginMSIPreambleData
\providecommand{\U}[1]{\protect\rule{.1in}{.1in}}
%EndMSIPreambleData
\usetikzlibrary{arrows.meta}
\usetikzlibrary{chains}
\newcounter{exer}
\newcounter{exera}
\numberwithin{exer}{subsection}
\theoremstyle{definition}
\newtheorem{theo}{Theorem}[subsection]
\newenvironment{theorem}[1][]
{\begin{theo}[#1]\begin{leftbar}}
{\end{leftbar}\end{theo}}
\newtheorem{lem}[theo]{Lemma}
\newenvironment{lemma}[1][]
{\begin{lem}[#1]\begin{leftbar}}
{\end{leftbar}\end{lem}}
\newtheorem{prop}[theo]{Proposition}
\newenvironment{proposition}[1][]
{\begin{prop}[#1]\begin{leftbar}}
{\end{leftbar}\end{prop}}
\newtheorem{defi}[theo]{Definition}
\newenvironment{definition}[1][]
{\begin{defi}[#1]\begin{leftbar}}
{\end{leftbar}\end{defi}}
\newtheorem{remk}[theo]{Remark}
\newenvironment{remark}[1][]
{\begin{remk}[#1]\begin{leftbar}}
{\end{leftbar}\end{remk}}
\newtheorem{coro}[theo]{Corollary}
\newenvironment{corollary}[1][]
{\begin{coro}[#1]\begin{leftbar}}
{\end{leftbar}\end{coro}}
\newtheorem{conv}[theo]{Convention}
\newenvironment{convention}[1][]
{\begin{conv}[#1]\begin{leftbar}}
{\end{leftbar}\end{conv}}
\newtheorem{quest}[theo]{Question}
\newenvironment{question}[1][]
{\begin{quest}[#1]\begin{leftbar}}
{\end{leftbar}\end{quest}}
\newtheorem{warn}[theo]{Warning}
\newenvironment{warning}[1][]
{\begin{warn}[#1]\begin{leftbar}}
{\end{leftbar}\end{warn}}
\newtheorem{conj}[theo]{Conjecture}
\newenvironment{conjecture}[1][]
{\begin{conj}[#1]\begin{leftbar}}
{\end{leftbar}\end{conj}}
\newtheorem{exam}[theo]{Example}
\newenvironment{example}[1][]
{\begin{exam}[#1]\begin{leftbar}}
{\end{leftbar}\end{exam}}
\newtheorem{exmp}[exer]{Exercise}
\newenvironment{exercise}[1][]
{\begin{exmp}[#1]\begin{leftbar}}
{\end{leftbar}\end{exmp}}
\newenvironment{statement}{\begin{quote}}{\end{quote}}
\newenvironment{fineprint}{\medskip \begin{small}}{\end{small} \medskip}
\iffalse
\newenvironment{proof}[1][Proof]{\noindent\textbf{#1.} }{\ \rule{0.5em}{0.5em}}
\newenvironment{question}[1][Question]{\noindent\textbf{#1.} }{\ \rule{0.5em}{0.5em}}
\newenvironment{warning}[1][Warning]{\noindent\textbf{#1.} }{\ \rule{0.5em}{0.5em}}
\newenvironment{teachingnote}[1][Teaching note]{\noindent\textbf{#1.} }{\ \rule{0.5em}{0.5em}}
\fi
\let\sumnonlimits\sum
\let\prodnonlimits\prod
\let\cupnonlimits\bigcup
\let\capnonlimits\bigcap
\renewcommand{\sum}{\sumnonlimits\limits}
\renewcommand{\prod}{\prodnonlimits\limits}
\renewcommand{\bigcup}{\cupnonlimits\limits}
\renewcommand{\bigcap}{\capnonlimits\limits}
\setlength\tablinesep{3pt}
\setlength\arraylinesep{3pt}
\setlength\extrarulesep{3pt}
\voffset=0cm
\hoffset=-0.7cm
\setlength\textheight{22.5cm}
\setlength\textwidth{15.5cm}
\newcommand\arxiv[1]{\href{http://www.arxiv.org/abs/#1}{\texttt{arXiv:#1}}}
\newenvironment{verlong}{}{}
\newenvironment{vershort}{}{}
\newenvironment{noncompile}{}{}
\newenvironment{teachingnote}{}{}
\excludecomment{verlong}
\includecomment{vershort}
\excludecomment{noncompile}
\excludecomment{teachingnote}
\newcommand{\CC}{\mathbb{C}}
\newcommand{\RR}{\mathbb{R}}
\newcommand{\QQ}{\mathbb{Q}}
\newcommand{\NN}{\mathbb{N}}
\newcommand{\ZZ}{\mathbb{Z}}
\newcommand{\KK}{\mathbb{K}}
\newcommand{\id}{\operatorname{id}}
\newcommand{\lcm}{\operatorname{lcm}}
\newcommand{\rev}{\operatorname{rev}}
\newcommand{\powset}[2][]{\ifthenelse{\equal{#2}{}}{\mathcal{P}\left(#1\right)}{\mathcal{P}_{#1}\left(#2\right)}}
\newcommand{\set}[1]{\left\{ #1 \right\}}
\newcommand{\abs}[1]{\left| #1 \right|}
\newcommand{\tup}[1]{\left( #1 \right)}
\newcommand{\ive}[1]{\left[ #1 \right]}
\newcommand{\floor}[1]{\left\lfloor #1 \right\rfloor}
\newcommand{\lf}[2]{#1^{\underline{#2}}}
\newcommand{\underbrack}[2]{\underbrace{#1}_{\substack{#2}}}
\newcommand{\horrule}[1]{\rule{\linewidth}{#1}}
\newcommand{\are}{\ar@{-}}
\newcommand{\nnn}{\nonumber\\}
\newcommand{\sslash}{\mathbin{/\mkern-6mu/}}
\newcommand{\numboxed}[2]{\underbrace{\boxed{#1}}_{\text{box } #2}}
\newcommand{\ig}[2]{\includegraphics[scale=#1]{#2.png}}
\newcommand{\UNFINISHED}{\begin{center} \Huge{\textbf{Unfinished material begins here.}} \end{center} }
\iffalse
\NOEXPAND{\today}{\today}
\NOEXPAND{\sslash}{\sslash}
\NOEXPAND{\numboxed}[2]{\numboxed}
\NOEXPAND{\UNFINISHED}{\UNFINISHED}
\fi
\ihead{Math 504 notes}
\ohead{page \thepage}
\cfoot{\today}
\begin{document}

\title{Math 504: Advanced Linear Algebra}
\author{Hugo Woerdeman, with edits by Darij Grinberg\thanks{Drexel University, Korman
Center, 15 S 33rd Street, Philadelphia PA, 19104, USA}}
\date{\today\ (unfinished!)}
\maketitle
\tableofcontents

\section*{Math 504 Lecture 6}

\section{Schur triangularization (cont'd)}

\subsection{Application: Cayley--Hamilton theorem}

Let us recall some properties of the characteristic polynomial of an $n\times
n$-matrix $A$:

\begin{definition}
Let $\mathbb{F}$ be a field. Let $A\in\mathbb{F}^{n\times n}$ be an $n\times
n$-matrix over $\mathbb{F}$.

The \textbf{characteristic polynomial} $p_{A}$ of $A$ is defined to be the
polynomial%
\[
\det\left(  tI_{n}-A\right)  \in\underbrace{\mathbb{F}\left[  t\right]
}_{\substack{\text{ring of all polynomials}\\\text{in the indeterminate
}t\\\text{with coefficients in }\mathbb{F}}}.
\]

\end{definition}

\begin{example}
Let $n=2$ and $A=\left(
\begin{array}
[c]{cc}%
a & b\\
c & d
\end{array}
\right)  $. Then,%
\begin{align*}
tI_{n}-A  & =tI_{2}-A=t\left(
\begin{array}
[c]{cc}%
1 & 0\\
0 & 1
\end{array}
\right)  -\left(
\begin{array}
[c]{cc}%
a & b\\
c & d
\end{array}
\right)  \\
& =\left(
\begin{array}
[c]{cc}%
t & 0\\
0 & t
\end{array}
\right)  -\left(
\begin{array}
[c]{cc}%
a & b\\
c & d
\end{array}
\right)  =\left(
\begin{array}
[c]{cc}%
t-a & -b\\
-c & t-d
\end{array}
\right)  ,
\end{align*}
so that%
\begin{align*}
p_{A}  & =\det\left(  tI_{n}-A\right)  =\det\left(
\begin{array}
[c]{cc}%
t-a & -b\\
-c & t-d
\end{array}
\right)  =\left(  t-a\right)  \left(  t-d\right)  -\left(  -b\right)  \left(
-c\right)  \\
& =t^{2}-\left(  a+d\right)  t+\left(  ad-bc\right)  .
\end{align*}

\end{example}

\begin{example}
Let $n=3$ and $A=\left(
\begin{array}
[c]{ccc}%
a & b & c\\
a^{\prime} & b^{\prime} & c^{\prime}\\
a^{\prime\prime} & b^{\prime\prime} & c^{\prime\prime}%
\end{array}
\right)  $. Then,%
\[
tI_{n}-A=tI_{3}-A=\left(
\begin{array}
[c]{ccc}%
t-a & -b & -c\\
-a^{\prime} & t-b^{\prime} & -c^{\prime}\\
-a^{\prime\prime} & -b^{\prime\prime} & t-c^{\prime\prime}%
\end{array}
\right)  ,
\]
so that%
\begin{align*}
p_{A}  & =\det\left(
\begin{array}
[c]{ccc}%
t-a & -b & -c\\
-a^{\prime} & t-b^{\prime} & -c^{\prime}\\
-a^{\prime\prime} & -b^{\prime\prime} & t-c^{\prime\prime}%
\end{array}
\right)  \\
& =\allowbreak t^{3}-\left(  a+b^{\prime}+c^{\prime\prime}\right)
t^{2}+\left(  ab^{\prime}-ba^{\prime}+ac^{\prime\prime}-ca^{\prime\prime
}+b^{\prime}c^{\prime\prime}-b^{\prime\prime}c^{\prime}\right)  \allowbreak
t\\
& \ \ \ \ \ \ \ \ \ \ -\left(  \allowbreak ab^{\prime}c^{\prime\prime
}-ab^{\prime\prime}c^{\prime}-ba^{\prime}c^{\prime\prime}+ba^{\prime\prime
}c^{\prime}+ca^{\prime}b^{\prime\prime}-\allowbreak ca^{\prime\prime}%
b^{\prime}\right)  .
\end{align*}

\end{example}

By the way, some authors define $p_{A}$ to be $\det\left(  A-tI_{n}\right)  $
instead of $\det\left(  tI_{n}-A\right)  $. This differs from our definition
only by a factor of $\left(  -1\right)  ^{n}$, so the difference is insignificant.

\begin{proposition}
[properties of the char. poly.]Let $\mathbb{F}$ be a field. Let $A\in
\mathbb{F}^{n\times n}$ be an $n\times n$-matrix over $\mathbb{F}$.

\textbf{(a)} The characteristic polynomial $p_{A}$ is a monic polynomial in
$t$ of degree $n$. (That is, its leading term is $t^{n}$.)

\textbf{(b)} The constant term of $p_{A}$ is $\left(  -1\right)  ^{n}\det A$.

\textbf{(c)} The $t^{n-1}$-coefficient of $p_{A}$ is $-\operatorname*{Tr}A$.
(Recall that $\operatorname*{Tr}A$ is defined to be the sum of all diagonal
entries of $A$; this is known as the \textbf{trace} of $A$.)
\end{proposition}

\begin{proof}
All of this should be more or less clear from the examples. Part \textbf{(b)}
follows from observing that the constant term of $p_{A}$ is $p_{A}\left(
0\right)  =\det\left(  0I_{n}-A\right)  =\det\left(  -A\right)  =\left(
-1\right)  ^{n}\det A$.

For details, I'll give references in the notes.
\end{proof}

\begin{theorem}
[Cayley--Hamilton theorem]Let $\mathbb{F}$ be a field. Let $A\in
\mathbb{F}^{n\times n}$ be an $n\times n$-matrix. Then,%
\[
p_{A}\left(  A\right)  =0.
\]
(The \textquotedblleft$0$\textquotedblright\ on the RHS is the zero matrix.)
\end{theorem}

\begin{example}
Let $n=2$ and $A=\left(
\begin{array}
[c]{cc}%
a & b\\
c & d
\end{array}
\right)  $. Then, as we know,%
\[
p_{A}=t^{2}-\left(  a+d\right)  t+\left(  ad-bc\right)  .
\]
Thus,%
\begin{align*}
p_{A}\left(  A\right)    & =A^{2}-\left(  a+d\right)  A+\left(  ad-bc\right)
I_{2}\\
& =\left(
\begin{array}
[c]{cc}%
a & b\\
c & d
\end{array}
\right)  ^{2}-\left(  a+d\right)  \left(
\begin{array}
[c]{cc}%
a & b\\
c & d
\end{array}
\right)  +\left(  ad-bc\right)  \left(
\begin{array}
[c]{cc}%
1 & 0\\
0 & 1
\end{array}
\right)  \\
& =\left(
\begin{array}
[c]{cc}%
0 & 0\\
0 & 0
\end{array}
\right)  =0.
\end{align*}

\end{example}

\begin{remark}
You cannot argue that $p_{A}\left(  A\right)  =\det\left(  AI_{n}-A\right)  $
\textquotedblleft by substituting $A$ for $t$ into $p_{A}=\det\left(
tI_{n}-A\right)  $\textquotedblright. Indeed, $tI_{n}-A$ is a matrix whose
entries are polynomials in $t$. If you substitute $A$ for $t$ into it, it will
become a matrix whose entries are matrices. First of all, it is not quite
clear how to take the determinant of such a matrix; second, this matrix is not
$AI_{n}-A$. For example, for $n=2$, plugging $A$ for $t$ in $tI_{n}-A$ gives%
\[
\left(
\begin{array}
[c]{cc}%
\left(
\begin{array}
[c]{cc}%
a & b\\
c & d
\end{array}
\right)  -a & -b\\
-c & \left(
\begin{array}
[c]{cc}%
a & b\\
c & d
\end{array}
\right)  -d
\end{array}
\right)  ,
\]
which doesn't quite look like $AI_{n}-A$ (which is the zero matrix). There
\textbf{is} a correct proof of the Cayley--Hamilton theorem along the lines of
\textquotedblleft substituting $A$ for $t$\textquotedblright, but it requires
a lot of work.
\end{remark}

There are various proofs of the Cayley--Hamilton theorem (I'll give references
in the notes). We will here only prove it for $\mathbb{F}=\mathbb{C}$:

\begin{proof}
[Proof of the Cayley--Hamilton theorem for $\mathbb{F}=\mathbb{C}$.] Assume
that $\mathbb{F}=\mathbb{C}$. The Schur triangularization theorem shows that
$A$ is unitarily similar to an upper-triangular matrix. Hence, $A$ is similar
to an upper-triangular matrix (because unitarily similar matrices always are
similar). In other words, there exists an invertible matrix $U$ and an
upper-triangular matrix $T$ such that $A=UTU^{-1}$. Consider these $U$ and $T$.

Now, let $\lambda_{1},\lambda_{2},\ldots,\lambda_{n}$ be the diagonal entries
of $T$. Then, by Proposition 2.3.4, these diagonal entries $\lambda
_{1},\lambda_{2},\ldots,\lambda_{n}$ are the eigenvalues of $A$ (with
algebraic multipliticies). Hence,%
\[
p_{A}=\left(  t-\lambda_{1}\right)  \left(  t-\lambda_{2}\right)
\cdots\left(  t-\lambda_{n}\right)
\]
(since $p_{A}$ is monic, and the roots of $p_{A}$ are precisely the
eigenvalues of $A$ with algebraic multiplicities).

Now, substituting $A$ for $t$ in the polynomial identity $p_{A}=\left(
t-\lambda_{1}\right)  \left(  t-\lambda_{2}\right)  \cdots\left(
t-\lambda_{n}\right)  $, we obtain
\[
p_{A}\left(  A\right)  =\left(  A-\lambda_{1}I_{n}\right)  \left(
A-\lambda_{2}I_{n}\right)  \cdots\left(  A-\lambda_{n}I_{n}\right)  .
\]
For each $i\in\left[  n\right]  $, we have%
\[
\underbrace{A}_{=UTU^{-1}}-\lambda_{i}\underbrace{I_{n}}_{=UU^{-1}}%
=UTU^{-1}-\lambda_{i}UU^{-1}=U\left(  T-\lambda_{i}I_{n}\right)  U^{-1}.
\]
Hence, the above equality becomes%
\begin{align*}
p_{A}\left(  A\right)    & =\left(  A-\lambda_{1}I_{n}\right)  \left(
A-\lambda_{2}I_{n}\right)  \cdots\left(  A-\lambda_{n}I_{n}\right)  \\
& =U\left(  T-\lambda_{1}I_{n}\right)  \underbrace{U^{-1}U}_{=I_{n}}\left(
T-\lambda_{2}I_{n}\right)  U^{-1}\cdots U\left(  T-\lambda_{n}I_{n}\right)
U^{-1}\\
& =U\left(  T-\lambda_{1}I_{n}\right)  \left(  T-\lambda_{2}I_{n}\right)
\cdots\left(  T-\lambda_{n}I_{n}\right)  U^{-1}.
\end{align*}
Thus, it suffices to show that%
\[
\left(  T-\lambda_{1}I_{n}\right)  \left(  T-\lambda_{2}I_{n}\right)
\cdots\left(  T-\lambda_{n}I_{n}\right)  =0.
\]


Let us show this on an example for $n=3$:%
\begin{align*}
& \ \ T=\left(
\begin{array}
[c]{ccc}%
\lambda_{1} & \ast & \ast\\
0 & \lambda_{2} & \ast\\
0 & 0 & \lambda_{3}%
\end{array}
\right)  \\
& \Longrightarrow\ \text{\ }T-\lambda_{1}I_{n}=\left(
\begin{array}
[c]{ccc}%
0 & \ast & \ast\\
0 & \lambda_{2}-\lambda_{1} & \ast\\
0 & 0 & \lambda_{3}-\lambda_{1}%
\end{array}
\right)  \\
& \Longrightarrow\ \ \left(  T-\lambda_{1}I_{n}\right)  \left(  T-\lambda
_{2}I_{n}\right)  =\left(
\begin{array}
[c]{ccc}%
0 & \ast & \ast\\
0 & \lambda_{2}-\lambda_{1} & \ast\\
0 & 0 & \lambda_{3}-\lambda_{1}%
\end{array}
\right)  \left(
\begin{array}
[c]{ccc}%
\lambda_{1}-\lambda_{2} & \ast & \ast\\
0 & 0 & \ast\\
0 & 0 & \lambda_{3}-\lambda_{1}%
\end{array}
\right)  \\
&
\ \ \ \ \ \ \ \ \ \ \ \ \ \ \ \ \ \ \ \ \ \ \ \ \ \ \ \ \ \ \ \ \ \ \ \ \ =\left(
%
\begin{array}
[c]{ccc}%
0 & 0 & \ast\\
0 & 0 & \ast\\
0 & 0 & \ast
\end{array}
\right)  \\
& \Longrightarrow\ \left(  T-\lambda_{1}I_{n}\right)  \left(  T-\lambda
_{2}I_{n}\right)  \left(  T-\lambda_{3}I_{n}\right)  =\left(
\begin{array}
[c]{ccc}%
0 & 0 & \ast\\
0 & 0 & \ast\\
0 & 0 & \ast
\end{array}
\right)  \left(
\begin{array}
[c]{ccc}%
\lambda_{1}-\lambda_{3} & \ast & \ast\\
0 & \lambda_{2}-\lambda_{3} & \ast\\
0 & 0 & 0
\end{array}
\right)  =\left(
\begin{array}
[c]{ccc}%
0 & 0 & 0\\
0 & 0 & 0\\
0 & 0 & 0
\end{array}
\right)  .
\end{align*}


The general proof follows the same pattern: Every time you add a new factor,
one more column of your matrix becomes $0$. Formally speaking, this means that
you are proving the following fact by induction on $j$:

For each $j\in\left\{  0,1,\ldots,n\right\}  $, the first $j$ columns of the
matrix%
\[
\left(  T-\lambda_{1}I_{n}\right)  \left(  T-\lambda_{2}I_{n}\right)
\cdots\left(  T-\lambda_{j}I_{n}\right)
\]
are $0$.

Once this is proved, we can apply this to $j=n$, and conclude that the first
$n$ columns of the matrix%
\[
\left(  T-\lambda_{1}I_{n}\right)  \left(  T-\lambda_{2}I_{n}\right)
\cdots\left(  T-\lambda_{n}I_{n}\right)
\]
are $0$. But this means that the whole matrix is $0$, qed.
\end{proof}

\subsection{Sylvester's equation}

\begin{definition}
Let $A\in\mathbb{C}^{n\times n}$. Then, the \textbf{spectrum} of $A$ is
defined to be the set of all eigenvalues of $A$. This spectrum is denoted by
$\sigma\left(  A\right)  $ (or by $\operatorname*{spec}A$).
\end{definition}

\begin{theorem}
Let $A$ be an $n\times n$-matrix, and let $B$ be an $m\times m$-matrix (both
with complex entries). Let $C$ be an $n\times m$-matrix. Then, the following
statements are equivalent:

\begin{itemize}
\item $\mathcal{U}$: There is a \textbf{unique} matrix $X\in\mathbb{C}%
^{n\times m}$ such that $AX-XB=C$.

\item $\mathcal{V}$: We have $\sigma\left(  A\right)  \cap\sigma\left(
B\right)  =\varnothing$.
\end{itemize}
\end{theorem}

\begin{example}
Let us take $n=1$ and $m=1$. In this case, $A$, $B$ and $C$ are $1\times
1$-matrices, so we can view them as scalars. Let us therefore write $a$, $b$
and $c$ for them. Then, the theorem says that the following statements are equivalent:

\begin{itemize}
\item $\mathcal{U}$: There is a \textbf{unique} complex number $x$ such that
$ax-xb=c$.

\item $\mathcal{V}$: We have $\left\{  a\right\}  \cap\left\{  b\right\}
=\varnothing$ (that is, $a\neq b$).
\end{itemize}

This is not surprising, because the equation $ax-xb=c$ has a unique solution
(namely, $x=\dfrac{c}{a-b}$) when $a\neq b$, and otherwise has either none or
infinitely many solution.
\end{example}

The equation $AX-XB=C$ in the Theorem is known as \textbf{Sylvester's
equation}. Because the $X$ is on different sides in $AX$ and in $XB$, it
cannot be factored out (matrices do not generally commute).

\begin{proof}
[Proof of the $\mathcal{V}\Longrightarrow\mathcal{U}$ part of the theorem.]
First, observe that the matrix space $\mathbb{C}^{n\times m}$ is itself a
$\mathbb{C}$-vector space of dimension $nm$.

Consider the map%
\begin{align*}
L:\mathbb{C}^{n\times m}  & \rightarrow\mathbb{C}^{n\times m},\\
X  & \mapsto AX-XB.
\end{align*}
This map $L$ is linear, because%
\begin{align*}
L\left(  \alpha X+\beta Y\right)    & =A\left(  \alpha X+\beta Y\right)
-\left(  \alpha X+\beta Y\right)  B\\
& =\alpha AX+\beta AY-\alpha XB-\beta YB\\
& =\alpha\left(  AX-XB\right)  +\beta\left(  AY-YB\right)  =\alpha L\left(
X\right)  +\beta L\left(  Y\right)  .
\end{align*}


Thus, $L$ is a linear map between two vector spaces that have the same
(finite) dimension. Hence, we have the following equivalence:
\begin{align*}
& \ \left(  L\text{ is surjective (= onto)}\right)  \\
& \Longleftrightarrow\ \left(  L\text{ is injective (= one-to-one)}\right)
\ \\
& \Longleftrightarrow\ \left(  L\text{ is bijective (= invertible)}\right)  .
\end{align*}


Now, statement $\mathcal{U}$ is saying that the matrix $C$ has a
\textbf{unique} preimage under $L$ (that is, there exists a unique
$X\in\mathbb{C}^{n\times m}$ such that $L\left(  X\right)  =C$). As we know
from general properties of linear maps, this is true whenever $L$ is
bijective, and false otherwise. So statement $\mathcal{U}$ is equivalent to
$L$ being bijective.

Now, let us prove that $\mathcal{V}\Longrightarrow\mathcal{U}$. To wit, we
will show that $L$ is \textbf{injective}. This will imply that $L$ is
bijective (by the above equivalence), and therefore statement $\mathcal{U}$
will follow.

In order to prove that a linear map is injective, it suffices to show that its
kernel (= nullspace) is $0$. So let $X\in\operatorname*{Ker}L$; we will show
that $X=0$.

From $X\in\operatorname*{Ker}L$, we get $L\left(  X\right)  =0$. Since
$L\left(  X\right)  =AX-XB$, this means that $AX-XB=0$. In other words,
$AX=XB$. Hence,%
\[
A^{2}X=A\underbrace{AX}_{=XB}=\underbrace{AX}_{=XB}B=XBB=XB^{2}.
\]
Similarly,%
\[
A^{3}X=XB^{3},\ \ \ \ \ \ \ \ \ \ A^{4}X=XB^{4},\ \ \ \ \ \ \ \ \ \ A^{5}%
X=XB^{5},\ \ \ \ \ \ \ \ \ \ \ldots.
\]
That is,%
\[
A^{k}X=XB^{k}\ \ \ \ \ \ \ \ \ \ \text{for each }k\in\mathbb{N}.
\]
(Strictly speaking, this is proved by induction on $k$.)

Therefore, I claim that%
\[
f\left(  A\right)  X=Xf\left(  B\right)  \ \ \ \ \ \ \ \ \ \ \text{for any
polynomial }f\in\mathbb{C}\left[  t\right]  .
\]
(Indeed, if we write the polynomial $f$ as $f=\sum\limits_{k=0}^{m}f_{k}t^{k}$
with $f_{k}\in\mathbb{C}$, then%
\[
f\left(  A\right)  X=\sum\limits_{k=0}^{m}f_{k}\underbrace{A^{k}X}_{=XB^{k}%
}=\sum\limits_{k=0}^{m}f_{k}XB^{k}=X\underbrace{\sum\limits_{k=0}^{m}%
f_{k}B^{k}}_{=f\left(  B\right)  }=Xf\left(  B\right)  ,
\]
as desired.)

Apply this claim to $f=p_{A}$. We obtain%
\[
p_{A}\left(  A\right)  X=Xp_{A}\left(  B\right)  =X\left(  B-\lambda_{1}%
I_{n}\right)  \left(  B-\lambda_{2}I_{n}\right)  \cdots\left(  B-\lambda
_{n}I_{n}\right)  ,
\]
where $\lambda_{1},\lambda_{2},\ldots,\lambda_{n}$ are the eigenvalues of $A$
(with algebraic multiplicities), because%
\[
p_{A}=\left(  t-\lambda_{1}\right)  \left(  t-\lambda_{2}\right)
\cdots\left(  t-\lambda_{n}\right)  .
\]
Thus,%
\[
X\left(  B-\lambda_{1}I_{n}\right)  \left(  B-\lambda_{2}I_{n}\right)
\cdots\left(  B-\lambda_{n}I_{n}\right)  =\underbrace{p_{A}\left(  A\right)
}_{\substack{=0\\\text{(by Cayley--Hamilton)}}}X=0.
\]
We want to prove that $X=0$. This would follow from this equation if we knew
that the factors
\[
B-\lambda_{1}I_{n},\ B-\lambda_{2}I_{n},\ \ldots,\ B-\lambda_{n}I_{n}%
\]
are invertible (because then we can cancel these factors). However, they are
indeed invertible, because each $\lambda_{i}$ is an eigenvalue of $A$ and
therefore \textbf{not} an eigenvalue of $B$ (since $\sigma\left(  A\right)
\cap\sigma\left(  B\right)  =\varnothing$). This completes the proof of
$\mathcal{V}\Longrightarrow\mathcal{U}$.
\end{proof}

Maybe $\mathcal{U}\Longrightarrow\mathcal{V}$ will be homework. Also a nice
exercise(?):%
\[
\sigma\left(  L\right)  =\sigma\left(  A\right)  -\sigma\left(  B\right)
=\left\{  \lambda-\mu\ \mid\ \lambda\in\sigma\left(  A\right)  \text{ and }%
\mu\in\sigma\left(  B\right)  \right\}  .
\]



\end{document}