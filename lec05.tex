\documentclass[numbers=enddot,12pt,final,onecolumn,notitlepage]{scrartcl}%
\usepackage[headsepline,footsepline,manualmark]{scrlayer-scrpage}
\usepackage[all,cmtip]{xy}
\usepackage{amssymb}
\usepackage{amsmath}
\usepackage{amsthm}
\usepackage{framed}
\usepackage{comment}
\usepackage{color}
\usepackage{hyperref}
\usepackage[sc]{mathpazo}
\usepackage[T1]{fontenc}
\usepackage{tikz}
\usepackage{needspace}
\usepackage{tabls}
\usepackage{wasysym}
\usepackage{easytable}
\usepackage{pythonhighlight}
%TCIDATA{OutputFilter=latex2.dll}
%TCIDATA{Version=5.50.0.2960}
%TCIDATA{LastRevised=Wednesday, September 29, 2021 11:47:28}
%TCIDATA{SuppressPackageManagement}
%TCIDATA{<META NAME="GraphicsSave" CONTENT="32">}
%TCIDATA{<META NAME="SaveForMode" CONTENT="1">}
%TCIDATA{BibliographyScheme=Manual}
%TCIDATA{Language=American English}
%BeginMSIPreambleData
\providecommand{\U}[1]{\protect\rule{.1in}{.1in}}
%EndMSIPreambleData
\usetikzlibrary{arrows.meta}
\usetikzlibrary{chains}
\newcounter{exer}
\newcounter{exera}
\numberwithin{exer}{subsection}
\theoremstyle{definition}
\newtheorem{theo}{Theorem}[subsection]
\newenvironment{theorem}[1][]
{\begin{theo}[#1]\begin{leftbar}}
{\end{leftbar}\end{theo}}
\newtheorem{lem}[theo]{Lemma}
\newenvironment{lemma}[1][]
{\begin{lem}[#1]\begin{leftbar}}
{\end{leftbar}\end{lem}}
\newtheorem{prop}[theo]{Proposition}
\newenvironment{proposition}[1][]
{\begin{prop}[#1]\begin{leftbar}}
{\end{leftbar}\end{prop}}
\newtheorem{defi}[theo]{Definition}
\newenvironment{definition}[1][]
{\begin{defi}[#1]\begin{leftbar}}
{\end{leftbar}\end{defi}}
\newtheorem{remk}[theo]{Remark}
\newenvironment{remark}[1][]
{\begin{remk}[#1]\begin{leftbar}}
{\end{leftbar}\end{remk}}
\newtheorem{coro}[theo]{Corollary}
\newenvironment{corollary}[1][]
{\begin{coro}[#1]\begin{leftbar}}
{\end{leftbar}\end{coro}}
\newtheorem{conv}[theo]{Convention}
\newenvironment{convention}[1][]
{\begin{conv}[#1]\begin{leftbar}}
{\end{leftbar}\end{conv}}
\newtheorem{quest}[theo]{Question}
\newenvironment{question}[1][]
{\begin{quest}[#1]\begin{leftbar}}
{\end{leftbar}\end{quest}}
\newtheorem{warn}[theo]{Warning}
\newenvironment{warning}[1][]
{\begin{warn}[#1]\begin{leftbar}}
{\end{leftbar}\end{warn}}
\newtheorem{conj}[theo]{Conjecture}
\newenvironment{conjecture}[1][]
{\begin{conj}[#1]\begin{leftbar}}
{\end{leftbar}\end{conj}}
\newtheorem{exam}[theo]{Example}
\newenvironment{example}[1][]
{\begin{exam}[#1]\begin{leftbar}}
{\end{leftbar}\end{exam}}
\newtheorem{exmp}[exer]{Exercise}
\newenvironment{exercise}[1][]
{\begin{exmp}[#1]\begin{leftbar}}
{\end{leftbar}\end{exmp}}
\newenvironment{statement}{\begin{quote}}{\end{quote}}
\newenvironment{fineprint}{\medskip \begin{small}}{\end{small} \medskip}
\iffalse
\newenvironment{proof}[1][Proof]{\noindent\textbf{#1.} }{\ \rule{0.5em}{0.5em}}
\newenvironment{question}[1][Question]{\noindent\textbf{#1.} }{\ \rule{0.5em}{0.5em}}
\newenvironment{warning}[1][Warning]{\noindent\textbf{#1.} }{\ \rule{0.5em}{0.5em}}
\newenvironment{teachingnote}[1][Teaching note]{\noindent\textbf{#1.} }{\ \rule{0.5em}{0.5em}}
\fi
\let\sumnonlimits\sum
\let\prodnonlimits\prod
\let\cupnonlimits\bigcup
\let\capnonlimits\bigcap
\renewcommand{\sum}{\sumnonlimits\limits}
\renewcommand{\prod}{\prodnonlimits\limits}
\renewcommand{\bigcup}{\cupnonlimits\limits}
\renewcommand{\bigcap}{\capnonlimits\limits}
\setlength\tablinesep{3pt}
\setlength\arraylinesep{3pt}
\setlength\extrarulesep{3pt}
\voffset=0cm
\hoffset=-0.7cm
\setlength\textheight{22.5cm}
\setlength\textwidth{15.5cm}
\newcommand\arxiv[1]{\href{http://www.arxiv.org/abs/#1}{\texttt{arXiv:#1}}}
\newenvironment{verlong}{}{}
\newenvironment{vershort}{}{}
\newenvironment{noncompile}{}{}
\newenvironment{teachingnote}{}{}
\excludecomment{verlong}
\includecomment{vershort}
\excludecomment{noncompile}
\excludecomment{teachingnote}
\newcommand{\CC}{\mathbb{C}}
\newcommand{\RR}{\mathbb{R}}
\newcommand{\QQ}{\mathbb{Q}}
\newcommand{\NN}{\mathbb{N}}
\newcommand{\ZZ}{\mathbb{Z}}
\newcommand{\KK}{\mathbb{K}}
\newcommand{\id}{\operatorname{id}}
\newcommand{\lcm}{\operatorname{lcm}}
\newcommand{\rev}{\operatorname{rev}}
\newcommand{\powset}[2][]{\ifthenelse{\equal{#2}{}}{\mathcal{P}\left(#1\right)}{\mathcal{P}_{#1}\left(#2\right)}}
\newcommand{\set}[1]{\left\{ #1 \right\}}
\newcommand{\abs}[1]{\left| #1 \right|}
\newcommand{\tup}[1]{\left( #1 \right)}
\newcommand{\ive}[1]{\left[ #1 \right]}
\newcommand{\floor}[1]{\left\lfloor #1 \right\rfloor}
\newcommand{\lf}[2]{#1^{\underline{#2}}}
\newcommand{\underbrack}[2]{\underbrace{#1}_{\substack{#2}}}
\newcommand{\horrule}[1]{\rule{\linewidth}{#1}}
\newcommand{\are}{\ar@{-}}
\newcommand{\nnn}{\nonumber\\}
\newcommand{\sslash}{\mathbin{/\mkern-6mu/}}
\newcommand{\numboxed}[2]{\underbrace{\boxed{#1}}_{\text{box } #2}}
\newcommand{\ig}[2]{\includegraphics[scale=#1]{#2.png}}
\newcommand{\UNFINISHED}{\begin{center} \Huge{\textbf{Unfinished material begins here.}} \end{center} }
\iffalse
\NOEXPAND{\today}{\today}
\NOEXPAND{\sslash}{\sslash}
\NOEXPAND{\numboxed}[2]{\numboxed}
\NOEXPAND{\UNFINISHED}{\UNFINISHED}
\fi
\ihead{Math 504 notes}
\ohead{page \thepage}
\cfoot{\today}
\begin{document}

\title{Math 504: Advanced Linear Algebra}
\author{Hugo Woerdeman, with edits by Darij Grinberg\thanks{Drexel University, Korman
Center, 15 S 33rd Street, Philadelphia PA, 19104, USA}}
\date{\today\ (unfinished!)}
\maketitle
\tableofcontents

\section*{Math 504 Lecture 5}

\section{Schur triangularization (cont'd)}

\subsection{Normal matrices}

\begin{definition}
A square matrix $A\in\mathbb{C}^{n\times n}$ is said to be \textbf{normal} if
$AA^{\ast}=A^{\ast}A$.
\end{definition}

In other words, a square matrix is normal if it commutes with its own
conjugate transpose. Here are some examples:

\begin{exam}
\textbf{(a)} Let $A=\left(
\begin{array}
[c]{cc}%
1 & -1\\
1 & 1
\end{array}
\right)  $. Then, $A^{\ast}=\left(
\begin{array}
[c]{cc}%
1 & 1\\
-1 & 1
\end{array}
\right)  $ and $AA^{\ast}=\left(
\begin{array}
[c]{cc}%
1 & -1\\
1 & 1
\end{array}
\right)  \left(
\begin{array}
[c]{cc}%
1 & 1\\
-1 & 1
\end{array}
\right)  =\allowbreak\left(
\begin{array}
[c]{cc}%
2 & 0\\
0 & 2
\end{array}
\right)  $ and $A^{\ast}A=\left(
\begin{array}
[c]{cc}%
1 & 1\\
-1 & 1
\end{array}
\right)  \left(
\begin{array}
[c]{cc}%
1 & -1\\
1 & 1
\end{array}
\right)  =\allowbreak\left(
\begin{array}
[c]{cc}%
2 & 0\\
0 & 2
\end{array}
\right)  $, so that $AA^{\ast}=A^{\ast}A$. Thus, $A$ is normal.

\textbf{(b)} Let $B=\left(
\begin{array}
[c]{cc}%
0 & i\\
0 & 0
\end{array}
\right)  $. Then, $B^{\ast}=\left(
\begin{array}
[c]{cc}%
0 & 0\\
-i & 0
\end{array}
\right)  $ and $BB^{\ast}=\left(
\begin{array}
[c]{cc}%
0 & i\\
0 & 0
\end{array}
\right)  \left(
\begin{array}
[c]{cc}%
0 & 0\\
-i & 0
\end{array}
\right)  =\allowbreak\left(
\begin{array}
[c]{cc}%
1 & 0\\
0 & 0
\end{array}
\right)  $ and $B^{\ast}B=\left(
\begin{array}
[c]{cc}%
0 & 0\\
-i & 0
\end{array}
\right)  \left(
\begin{array}
[c]{cc}%
0 & i\\
0 & 0
\end{array}
\right)  =\allowbreak\left(
\begin{array}
[c]{cc}%
0 & 0\\
0 & 1
\end{array}
\right)  $, so that $BB^{\ast}\neq B^{\ast}B$. Thus, $B$ is not normal.

\textbf{(c)} For any $a,b\in\mathbb{C}$, the matrix $\left(
\begin{array}
[c]{cc}%
a & b\\
b & a
\end{array}
\right)  $ is normal.
\end{exam}

The class of normal matrices includes several known classes. Recall:

\begin{itemize}
\item A square matrix $A\in\mathbb{C}^{n\times n}$ is unitary if and only if
$AA^{\ast}=A^{\ast}A=I_{n}$.

\item A square matrix $A\in\mathbb{C}^{n\times n}$ is \textbf{Hermitian} if
and only if $A^{\ast}=A$.

\item A square matrix $A\in\mathbb{C}^{n\times n}$ is \textbf{skew-Hermitian}
if and only if $A^{\ast}=-A$.
\end{itemize}

\begin{proposition}
\textbf{(a)} Every Hermitian matrix is normal.

\textbf{(b)} Every skew-Hermitian matrix is normal.

\textbf{(c)} Every unitary matrix is normal.

\textbf{(d)} Every diagonal matrix is normal.
\end{proposition}

\begin{proof}
\textbf{(a)} If $A$ is Hermitian, then $A^{\ast}=A$, so that%
\[
A\underbrace{A^{\ast}}_{=A}=\underbrace{A}_{=A^{\ast}}A=A^{\ast}A,
\]
so that $A$ is normal.

\textbf{(b)} Similar.

\textbf{(c)} Trivial.

\textbf{(d)} LTTR.
\end{proof}

Unlike the unitary matrices, the normal matrices are not closed under
multiplication (i.e., $A$ and $B$ can be unitary without $AB$ being unitary).

Here are two more ways to construct normal matrices out of existing normal matrices:

\begin{proposition}
Let $A\in\mathbb{C}^{n\times n}$ be a normal matrix.

\textbf{(a)} If $\lambda\in\mathbb{C}$ is arbitrary, then $\lambda I_{n}+A$ is normal.

\textbf{(b)} If $U\in\mathbb{C}^{n\times n}$ is a unitary matrix, then the
matrix $UAU^{\ast}$ is normal.
\end{proposition}

\begin{proof}
We have $AA^{\ast}=A^{\ast}A$ (since $A$ is normal).

\textbf{(a)} Let $\lambda\in\mathbb{C}$. Then,%
\[
\left(  \lambda I_{n}+A\right)  ^{\ast}=\left(  \lambda I_{n}\right)  ^{\ast
}+A^{\ast}=\overline{\lambda}I_{n}+A^{\ast}.
\]
Hence,%
\[
\left(  \lambda I_{n}+A\right)  \left(  \lambda I_{n}+A\right)  ^{\ast
}=\left(  \lambda I_{n}+A\right)  \left(  \overline{\lambda}I_{n}+A^{\ast
}\right)  =\lambda\overline{\lambda}I_{n}+\lambda A^{\ast}+\overline{\lambda
}A+AA^{\ast}%
\]
and similarly%
\[
\left(  \lambda I_{n}+A\right)  ^{\ast}\left(  \lambda I_{n}+A\right)
=\overline{\lambda}\lambda I_{n}+\lambda A^{\ast}+\overline{\lambda}A+A^{\ast
}A.
\]
The right hand sides are equal, since $\lambda\overline{\lambda}%
=\overline{\lambda}\lambda$ and $AA^{\ast}=A^{\ast}A$. Thus, the left hand
sides are equal, too. In other words,%
\[
\left(  \lambda I_{n}+A\right)  \left(  \lambda I_{n}+A\right)  ^{\ast
}=\left(  \lambda I_{n}+A\right)  ^{\ast}\left(  \lambda I_{n}+A\right)  .
\]
So $\lambda I_{n}+A$ is normal.

\textbf{(b)} Let $U$ be a unitary matrix. Then, $U^{\ast}U=UU^{\ast}=I_{n}$.
Now,%
\[
\left(  UAU^{\ast}\right)  ^{\ast}=\underbrace{\left(  U^{\ast}\right)
^{\ast}}_{=U}A^{\ast}U^{\ast}=UA^{\ast}U^{\ast}.
\]
Thus,%
\[
\left(  UAU^{\ast}\right)  \left(  UAU^{\ast}\right)  ^{\ast}=\left(
UAU^{\ast}\right)  \left(  UA^{\ast}U^{\ast}\right)  =UA\underbrace{U^{\ast}%
U}_{=I_{n}}A^{\ast}U^{\ast}=UAA^{\ast}U^{\ast}.
\]
Similarly,%
\[
\left(  UAU^{\ast}\right)  ^{\ast}\left(  UAU^{\ast}\right)  =UA^{\ast
}AU^{\ast}.
\]
Again, the right hand sides are equal, since $AA^{\ast}=A^{\ast}A$. So the
left hand sides are equal, and this shows that $UAU^{\ast}$ is normal.
\end{proof}

We will now show the following:

\begin{lemma}
Let $T\in\mathbb{C}^{n\times n}$ be a triangular matrix. Then, $T$ is normal
if and only if $T$ is diagonal.
\end{lemma}

\begin{proof}
$\Longleftarrow:$ If $T$ is diagonal, then $T$ is normal, as we have already seen.

$\Longrightarrow:$ Assume that $T$ is normal. We must show that $T$ is diagonal.

WLOG assume that $T$ is upper-triangular (since the other case is analogous). 

Write $T$ in the form%
\[
T=\left(
\begin{array}
[c]{cccc}%
t_{1,1} & t_{1,2} & \cdots & t_{1,n}\\
& t_{2,2} & \cdots & t_{2,n}\\
&  & \ddots & \vdots\\
&  &  & t_{n,n}%
\end{array}
\right)  ,
\]
where the invisible entries are $0$'s. (We can do this, since $T$ is
upper-triangular.) Thus,%
\[
T^{\ast}=\left(
\begin{array}
[c]{cccc}%
\overline{t_{1,1}} &  &  & \\
\overline{t_{1,2}} & \overline{t_{2,2}} &  & \\
\vdots & \vdots & \ddots & \\
\overline{t_{1,n}} & \overline{t_{2,n}} & \cdots & \overline{t_{n,n}}%
\end{array}
\right)  .
\]


Since $T$ is normal, we have $TT^{\ast}=T^{\ast}T$. Now let's look at the
entries of this matrix. We have%
\begin{align*}
\left(  TT^{\ast}\right)  _{1,1}  & =t_{1,1}\overline{t_{1,1}}+t_{1,2}%
\overline{t_{1,2}}+\cdots+t_{1,n}\overline{t_{1,n}}=\left\vert t_{1,1}%
\right\vert ^{2}+\left\vert t_{1,2}\right\vert ^{2}+\cdots+\left\vert
t_{1,n}\right\vert ^{2},\ \ \ \ \ \ \ \ \ \ \text{but}\\
\left(  T^{\ast}T\right)  _{1,1}  & =\overline{t_{1,1}}t_{1,1}=\left\vert
t_{1,1}\right\vert ^{2}.
\end{align*}
However, the left hand sides of these must be equal, since $TT^{\ast}=T^{\ast
}T$. Thus, the right hand sides are equal too. That is,%
\[
\left\vert t_{1,1}\right\vert ^{2}+\left\vert t_{1,2}\right\vert ^{2}%
+\cdots+\left\vert t_{1,n}\right\vert ^{2}=\left\vert t_{1,1}\right\vert ^{2}.
\]
Thus, $\left\vert t_{1,2}\right\vert ^{2}+\cdots+\left\vert t_{1,n}\right\vert
^{2}=0$, so that $t_{1,2}=t_{1,3}=\cdots=t_{1,n}=0$ (because $\left\vert
t_{1,i}\right\vert ^{2}\geq0$ for all $i$).

We continue with the $2,2$-entries:%
\begin{align*}
\left(  TT^{\ast}\right)  _{2,2}  & =t_{2,2}\overline{t_{2,2}}+t_{2,3}%
\overline{t_{2,3}}+\cdots+t_{2,n}\overline{t_{2,n}}=\left\vert t_{2,2}%
\right\vert ^{2}+\left\vert t_{2,3}\right\vert ^{2}+\cdots+\left\vert
t_{2,n}\right\vert ^{2}\ \ \ \ \ \ \ \ \ \ \text{and}\\
\left(  T^{\ast}T\right)  _{2,2}  & =\overline{t_{1,2}}t_{1,2}+\overline
{t_{2,2}}t_{2,2}=\underbrace{\left\vert t_{1,2}\right\vert ^{2}}%
_{\substack{=0\\\text{(since }t_{1,2}=0\text{)}}}+\left\vert t_{2,2}%
\right\vert ^{2}=\left\vert t_{2,2}\right\vert ^{2}.
\end{align*}
Comparing these equalities, we get%
\[
\left\vert t_{2,2}\right\vert ^{2}+\left\vert t_{2,3}\right\vert ^{2}%
+\cdots+\left\vert t_{2,n}\right\vert ^{2}=\left\vert t_{2,2}\right\vert ^{2}.
\]
As before, this lets us conclude that $t_{2,3}=t_{2,4}=\cdots=t_{2,n}=0$.

Keep going like this to prove that
\[
t_{i,j}=0\ \ \ \ \ \ \ \ \ \ \text{for all }i<j.
\]
Strictly speaking, this is a strong induction on $i$. In the induction step,
use the fact that $t_{k,i}=0$ for all $k<i$ (this follows from the induction hypothesis).

This shows that $T$ is a diagonal matrix, qed.
\end{proof}

\subsection{The spectral theorem}

The spectral theorem provides an answer to the question \textquotedblleft what
are normal matrices \textbf{really}?\textquotedblright.

\begin{theorem}
[spectral theorem for normal matrices]Let $A\in\mathbb{C}^{n\times n}$ be a
normal matrix. Then:

\textbf{(a)} There exists a unitary matrix $U\in\operatorname*{U}%
\nolimits_{n}\left(  \mathbb{C}\right)  $ and a diagonal matrix $D\in
\mathbb{C}^{n\times n}$ such that
\[
A=UDU^{\ast}.
\]
In other words, $A$ is unitarily similar to a diagonal matrix.

\textbf{(b)} Let $U\in\operatorname*{U}\nolimits_{n}\left(  \mathbb{C}\right)
$ be a unitary matrix and $D\in\mathbb{C}^{n\times n}$ be a diagonal matrix
such that $A=UDU^{\ast}$. Then, the diagonal entries of $D$ are the
eigenvalues of $A$. Moreover, the columns of $U$ are eigenvectors of $A$.
Thus, there exists an orthonormal basis of $\mathbb{C}^{n}$ consisting of
eigenvectors of $A$.
\end{theorem}

\begin{proof}
\textbf{(a)} The Schur triangularization theorem tells us that we can write
$A$ in the form%
\[
A=UTU^{\ast}%
\]
for a unitary matrix $U$ and an upper-triangular matrix $T$. Consider these
$U$ and $T$. If we can show that $T$ is diagonal, then we are done.

From $A=UTU^{\ast}$, we see that $A$ is unitarily equivalent to $T$. Thus, $T$
is unitarily equivalent to $A$. Hence, our proposition above yields that $T$
is normal, since $A$ is normal. Now, the Lemma we just proved yields that $T$
is diagonal (because $T$ is triangular and normal). So part \textbf{(a)} is proven.

\textbf{(b)} The matrix $D$ is unitarily similar to $A$ (since $A=UDU^{\ast}%
$), thus similar to $A$. Hence, $D$ has the same eigenvalues as $A$. However,
$D$ is a diagonal matrix, so its eigenvalues are its diagonal entries. So the
diagonal entries of $D$ must be the eigenvalues of $A$.

What about $U$ ? The columns of $U$ are $Ue_{1},Ue_{2},\ldots,Ue_{n}$, where
$\left(  e_{1},e_{2},\ldots,e_{n}\right)  $ is the standard basis of
$\mathbb{C}^{n}$.

[In general, for example, $\left(
\begin{array}
[c]{ccc}%
a & b & c\\
d & e & f\\
g & h & i
\end{array}
\right)  e_{2}=\left(
\begin{array}
[c]{ccc}%
a & b & c\\
d & e & f\\
g & h & i
\end{array}
\right)  \left(
\begin{array}
[c]{c}%
0\\
1\\
0
\end{array}
\right)  =\left(
\begin{array}
[c]{c}%
b\\
e\\
h
\end{array}
\right)  $.]

Now, I claim that each $Ue_{i}$ is an eigenvector of $A$. Indeed,
\[
\underbrace{A}_{=UDU^{\ast}}\cdot Ue_{i}=UD\underbrace{U^{\ast}U}_{=I_{n}%
}e_{i}=U\underbrace{De_{i}}_{\substack{=\lambda e_{i}\\\text{where }%
\lambda\text{ is the }\left(  i,i\right)  \text{-th entry of }D\\\text{(since
}D\text{ is diagonal)}}}=U\cdot\lambda e_{i}=\lambda\cdot Ue_{i}.
\]


Thus, we conclude that the $n$ columns of $U$ are eigenvectors of $A$.

Since $U$ is unitary, these $n$ columns form an orthonormal basis of
$\mathbb{C}^{n}$. Thus, we have found an orthonormal basis of $\mathbb{C}^{n}$
that consists of eigenvectors of $A$. Qed.
\end{proof}

The decomposition $A=UDU^{\ast}$ in the spectral theorem (or, to be more
precise, the pair $\left(  U,D\right)  $) is called a \textbf{spectral
decomposition} of $A$.

Only normal matrices have a spectral decomposition:

\begin{corollary}
An $n\times n$-matrix $A\in\mathbb{C}^{n\times n}$ is normal if and only if it
is unitarily similar to a diagonal matrix.
\end{corollary}

\begin{proof}
$\Longrightarrow:$ This is just part \textbf{(a)} of the spectral theorem.

$\Longleftarrow:$ Assume that $A$ is unitarily similar to a diagonal matrix.
Thus, $A$ is unitarily similar to a normal matrix (since diagonal matrices are
normal), and thus itself normal. Qed.
\end{proof}

We can similarly characterize Hermitian matrices:

\begin{proposition}
Let $A\in\mathbb{C}^{n\times n}$ be a Hermitian matrix, and let $\left(
U,D\right)  $ be a spectral decomposition of $A$. Then, the diagonal entries
of $D$ are real.
\end{proposition}

\begin{proof}
We have $A=UDU^{\ast}$, thus $A^{\ast}=\left(  UDU^{\ast}\right)  ^{\ast
}=UD^{\ast}U^{\ast}$. However, since $A$ is Hermitian, we have $A^{\ast}=A$.
In other words, $UD^{\ast}U^{\ast}=UDU^{\ast}$. We can cancel $U$ and
$U^{\ast}$ from this equality (since $U$ is unitary), and get $D^{\ast}=D$.
Since $D$ is diagonal, this is simply saying that each diagonal entry of $D$
equals its own conjugate, i.e., is real.
\end{proof}

\begin{corollary}
An $n\times n$-matrix $A\in\mathbb{C}^{n\times n}$ is Hermitian if and only if
it is unitarily similar to a diagonal matrix with real entries.
\end{corollary}

\begin{proof}
$\Longrightarrow:$ Follows from the preceding proposition + the spectral theorem.

$\Longleftarrow:$ If $A$ is unitarily similar to a diagonal matrix with real
entries, then $A=UDU^{\ast}$ where $D$ is a diagonal matrix with real entries.
Thus, $A^{\ast}=UD^{\ast}U^{\ast}$. However, $D^{\ast}=D$. So it follows that
$A^{\ast}=A$, so $A$ is Hermitian.
\end{proof}

The corollary we just proved is the complex analogue of the classical
\textquotedblleft real spectral theorem\textquotedblright, which says that a
symmetric matrix $A\in\mathbb{R}^{n\times n}$ is similar to a diagonal matrix
with real entries via an orthogonal matrix with real entries.

Similarly, we can handle skew-Hermitian matrices:

\begin{proposition}
Let $A\in\mathbb{C}^{n\times n}$ be a skew-Hermitian matrix, and let $\left(
U,D\right)  $ be a spectral decomposition of $A$. Then, the diagonal entries
of $D$ are purely imaginary.
\end{proposition}

\begin{corollary}
An $n\times n$-matrix $A\in\mathbb{C}^{n\times n}$ is skew-Hermitian if and
only if it is unitarily similar to a diagonal matrix with purely imaginary entries.
\end{corollary}

Likewise, we can handle unitary matrices:

\begin{proposition}
Let $A\in\mathbb{C}^{n\times n}$ be a unitary matrix, and let $\left(
U,D\right)  $ be a spectral decomposition of $A$. Then, the diagonal entries
of $D$ are complex numbers with absolute value $1$.
\end{proposition}

\begin{corollary}
An $n\times n$-matrix $A\in\mathbb{C}^{n\times n}$ is unitary if and only if
it is unitarily similar to a diagonal matrix whose diagonal entries have
absolute value $1$.
\end{corollary}




\end{document}