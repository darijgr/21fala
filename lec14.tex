\documentclass[numbers=enddot,12pt,final,onecolumn,notitlepage]{scrartcl}%
\usepackage[headsepline,footsepline,manualmark]{scrlayer-scrpage}
\usepackage[all,cmtip]{xy}
\usepackage{amssymb}
\usepackage{amsmath}
\usepackage{amsthm}
\usepackage{framed}
\usepackage{comment}
\usepackage{color}
\usepackage{hyperref}
\usepackage[sc]{mathpazo}
\usepackage[T1]{fontenc}
\usepackage{tikz}
\usepackage{needspace}
\usepackage{tabls}
\usepackage{wasysym}
\usepackage{easytable}
\usepackage{pythonhighlight}
%TCIDATA{OutputFilter=latex2.dll}
%TCIDATA{Version=5.50.0.2960}
%TCIDATA{LastRevised=Friday, October 22, 2021 11:49:33}
%TCIDATA{SuppressPackageManagement}
%TCIDATA{<META NAME="GraphicsSave" CONTENT="32">}
%TCIDATA{<META NAME="SaveForMode" CONTENT="1">}
%TCIDATA{BibliographyScheme=Manual}
%TCIDATA{Language=American English}
%BeginMSIPreambleData
\providecommand{\U}[1]{\protect\rule{.1in}{.1in}}
%EndMSIPreambleData
\usetikzlibrary{arrows.meta}
\usetikzlibrary{chains}
\newcounter{exer}
\newcounter{exera}
\numberwithin{exer}{subsection}
\theoremstyle{definition}
\newtheorem{theo}{Theorem}[subsection]
\newenvironment{theorem}[1][]
{\begin{theo}[#1]\begin{leftbar}}
{\end{leftbar}\end{theo}}
\newtheorem{lem}[theo]{Lemma}
\newenvironment{lemma}[1][]
{\begin{lem}[#1]\begin{leftbar}}
{\end{leftbar}\end{lem}}
\newtheorem{prop}[theo]{Proposition}
\newenvironment{proposition}[1][]
{\begin{prop}[#1]\begin{leftbar}}
{\end{leftbar}\end{prop}}
\newtheorem{defi}[theo]{Definition}
\newenvironment{definition}[1][]
{\begin{defi}[#1]\begin{leftbar}}
{\end{leftbar}\end{defi}}
\newtheorem{remk}[theo]{Remark}
\newenvironment{remark}[1][]
{\begin{remk}[#1]\begin{leftbar}}
{\end{leftbar}\end{remk}}
\newtheorem{coro}[theo]{Corollary}
\newenvironment{corollary}[1][]
{\begin{coro}[#1]\begin{leftbar}}
{\end{leftbar}\end{coro}}
\newtheorem{conv}[theo]{Convention}
\newenvironment{convention}[1][]
{\begin{conv}[#1]\begin{leftbar}}
{\end{leftbar}\end{conv}}
\newtheorem{quest}[theo]{Question}
\newenvironment{question}[1][]
{\begin{quest}[#1]\begin{leftbar}}
{\end{leftbar}\end{quest}}
\newtheorem{warn}[theo]{Warning}
\newenvironment{warning}[1][]
{\begin{warn}[#1]\begin{leftbar}}
{\end{leftbar}\end{warn}}
\newtheorem{conj}[theo]{Conjecture}
\newenvironment{conjecture}[1][]
{\begin{conj}[#1]\begin{leftbar}}
{\end{leftbar}\end{conj}}
\newtheorem{exam}[theo]{Example}
\newenvironment{example}[1][]
{\begin{exam}[#1]\begin{leftbar}}
{\end{leftbar}\end{exam}}
\newtheorem{exmp}[exer]{Exercise}
\newenvironment{exercise}[1][]
{\begin{exmp}[#1]\begin{leftbar}}
{\end{leftbar}\end{exmp}}
\newenvironment{statement}{\begin{quote}}{\end{quote}}
\newenvironment{fineprint}{\medskip \begin{small}}{\end{small} \medskip}
\iffalse
\newenvironment{proof}[1][Proof]{\noindent\textbf{#1.} }{\ \rule{0.5em}{0.5em}}
\newenvironment{question}[1][Question]{\noindent\textbf{#1.} }{\ \rule{0.5em}{0.5em}}
\newenvironment{warning}[1][Warning]{\noindent\textbf{#1.} }{\ \rule{0.5em}{0.5em}}
\newenvironment{teachingnote}[1][Teaching note]{\noindent\textbf{#1.} }{\ \rule{0.5em}{0.5em}}
\fi
\let\sumnonlimits\sum
\let\prodnonlimits\prod
\let\cupnonlimits\bigcup
\let\capnonlimits\bigcap
\renewcommand{\sum}{\sumnonlimits\limits}
\renewcommand{\prod}{\prodnonlimits\limits}
\renewcommand{\bigcup}{\cupnonlimits\limits}
\renewcommand{\bigcap}{\capnonlimits\limits}
\setlength\tablinesep{3pt}
\setlength\arraylinesep{3pt}
\setlength\extrarulesep{3pt}
\voffset=0cm
\hoffset=-0.7cm
\setlength\textheight{22.5cm}
\setlength\textwidth{15.5cm}
\newcommand\arxiv[1]{\href{http://www.arxiv.org/abs/#1}{\texttt{arXiv:#1}}}
\newenvironment{verlong}{}{}
\newenvironment{vershort}{}{}
\newenvironment{noncompile}{}{}
\newenvironment{teachingnote}{}{}
\excludecomment{verlong}
\includecomment{vershort}
\excludecomment{noncompile}
\excludecomment{teachingnote}
\newcommand{\CC}{\mathbb{C}}
\newcommand{\RR}{\mathbb{R}}
\newcommand{\QQ}{\mathbb{Q}}
\newcommand{\NN}{\mathbb{N}}
\newcommand{\ZZ}{\mathbb{Z}}
\newcommand{\KK}{\mathbb{K}}
\newcommand{\id}{\operatorname{id}}
\newcommand{\lcm}{\operatorname{lcm}}
\newcommand{\rev}{\operatorname{rev}}
\newcommand{\powset}[2][]{\ifthenelse{\equal{#2}{}}{\mathcal{P}\left(#1\right)}{\mathcal{P}_{#1}\left(#2\right)}}
\newcommand{\set}[1]{\left\{ #1 \right\}}
\newcommand{\abs}[1]{\left| #1 \right|}
\newcommand{\tup}[1]{\left( #1 \right)}
\newcommand{\ive}[1]{\left[ #1 \right]}
\newcommand{\floor}[1]{\left\lfloor #1 \right\rfloor}
\newcommand{\lf}[2]{#1^{\underline{#2}}}
\newcommand{\underbrack}[2]{\underbrace{#1}_{\substack{#2}}}
\newcommand{\horrule}[1]{\rule{\linewidth}{#1}}
\newcommand{\are}{\ar@{-}}
\newcommand{\nnn}{\nonumber\\}
\newcommand{\sslash}{\mathbin{/\mkern-6mu/}}
\newcommand{\numboxed}[2]{\underbrace{\boxed{#1}}_{\text{box } #2}}
\newcommand{\ig}[2]{\includegraphics[scale=#1]{#2.png}}
\newcommand{\UNFINISHED}{\begin{center} \Huge{\textbf{Unfinished material begins here.}} \end{center} }
\iffalse
\NOEXPAND{\today}{\today}
\NOEXPAND{\sslash}{\sslash}
\NOEXPAND{\numboxed}[2]{\numboxed}
\NOEXPAND{\UNFINISHED}{\UNFINISHED}
\fi
\ihead{Math 504 notes}
\ohead{page \thepage}
\cfoot{\today}
\begin{document}

\title{Math 504: Advanced Linear Algebra}
\author{Hugo Woerdeman, with edits by Darij Grinberg\thanks{Drexel University, Korman
Center, 15 S 33rd Street, Philadelphia PA, 19104, USA}}
\date{\today\ (unfinished!)}
\maketitle
\tableofcontents

\section*{Math 504 Lecture 14}

\section{Jordan canonical (aka normal) form (cont'd)}

\subsection{The centralizer of a matrix}

Here is a fairly natural question: Which matrices commute with a given square
matrix $A$ ?

\begin{proposition}
Let $\mathbb{F}$ be a field. Let $A\in\mathbb{F}^{n\times n}$ be an $n\times
n$-matrix. Let $f$ and $g$ be two polynomials in a single variable $t$ over
$\mathbb{F}$. Then, $f\left(  A\right)  $ commutes with $g\left(  A\right)  $.
\end{proposition}

\begin{proof}
Write $f\left(  t\right)  $ as $f\left(  t\right)  =\sum_{i=0}^{n}f_{i}t^{i}$,
and write $g\left(  t\right)  $ as $g\left(  t\right)  =\sum_{j=0}^{m}%
g_{j}t^{j}$. Then,%
\[
f\left(  A\right)  =\sum_{i=0}^{n}f_{i}A^{i}\ \ \ \ \ \ \ \ \ \ \text{and}%
\ \ \ \ \ \ \ \ \ \ g\left(  A\right)  =\sum_{j=0}^{m}g_{j}A^{j}.
\]
Thus,%
\[
f\left(  A\right)  \cdot g\left(  A\right)  =\left(  \sum_{i=0}^{n}f_{i}%
A^{i}\right)  \cdot\left(  \sum_{j=0}^{m}g_{j}A^{j}\right)  =\sum_{i=0}%
^{n}\ \ \sum_{j=0}^{m}f_{i}g_{j}\underbrace{A^{i}A^{j}}_{=A^{i+j}}=\sum
_{i=0}^{n}\ \ \sum_{j=0}^{m}f_{i}g_{j}A^{i+j}.
\]
A similar computation shows that%
\[
g\left(  A\right)  \cdot f\left(  A\right)  =\sum_{i=0}^{n}\ \ \sum_{j=0}%
^{m}f_{i}g_{j}A^{i+j}.
\]
Comparing these two, we obtain $f\left(  A\right)  \cdot g\left(  A\right)
=g\left(  A\right)  \cdot f\left(  A\right)  $, qed.
\end{proof}

Thus, in particular, $f\left(  A\right)  $ commutes with $A$ for any
polynomial $f$ (because $A=g\left(  A\right)  $ for $g\left(  t\right)  =t$).

But are there other matrices that commute with $A$ ?

There certainly can be. For instance, if $A=\lambda I_{n}$ for some
$\lambda\in\mathbb{F}$, then \textbf{every} $n\times n$-matrix commutes with
$A$ (but very few matrices are of the form $f\left(  A\right)  $ for some
polynomial $f$). This is, in a sense, the \textquotedblleft best case
scentario\textquotedblright. Only for $A=\lambda I_{n}$ is it true that every
$n\times n$-matrix commutes with $A$.

Let us study the general case now.

\begin{definition}
Let $A\in\mathbb{F}^{n\times n}$ be an $n\times n$-matrix. The
\textbf{centralizer} of $A$ is defined to be the set of all $n\times
n$-matrices $B\in\mathbb{F}^{n\times n}$ such that $AB=BA$. We denote this set
by $\operatorname*{Cent}A$.
\end{definition}

We thus want to know what $\operatorname*{Cent}A$ is.

We begin with some general properties:

\begin{proposition}
Let $A\in\mathbb{F}^{n\times n}$ be an $n\times n$-matrix. Then,
$\operatorname*{Cent}A$ is a subset of $\mathbb{F}^{n\times n}$ that is closed
under addition, scaling and multiplication and contains $\lambda I_{n}$ for
all $\lambda\in\mathbb{F}$. In other words:

\textbf{(a)} For any $B,C\in\operatorname*{Cent}A$, we have $B+C\in
\operatorname*{Cent}A$.

\textbf{(b)} For any $B\in\operatorname*{Cent}A$ and $\lambda\in\mathbb{F}$,
we have $\lambda B\in\operatorname*{Cent}A$.

\textbf{(c)} For any $B,C\in\operatorname*{Cent}A$, we have $BC\in
\operatorname*{Cent}A$.

\textbf{(d)} For any $\lambda\in\mathbb{F}$, we have $\lambda I_{n}%
\in\operatorname*{Cent}A$.
\end{proposition}

This implies, in particular, that $\operatorname*{Cent}A$ is a vector subspace
of $\mathbb{F}^{n\times n}$. Furthermore, it shows that $\operatorname*{Cent}%
A$ is an $\mathbb{F}$-subalgebra of $\mathbb{F}^{n\times n}$ (in particular, a
subring of $\mathbb{F}^{n\times n}$).

\begin{proof}
[Proof of the Proposition.] Let me just show part \textbf{(c)}; the other
parts are even easier.

\textbf{(c)} Let $B,C\in\operatorname*{Cent}A$. Thus, $AB=BA$ and $AC=CA$.
Now,%
\[
\underbrace{AB}_{=BA}C=B\underbrace{AC}_{=CA}=BCA.
\]
This shows that $BC\in\operatorname*{Cent}A$. Thus, part \textbf{(c)} is proved.
\end{proof}

Now, as an example, let us compute $\operatorname*{Cent}A$ in the case when
$A$ is a single Jordan cell $J_{n}\left(  0\right)  $. So we fix an $n>0$, and
we set%
\[
A:=J_{n}\left(  0\right)  =\left(
\begin{array}
[c]{ccccc}%
0 & 1 & 0 & \cdots & 0\\
0 & 0 & 1 & \cdots & 0\\
0 & 0 & 0 & \cdots & 0\\
\vdots & \vdots & \vdots & \ddots & \vdots\\
0 & 0 & 0 & \cdots & 0
\end{array}
\right)  .
\]
Let $B\in\mathbb{F}^{n\times n}$ be arbitrary. We want to know when
$B\in\operatorname*{Cent}A$. In other words, we want to know when $AB=BA$.

We have
\begin{align*}
AB  & =\left(
\begin{array}
[c]{ccccc}%
0 & 1 & 0 & \cdots & 0\\
0 & 0 & 1 & \cdots & 0\\
0 & 0 & 0 & \cdots & 0\\
\vdots & \vdots & \vdots & \ddots & \vdots\\
0 & 0 & 0 & \cdots & 0
\end{array}
\right)  \left(
\begin{array}
[c]{ccccc}%
B_{1,1} & B_{1,2} & B_{1,3} & \cdots & B_{1,n}\\
B_{2,1} & B_{2,2} & B_{2,3} & \cdots & B_{2,n}\\
B_{3,1} & B_{3,2} & B_{3,3} & \cdots & B_{3,n}\\
\vdots & \vdots & \vdots & \ddots & \vdots\\
B_{n,1} & B_{n,2} & B_{n,3} & \cdots & B_{n,n}%
\end{array}
\right)  \\
& =\left(
\begin{array}
[c]{ccccc}%
B_{2,1} & B_{2,2} & B_{2,3} & \cdots & B_{2,n}\\
B_{3,1} & B_{3,2} & B_{3,3} & \cdots & B_{3,n}\\
\vdots & \vdots & \vdots & \ddots & \vdots\\
B_{n,1} & B_{n,2} & B_{n,3} & \cdots & B_{n,n}\\
0 & 0 & 0 & \cdots & 0
\end{array}
\right)
\end{align*}
and%
\begin{align*}
BA  & =\left(
\begin{array}
[c]{ccccc}%
B_{1,1} & B_{1,2} & B_{1,3} & \cdots & B_{1,n}\\
B_{2,1} & B_{2,2} & B_{2,3} & \cdots & B_{2,n}\\
B_{3,1} & B_{3,2} & B_{3,3} & \cdots & B_{3,n}\\
\vdots & \vdots & \vdots & \ddots & \vdots\\
B_{n,1} & B_{n,2} & B_{n,3} & \cdots & B_{n,n}%
\end{array}
\right)  \left(
\begin{array}
[c]{ccccc}%
0 & 1 & 0 & \cdots & 0\\
0 & 0 & 1 & \cdots & 0\\
0 & 0 & 0 & \cdots & 0\\
\vdots & \vdots & \vdots & \ddots & \vdots\\
0 & 0 & 0 & \cdots & 0
\end{array}
\right)  \\
& =\left(
\begin{array}
[c]{ccccc}%
0 & B_{1,1} & B_{1,2} & \cdots & B_{1,n-1}\\
0 & B_{2,1} & B_{2,2} & \cdots & B_{2,n-1}\\
0 & B_{3,1} & B_{3,2} & \cdots & B_{3,n-1}\\
\vdots & \vdots & \vdots & \ddots & \ddots\\
0 & B_{n,1} & B_{n,2} & \cdots & B_{n,n-1}%
\end{array}
\right)  .
\end{align*}
Thus, $AB=BA$ holds if and only if
\[
\left(
\begin{array}
[c]{ccccc}%
B_{2,1} & B_{2,2} & B_{2,3} & \cdots & B_{2,n}\\
B_{3,1} & B_{3,2} & B_{3,3} & \cdots & B_{3,n}\\
\vdots & \vdots & \vdots & \ddots & \vdots\\
B_{n,1} & B_{n,2} & B_{n,3} & \cdots & B_{n,n}\\
0 & 0 & 0 & \cdots & 0
\end{array}
\right)  =\left(
\begin{array}
[c]{ccccc}%
0 & B_{1,1} & B_{1,2} & \cdots & B_{1,n-1}\\
0 & B_{2,1} & B_{2,2} & \cdots & B_{2,n-1}\\
0 & B_{3,1} & B_{3,2} & \cdots & B_{3,n-1}\\
\vdots & \vdots & \vdots & \ddots & \ddots\\
0 & B_{n,1} & B_{n,2} & \cdots & B_{n,n-1}%
\end{array}
\right)  ,
\]
i.e., if%
\begin{align*}
B_{2,j}  & =B_{1,j-1}\ \ \ \ \ \ \ \ \ \ \text{for all }j\in\left[  n\right]
\text{ (where }B_{1,0}:=0\text{);}\\
B_{3,j}  & =B_{2,j-1}\ \ \ \ \ \ \ \ \ \ \text{for all }j\in\left[  n\right]
\text{ (where }B_{2,0}:=0\text{);}\\
B_{4,j}  & =B_{3,j-1}\ \ \ \ \ \ \ \ \ \ \text{for all }j\in\left[  n\right]
\text{ (where }B_{3,0}:=0\text{);}\\
& \ldots;\\
B_{n,j}  & =B_{n-1,j-1}\ \ \ \ \ \ \ \ \ \ \text{for all }j\in\left[
n\right]  \text{ (where }B_{n-1,0}:=0\text{);}\\
0  & =B_{n,j}\ \ \ \ \ \ \ \ \ \ \text{for all }j\in\left[  n-1\right]  .
\end{align*}
The latter system of equations can be restated as follows:%
\begin{align*}
& \ldots;\\
B_{n,n-2}  & =B_{n-1,n-3}=B_{n-2,n-4}=\cdots=B_{3,1}=0;\\
B_{n,n-1}  & =B_{n-1,n-2}=B_{n-2,n-3}=\cdots=B_{2,1}=0;\\
B_{n,n}  & =B_{n-1,n-1}=B_{n-2,n-2}=\cdots=B_{1,1};\\
B_{n-1,n}  & =B_{n-2,n-1}=B_{n-3,n-2}=\cdots=B_{1,2};\\
B_{n-2,n}  & =B_{n-3,n-1}=B_{n-4,n-2}=\cdots=B_{1,3};\\
& \ldots.
\end{align*}
In other words, it means that the matrix $B$ looks as follows:%
\[
B=\left(
\begin{array}
[c]{ccccc}%
b_{0} & b_{1} & b_{2} & \cdots & b_{n-1}\\
& b_{0} & b_{1} & \cdots & b_{n-2}\\
&  & b_{0} & \cdots & b_{n-3}\\
&  &  & \ddots & \vdots\\
&  &  &  & b_{0}%
\end{array}
\right)
\]
(where the empty cells have entries equal to $0$). This is called an
\textbf{upper-triangular Toeplitz matrix}. We can also rewrite it as%
\[
B=b_{0}I_{n}+b_{1}A+b_{2}A^{2}+\cdots+b_{n-1}A^{n-1}.
\]


So we have proved the following:

\begin{theorem}
Let $n>0$. Let $A=J_{n}\left(  0\right)  $. Then,
\begin{align*}
\operatorname*{Cent}A  & =\left\{  \left(
\begin{array}
[c]{ccccc}%
b_{0} & b_{1} & b_{2} & \cdots & b_{n-1}\\
& b_{0} & b_{1} & \cdots & b_{n-2}\\
&  & b_{0} & \cdots & b_{n-3}\\
&  &  & \ddots & \vdots\\
&  &  &  & b_{0}%
\end{array}
\right)  \ \mid\ b_{0},b_{1},\ldots,b_{n-1}\in\mathbb{F}\right\}  \\
& =\left\{  b_{0}I_{n}+b_{1}A+b_{2}A^{2}+\cdots+b_{n-1}A^{n-1}\ \mid
\ b_{0},b_{1},\ldots,b_{n-1}\in\mathbb{F}\right\}  \\
& =\left\{  f\left(  A\right)  \ \mid\ f\in\mathbb{F}\left[  t\right]  \text{
is a polynomial of degree }\leq n-1\right\}  .
\end{align*}

\end{theorem}

So this is the worst-case scenario: The only matrices commuting with $A$ are
the matrices of the form $f\left(  A\right)  $ (which, as we recall, must
always commute with $A$).

What happens for an arbitrary $A$ ? Is the answer closer to the best-case
scenario or to the worst-case scenario? The answer is that the worst-case
scenario holds for a randomly chosen matrix, but we can actually answer the
question \textquotedblleft what is $\operatorname*{Cent}A$
exactly\textquotedblright\ if we know the Jordan canonical form of $A$.

We start with simple propositions:

\begin{proposition}
Let $A\in\mathbb{F}^{n\times n}$ and $\lambda\in\mathbb{F}$. Then,
$\operatorname*{Cent}\left(  A-\lambda I_{n}\right)  =\operatorname*{Cent}A$.
\end{proposition}

\begin{proof}
Exercise (diff. [1]).
\end{proof}

\begin{proposition}
Let $A$, $B$ and $S$ be three $n\times n$-matrices such that $S$ is
invertible. Then,
\[
\left(  B\in\operatorname*{Cent}A\right)  \ \Longleftrightarrow\ \left(
SBS^{-1}\in\operatorname*{Cent}\left(  SAS^{-1}\right)  \right)  .
\]

\end{proposition}

\begin{proof}
Exercise (diff. [1]).
\end{proof}

Thus, if $A$ is a matrix with complex entries, and if we want to compute
$\operatorname*{Cent}A$, it suffices to compute $\operatorname*{Cent}J$, where
$J$ is the JCF of $A$.

Therefore, we now focus on centralizers of Jordan matrices.

\begin{proposition}
Let $A_{1},A_{2},\ldots,A_{k}$ be square matrices with complex entries. Assume
that the spectra of these matrices are disjoint -- i.e., if $i\neq j$, then
$\sigma\left(  A_{i}\right)  \cap\sigma\left(  A_{j}\right)  =\varnothing$.

Then,%
\begin{align*}
& \operatorname*{Cent}\left(
\begin{array}
[c]{cccc}%
A_{1} &  &  & \\
& A_{2} &  & \\
&  & \ddots & \\
&  &  & A_{k}%
\end{array}
\right)  \\
& =\left\{  \left(
\begin{array}
[c]{cccc}%
B_{1} &  &  & \\
& B_{2} &  & \\
&  & \ddots & \\
&  &  & B_{k}%
\end{array}
\right)  \ \mid\ B_{i}\in\operatorname*{Cent}\left(  A_{i}\right)  \text{ for
each }i\in\left[  k\right]  \right\}  .
\end{align*}

\end{proposition}

\begin{proof}
The $\supseteq$ inclusion is obvious. We thus need to prove the $\subseteq$
inclusion only.

Let $A_{i}$ be an $n_{i}\times n_{i}$-matrix for each $i\in\left[  k\right]  $.

Let $B\in\operatorname*{Cent}\left(
\begin{array}
[c]{cccc}%
A_{1} &  &  & \\
& A_{2} &  & \\
&  & \ddots & \\
&  &  & A_{k}%
\end{array}
\right)  $. We want to show that $B$ has the form $\left(
\begin{array}
[c]{cccc}%
B_{1} &  &  & \\
& B_{2} &  & \\
&  & \ddots & \\
&  &  & B_{k}%
\end{array}
\right)  $ where $B_{i}\in\operatorname*{Cent}\left(  A_{i}\right)  $ for each
$i\in\left[  k\right]  $.

Write $B$ as a block matrix%
\[
B=\left(
\begin{array}
[c]{cccc}%
B\left(  1,1\right)   & B\left(  1,2\right)   & \cdots & B\left(  1,k\right)
\\
B\left(  2,1\right)   & B\left(  2,2\right)   & \cdots & B\left(  2,k\right)
\\
\vdots & \vdots & \ddots & \vdots\\
B\left(  k,1\right)   & B\left(  k,2\right)   & \cdots & B\left(  k,k\right)
\end{array}
\right)  ,
\]
where each $B\left(  i,j\right)  $ is an $n_{i}\times n_{j}$-matrix. Then, by
the rule for multiplying block matrices, we have%
\begin{align*}
& \left(
\begin{array}
[c]{cccc}%
A_{1} &  &  & \\
& A_{2} &  & \\
&  & \ddots & \\
&  &  & A_{k}%
\end{array}
\right)  \left(
\begin{array}
[c]{cccc}%
B\left(  1,1\right)   & B\left(  1,2\right)   & \cdots & B\left(  1,k\right)
\\
B\left(  2,1\right)   & B\left(  2,2\right)   & \cdots & B\left(  2,k\right)
\\
\vdots & \vdots & \ddots & \vdots\\
B\left(  k,1\right)   & B\left(  k,2\right)   & \cdots & B\left(  k,k\right)
\end{array}
\right)  \\
& =\left(
\begin{array}
[c]{cccc}%
A_{1}B\left(  1,1\right)   & A_{1}B\left(  1,2\right)   & \cdots &
A_{1}B\left(  1,k\right)  \\
A_{2}B\left(  2,1\right)   & A_{2}B\left(  2,2\right)   & \cdots &
A_{2}B\left(  2,k\right)  \\
\vdots & \vdots & \ddots & \vdots\\
A_{k}B\left(  k,1\right)   & A_{k}B\left(  k,2\right)   & \cdots &
A_{k}B\left(  k,k\right)
\end{array}
\right)
\end{align*}
and%
\begin{align*}
& \left(
\begin{array}
[c]{cccc}%
B\left(  1,1\right)   & B\left(  1,2\right)   & \cdots & B\left(  1,k\right)
\\
B\left(  2,1\right)   & B\left(  2,2\right)   & \cdots & B\left(  2,k\right)
\\
\vdots & \vdots & \ddots & \vdots\\
B\left(  k,1\right)   & B\left(  k,2\right)   & \cdots & B\left(  k,k\right)
\end{array}
\right)  \left(
\begin{array}
[c]{cccc}%
A_{1} &  &  & \\
& A_{2} &  & \\
&  & \ddots & \\
&  &  & A_{k}%
\end{array}
\right)  \\
& =\left(
\begin{array}
[c]{cccc}%
B\left(  1,1\right)  A_{1} & B\left(  1,2\right)  A_{2} & \cdots & B\left(
1,k\right)  A_{k}\\
B\left(  2,1\right)  A_{1} & B\left(  2,2\right)  A_{2} & \cdots & B\left(
2,k\right)  A_{k}\\
\vdots & \vdots & \ddots & \vdots\\
B\left(  k,1\right)  A_{1} & B\left(  k,2\right)  A_{2} & \cdots & B\left(
k,k\right)  A_{k}%
\end{array}
\right)  .
\end{align*}
However, these two matrices must be equal, since $\left(
\begin{array}
[c]{cccc}%
B\left(  1,1\right)   & B\left(  1,2\right)   & \cdots & B\left(  1,k\right)
\\
B\left(  2,1\right)   & B\left(  2,2\right)   & \cdots & B\left(  2,k\right)
\\
\vdots & \vdots & \ddots & \vdots\\
B\left(  k,1\right)   & B\left(  k,2\right)   & \cdots & B\left(  k,k\right)
\end{array}
\right)  \in\operatorname*{Cent}\left(
\begin{array}
[c]{cccc}%
A_{1} &  &  & \\
& A_{2} &  & \\
&  & \ddots & \\
&  &  & A_{k}%
\end{array}
\right)  $. Thus, we have%
\[
\left(
\begin{array}
[c]{cccc}%
A_{1}B\left(  1,1\right)   & A_{1}B\left(  1,2\right)   & \cdots &
A_{1}B\left(  1,k\right)  \\
A_{2}B\left(  2,1\right)   & A_{2}B\left(  2,2\right)   & \cdots &
A_{2}B\left(  2,k\right)  \\
\vdots & \vdots & \ddots & \vdots\\
A_{k}B\left(  k,1\right)   & A_{k}B\left(  k,2\right)   & \cdots &
A_{k}B\left(  k,k\right)
\end{array}
\right)  =\left(
\begin{array}
[c]{cccc}%
B\left(  1,1\right)  A_{1} & B\left(  1,2\right)  A_{2} & \cdots & B\left(
1,k\right)  A_{k}\\
B\left(  2,1\right)  A_{1} & B\left(  2,2\right)  A_{2} & \cdots & B\left(
2,k\right)  A_{k}\\
\vdots & \vdots & \ddots & \vdots\\
B\left(  k,1\right)  A_{1} & B\left(  k,2\right)  A_{2} & \cdots & B\left(
k,k\right)  A_{k}%
\end{array}
\right)  .
\]
Comparing blocks, we can rewrite this as%
\[
A_{i}B\left(  i,j\right)  =B\left(  i,j\right)  A_{j}%
\ \ \ \ \ \ \ \ \ \ \text{for all }i,j\in\left[  k\right]  .
\]


Now, let $i,j\in\left[  k\right]  $ be distinct. Consider this equality
$A_{i}B\left(  i,j\right)  =B\left(  i,j\right)  A_{j}$. We can rewrite it as
$A_{i}B\left(  i,j\right)  -B\left(  i,j\right)  A_{j}=0$. Thus, $B\left(
i,j\right)  $ is an $n_{i}\times n_{j}$-matrix $X$ satisfying $A_{i}%
X-XA_{j}=0$. However, because $\sigma\left(  A_{i}\right)  \cap\sigma\left(
A_{j}\right)  =\varnothing$, a theorem we proved before (the Sylvester matrix
equation) tells us that there is a \textbf{unique} $n_{i}\times n_{j}$-matrix
$X$ satisfying $A_{i}X-XA_{j}=0$. Clearly, this unique matrix $X$ must be the
$0$ matrix (since the $0$ matrix satisfies $A_{i}0-0A_{j}=0$). So we conclude
that $B\left(  i,j\right)  $ is the $0$ matrix. In other words, $B\left(
i,j\right)  =0$.

So we have shown that $B\left(  i,j\right)  =0$ whenever $i$ and $j$ are
distinct. Thus,%
\[
B=\left(
\begin{array}
[c]{cccc}%
B\left(  1,1\right)   & B\left(  1,2\right)   & \cdots & B\left(  1,k\right)
\\
B\left(  2,1\right)   & B\left(  2,2\right)   & \cdots & B\left(  2,k\right)
\\
\vdots & \vdots & \ddots & \vdots\\
B\left(  k,1\right)   & B\left(  k,2\right)   & \cdots & B\left(  k,k\right)
\end{array}
\right)  =\left(
\begin{array}
[c]{cccc}%
B\left(  1,1\right)   &  &  & \\
& B\left(  2,2\right)   &  & \\
&  & \ddots & \\
&  &  & B\left(  k,k\right)
\end{array}
\right)  .
\]
This shows that $B$ is block-diagonal. Now, applying the equation%
\[
A_{i}B\left(  i,j\right)  =B\left(  i,j\right)  A_{j}%
\ \ \ \ \ \ \ \ \ \ \text{for all }i,j\in\left[  k\right]
\]
to $j=i$, we obtain $A_{i}B\left(  i,i\right)  =B\left(  i,i\right)  A_{i}$,
which of course means that $B\left(  i,i\right)  \in\operatorname*{Cent}%
\left(  A_{i}\right)  $. Thus, $B$ has the form $\left(
\begin{array}
[c]{cccc}%
B_{1} &  &  & \\
& B_{2} &  & \\
&  & \ddots & \\
&  &  & B_{k}%
\end{array}
\right)  $ where $B_{i}\in\operatorname*{Cent}\left(  A_{i}\right)  $ for each
$i\in\left[  k\right]  $. Proof complete.
\end{proof}

So we only need to compute $\operatorname*{Cent}J$ when $J$ is a Jordan matrix
with only one eigenvalue.

We can WLOG assume that this eigenvalue is $0$, since we know that
$\operatorname*{Cent}\left(  A-\lambda I_{n}\right)  =\operatorname*{Cent}A$.

So we only need to compute $\operatorname*{Cent}J$ when $J$ is a Jordan matrix
with zeroes on its diagonal.

If $J$ is just a single Jordan cell, we already know the result (by the above
theorem which describes $\operatorname*{Cent}A$ for $A=J_{n}\left(  0\right)
$). In the general case, we have the following:

\begin{proposition}
Let $J$ be a Jordan matrix whose Jordan blocks are%
\[
J_{n_{1}}\left(  0\right)  ,\ \ J_{n_{2}}\left(  0\right)  ,\ \ \ldots
,\ \ J_{n_{k}}\left(  0\right)  .
\]
Let $B$ be an $n\times n$-matrix, written as a block matrix%
\[
B=\left(
\begin{array}
[c]{cccc}%
B\left(  1,1\right)   & B\left(  1,2\right)   & \cdots & B\left(  1,k\right)
\\
B\left(  2,1\right)   & B\left(  2,2\right)   & \cdots & B\left(  2,k\right)
\\
\vdots & \vdots & \ddots & \vdots\\
B\left(  k,1\right)   & B\left(  k,2\right)   & \cdots & B\left(  k,k\right)
\end{array}
\right)  ,
\]
where each $B\left(  i,j\right)  $ is an $n_{i}\times n_{j}$-matrix. Then,
$B\in\operatorname*{Cent}J$ if and only if each of the $k^{2}$ blocks
$B\left(  i,j\right)  $ is an \textbf{upper-triangular Toeplitz matrix in the
wide sense}.

Here, we say that a matrix is an \textbf{upper-triangular Toeplitz matrix in
the wide sense} if it

\begin{itemize}
\item has the form $\left(
\begin{array}
[c]{cc}%
0 & U
\end{array}
\right)  $, where $U$ is an upper-triangular Toeplitz (square) matrix and $0$
is a zero matrix, or

\item has the form $\left(
\begin{array}
[c]{c}%
U\\
0
\end{array}
\right)  $, where $U$ is an upper-triangular Toeplitz (square) matrix and $0$
is a zero matrix.
\end{itemize}

(The zero matrices are allowed to be empty.)
\end{proposition}

\begin{proof}
Essentially the same argument that we used to prove the theorem about
$J_{n}\left(  0\right)  $, just with a lot more bookkeeping involved.
\end{proof}

We can summarize our results into a single theorem:

\begin{theorem}
Let $A\in\mathbb{C}^{n\times n}$ be an $n\times n$-matrix with Jordan
canonical form $J$. Then, $\operatorname*{Cent}A$ is a vector subspace of
$\mathbb{C}^{n\times n}$ with dimension%
\[
\sum_{\lambda\in\sigma\left(  A\right)  }g_{\lambda}\left(  A\right)  .
\]
Here, for each eigenvalue $\lambda$ of $A$, the number $g_{\lambda}\left(
A\right)  $ is a nonnegative integer defined as follows: Let $n_{1}%
,n_{2},\ldots,n_{k}$ be the sizes of the Jordan blocks at eigenvalue $\lambda$
that appear in $J$; then, we set%
\[
g_{\lambda}\left(  A\right)  :=\sum_{i=1}^{k}\ \ \sum_{j=1}^{k}\min\left\{
n_{i},n_{j}\right\}  .
\]

\end{theorem}

\begin{proof}
Combine our above results and count the degrees of freedom.
\end{proof}

Now, let us return to the worst-case scenario: When is $\operatorname*{Cent}%
A=\left\{  f\left(  A\right)  \ \mid\ f\in\mathbb{C}\left[  t\right]
\right\}  $ ? We can answer this, too, although the proof takes longer.

\begin{definition}
An $n\times n$-matrix $A\in\mathbb{F}^{n\times n}$ is said to be
\textbf{nonderogatory} if $q_{A}=p_{A}$ (that is, the minimal polynomial of
$A$ equals the characteristic polynomial of $A$).
\end{definition}

A randomly chosen matrix with complex entries will be nonderogatory with
probability $1$; but there are exceptions. It is easy to see that if a matrix
$A$ has $n$ distinct eigenvalues, then $A$ is nonderogatory, but this is not
an \textquotedblleft if and only if\textquotedblright; a single Jordan cell is
also nonderogatory.

\begin{proposition}
An $n\times n$-matrix $A\in\mathbb{C}^{n\times n}$ is nonderogatory if and
only if its Jordan canonical form has exactly one Jordan block for each eigenvalue.
\end{proposition}

\begin{proof}
HW (difficulty [2]).
\end{proof}

\begin{theorem}
Let $A\in\mathbb{C}^{n\times n}$ be an $n\times n$-matrix. Then,
\[
\operatorname*{Cent}A=\left\{  f\left(  A\right)  \ \mid\ f\in\mathbb{C}%
\left[  t\right]  \right\}
\]
if and only if $f$ is nonderogatory. Moreover, in this case,%
\[
\operatorname*{Cent}A=\left\{  f\left(  A\right)  \ \mid\ f\in\mathbb{C}%
\left[  t\right]  \text{ is a polynomial of degree }\leq n-1\right\}  .
\]

\end{theorem}

\begin{proof}
Later or exercises?
\end{proof}


\end{document}