\documentclass[numbers=enddot,12pt,final,onecolumn,notitlepage]{scrartcl}%
\usepackage[headsepline,footsepline,manualmark]{scrlayer-scrpage}
\usepackage[all,cmtip]{xy}
\usepackage{amssymb}
\usepackage{amsmath}
\usepackage{amsthm}
\usepackage{framed}
\usepackage{comment}
\usepackage{color}
\usepackage{hyperref}
\usepackage[sc]{mathpazo}
\usepackage[T1]{fontenc}
\usepackage{tikz}
\usepackage{needspace}
\usepackage{tabls}
\usepackage{wasysym}
\usepackage{easytable}
\usepackage{pythonhighlight}
%TCIDATA{OutputFilter=latex2.dll}
%TCIDATA{Version=5.50.0.2960}
%TCIDATA{LastRevised=Wednesday, October 20, 2021 11:50:12}
%TCIDATA{SuppressPackageManagement}
%TCIDATA{<META NAME="GraphicsSave" CONTENT="32">}
%TCIDATA{<META NAME="SaveForMode" CONTENT="1">}
%TCIDATA{BibliographyScheme=Manual}
%TCIDATA{Language=American English}
%BeginMSIPreambleData
\providecommand{\U}[1]{\protect\rule{.1in}{.1in}}
%EndMSIPreambleData
\usetikzlibrary{arrows.meta}
\usetikzlibrary{chains}
\newcounter{exer}
\newcounter{exera}
\numberwithin{exer}{subsection}
\theoremstyle{definition}
\newtheorem{theo}{Theorem}[subsection]
\newenvironment{theorem}[1][]
{\begin{theo}[#1]\begin{leftbar}}
{\end{leftbar}\end{theo}}
\newtheorem{lem}[theo]{Lemma}
\newenvironment{lemma}[1][]
{\begin{lem}[#1]\begin{leftbar}}
{\end{leftbar}\end{lem}}
\newtheorem{prop}[theo]{Proposition}
\newenvironment{proposition}[1][]
{\begin{prop}[#1]\begin{leftbar}}
{\end{leftbar}\end{prop}}
\newtheorem{defi}[theo]{Definition}
\newenvironment{definition}[1][]
{\begin{defi}[#1]\begin{leftbar}}
{\end{leftbar}\end{defi}}
\newtheorem{remk}[theo]{Remark}
\newenvironment{remark}[1][]
{\begin{remk}[#1]\begin{leftbar}}
{\end{leftbar}\end{remk}}
\newtheorem{coro}[theo]{Corollary}
\newenvironment{corollary}[1][]
{\begin{coro}[#1]\begin{leftbar}}
{\end{leftbar}\end{coro}}
\newtheorem{conv}[theo]{Convention}
\newenvironment{convention}[1][]
{\begin{conv}[#1]\begin{leftbar}}
{\end{leftbar}\end{conv}}
\newtheorem{quest}[theo]{Question}
\newenvironment{question}[1][]
{\begin{quest}[#1]\begin{leftbar}}
{\end{leftbar}\end{quest}}
\newtheorem{warn}[theo]{Warning}
\newenvironment{warning}[1][]
{\begin{warn}[#1]\begin{leftbar}}
{\end{leftbar}\end{warn}}
\newtheorem{conj}[theo]{Conjecture}
\newenvironment{conjecture}[1][]
{\begin{conj}[#1]\begin{leftbar}}
{\end{leftbar}\end{conj}}
\newtheorem{exam}[theo]{Example}
\newenvironment{example}[1][]
{\begin{exam}[#1]\begin{leftbar}}
{\end{leftbar}\end{exam}}
\newtheorem{exmp}[exer]{Exercise}
\newenvironment{exercise}[1][]
{\begin{exmp}[#1]\begin{leftbar}}
{\end{leftbar}\end{exmp}}
\newenvironment{statement}{\begin{quote}}{\end{quote}}
\newenvironment{fineprint}{\medskip \begin{small}}{\end{small} \medskip}
\iffalse
\newenvironment{proof}[1][Proof]{\noindent\textbf{#1.} }{\ \rule{0.5em}{0.5em}}
\newenvironment{question}[1][Question]{\noindent\textbf{#1.} }{\ \rule{0.5em}{0.5em}}
\newenvironment{warning}[1][Warning]{\noindent\textbf{#1.} }{\ \rule{0.5em}{0.5em}}
\newenvironment{teachingnote}[1][Teaching note]{\noindent\textbf{#1.} }{\ \rule{0.5em}{0.5em}}
\fi
\let\sumnonlimits\sum
\let\prodnonlimits\prod
\let\cupnonlimits\bigcup
\let\capnonlimits\bigcap
\renewcommand{\sum}{\sumnonlimits\limits}
\renewcommand{\prod}{\prodnonlimits\limits}
\renewcommand{\bigcup}{\cupnonlimits\limits}
\renewcommand{\bigcap}{\capnonlimits\limits}
\setlength\tablinesep{3pt}
\setlength\arraylinesep{3pt}
\setlength\extrarulesep{3pt}
\voffset=0cm
\hoffset=-0.7cm
\setlength\textheight{22.5cm}
\setlength\textwidth{15.5cm}
\newcommand\arxiv[1]{\href{http://www.arxiv.org/abs/#1}{\texttt{arXiv:#1}}}
\newenvironment{verlong}{}{}
\newenvironment{vershort}{}{}
\newenvironment{noncompile}{}{}
\newenvironment{teachingnote}{}{}
\excludecomment{verlong}
\includecomment{vershort}
\excludecomment{noncompile}
\excludecomment{teachingnote}
\newcommand{\CC}{\mathbb{C}}
\newcommand{\RR}{\mathbb{R}}
\newcommand{\QQ}{\mathbb{Q}}
\newcommand{\NN}{\mathbb{N}}
\newcommand{\ZZ}{\mathbb{Z}}
\newcommand{\KK}{\mathbb{K}}
\newcommand{\id}{\operatorname{id}}
\newcommand{\lcm}{\operatorname{lcm}}
\newcommand{\rev}{\operatorname{rev}}
\newcommand{\powset}[2][]{\ifthenelse{\equal{#2}{}}{\mathcal{P}\left(#1\right)}{\mathcal{P}_{#1}\left(#2\right)}}
\newcommand{\set}[1]{\left\{ #1 \right\}}
\newcommand{\abs}[1]{\left| #1 \right|}
\newcommand{\tup}[1]{\left( #1 \right)}
\newcommand{\ive}[1]{\left[ #1 \right]}
\newcommand{\floor}[1]{\left\lfloor #1 \right\rfloor}
\newcommand{\lf}[2]{#1^{\underline{#2}}}
\newcommand{\underbrack}[2]{\underbrace{#1}_{\substack{#2}}}
\newcommand{\horrule}[1]{\rule{\linewidth}{#1}}
\newcommand{\are}{\ar@{-}}
\newcommand{\nnn}{\nonumber\\}
\newcommand{\sslash}{\mathbin{/\mkern-6mu/}}
\newcommand{\numboxed}[2]{\underbrace{\boxed{#1}}_{\text{box } #2}}
\newcommand{\ig}[2]{\includegraphics[scale=#1]{#2.png}}
\newcommand{\UNFINISHED}{\begin{center} \Huge{\textbf{Unfinished material begins here.}} \end{center} }
\iffalse
\NOEXPAND{\today}{\today}
\NOEXPAND{\sslash}{\sslash}
\NOEXPAND{\numboxed}[2]{\numboxed}
\NOEXPAND{\UNFINISHED}{\UNFINISHED}
\fi
\ihead{Math 504 notes}
\ohead{page \thepage}
\cfoot{\today}
\begin{document}

\title{Math 504: Advanced Linear Algebra}
\author{Hugo Woerdeman, with edits by Darij Grinberg\thanks{Drexel University, Korman
Center, 15 S 33rd Street, Philadelphia PA, 19104, USA}}
\date{\today\ (unfinished!)}
\maketitle
\tableofcontents

\section*{Math 504 Lecture 13}

\section{Jordan canonical (aka normal) form (cont'd)}

\subsection{The companion matrix}

For each $n\times n$-matrix $A$, we have defined its characteristic polynomial
$p_{A}$ and its minimal polynomial $q_{A}$. What variety of polynomials do we
get this way? Do all characteristic polynomials share some property, or can
any monic polynomial be a characteristic polynomial? The same question for
minimal polynomials?

Poll:

\begin{enumerate}
\item Any monic polynomial can be a characteristic polynomial.

\item Characteristic polynomials have some special property among monic polynomials.
\end{enumerate}

The answer is 1.

\begin{definition}
Let $\mathbb{F}$ be a field, and let $n\in\mathbb{N}$.

Let $p\left(  t\right)  =t^{n}+p_{n-1}t^{n-1}+p_{n-2}t^{n-2}+\cdots+p_{1}%
t^{1}+p_{0}t^{0}$ be a monic polynomial of degree $n$ with coefficients in
$\mathbb{F}$. Then, the \textbf{companion matrix} of $p\left(  t\right)  $ is
defined to be the matrix%
\[
C_{p}:=\left(
\begin{array}
[c]{cccccc}%
0 &  &  & \cdots &  & -p_{0}\\
1 & 0 &  & \cdots &  & -p_{1}\\
& 1 & 0 & \cdots &  & -p_{2}\\
\vdots & \vdots & \vdots & \ddots & \vdots & \vdots\\
&  &  & \cdots & 0 & -p_{n-2}\\
&  &  & \cdots & 1 & -p_{n-1}%
\end{array}
\right)  \in\mathbb{F}^{n\times n}.
\]
This is the $n\times n$-matrix whose first $n-1$ columns are the standard
basis vectors $e_{2},e_{3},\ldots,e_{n}$, and whose last column is $\left(
-p_{0},-p_{1},\ldots,-p_{n-1}\right)  ^{T}$.
\end{definition}

\begin{proposition}
For any monic polynomial $p\left(  t\right)  $, we have%
\[
p_{C_{p}}\left(  t\right)  =q_{C_{p}}\left(  t\right)  =p\left(  t\right)  .
\]
(The two $p$'s in \textquotedblleft$p_{C_{p}}$\textquotedblright\ stand for
different things: The first stands for \textquotedblleft characteristic
polynomial\textquotedblright, while the second is the $p\left(  t\right)  $ we
have given.)
\end{proposition}

\begin{proof}
Let us first show that $p_{C_{p}}\left(  t\right)  =p\left(  t\right)  $. To
do so, we induct on $n$. Recall that%
\begin{align*}
p_{C_{p}}\left(  t\right)    & =\det\left(  tI_{n}-C_{p}\right)  \\
& =\det\left(
\begin{array}
[c]{cccccc}%
t &  &  & \cdots &  & p_{0}\\
-1 & t &  & \cdots &  & p_{1}\\
& -1 & t & \cdots &  & p_{2}\\
\vdots & \vdots & \vdots & \ddots & \vdots & \vdots\\
&  &  & \cdots & t & p_{n-2}\\
&  &  & \cdots & -1 & t+p_{n-1}%
\end{array}
\right)  .
\end{align*}
We compute this determinant by Laplace expansion along the first row:%
\begin{align*}
& \det\left(
\begin{array}
[c]{cccccc}%
t &  &  & \cdots &  & p_{0}\\
-1 & t &  & \cdots &  & p_{1}\\
& -1 & t & \cdots &  & p_{2}\\
\vdots & \vdots & \vdots & \ddots & \vdots & \vdots\\
&  &  & \cdots & t & p_{n-2}\\
&  &  & \cdots & -1 & t+p_{n-1}%
\end{array}
\right)  \\
& =t\ \underbrace{\det\left(
\begin{array}
[c]{ccccc}%
t &  & \cdots &  & p_{1}\\
-1 & t & \cdots &  & p_{2}\\
\vdots & \vdots & \ddots & \vdots & \vdots\\
&  & \cdots & t & p_{n-2}\\
&  & \cdots & -1 & t+p_{n-1}%
\end{array}
\right)  }_{\substack{=t^{n-1}+p_{n-1}t^{n-2}+\cdots+p_{2}t^{1}+p_{1}%
t^{0}\\\text{(by the induction hypothesis)}}}+\left(  -1\right)  ^{n+1}%
p_{0}\ \underbrace{\det\left(
\begin{array}
[c]{ccccc}%
-1 & t &  & \cdots & \\
& -1 & t & \cdots & \\
\vdots & \vdots & \vdots & \ddots & \vdots\\
&  &  & \cdots & t\\
&  &  & \cdots & -1
\end{array}
\right)  }_{\substack{=\left(  -1\right)  ^{n-1}\\\text{(since this matrix is
upper-triangular}\\\text{of size }n-1\text{)}}}\\
& =t\left(  t^{n-1}+p_{n-1}t^{n-2}+\cdots+p_{2}t^{1}+p_{1}t^{0}\right)
+\underbrace{\left(  -1\right)  ^{n+1}p_{0}\left(  -1\right)  ^{n-1}}_{=p_{0}%
}\\
& =t\left(  t^{n-1}+p_{n-1}t^{n-2}+\cdots+p_{2}t^{1}+p_{1}t^{0}\right)
+p_{0}\\
& =t^{n}+p_{n-1}t^{n-1}+p_{2}t^{2}+p_{1}t^{1}+p_{0}=p\left(  t\right)  .
\end{align*}


Thus, $p_{C_{p}}\left(  t\right)  =p\left(  t\right)  $ is proved.

Now, let us show that $q_{C_{p}}\left(  t\right)  =p\left(  t\right)  $.
Indeed, both $q_{C_{p}}\left(  t\right)  $ and $p\left(  t\right)  $ are monic
polynomials, and we know from last lecture that $q_{C_{p}}\left(  t\right)
\mid p_{C_{p}}\left(  t\right)  =p\left(  t\right)  $. Hence, if $q_{C_{p}%
}\left(  t\right)  \neq p\left(  t\right)  $, then $q_{C_{p}}\left(  t\right)
$ is a proper divisor of $p\left(  t\right)  $, thus has degree $<n$ (since
$p\left(  t\right)  $ has degree $n$). So we just need to rule out the
possibility that $q_{C_{p}}\left(  t\right)  $ has degree $<n$.

Indeed, assume (for the sake of contradiction) that $q_{C_{p}}\left(
t\right)  $ has degree $<n$. Thus, $q_{C_{p}}\left(  t\right)  =a_{k}%
t^{k}+a_{k-1}t^{k-1}+\cdots+a_{0}t^{0}$ with $k<n$ and $a_{k}=1$ (since
$q_{C_{p}}$ is monic of degree $<n$). However, the definition of $q_{C_{p}}$
yields $q_{C_{p}}\left(  C_{p}\right)  =0$. In other words,%
\[
a_{k}C_{p}^{k}+a_{k-1}C_{p}^{k-1}+\cdots+a_{0}C_{p}^{0}=0.
\]


However, let us look at what $C_{p}$ does to the standard basis vector
$e_{1}=\left(  1,0,0,0,\ldots,0\right)  ^{T}$. We have%
\begin{align*}
C_{p}^{0}e_{1}  & =e_{1};\\
C_{p}^{1}e_{1}  & =C_{p}e_{1}=e_{2};\\
C_{p}^{2}e_{1}  & =C_{p}e_{2}=e_{3};\\
& \ldots;\\
C_{p}^{n-1}e_{1}  & =e_{n}.
\end{align*}
Thus, applying our equality%
\[
a_{k}C_{p}^{k}+a_{k-1}C_{p}^{k-1}+\cdots+a_{0}C_{p}^{0}=0
\]
to $e_{1}$, we obtain%
\[
a_{k}e_{k+1}+a_{k-1}e_{k}+\cdots+a_{0}e_{1}=0\ \ \ \ \ \ \ \ \ \ \left(
\text{since }k<n\right)  .
\]
But this is absurd, since $e_{1},e_{2},\ldots,e_{n}$ are linearly independent.
So we found a contradiction, and thus we conclude that $q_{C_{p}}\left(
t\right)  $ has degree $\geq n$. So, by the above, we obtain $q_{C_{p}}\left(
t\right)  =p\left(  t\right)  $.
\end{proof}

\begin{remark}
For algebraists: The companion matrix $C_{p}$ has a natural meaning. To wit,
consider the quotient ring $\mathbb{F}\left[  t\right]  /\left(  p\left(
t\right)  \right)  $ as an $n$-dimensional $\mathbb{F}$-vector space with
basis $\left(  \overline{t^{0}},\overline{t^{1}},\ldots,\overline{t^{n-1}%
}\right)  $. Then, the companion matrix $C_{p}$ represents the endomorphism
\textquotedblleft multiply by $t$\textquotedblright\ (that is, the
endomorphism that sends each $\overline{f\left(  t\right)  }$ to
$\overline{t\cdot f\left(  t\right)  }$) in this basis.
\end{remark}

\subsection{The Jordan--Chevalley decomposition}

Recall that:

\begin{itemize}
\item A matrix $A\in\mathbb{C}^{n\times n}$ is said to be
\textbf{diagonalizable} if it is similar to a diagonal matrix.

\item A matrix $A\in\mathbb{C}^{n\times n}$ is said to be \textbf{nilpotent}
if some power of it is the zero matrix (i.e., if $A^{k}=0$ for some
$k\in\mathbb{N}$). Actually (this will be a HW problem), for an $n\times
n$-matrix $A$ to be nilpotent, it is necessary and sufficient that $A^{n}=0$.
\end{itemize}

\begin{theorem}
[Jordan--Chevalley decomposition]Let $A\in\mathbb{C}^{n\times n}$ be an
$n\times n$-matrix.

\textbf{(a)} Then, there exists a unique pair $\left(  D,N\right)  $
consisting of

\begin{itemize}
\item a diagonalizable matrix $D\in\mathbb{C}^{n\times n}$ and

\item a nilpotent matrix $N\in\mathbb{C}^{n\times n}$
\end{itemize}

such that $DN=ND$ and $A=D+N$.

\textbf{(b)} Both $D$ and $N$ in this pair can be written as polynomials in
$A$. In other words, there exist two polynomials $f,g\in\mathbb{C}\left[
t\right]  $ such that $D=f\left(  A\right)  $ and $N=g\left(  A\right)  $.
\end{theorem}

The pair $\left(  D,N\right)  $ in this theorem is known as the
\textbf{Jordan--Chevalley decomposition} (or the \textbf{Dunford
decomposition}) of $A$.

\begin{proof}
[Partial proof.] We will show the following two claims:

\begin{statement}
\textit{Claim 1:} There exists a pair $\left(  D,N\right)  $ as in part
\textbf{(a)} of the theorem.
\end{statement}

\begin{statement}
\textit{Claim 2:} The $D$ and $N$ in this particular pair can be written as
polynomials in $A$.
\end{statement}

To prove both Claims 1 and 2, we can WLOG assume that $A$ is a Jordan matrix.
Indeed, if $A=SJS^{-1}$ for some invertible $S$, and if $\left(  D^{\prime
},N^{\prime}\right)  $ is a Jordan--Chevalley decomposition of $J$, then
$\left(  SD^{\prime}S^{-1},SN^{\prime}S^{-1}\right)  $ is a Jordan--Chevalley
decomposition of $A$. Conjugation of matrices preserves all the properties we
need (such as being a polynomial, commuting, etc.), so we only need to prove
the claims for $J$.

So we WLOG assume that $A$ is a Jordan matrix. Thus,%
\[
A=\left(
\begin{array}
[c]{cccc}%
J_{k_{1}}\left(  \lambda_{1}\right)   &  &  & \\
& J_{k_{2}}\left(  \lambda_{2}\right)   &  & \\
&  & \ddots & \\
&  &  & J_{k_{p}}\left(  \lambda_{p}\right)
\end{array}
\right)
\]
for some $\lambda_{1},\lambda_{2},\ldots,\lambda_{p}$ and some $k_{1}%
,k_{2},\ldots,k_{p}$. Now, we want to decompose this $A$ as $D+N$ with $D$
diagonal and $N$ nilpotent and $DN=ND$. We do this by setting%
\[
D:=\left(
\begin{array}
[c]{cccc}%
\lambda_{1}I_{k_{1}} &  &  & \\
& \lambda_{2}I_{k_{2}} &  & \\
&  & \ddots & \\
&  &  & \lambda_{p}I_{k_{p}}%
\end{array}
\right)  \ \ \ \ \ \ \ \ \ \ \text{and}\ \ \ \ \ \ \ \ \ \ N:=\left(
\begin{array}
[c]{cccc}%
J_{k_{1}}\left(  0\right)   &  &  & \\
& J_{k_{2}}\left(  0\right)   &  & \\
&  & \ddots & \\
&  &  & J_{k_{p}}\left(  0\right)
\end{array}
\right)  .
\]
It is easy to check that $A=D+N$ and $DN=ND$ (since block-diagonal matrices
can be multiplied block by block). Clearly, $D$ is diagonalizable (since $D$
is diagonal) and $N$ is nilpotent (since $N$ is strictly upper-triangular).
Thus, Claim 1 is proved.

Now, we need to prove Claim 2 -- i.e., we need to prove that $D$ and $N$ can
be written as polynomials in $A$. Since $A=D+N$, it suffices to show this for
$D$ (since $N=A-D$).

Our above construction of $D$ shows that $D$ is simply $A$ with its
non-diagonal entries removed. Let $\mu_{1},\mu_{2},\ldots,\mu_{m}$ be the
\textbf{distinct} eigenvalues (i.e., diagonal entries) of $A$. For each
$i\in\left[  m\right]  $, let $\ell_{i}$ be the size of the largest Jordan
block of $A$ at eigenvalue $\mu_{i}$. (Thus, the minimal polynomial of $A$ is
$\prod\limits_{i=1}^{m}\left(  t-\mu_{i}\right)  ^{\ell_{i}}$.)

Now, define the polynomial%
\[
f\left(  t\right)  :=\sum_{i=1}^{m}\mu_{i}\prod\limits_{j\neq i}\left(
\dfrac{t-\mu_{j}}{\mu_{i}-\mu_{j}}\right)  ^{\ell_{j}}\in\mathbb{C}\left[
t\right]  ,
\]
where the product sign $\prod\limits_{j\neq i}$ means a product over all
$j\in\left[  m\right]  $ except for $j=i$. We claim that $f\left(  A\right)
=D$. In other words, we claim that applying $f$ to $A$ has the effect of
cleaning out all off-diagonal entries (while the diagonal entries remain as
they are).

To prove this claim, we recall that%
\[
A=\left(
\begin{array}
[c]{cccc}%
J_{k_{1}}\left(  \lambda_{1}\right)   &  &  & \\
& J_{k_{2}}\left(  \lambda_{2}\right)   &  & \\
&  & \ddots & \\
&  &  & J_{k_{p}}\left(  \lambda_{p}\right)
\end{array}
\right)  .
\]
Hence,%
\[
f\left(  A\right)  =\left(
\begin{array}
[c]{cccc}%
f\left(  J_{k_{1}}\left(  \lambda_{1}\right)  \right)   &  &  & \\
& f\left(  J_{k_{2}}\left(  \lambda_{2}\right)  \right)   &  & \\
&  & \ddots & \\
&  &  & f\left(  J_{k_{p}}\left(  \lambda_{p}\right)  \right)
\end{array}
\right)  .
\]
So we need to show that%
\[
f\left(  J_{k_{u}}\left(  \lambda_{u}\right)  \right)  =\lambda_{u}I_{k_{u}%
}\ \ \ \ \ \ \ \ \ \ \text{for each }u\in\left[  p\right]  .
\]
To prove this, we fix a $u\in\left[  p\right]  $, and we set $B:=J_{k_{u}%
}\left(  \lambda_{u}\right)  $. Thus, $B=J_{k}\left(  \mu_{v}\right)  $ for
some $v\in\left[  p\right]  $ and some positive $k\leq\ell_{v}$.

Substituting $B$ for $t$ into%
\[
f\left(  t\right)  :=\sum_{i=1}^{m}\mu_{i}\prod\limits_{j\neq i}\left(
\dfrac{t-\mu_{j}}{\mu_{i}-\mu_{j}}\right)  ^{\ell_{j}},
\]
we obtain%
\[
f\left(  B\right)  =\sum_{i=1}^{m}\mu_{i}\prod\limits_{j\neq i}\left(
\dfrac{B-\mu_{j}I}{\mu_{i}-\mu_{j}}\right)  ^{\ell_{j}}.
\]
We take a closer look at the addends of the sum. For each $i\in\left[
m\right]  $ that is distinct from $v$, the product $\prod\limits_{j\neq
i}\left(  \dfrac{B-\mu_{j}I}{\mu_{i}-\mu_{j}}\right)  ^{\ell_{j}}$ contains a
factor $\left(  \dfrac{B-\mu_{v}I}{\mu_{i}-\mu_{v}}\right)  ^{\ell_{v}}=0$,
because the $k\times k$-matrix $\dfrac{B-\mu_{v}I}{\mu_{i}-\mu_{v}}$ is
strictly upper-triangular and $k\leq\ell_{v}$. So the entire addend $\mu
_{i}\prod\limits_{j\neq i}\left(  \dfrac{B-\mu_{j}I}{\mu_{i}-\mu_{j}}\right)
^{\ell_{j}}$ is $0$ whenever $i$ is distinct from $v$. Thus, all these addends
disappear except for the addend for $i=v$. So the above formula for $f\left(
B\right)  $ simplifies to%
\[
f\left(  B\right)  =\mu_{v}\prod\limits_{j\neq v}\left(  \dfrac{B-\mu_{j}%
I}{\mu_{v}-\mu_{j}}\right)  ^{\ell_{j}}.
\]
This should be $\mu_{v}I$.

I'm just seeing this isn't exactly the case. TODO: Fix this.

[Alternatively, use the Chinese Remainder Theorem for polynomials to find a
polynomial $f$ that satisfies $f\left(  J_{k_{u}}\left(  \lambda_{u}\right)
\right)  =\lambda_{u}I_{k_{u}}$. This polynomial $f$ should satisfy
$f\equiv\left(  t-\lambda_{u}\right)  ^{k_{u}}+\lambda_{u}$ for each $u$.]
\end{proof}

\subsection{The real Jordan canonical form}

Given a matrix $A\in\mathbb{R}^{n\times n}$ with real entries, its Jordan
canonical form doesn't necessarily have real entries. Indeed, the eigenvalues
of $A$ don't have to be real. Sometimes, we want to find a \textquotedblleft
simple\textquotedblright\ form for $A$ that does have real entries. What
follows is a way to tweak the Jordan canonical form to this use case.

We observe the following:

\begin{lemma}
Let $A\in\mathbb{R}^{n\times n}$ and $\lambda\in\mathbb{C}$. Then, the
\textquotedblleft Jordan structure of $A$ at $\lambda$\textquotedblright%
\ (meaning the multiset of the sizes of the Jordan blocks of $A$ at $\lambda$)
equals the Jordan structure of $A$ at $\overline{\lambda}$. In other words,
for each $p>0$, we have%
\begin{align*}
& \left(  \text{the number of Jordan blocks of }A\text{ at }\lambda\text{
having size }p\right)  \\
& =\left(  \text{the number of Jordan blocks of }A\text{ at }\overline
{\lambda}\text{ having size }p\right)  .
\end{align*}


In other words, Jordan blocks at $\lambda$ and Jordan blocks at $\overline
{\lambda}$ come in pairs of equal sizes.
\end{lemma}

\begin{proof}
HW.
\end{proof}

So we can try to combine each Jordan block at $\lambda$ with an equally sized
Jordan block at $\overline{\lambda}$ and hope that something real comes out
somehow, in the same way as $\left(  t-\lambda\right)  \left(  t-\overline
{\lambda}\right)  =t^{2}-2\operatorname*{Re}\lambda+\left\vert \lambda
\right\vert ^{2}\in\mathbb{R}\left[  t\right]  $.

How to do this?%
\begin{align*}
& \left(
\begin{array}
[c]{cccccccc}%
\lambda & 1 &  &  &  &  &  & \\
& \lambda & \ddots &  &  &  &  & \\
&  & \ddots & 1 &  &  &  & \\
&  &  & \lambda &  &  &  & \\
&  &  &  & \overline{\lambda} & 1 &  & \\
&  &  &  &  & \overline{\lambda} & \ddots & \\
&  &  &  &  &  & \ddots & 1\\
&  &  &  &  &  &  & \overline{\lambda}%
\end{array}
\right)  \\
& \sim\left(
\begin{array}
[c]{cccccccc}%
\lambda &  & 1 &  &  &  &  & \\
& \overline{\lambda} &  & 1 &  &  &  & \\
&  & \lambda &  & 1 &  &  & \\
&  &  & \overline{\lambda} &  & 1 &  & \\
&  &  &  & \ddots &  &  & \\
&  &  &  &  & \ddots &  & \\
&  &  &  &  &  & \lambda & \\
&  &  &  &  &  &  & \overline{\lambda}%
\end{array}
\right)  =\left(
\begin{array}
[c]{cccc}%
L & I_{2} &  & \\
& L & I_{2} & \\
&  & \ddots & \\
&  &  & L
\end{array}
\right)  ,
\end{align*}
where $L$ is the $2\times2$-matrix $\left(
\begin{array}
[c]{cc}%
\lambda & \\
& \overline{\lambda}%
\end{array}
\right)  $. However,%
\[
\left(
\begin{array}
[c]{cc}%
\lambda & \\
& \overline{\lambda}%
\end{array}
\right)  \sim\left(
\begin{array}
[c]{cc}%
a & b\\
-b & a
\end{array}
\right)  ,
\]
where $a=\operatorname*{Re}\lambda$ and $b=\operatorname*{Im}\lambda$ (so that
$\lambda=a+bi$). (HW!) So our matrix is similar to%
\[
\left(
\begin{array}
[c]{cccccccc}%
a & b & 1 &  &  &  &  & \\
-b & a &  & 1 &  &  &  & \\
&  & a & b & 1 &  &  & \\
&  & -b & a &  & 1 &  & \\
&  &  &  & \ddots &  &  & \\
&  &  &  &  & \ddots &  & \\
&  &  &  &  &  & a & b\\
&  &  &  &  &  & -b & a
\end{array}
\right)  .
\]



\end{document}