\documentclass[numbers=enddot,12pt,final,onecolumn,notitlepage]{scrartcl}%
\usepackage[headsepline,footsepline,manualmark]{scrlayer-scrpage}
\usepackage[all,cmtip]{xy}
\usepackage{amssymb}
\usepackage{amsmath}
\usepackage{amsthm}
\usepackage{framed}
\usepackage{comment}
\usepackage{color}
\usepackage{hyperref}
\usepackage[sc]{mathpazo}
\usepackage[T1]{fontenc}
\usepackage{tikz}
\usepackage{needspace}
\usepackage{tabls}
\usepackage{wasysym}
\usepackage{easytable}
\usepackage{pythonhighlight}
%TCIDATA{OutputFilter=latex2.dll}
%TCIDATA{Version=5.50.0.2960}
%TCIDATA{LastRevised=Monday, September 27, 2021 11:50:44}
%TCIDATA{SuppressPackageManagement}
%TCIDATA{<META NAME="GraphicsSave" CONTENT="32">}
%TCIDATA{<META NAME="SaveForMode" CONTENT="1">}
%TCIDATA{BibliographyScheme=Manual}
%TCIDATA{Language=American English}
%BeginMSIPreambleData
\providecommand{\U}[1]{\protect\rule{.1in}{.1in}}
%EndMSIPreambleData
\usetikzlibrary{arrows.meta}
\usetikzlibrary{chains}
\newcounter{exer}
\newcounter{exera}
\numberwithin{exer}{subsection}
\theoremstyle{definition}
\newtheorem{theo}{Theorem}[subsection]
\newenvironment{theorem}[1][]
{\begin{theo}[#1]\begin{leftbar}}
{\end{leftbar}\end{theo}}
\newtheorem{lem}[theo]{Lemma}
\newenvironment{lemma}[1][]
{\begin{lem}[#1]\begin{leftbar}}
{\end{leftbar}\end{lem}}
\newtheorem{prop}[theo]{Proposition}
\newenvironment{proposition}[1][]
{\begin{prop}[#1]\begin{leftbar}}
{\end{leftbar}\end{prop}}
\newtheorem{defi}[theo]{Definition}
\newenvironment{definition}[1][]
{\begin{defi}[#1]\begin{leftbar}}
{\end{leftbar}\end{defi}}
\newtheorem{remk}[theo]{Remark}
\newenvironment{remark}[1][]
{\begin{remk}[#1]\begin{leftbar}}
{\end{leftbar}\end{remk}}
\newtheorem{coro}[theo]{Corollary}
\newenvironment{corollary}[1][]
{\begin{coro}[#1]\begin{leftbar}}
{\end{leftbar}\end{coro}}
\newtheorem{conv}[theo]{Convention}
\newenvironment{convention}[1][]
{\begin{conv}[#1]\begin{leftbar}}
{\end{leftbar}\end{conv}}
\newtheorem{quest}[theo]{Question}
\newenvironment{question}[1][]
{\begin{quest}[#1]\begin{leftbar}}
{\end{leftbar}\end{quest}}
\newtheorem{warn}[theo]{Warning}
\newenvironment{warning}[1][]
{\begin{warn}[#1]\begin{leftbar}}
{\end{leftbar}\end{warn}}
\newtheorem{conj}[theo]{Conjecture}
\newenvironment{conjecture}[1][]
{\begin{conj}[#1]\begin{leftbar}}
{\end{leftbar}\end{conj}}
\newtheorem{exam}[theo]{Example}
\newenvironment{example}[1][]
{\begin{exam}[#1]\begin{leftbar}}
{\end{leftbar}\end{exam}}
\newtheorem{exmp}[exer]{Exercise}
\newenvironment{exercise}[1][]
{\begin{exmp}[#1]\begin{leftbar}}
{\end{leftbar}\end{exmp}}
\newenvironment{statement}{\begin{quote}}{\end{quote}}
\newenvironment{fineprint}{\medskip \begin{small}}{\end{small} \medskip}
\iffalse
\newenvironment{proof}[1][Proof]{\noindent\textbf{#1.} }{\ \rule{0.5em}{0.5em}}
\newenvironment{question}[1][Question]{\noindent\textbf{#1.} }{\ \rule{0.5em}{0.5em}}
\newenvironment{warning}[1][Warning]{\noindent\textbf{#1.} }{\ \rule{0.5em}{0.5em}}
\newenvironment{teachingnote}[1][Teaching note]{\noindent\textbf{#1.} }{\ \rule{0.5em}{0.5em}}
\fi
\let\sumnonlimits\sum
\let\prodnonlimits\prod
\let\cupnonlimits\bigcup
\let\capnonlimits\bigcap
\renewcommand{\sum}{\sumnonlimits\limits}
\renewcommand{\prod}{\prodnonlimits\limits}
\renewcommand{\bigcup}{\cupnonlimits\limits}
\renewcommand{\bigcap}{\capnonlimits\limits}
\setlength\tablinesep{3pt}
\setlength\arraylinesep{3pt}
\setlength\extrarulesep{3pt}
\voffset=0cm
\hoffset=-0.7cm
\setlength\textheight{22.5cm}
\setlength\textwidth{15.5cm}
\newcommand\arxiv[1]{\href{http://www.arxiv.org/abs/#1}{\texttt{arXiv:#1}}}
\newenvironment{verlong}{}{}
\newenvironment{vershort}{}{}
\newenvironment{noncompile}{}{}
\newenvironment{teachingnote}{}{}
\excludecomment{verlong}
\includecomment{vershort}
\excludecomment{noncompile}
\excludecomment{teachingnote}
\newcommand{\CC}{\mathbb{C}}
\newcommand{\RR}{\mathbb{R}}
\newcommand{\QQ}{\mathbb{Q}}
\newcommand{\NN}{\mathbb{N}}
\newcommand{\ZZ}{\mathbb{Z}}
\newcommand{\KK}{\mathbb{K}}
\newcommand{\id}{\operatorname{id}}
\newcommand{\lcm}{\operatorname{lcm}}
\newcommand{\rev}{\operatorname{rev}}
\newcommand{\powset}[2][]{\ifthenelse{\equal{#2}{}}{\mathcal{P}\left(#1\right)}{\mathcal{P}_{#1}\left(#2\right)}}
\newcommand{\set}[1]{\left\{ #1 \right\}}
\newcommand{\abs}[1]{\left| #1 \right|}
\newcommand{\tup}[1]{\left( #1 \right)}
\newcommand{\ive}[1]{\left[ #1 \right]}
\newcommand{\floor}[1]{\left\lfloor #1 \right\rfloor}
\newcommand{\lf}[2]{#1^{\underline{#2}}}
\newcommand{\underbrack}[2]{\underbrace{#1}_{\substack{#2}}}
\newcommand{\horrule}[1]{\rule{\linewidth}{#1}}
\newcommand{\are}{\ar@{-}}
\newcommand{\nnn}{\nonumber\\}
\newcommand{\sslash}{\mathbin{/\mkern-6mu/}}
\newcommand{\numboxed}[2]{\underbrace{\boxed{#1}}_{\text{box } #2}}
\newcommand{\ig}[2]{\includegraphics[scale=#1]{#2.png}}
\newcommand{\UNFINISHED}{\begin{center} \Huge{\textbf{Unfinished material begins here.}} \end{center} }
\iffalse
\NOEXPAND{\today}{\today}
\NOEXPAND{\sslash}{\sslash}
\NOEXPAND{\numboxed}[2]{\numboxed}
\NOEXPAND{\UNFINISHED}{\UNFINISHED}
\fi
\ihead{Math 504 notes}
\ohead{page \thepage}
\cfoot{\today}
\begin{document}

\title{Math 504: Advanced Linear Algebra}
\author{Hugo Woerdeman, with edits by Darij Grinberg\thanks{Drexel University, Korman
Center, 15 S 33rd Street, Philadelphia PA, 19104, USA}}
\date{\today\ (unfinished!)}
\maketitle
\tableofcontents

\section*{Math 504 Lecture 4}

\section{Schur triangularization (cont'd)}

\subsection{Schur triangularization}

We are now ready for one more matrix decomposition, the so-called
\textbf{Schur triangularization} (aka \textbf{Schur decomposition}):

\begin{theorem}
[Schur triangularization theorem]Let $A\in\mathbb{C}^{n\times n}$. Then, there
exists a unitary matrix $U\in\operatorname*{U}\nolimits_{n}\left(
\mathbb{C}\right)  $ and an upper-triangular matrix $T\in\mathbb{C}^{n\times
n}$ such that $A=UTU^{\ast}$.

In other words, $A$ is unitary similar to some upper-triangular matrix.
\end{theorem}

\begin{example}
Let $A=\left(
\begin{array}
[c]{cc}%
1 & 3\\
-3 & 7
\end{array}
\right)  $. Then, a Schur triangularization of $A$ is $A=UTU^{\ast}$, where%
\[
U=\dfrac{1}{\sqrt{2}}\left(
\begin{array}
[c]{cc}%
1 & -1\\
1 & 1
\end{array}
\right)  \ \ \ \ \ \ \ \ \ \ \text{and}\ \ \ \ \ \ \ \ \ \ T=\left(
\begin{array}
[c]{cc}%
4 & 6\\
0 & 4
\end{array}
\right)  .
\]

\end{example}

\begin{proof}
Induction on $n$.

\textit{Base case} ($n=0$) is trivial.

\textit{Induction step:} (From $n-1$ to $n$:)

Suppose that we have proved the theorem for $\left(  n-1\right)  \times\left(
n-1\right)  $-matrices.

Let us now prove it for $n\times n$-matrices.

So let $A\in\mathbb{C}^{n\times n}$. Since $n>0$, this matrix $A$ has at least
one eigenvalue (by the FTA).

Fix some eigenvalue $\lambda$ of $A$, and let $x\neq0$ be an eigenvector for
this eigenvalue.

Let $u_{1}=\dfrac{1}{\left\vert \left\vert x\right\vert \right\vert }x$.
Choose vectors $u_{2},u_{3},\ldots,u_{n}$ such that $\left(  u_{1}%
,u_{2},\ldots,u_{n}\right)  $ is an orthonormal basis of $\mathbb{C}^{n}$. (We
can do this, by a lemma from the lectures before.)

Let $U$ be the matrix with columns $u_{1},u_{2},\ldots,u_{n}$. Then, $U$ is
unitary (by Theorem 1.5.3 $\mathcal{E}\Longrightarrow\mathcal{A}$). Hence,
$U^{\ast}$ is unitary.

Now, by standard properties of matrix multiplication, we have%
\[
\left(  AU\right)  _{\bullet,1}=A\underbrace{U_{\bullet,1}}_{=u_{1}}%
=Au_{1}=\lambda u_{1}%
\]
(since $u_{1}$, being a scalar multiple of $x$, is an eigenvector of $A$ for
eigenvalue $\lambda$). Thus,%
\begin{align*}
\left(  U^{\ast}AU\right)  _{\bullet,1}  & =U^{\ast}\left(  AU\right)
_{\bullet,1}=U^{\ast}\lambda u_{1}=\lambda U^{\ast}\underbrace{u_{1}%
}_{=U_{\bullet,1}}\\
& =\lambda\underbrace{U^{\ast}U_{\bullet,1}}_{=\left(  U^{\ast}U\right)
_{\bullet,1}}=\lambda\left(  \underbrace{U^{\ast}U}_{=I_{n}}\right)
_{\bullet,1}=\lambda\left(  I_{n}\right)  _{\bullet,1}.
\end{align*}
In other words,%
\[
U^{\ast}AU=\left(
\begin{array}
[c]{ccccc}%
\lambda & \ast & \ast & \cdots & \ast\\
0 & \ast & \ast & \cdots & \ast\\
0 & \ast & \ast & \cdots & \ast\\
\vdots & \vdots & \vdots & \ddots & \vdots\\
0 & \ast & \ast & \cdots & \ast
\end{array}
\right)  ,
\]
where the asterisk ($\ast$) means an entry that you don't know or don't care about.

Let us write this in block-matrix notation:%
\[
U^{\ast}AU=\left(
\begin{array}
[c]{cc}%
\lambda & p\\
0 & B
\end{array}
\right)  ,
\]
where $p\in\mathbb{C}^{1\times\left(  n-1\right)  }$ is a row vector and
$B\in\mathbb{C}^{\left(  n-1\right)  \times\left(  n-1\right)  }$ is a matrix.
(The \textquotedblleft$0$\textquotedblright\ here is actually the zero vector
in $\mathbb{C}^{n-1}$.)

Now, by the induction hypothesis, $B$ is unitary similar to an
upper-triangular matrix. In other words, there is a unitary matrix
$V\in\mathbb{C}^{\left(  n-1\right)  \times\left(  n-1\right)  }$ and an
upper-triangular $S$ such that $VBV^{\ast}=S$. Consider these $V$ and $S$.

Now, let $W=\left(
\begin{array}
[c]{cc}%
1 & 0\\
0 & V
\end{array}
\right)  $. This is a block-diagonal matrix, with $\left(
\begin{array}
[c]{c}%
1
\end{array}
\right)  $ and $V$ being its diagonal blocks. Hence, $W^{\ast}=\left(
\begin{array}
[c]{cc}%
1 & 0\\
0 & V^{\ast}%
\end{array}
\right)  $. Moreover, $W$ is a unitary matrix (since it is a block-diagonal
matrix whose diagonal blocks are unitary). Thus, $WU^{\ast}$ is unitary (being
the product of the two unitary matrices $W$ and $U^{\ast}$).

Now,%
\begin{align*}
& \underbrace{W}_{=\left(
\begin{array}
[c]{cc}%
1 & 0\\
0 & V
\end{array}
\right)  }\underbrace{U^{\ast}AU}_{=\left(
\begin{array}
[c]{cc}%
\lambda & p\\
0 & B
\end{array}
\right)  }\underbrace{W^{\ast}}_{=\left(
\begin{array}
[c]{cc}%
1 & 0\\
0 & V^{\ast}%
\end{array}
\right)  }\\
& =\left(
\begin{array}
[c]{cc}%
1 & 0\\
0 & V
\end{array}
\right)  \left(
\begin{array}
[c]{cc}%
\lambda & p\\
0 & B
\end{array}
\right)  \left(
\begin{array}
[c]{cc}%
1 & 0\\
0 & V^{\ast}%
\end{array}
\right)  \\
& =\left(
\begin{array}
[c]{cc}%
1\cdot\lambda\cdot1 & 1\cdot p\cdot V^{\ast}\\
V\cdot0\cdot1 & V\cdot B\cdot V^{\ast}%
\end{array}
\right)  =\left(
\begin{array}
[c]{cc}%
\lambda & pV^{\ast}\\
0 & VBV^{\ast}%
\end{array}
\right)  =\left(
\begin{array}
[c]{cc}%
\lambda & pV^{\ast}\\
0 & S
\end{array}
\right)  .
\end{align*}
This matrix is upper-triangular (since $S$ is upper-triangular). However,
$WU^{\ast}AUW^{\ast}$ is unitary similar to $A$, because%
\[
WU^{\ast}A\underbrace{UW^{\ast}}_{\substack{=U^{\ast\ast}W^{\ast}\\=\left(
WU^{\ast}\right)  ^{\ast}}}=\underbrace{WU^{\ast}}_{\text{unitary}}A\left(
WU^{\ast}\right)  ^{\ast}.
\]
Thus, we have found an upper-triangular matrix (namely, $WU^{\ast}AUW^{\ast}$)
that is unitary similar to $A$. Qed.
\end{proof}

So much for making a single matrix triangular.

Can we make a whole bunch of matrices triangular simultaneously (using the
same unitary $U$)?

Recall that two square matrices $A$ and $B$ are said to \emph{commute} if
$AB=BA$. For example, any two powers of a single matrix commute (because if
$A$ is a square matrix, then $A^{k}A^{\ell}=A^{\ell}A^{k}$).

\begin{lemma}
Let $n>0$. Let $\mathcal{F}$ be a subset of $\mathbb{C}^{n\times n}$. Assume
that any two matrices in $\mathcal{F}$ commute (i.e., for any $A,B\in
\mathcal{F}$, we have $AB=BA$).

Then, there exists a nonzero $x\in\mathbb{C}^{n}$ such that $x$ is an
eigenvector of each $A\in\mathcal{F}$.
\end{lemma}

In short: A family of pairwise commuting matrices always has a common eigenvector.

\begin{proof}
An $\mathcal{F}$\textbf{-invariant subspace} of $\mathbb{C}^{n}$ shall mean a
vector subspace $V$ of $\mathbb{C}^{n}$ such that%
\[
AV\subseteq V\ \ \ \ \ \ \ \ \ \ \text{for each }A\in\mathcal{F}.
\]
Here,%
\[
AV:=\left\{  Av\ \mid\ v\in V\right\}  .
\]


For example, $\left\{  0\right\}  $ and $\mathbb{C}^{n}$ itself are two
$\mathcal{F}$-invariant subspaces. There might be further $\mathcal{F}%
$-invariant subspaces between these two.

[\textbf{Example:} If $n=2$ and $\mathcal{F}=\left\{  \left(
\begin{array}
[c]{cc}%
1 & a\\
0 & 2
\end{array}
\right)  \ \mid\ a\in\mathbb{R}\right\}  $, then $\operatorname*{span}\left(
e_{1}\right)  $ is an $\mathcal{F}$-invariant subspace.]

Let $W$ be an $\mathcal{F}$-invariant subspace of $\mathbb{C}^{n}$ that has
lowest possible positive dimension.

We will show that each $x\in W$ is an eigenvector of each $A\in\mathcal{F}$.

Indeed, fix any $A\in\mathcal{F}$. Then, $AW\subseteq W$ (since $W$ is
$\mathcal{F}$-invariant). Thus, $A$ restricts to a $\mathbb{C}$-linear map
$A\mid_{W}:W\rightarrow W$. Since $\dim W>0$, this $\mathbb{C}$-linear map
$A\mid_{W}$ has an eigenvalue $\lambda$ and a corresponding nonzero
eigenvector $w\neq0$. So $w\in W$ is a nonzero vector satisfying $Aw=\lambda
w$.

Let%
\[
W_{A,\lambda}:=\left\{  y\in W\ \mid\ Ay=\lambda y\right\}  .
\]
This $W_{A,\lambda}$ is a vector subspace of $W$. It has positive dimension,
because it contains the nonzero vector $w$. Furthermore, I claim that this
subspace $W_{A,\lambda}$ is $\mathcal{F}$-invariant.

[\textit{Proof:} Let $B\in\mathcal{F}$ be arbitrary. We must prove that
$BW_{A,\lambda}\subseteq W_{A,\lambda}$. In other words, we must prove that
$Bz\in W_{A,\lambda}$ for each $z\in W_{A,\lambda}$.

Indeed, let $z\in W_{A,\lambda}$. Then, we must show that $Bz\in W_{A,\lambda
}$. It is clear that $Bz\in W$, since $z\in W_{A,\lambda}\subseteq W$ and
because $W$ is $\mathcal{F}$-invariant. Furthermore, we have $ABz=\lambda Bz$
because%
\[
\underbrace{AB}_{=BA}z=B\underbrace{Az}_{\substack{=\lambda z\\\text{(since
}z\in W_{A,\lambda}\text{)}}}=B\lambda z=\lambda Bz.
\]
So we conclude that $Bz\in W_{A,\lambda}$. This shows that $W_{A,\lambda}$ is
$\mathcal{F}$-invariant.]

So $W_{A,\lambda}$ is an $\mathcal{F}$-invariant subspace of $\mathbb{C}^{n}$
of positive dimension that is a subspace of $W$. However, $W$ is an
$\mathcal{F}$-invariant subspace of smallest positive dimension. Thus,
$W_{A,\lambda}$ must have the same dimension as $W$. Hence, $W_{A,\lambda}=W$
(since a subspace of $W$ having the same dimension as $W$ must just be $W$
itself). This shows that any vector in $W$ is an eigenvector of $A$ (since it
belongs to $W_{A,\lambda}=\left\{  y\in W\ \mid\ Ay=\lambda y\right\}  $).

Forget that we fixed $A$. We thus have shown that any vector in $W$ is an
eigenvector of each $A\in\mathcal{F}$. Since $W$ has positive dimension, there
exists some nonzero vector in $W$. Thus, this vector is an eigenvector of each
$A\in\mathcal{F}$. This proves the lemma.
\end{proof}

\begin{theorem}
[Schur triangularization for many commuting matrices]Let $\mathcal{F}$ be a
subset of $\mathbb{C}^{n\times n}$. Assume that any two matrices in
$\mathcal{F}$ commute (i.e., for any $A,B\in\mathcal{F}$, we have $AB=BA$).

Then, there exists a unitary $n\times n$-matrix $U$ such that%
\[
UAU^{\ast}\text{ is upper-triangular for all }A\in\mathcal{F}.
\]

\end{theorem}

\begin{proof}
Same induction as for the previous theorem. But now, instead of picking an
eigenvector of a single matrix $A$, we pick a common eigenvector for all
matrices in $\mathcal{F}$. The existence of such an eigenvector is guaranteed
by the preceding lemma.
\end{proof}

Note that the theorem has no converse. Indeed, it is well possible that a set
$\mathcal{F}$ of matrices can be simultaneously triangularized by one and the
same unitary matrix $U$, but these matrices do not pairwise commute. For
example, $\mathcal{F}=\left\{  \left(
\begin{array}
[c]{ccc}%
0 & 1 & 0\\
0 & 0 & 0\\
0 & 0 & 0
\end{array}
\right)  ,\left(
\begin{array}
[c]{ccc}%
0 & 0 & 0\\
0 & 0 & 1\\
0 & 0 & 0
\end{array}
\right)  \right\}  $.




\end{document}