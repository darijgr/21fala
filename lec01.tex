\documentclass[numbers=enddot,12pt,final,onecolumn,notitlepage]{scrartcl}%
\usepackage[headsepline,footsepline,manualmark]{scrlayer-scrpage}
\usepackage[all,cmtip]{xy}
\usepackage{amssymb}
\usepackage{amsmath}
\usepackage{amsthm}
\usepackage{framed}
\usepackage{comment}
\usepackage{color}
\usepackage{hyperref}
\usepackage[sc]{mathpazo}
\usepackage[T1]{fontenc}
\usepackage{tikz}
\usepackage{needspace}
\usepackage{tabls}
\usepackage{wasysym}
\usepackage{easytable}
\usepackage{pythonhighlight}
%TCIDATA{OutputFilter=latex2.dll}
%TCIDATA{Version=5.50.0.2960}
%TCIDATA{LastRevised=Monday, September 20, 2021 11:50:38}
%TCIDATA{SuppressPackageManagement}
%TCIDATA{<META NAME="GraphicsSave" CONTENT="32">}
%TCIDATA{<META NAME="SaveForMode" CONTENT="1">}
%TCIDATA{BibliographyScheme=Manual}
%TCIDATA{Language=American English}
%BeginMSIPreambleData
\providecommand{\U}[1]{\protect\rule{.1in}{.1in}}
%EndMSIPreambleData
\usetikzlibrary{arrows.meta}
\usetikzlibrary{chains}
\newcounter{exer}
\newcounter{exera}
\numberwithin{exer}{subsection}
\theoremstyle{definition}
\newtheorem{theo}{Theorem}[subsection]
\newenvironment{theorem}[1][]
{\begin{theo}[#1]\begin{leftbar}}
{\end{leftbar}\end{theo}}
\newtheorem{lem}[theo]{Lemma}
\newenvironment{lemma}[1][]
{\begin{lem}[#1]\begin{leftbar}}
{\end{leftbar}\end{lem}}
\newtheorem{prop}[theo]{Proposition}
\newenvironment{proposition}[1][]
{\begin{prop}[#1]\begin{leftbar}}
{\end{leftbar}\end{prop}}
\newtheorem{defi}[theo]{Definition}
\newenvironment{definition}[1][]
{\begin{defi}[#1]\begin{leftbar}}
{\end{leftbar}\end{defi}}
\newtheorem{remk}[theo]{Remark}
\newenvironment{remark}[1][]
{\begin{remk}[#1]\begin{leftbar}}
{\end{leftbar}\end{remk}}
\newtheorem{coro}[theo]{Corollary}
\newenvironment{corollary}[1][]
{\begin{coro}[#1]\begin{leftbar}}
{\end{leftbar}\end{coro}}
\newtheorem{conv}[theo]{Convention}
\newenvironment{convention}[1][]
{\begin{conv}[#1]\begin{leftbar}}
{\end{leftbar}\end{conv}}
\newtheorem{quest}[theo]{Question}
\newenvironment{question}[1][]
{\begin{quest}[#1]\begin{leftbar}}
{\end{leftbar}\end{quest}}
\newtheorem{warn}[theo]{Warning}
\newenvironment{warning}[1][]
{\begin{warn}[#1]\begin{leftbar}}
{\end{leftbar}\end{warn}}
\newtheorem{conj}[theo]{Conjecture}
\newenvironment{conjecture}[1][]
{\begin{conj}[#1]\begin{leftbar}}
{\end{leftbar}\end{conj}}
\newtheorem{exam}[theo]{Example}
\newenvironment{example}[1][]
{\begin{exam}[#1]\begin{leftbar}}
{\end{leftbar}\end{exam}}
\newtheorem{exmp}[exer]{Exercise}
\newenvironment{exercise}[1][]
{\begin{exmp}[#1]\begin{leftbar}}
{\end{leftbar}\end{exmp}}
\newenvironment{statement}{\begin{quote}}{\end{quote}}
\newenvironment{fineprint}{\medskip \begin{small}}{\end{small} \medskip}
\iffalse
\newenvironment{proof}[1][Proof]{\noindent\textbf{#1.} }{\ \rule{0.5em}{0.5em}}
\newenvironment{question}[1][Question]{\noindent\textbf{#1.} }{\ \rule{0.5em}{0.5em}}
\newenvironment{warning}[1][Warning]{\noindent\textbf{#1.} }{\ \rule{0.5em}{0.5em}}
\newenvironment{teachingnote}[1][Teaching note]{\noindent\textbf{#1.} }{\ \rule{0.5em}{0.5em}}
\fi
\let\sumnonlimits\sum
\let\prodnonlimits\prod
\let\cupnonlimits\bigcup
\let\capnonlimits\bigcap
\renewcommand{\sum}{\sumnonlimits\limits}
\renewcommand{\prod}{\prodnonlimits\limits}
\renewcommand{\bigcup}{\cupnonlimits\limits}
\renewcommand{\bigcap}{\capnonlimits\limits}
\setlength\tablinesep{3pt}
\setlength\arraylinesep{3pt}
\setlength\extrarulesep{3pt}
\voffset=0cm
\hoffset=-0.7cm
\setlength\textheight{22.5cm}
\setlength\textwidth{15.5cm}
\newcommand\arxiv[1]{\href{http://www.arxiv.org/abs/#1}{\texttt{arXiv:#1}}}
\newenvironment{verlong}{}{}
\newenvironment{vershort}{}{}
\newenvironment{noncompile}{}{}
\newenvironment{teachingnote}{}{}
\excludecomment{verlong}
\includecomment{vershort}
\excludecomment{noncompile}
\excludecomment{teachingnote}
\newcommand{\CC}{\mathbb{C}}
\newcommand{\RR}{\mathbb{R}}
\newcommand{\QQ}{\mathbb{Q}}
\newcommand{\NN}{\mathbb{N}}
\newcommand{\ZZ}{\mathbb{Z}}
\newcommand{\KK}{\mathbb{K}}
\newcommand{\id}{\operatorname{id}}
\newcommand{\lcm}{\operatorname{lcm}}
\newcommand{\rev}{\operatorname{rev}}
\newcommand{\powset}[2][]{\ifthenelse{\equal{#2}{}}{\mathcal{P}\left(#1\right)}{\mathcal{P}_{#1}\left(#2\right)}}
\newcommand{\set}[1]{\left\{ #1 \right\}}
\newcommand{\abs}[1]{\left| #1 \right|}
\newcommand{\tup}[1]{\left( #1 \right)}
\newcommand{\ive}[1]{\left[ #1 \right]}
\newcommand{\floor}[1]{\left\lfloor #1 \right\rfloor}
\newcommand{\lf}[2]{#1^{\underline{#2}}}
\newcommand{\underbrack}[2]{\underbrace{#1}_{\substack{#2}}}
\newcommand{\horrule}[1]{\rule{\linewidth}{#1}}
\newcommand{\are}{\ar@{-}}
\newcommand{\nnn}{\nonumber\\}
\newcommand{\sslash}{\mathbin{/\mkern-6mu/}}
\newcommand{\numboxed}[2]{\underbrace{\boxed{#1}}_{\text{box } #2}}
\newcommand{\ig}[2]{\includegraphics[scale=#1]{#2.png}}
\newcommand{\UNFINISHED}{\begin{center} \Huge{\textbf{Unfinished material begins here.}} \end{center} }
\iffalse
\NOEXPAND{\today}{\today}
\NOEXPAND{\sslash}{\sslash}
\NOEXPAND{\numboxed}[2]{\numboxed}
\NOEXPAND{\UNFINISHED}{\UNFINISHED}
\fi
\ihead{Math 504 notes}
\ohead{page \thepage}
\cfoot{\today}
\begin{document}

\title{Math 504: Advanced Linear Algebra}
\author{Hugo Woerdeman, with edits by Darij Grinberg\thanks{Drexel University, Korman
Center, 15 S 33rd Street, Philadelphia PA, 19104, USA}}
\date{\today\ (unfinished!)}
\maketitle
\tableofcontents

\section*{Math 504 Lecture 1}

Chris volunteered to scribe Lecture 3.

\section{Unitary matrices ([HorJoh13, \S 2.1])}

\subsection{Inner product}

For any $z\in\mathbb{C}$, we let $\overline{z}$ be the complex conjugate of
$z$. So $\overline{a+bi}=a-bi$ if $a,b\in\mathbb{R}$. 

\begin{definition}
For any two vectors $u=\left(
\begin{array}
[c]{c}%
u_{1}\\
u_{2}\\
\vdots\\
u_{n}%
\end{array}
\right)  \in\mathbb{C}^{n}$ and $v=\left(
\begin{array}
[c]{c}%
v_{1}\\
v_{2}\\
\vdots\\
v_{n}%
\end{array}
\right)  \in\mathbb{C}^{n}$, we define the scalar%
\[
\left\langle u,v\right\rangle :=u_{1}\overline{v_{1}}+u_{2}\overline{v_{2}%
}+\cdots+u_{n}\overline{v_{n}}\in\mathbb{C}.
\]
This scalar $\left\langle u,v\right\rangle $ is called the \textbf{inner
product} (or \textbf{dot product}) of $u$ and $v$.
\end{definition}

For example,%
\begin{align*}
\left\langle \left(
\begin{array}
[c]{c}%
1+i\\
2+3i
\end{array}
\right)  ,\ \ \left(
\begin{array}
[c]{c}%
-i\\
4+i
\end{array}
\right)  \right\rangle  & =\left(  1+i\right)  \overline{\left(  -i\right)
}+\left(  2+3i\right)  \overline{\left(  4+i\right)  }\\
& =\left(  1+i\right)  i+\left(  2+3i\right)  \left(  4-i\right)
=\cdots=10+11i.
\end{align*}


\begin{definition}
For any column vector $v=\left(
\begin{array}
[c]{c}%
v_{1}\\
v_{2}\\
\vdots\\
v_{n}%
\end{array}
\right)  \in\mathbb{C}^{n}$, we define the row vector%
\[
v^{\ast}:=\left(
\begin{array}
[c]{cccc}%
\overline{v_{1}} & \overline{v_{2}} & \cdots & \overline{v_{n}}%
\end{array}
\right)  .
\]

\end{definition}

\begin{proposition}
Let $u,v\in\mathbb{C}^{n}$. Then:

\textbf{(a)} We have $\left\langle u,v\right\rangle =\overline{\left\langle
v,u\right\rangle }$.

\textbf{(b)} We have $\left\langle u,v\right\rangle =v^{\ast}u$.

\textbf{(c)} We have $\left\langle u+u^{\prime},v\right\rangle =\left\langle
u,v\right\rangle +\left\langle u^{\prime},v\right\rangle $ for any $u^{\prime
}\in\mathbb{C}^{n}$.

\textbf{(d)} We have $\left\langle u,v+v^{\prime}\right\rangle =\left\langle
u,v\right\rangle +\left\langle u,v^{\prime}\right\rangle $ for any $v^{\prime
}\in\mathbb{C}^{n}$.

\textbf{(e)} We have $\left\langle \lambda u,v\right\rangle =\lambda
\left\langle u,v\right\rangle $ for any $\lambda\in\mathbb{C}$.

\textbf{(f)} We have $\left\langle u,\lambda v\right\rangle =\overline
{\lambda}\left\langle u,v\right\rangle $ for any $\lambda\in\mathbb{C}$.
\end{proposition}

\begin{proposition}
Let $x\in\mathbb{C}^{n}$. Then:

\textbf{(a)} The number $\left\langle x,x\right\rangle $ is a nonnegative real.

\textbf{(b)} If $x$ is nonzero, then $\left\langle x,x\right\rangle $ is a
positive real.
\end{proposition}

\begin{proof}%
\[
\left\langle x,x\right\rangle =x_{1}\overline{x_{1}}+x_{2}\overline{x_{2}%
}+\cdots+x_{n}\overline{x_{n}}=\left\vert x_{1}\right\vert ^{2}+\left\vert
x_{2}\right\vert ^{2}+\cdots+\left\vert x_{n}\right\vert ^{2},
\]
since $z\overline{z}=\left\vert z\right\vert ^{2}$ for any $z\in\mathbb{C}$.
Note that $\left\vert x_{1}\right\vert ,\left\vert x_{2}\right\vert
,\ldots,\left\vert x_{n}\right\vert $ are reals. Thus, $\left\vert
x_{1}\right\vert ^{2}+\left\vert x_{2}\right\vert ^{2}+\cdots+\left\vert
x_{n}\right\vert ^{2}$ is clearly a nonnegative real. Furthermore, if $x$ is
nonzero, then at least one $x_{i}$ is nonzero, and therefore $\left\vert
x_{1}\right\vert ^{2}+\left\vert x_{2}\right\vert ^{2}+\cdots+\left\vert
x_{n}\right\vert ^{2}$ is positive.
\end{proof}

\begin{definition}
Let $x\in\mathbb{C}^{n}$. We define the \textbf{length} (aka \textbf{norm}) of
$x$ to be the nonnegative real number%
\[
\left\vert \left\vert x\right\vert \right\vert :=\sqrt{\left\langle
x,x\right\rangle }.
\]

\end{definition}

For example, if $x=\left(  1,1\right)  $, then $\left\langle x,x\right\rangle
=1\overline{1}+1\overline{1}=2$, so $\left\vert \left\vert x\right\vert
\right\vert =\sqrt{2}$.

\begin{proposition}
For any $\lambda\in\mathbb{C}$ and $x\in\mathbb{C}^{n}$, we have $\left\vert
\left\vert \lambda x\right\vert \right\vert =\left\vert \lambda\right\vert
\cdot\left\vert \left\vert x\right\vert \right\vert $.
\end{proposition}

\subsection{Orthogonality and orthonormality}

\begin{definition}
Let $x\in\mathbb{C}^{n}$ and $y\in\mathbb{C}^{n}$ be two vectors. We say that
$x$ is \textbf{orthogonal} to $y$ if and only if $\left\langle
x,y\right\rangle =0$. The shorthand for this is \textquotedblleft$x\perp
y$\textquotedblright. (%
%TCIMACRO{\TEXTsymbol{\backslash}}%
%BeginExpansion
$\backslash$%
%EndExpansion
perp)
\end{definition}

The relation $\perp$ is symmetric:

\begin{proposition}
Let $x\in\mathbb{C}^{n}$ and $y\in\mathbb{C}^{n}$ be two vectors. Then,
$x\perp y$ if and only if $y\perp x$.
\end{proposition}

\begin{proof}
Recall that $\left\langle u,v\right\rangle =\overline{\left\langle
v,u\right\rangle }$.
\end{proof}

\begin{definition}
Let $\left(  u_{1},u_{2},\ldots,u_{k}\right)  $ be a tuple of vectors in
$\mathbb{C}^{n}$. Then:

\textbf{(a)} We say that the tuple $\left(  u_{1},u_{2},\ldots,u_{k}\right)  $
is \textbf{orthogonal} if we have%
\[
u_{p}\perp u_{q}\ \ \ \ \ \ \ \ \ \ \text{for all }p\neq q.
\]


\textbf{(b)} We say that the tuple $\left(  u_{1},u_{2},\ldots,u_{k}\right)  $
is \textbf{orthonormal} if it is orthogonal and%
\[
\left\vert \left\vert u_{1}\right\vert \right\vert =\left\vert \left\vert
u_{2}\right\vert \right\vert =\cdots=\left\vert \left\vert u_{k}\right\vert
\right\vert =1.
\]

\end{definition}

\begin{example}
\textbf{(a)} The tuple%
\[
\left(  \left(
\begin{array}
[c]{c}%
1\\
0\\
0
\end{array}
\right)  ,\ \ \left(
\begin{array}
[c]{c}%
0\\
1\\
0
\end{array}
\right)  ,\ \ \left(
\begin{array}
[c]{c}%
0\\
0\\
1
\end{array}
\right)  \right)
\]
is orthonormal.

\textbf{(b)} More generally: Let $n\in\mathbb{N}$. Let $e_{1},e_{2}%
,\ldots,e_{n}\in\mathbb{C}^{n}$ be the vectors defined by%
\[
e_{i}=\left(
\begin{array}
[c]{c}%
0\\
0\\
\vdots\\
0\\
1\\
0\\
0\\
\vdots\\
0
\end{array}
\right)  \ \ \ \ \ \ \ \ \ \ \text{with the }1\text{ being in the }i\text{-th
position.}%
\]
Then, the tuple $\left(  e_{1},e_{2},\ldots,e_{n}\right)  $ is orthonormal. It
is furthermore a basis of $\mathbb{C}^{n}$, and is known as the
\textbf{standard basis}.

\textbf{(c)} The pair $\left(  \left(
\begin{array}
[c]{c}%
1\\
-i\\
2
\end{array}
\right)  ,\ \ \left(
\begin{array}
[c]{c}%
0\\
2i\\
1
\end{array}
\right)  \right)  $ of vectors in $\mathbb{C}^{3}$ is orthogonal, but not orthonormal.

\textbf{(d)} The pair $\left(  \dfrac{1}{\sqrt{6}}\left(
\begin{array}
[c]{c}%
1\\
-i\\
2
\end{array}
\right)  ,\ \ \dfrac{1}{\sqrt{5}}\left(
\begin{array}
[c]{c}%
0\\
2i\\
1
\end{array}
\right)  \right)  $ of vectors in $\mathbb{C}^{3}$ is orthonormal.
\end{example}

\begin{proposition}
Let $\left(  u_{1},u_{2},\ldots,u_{k}\right)  $ be an orthogonal tuple of
nonzero vectors in $\mathbb{C}^{n}$. Then, the tuple%
\[
\left(  \dfrac{1}{\left\vert \left\vert u_{1}\right\vert \right\vert }%
u_{1},\ \ \dfrac{1}{\left\vert \left\vert u_{2}\right\vert \right\vert }%
u_{2},\ \ \ldots,\ \ \dfrac{1}{\left\vert \left\vert u_{k}\right\vert
\right\vert }u_{k}\right)
\]
is orthonormal.
\end{proposition}

\begin{proposition}
Any orthogonal tuple of nonzero vectors in $\mathbb{C}^{n}$ is linearly independent.
\end{proposition}

\begin{proof}
See notes.
\end{proof}

\begin{lemma}
Let $k<n$. Let $a_{1},a_{2},\ldots,a_{k}$ be $k$ vectors in $\mathbb{C}^{n}$.
Then, there exists a nonzero vector $b$ that is orthogonal to each of
$a_{1},a_{2},\ldots,a_{k}$.
\end{lemma}

\begin{proof}
(See notes for details.) Write each vector $a_{i}$ as $a_{i}=\left(
\begin{array}
[c]{c}%
a_{i,1}\\
a_{i,2}\\
\vdots\\
a_{i,n}%
\end{array}
\right)  $. Let $b=\left(
\begin{array}
[c]{c}%
b_{1}\\
b_{2}\\
\vdots\\
b_{n}%
\end{array}
\right)  \in\mathbb{C}^{n}$ be a vector whose entries are so far undetermined.
To ensure that $b$ is orthogonal to $a_{i}$, we need $\left\langle
b,a_{i}\right\rangle =0$. In other words, we need%
\[
b_{1}\overline{a_{i,1}}+b_{2}\overline{a_{i,2}}+\cdots+b_{n}\overline{a_{i,n}%
}=0.
\]
This has to hold for each $i\in\left[  k\right]  :=\left\{  1,2,\ldots
,k\right\}  $. This is a system of $k$ homogeneous linear equations in the $n$
unknowns $b_{1},b_{2},\ldots,b_{n}$. Since there are fewer equations than
there are unknowns, there exists a nonzero solution. IOW, there exists a
nonzero vector $b$ orthogonal to all of $a_{i}$.
\end{proof}

\subsection{Conjugate transposes}

Generalizing our notation $v^{\ast}$, we can define $A^{\ast}$ for any matrix
$A$:

\begin{definition}
Let $A=\left(
\begin{array}
[c]{ccc}%
a_{1,1} & \cdots & a_{1,m}\\
\vdots & \ddots & \vdots\\
a_{n,1} & \cdots & a_{n,m}%
\end{array}
\right)  $ be any matrix. Then, we define the matrix%
\[
A^{\ast}=\left(
\begin{array}
[c]{ccc}%
\overline{a_{1,1}} & \cdots & \overline{a_{n,1}}\\
\vdots & \ddots & \vdots\\
\overline{a_{1,m}} & \cdots & \overline{a_{n,m}}%
\end{array}
\right)  .
\]
This matrix $A^{\ast}$ is called the \textbf{conjugate transpose} of $A$.
\end{definition}

IOW, $A^{\ast}$ is obtained from $A$ by transposing the matrix and conjugating
all entries.

\begin{example}%
\[
\left(
\begin{array}
[c]{ccc}%
1+i & 2-3i & i\\
6 & 0 & 10+i
\end{array}
\right)  ^{\ast}=\left(
\begin{array}
[c]{cc}%
1-i & 6\\
2+3i & 0\\
-i & 10-i
\end{array}
\right)  .
\]

\end{example}

\begin{proposition}
\textbf{(a)} If $A,B\in\mathbb{C}^{n\times m}$ are two matrices, then $\left(
A+B\right)  ^{\ast}=A^{\ast}+B^{\ast}$.

\textbf{(b)} If $A$ is a matrix and $\lambda\in\mathbb{C}$, then $\left(
\lambda A\right)  ^{\ast}=\overline{\lambda}A^{\ast}$.

\textbf{(c)} If $A$ and $B$ are two matrices that can be multiplied, then
$\left(  AB\right)  ^{\ast}=B^{\ast}A^{\ast}$.

\textbf{(d)} If $A$ is any matrix, then $\left(  A^{\ast}\right)  ^{\ast}=A$.
\end{proposition}

\subsection{Isometries}

\begin{definition}
An $n\times k$-matrix $A$ is said to be an \textbf{isometry} if $A^{\ast
}A=I_{k}$. (The notation $I_{k}$ means the $k\times k$ identity matrix.)
\end{definition}

\begin{proposition}
An $n\times k$-matrix $A$ is an isometry if and only if its columns form an
orthonormal tuple of vectors.
\end{proposition}

\begin{proof}
Let $A$ be an $n\times k$-matrix, and let $a_{1},a_{2},\ldots,a_{k}$ be its
$k$ columns. Thus,%
\[
A=\left(
\begin{array}
[c]{ccc}%
\mid &  & \mid\\
a_{1} & \cdots & a_{k}\\
\mid &  & \mid
\end{array}
\right)  \ \ \ \ \ \ \ \ \ \ \text{and therefore}\ \ \ \ \ \ \ \ \ \ A^{\ast
}=\left(
\begin{array}
[c]{ccc}%
- & a_{1}^{\ast} & -\\
& \vdots & \\
- & a_{k}^{\ast} & -
\end{array}
\right)  .
\]
Hence,%
\begin{align*}
A^{\ast}A  & =\left(
\begin{array}
[c]{ccc}%
- & a_{1}^{\ast} & -\\
& \vdots & \\
- & a_{k}^{\ast} & -
\end{array}
\right)  \left(
\begin{array}
[c]{ccc}%
\mid &  & \mid\\
a_{1} & \cdots & a_{k}\\
\mid &  & \mid
\end{array}
\right)  \\
& =\left(
\begin{array}
[c]{ccc}%
a_{1}^{\ast}a_{1} & \cdots & a_{1}^{\ast}a_{k}\\
\vdots & \ddots & \vdots\\
a_{k}^{\ast}a_{1} & \cdots & a_{k}^{\ast}a_{k}%
\end{array}
\right)  =\left(
\begin{array}
[c]{ccc}%
\left\vert \left\vert a_{1}\right\vert \right\vert ^{2} & \cdots &
\left\langle a_{1},a_{k}\right\rangle \\
\vdots & \ddots & \vdots\\
\left\langle a_{k},a_{1}\right\rangle  & \cdots & \left\vert \left\vert
a_{k}\right\vert \right\vert ^{2}%
\end{array}
\right)  .
\end{align*}
On the other hand,%
\[
I_{k}=\left(
\begin{array}
[c]{ccc}%
1 & \cdots & 0\\
\vdots & \ddots & \vdots\\
0 & \cdots & 1
\end{array}
\right)  .
\]
Thus, $A^{\ast}A=I_{k}$ holds if and only if we have%
\[
\left\langle a_{p},a_{q}\right\rangle =0\ \ \ \ \ \ \ \ \ \ \text{for all
}p\neq q
\]
and%
\[
\left\vert \left\vert a_{p}\right\vert \right\vert ^{2}%
=1\ \ \ \ \ \ \ \ \ \ \text{for each }p.
\]
IOW, it holds if and only if we have%
\[
a_{p}\perp a_{q}\ \ \ \ \ \ \ \ \ \ \text{for all }p\neq q
\]
and%
\[
\left\vert \left\vert a_{1}\right\vert \right\vert =\left\vert \left\vert
a_{2}\right\vert \right\vert =\cdots=\left\vert \left\vert a_{k}\right\vert
\right\vert =1.
\]
IOW, it holds if and only if $\left(  a_{1},a_{2},\ldots,a_{k}\right)  $ is
orthonormal. Qed.
\end{proof}

Isometries are called isometries because they preserve lengths:

\begin{proposition}
Let $A\in\mathbb{C}^{n\times k}$ be an isometry. Then, each $x\in
\mathbb{C}^{k}$ satisfies $\left\vert \left\vert Ax\right\vert \right\vert
=\left\vert \left\vert x\right\vert \right\vert $.
\end{proposition}

\begin{proof}
We have $A^{\ast}A=I_{k}$ (since $A$ is an isometry). Let $x\in\mathbb{C}^{k}%
$. By definition, $\left\vert \left\vert Ax\right\vert \right\vert
=\sqrt{\left\langle Ax,Ax\right\rangle }$, so that%
\begin{align*}
\left\vert \left\vert Ax\right\vert \right\vert ^{2}  & =\left\langle
Ax,Ax\right\rangle =\underbrace{\left(  Ax\right)  ^{\ast}}_{=x^{\ast}A^{\ast
}}Ax\ \ \ \ \ \ \ \ \ \ \left(  \text{since }\left\langle u,v\right\rangle
=v^{\ast}u\right)  \\
& =x^{\ast}\underbrace{A^{\ast}A}_{=I_{k}}x=x^{\ast}x=\left\langle
x,x\right\rangle =\left\vert \left\vert x\right\vert \right\vert ^{2}.
\end{align*}
Hence, $\left\vert \left\vert Ax\right\vert \right\vert =\left\vert \left\vert
x\right\vert \right\vert $.
\end{proof}

\subsection{Unitary matrices}

\begin{definition}
A matrix $U\in\mathbb{C}^{n\times k}$ is said to be \textbf{unitary} if both
$U$ and $U^{\ast}$ are isometries.
\end{definition}

\begin{example}
\textbf{(a)} The matrix $A=\dfrac{1}{\sqrt{2}}\left(
\begin{array}
[c]{cc}%
1 & 1\\
1 & -1
\end{array}
\right)  $ is unitary. Indeed, it is easy to see that $A$ is an isometry, but
$A^{\ast}$ is therefore also an isometry, since $A^{\ast}=A$. Thus, $A$ is unitary.

\textbf{(b)} A $1\times1$-matrix $\left(
\begin{array}
[c]{c}%
\lambda
\end{array}
\right)  $ is unitary if and only if $\left\vert \lambda\right\vert =1$.

\textbf{(c)} For any $n\in\mathbb{N}$, the identity matrix $I_{n}$ is an
isometry and thus unitary.
\end{example}


\end{document}