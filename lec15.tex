\documentclass[numbers=enddot,12pt,final,onecolumn,notitlepage]{scrartcl}%
\usepackage[headsepline,footsepline,manualmark]{scrlayer-scrpage}
\usepackage[all,cmtip]{xy}
\usepackage{amssymb}
\usepackage{amsmath}
\usepackage{amsthm}
\usepackage{framed}
\usepackage{comment}
\usepackage{color}
\usepackage{hyperref}
\usepackage[sc]{mathpazo}
\usepackage[T1]{fontenc}
\usepackage{tikz}
\usepackage{needspace}
\usepackage{tabls}
\usepackage{wasysym}
\usepackage{easytable}
\usepackage{pythonhighlight}
%TCIDATA{OutputFilter=latex2.dll}
%TCIDATA{Version=5.50.0.2960}
%TCIDATA{LastRevised=Monday, October 25, 2021 11:51:12}
%TCIDATA{SuppressPackageManagement}
%TCIDATA{<META NAME="GraphicsSave" CONTENT="32">}
%TCIDATA{<META NAME="SaveForMode" CONTENT="1">}
%TCIDATA{BibliographyScheme=Manual}
%TCIDATA{Language=American English}
%BeginMSIPreambleData
\providecommand{\U}[1]{\protect\rule{.1in}{.1in}}
%EndMSIPreambleData
\usetikzlibrary{arrows.meta}
\usetikzlibrary{chains}
\newcounter{exer}
\newcounter{exera}
\numberwithin{exer}{subsection}
\theoremstyle{definition}
\newtheorem{theo}{Theorem}[subsection]
\newenvironment{theorem}[1][]
{\begin{theo}[#1]\begin{leftbar}}
{\end{leftbar}\end{theo}}
\newtheorem{lem}[theo]{Lemma}
\newenvironment{lemma}[1][]
{\begin{lem}[#1]\begin{leftbar}}
{\end{leftbar}\end{lem}}
\newtheorem{prop}[theo]{Proposition}
\newenvironment{proposition}[1][]
{\begin{prop}[#1]\begin{leftbar}}
{\end{leftbar}\end{prop}}
\newtheorem{defi}[theo]{Definition}
\newenvironment{definition}[1][]
{\begin{defi}[#1]\begin{leftbar}}
{\end{leftbar}\end{defi}}
\newtheorem{remk}[theo]{Remark}
\newenvironment{remark}[1][]
{\begin{remk}[#1]\begin{leftbar}}
{\end{leftbar}\end{remk}}
\newtheorem{coro}[theo]{Corollary}
\newenvironment{corollary}[1][]
{\begin{coro}[#1]\begin{leftbar}}
{\end{leftbar}\end{coro}}
\newtheorem{conv}[theo]{Convention}
\newenvironment{convention}[1][]
{\begin{conv}[#1]\begin{leftbar}}
{\end{leftbar}\end{conv}}
\newtheorem{quest}[theo]{Question}
\newenvironment{question}[1][]
{\begin{quest}[#1]\begin{leftbar}}
{\end{leftbar}\end{quest}}
\newtheorem{warn}[theo]{Warning}
\newenvironment{warning}[1][]
{\begin{warn}[#1]\begin{leftbar}}
{\end{leftbar}\end{warn}}
\newtheorem{conj}[theo]{Conjecture}
\newenvironment{conjecture}[1][]
{\begin{conj}[#1]\begin{leftbar}}
{\end{leftbar}\end{conj}}
\newtheorem{exam}[theo]{Example}
\newenvironment{example}[1][]
{\begin{exam}[#1]\begin{leftbar}}
{\end{leftbar}\end{exam}}
\newtheorem{exmp}[exer]{Exercise}
\newenvironment{exercise}[1][]
{\begin{exmp}[#1]\begin{leftbar}}
{\end{leftbar}\end{exmp}}
\newenvironment{statement}{\begin{quote}}{\end{quote}}
\newenvironment{fineprint}{\medskip \begin{small}}{\end{small} \medskip}
\iffalse
\newenvironment{proof}[1][Proof]{\noindent\textbf{#1.} }{\ \rule{0.5em}{0.5em}}
\newenvironment{question}[1][Question]{\noindent\textbf{#1.} }{\ \rule{0.5em}{0.5em}}
\newenvironment{warning}[1][Warning]{\noindent\textbf{#1.} }{\ \rule{0.5em}{0.5em}}
\newenvironment{teachingnote}[1][Teaching note]{\noindent\textbf{#1.} }{\ \rule{0.5em}{0.5em}}
\fi
\let\sumnonlimits\sum
\let\prodnonlimits\prod
\let\cupnonlimits\bigcup
\let\capnonlimits\bigcap
\renewcommand{\sum}{\sumnonlimits\limits}
\renewcommand{\prod}{\prodnonlimits\limits}
\renewcommand{\bigcup}{\cupnonlimits\limits}
\renewcommand{\bigcap}{\capnonlimits\limits}
\setlength\tablinesep{3pt}
\setlength\arraylinesep{3pt}
\setlength\extrarulesep{3pt}
\voffset=0cm
\hoffset=-0.7cm
\setlength\textheight{22.5cm}
\setlength\textwidth{15.5cm}
\newcommand\arxiv[1]{\href{http://www.arxiv.org/abs/#1}{\texttt{arXiv:#1}}}
\newenvironment{verlong}{}{}
\newenvironment{vershort}{}{}
\newenvironment{noncompile}{}{}
\newenvironment{teachingnote}{}{}
\excludecomment{verlong}
\includecomment{vershort}
\excludecomment{noncompile}
\excludecomment{teachingnote}
\newcommand{\CC}{\mathbb{C}}
\newcommand{\RR}{\mathbb{R}}
\newcommand{\QQ}{\mathbb{Q}}
\newcommand{\NN}{\mathbb{N}}
\newcommand{\ZZ}{\mathbb{Z}}
\newcommand{\KK}{\mathbb{K}}
\newcommand{\id}{\operatorname{id}}
\newcommand{\lcm}{\operatorname{lcm}}
\newcommand{\rev}{\operatorname{rev}}
\newcommand{\powset}[2][]{\ifthenelse{\equal{#2}{}}{\mathcal{P}\left(#1\right)}{\mathcal{P}_{#1}\left(#2\right)}}
\newcommand{\set}[1]{\left\{ #1 \right\}}
\newcommand{\abs}[1]{\left| #1 \right|}
\newcommand{\tup}[1]{\left( #1 \right)}
\newcommand{\ive}[1]{\left[ #1 \right]}
\newcommand{\floor}[1]{\left\lfloor #1 \right\rfloor}
\newcommand{\lf}[2]{#1^{\underline{#2}}}
\newcommand{\underbrack}[2]{\underbrace{#1}_{\substack{#2}}}
\newcommand{\horrule}[1]{\rule{\linewidth}{#1}}
\newcommand{\are}{\ar@{-}}
\newcommand{\nnn}{\nonumber\\}
\newcommand{\sslash}{\mathbin{/\mkern-6mu/}}
\newcommand{\numboxed}[2]{\underbrace{\boxed{#1}}_{\text{box } #2}}
\newcommand{\ig}[2]{\includegraphics[scale=#1]{#2.png}}
\newcommand{\UNFINISHED}{\begin{center} \Huge{\textbf{Unfinished material begins here.}} \end{center} }
\iffalse
\NOEXPAND{\today}{\today}
\NOEXPAND{\sslash}{\sslash}
\NOEXPAND{\numboxed}[2]{\numboxed}
\NOEXPAND{\UNFINISHED}{\UNFINISHED}
\fi
\ihead{Math 504 notes}
\ohead{page \thepage}
\cfoot{\today}
\begin{document}

\title{Math 504: Advanced Linear Algebra}
\author{Hugo Woerdeman, with edits by Darij Grinberg\thanks{Drexel University, Korman
Center, 15 S 33rd Street, Philadelphia PA, 19104, USA}}
\date{\today\ (unfinished!)}
\maketitle
\tableofcontents

\section*{Math 504 Lecture 15}

\section{Hermitian matrices}

\textbf{Recall:} A \textbf{Hermitian matrix} is an $n\times n$-matrix
$A\in\mathbb{C}^{n\times n}$ such that $A^{\ast}=A$.

Note that this is the complex analogue of real symmetric matrices
($A\in\mathbb{R}^{n\times n}$ such that $A^{T}=A$).

If $A$ is a Hermitian matrix, then $A_{i,i}\in\mathbb{R}$ and $A_{i,j}%
=\overline{A_{j,i}}$.

For instance, the matrix $\left(
\begin{array}
[c]{ccc}%
-1 & i & 2\\
-i & 5 & 1-i\\
2 & 1+i & 0
\end{array}
\right)  $ is Hermitian.

\subsection{Basics}

\begin{theorem}
Let $A\in\mathbb{C}^{n\times n}$ be an $n\times n$-matrix. Then, the following
are equivalent:

\begin{itemize}
\item $\mathcal{A}$: The matrix $A$ is Hermitian (i.e., we have $A^{\ast}=A$).

\item $\mathcal{B}$: We have $A=UDU^{\ast}$ for some unitary $U$ and some real
diagonal $D$ (that is, $D$ is a diagonal matrix with real entries).

\item $\mathcal{C}$: The matrix $A$ is normal and its eigenvalues are real.

\item $\mathcal{D}$: We have $\left\langle Ax,x\right\rangle \in\mathbb{R}$
for each $x\in\mathbb{C}^{n}$.

\item $\mathcal{E}$: The matrix $S^{\ast}AS$ is Hermitian for all
$S\in\mathbb{C}^{n\times k}$ (for all $k\in\mathbb{N}$).
\end{itemize}
\end{theorem}

To prove this, we will need a lemma:

\begin{lemma}
Let $M\in\mathbb{C}^{n\times n}$ be an $n\times n$-matrix. Assume that
$\left\langle Mx,x\right\rangle =0$ for each $x\in\mathbb{C}^{n}$. Then, $M=0$.
\end{lemma}

\begin{proof}
If $x\in\mathbb{C}^{n}$ has entries $x_{1},x_{2},\ldots,x_{n}$, then%
\begin{align*}
\left\langle Mx,x\right\rangle  & =\left\langle \left(
\begin{array}
[c]{c}%
M_{1,1}x_{1}+M_{1,2}x_{2}+\cdots+M_{1,n}x_{n}\\
M_{2,1}x_{1}+M_{2,2}x_{2}+\cdots+M_{2,n}x_{n}\\
\vdots\\
M_{n,1}x_{1}+M_{n,2}x_{2}+\cdots+M_{n,n}x_{n}%
\end{array}
\right)  ,\left(
\begin{array}
[c]{c}%
x_{1}\\
x_{2}\\
\vdots\\
x_{n}%
\end{array}
\right)  \right\rangle \\
& =\sum_{i=1}^{n}\left(  M_{i,1}x_{1}+M_{i,2}x_{2}+\cdots+M_{i,n}x_{n}\right)
\overline{x_{i}}\\
& =\sum_{i=1}^{n}\ \ \sum_{j=1}^{n}M_{i,j}x_{j}\overline{x_{i}}=\sum_{i=1}%
^{n}\ \ \sum_{j=1}^{n}M_{i,j}\overline{x_{i}}x_{j}.
\end{align*}
So this is always $=0$ by assumption, no matter what $x$ is. So we have shown
that%
\[
\sum_{i=1}^{n}\ \ \sum_{j=1}^{n}M_{i,j}\overline{x_{i}}x_{j}%
=0\ \ \ \ \ \ \ \ \ \ \text{for every }x=\left(
\begin{array}
[c]{c}%
x_{1}\\
x_{2}\\
\vdots\\
x_{n}%
\end{array}
\right)  \in\mathbb{C}^{n}.
\]
In particular:

\begin{itemize}
\item We can apply this to $x=e_{1}=\left(  1,0,0,\ldots,0\right)  ^{T}$, and
we obtain $M_{1,1}\cdot\overline{1}\cdot1=0$, which means $M_{1,1}=0$.
Similarly, we can find $M_{i,i}=0$ for all $i\in\left[  n\right]  $.

\item We can apply this to $x=e_{1}+e_{2}=\left(  1,1,0,0,\ldots,0\right)
^{T}$, and we obtain
\[
M_{1,1}\cdot\overline{1}\cdot1+M_{1,2}\cdot\overline{1}\cdot1+M_{2,1}%
\cdot\overline{1}\cdot1\cdot M_{2,2}\cdot\overline{1}\cdot1=0.
\]
This simplifies to%
\[
M_{1,1}+M_{1,2}+M_{2,1}+M_{2,2}=0.
\]
However, the previous bullet point yields $M_{1,1}=0$ and $M_{2,2}=0$, so this
simplifies further to%
\[
M_{1,2}+M_{2,1}=0.
\]


\item We can apply this to $x=e_{1}+ie_{2}=\left(  1,i,0,0,\ldots,0\right)
^{T}$, and we obtain
\[
M_{1,1}\cdot\overline{1}\cdot1+M_{1,2}\cdot\overline{1}\cdot i+M_{2,1}%
\cdot\overline{i}\cdot1\cdot M_{2,2}\cdot\overline{i}\cdot i=0.
\]
This simplifies to%
\[
M_{1,1}+iM_{1,2}-iM_{2,1}+M_{2,2}=0.
\]
However, we know that $M_{1,1}=0$ and $M_{2,2}=0$, so this simplifies further
to%
\[
iM_{1,2}-iM_{2,1}=0.
\]
Thus,
\[
M_{1,2}-M_{2,1}=0.
\]
Adding this to%
\[
M_{1,2}+M_{2,1}=0,
\]
we obtain $2M_{1,2}=0$. In other words, $M_{1,2}=0$. Similarly, we can show
that $M_{i,j}=0$ for all $i\neq j$.
\end{itemize}

So we have now shown that all entries of $M$ are $0$. In other words, $M=0$.
This proves the lemma.
\end{proof}

Now we can prove the theorem:

\begin{proof}
[Proof of Theorem.] The implication $\mathcal{A}\Longrightarrow\mathcal{B}$
follows from the spectral theorem. So does the implication $\mathcal{A}%
\Longrightarrow\mathcal{C}$. The implication $\mathcal{B}\Longrightarrow
\mathcal{A}$ follows from a corollary of the spectral theorem. Finally,
$\mathcal{C}\Longrightarrow\mathcal{B}$ also follows from the spectral
theorem. So we only need to prove the equivalence $\mathcal{A}%
\Longleftrightarrow\mathcal{D}\Longleftrightarrow\mathcal{E}$.

\begin{itemize}
\item \textit{Proof of }$\mathcal{A}\Longrightarrow\mathcal{D}$\textit{:}
Assume that $\mathcal{A}$ holds. Thus, $A=A^{\ast}$. Now, let $x\in
\mathbb{C}^{n}$. Then, $\left\langle Ax,x\right\rangle =\overline{\left\langle
x,Ax\right\rangle }$ (by the rules for inner products). However, by the
formula $\left\langle u,v\right\rangle =v^{\ast}u$, we have%
\[
\left\langle Ax,x\right\rangle =x^{\ast}Ax\ \ \ \ \ \ \ \ \ \ \text{and}%
\ \ \ \ \ \ \ \ \ \ \left\langle x,Ax\right\rangle =\underbrace{\left(
Ax\right)  ^{\ast}}_{=x^{\ast}A^{\ast}}x=x^{\ast}\underbrace{A^{\ast}}%
_{=A}x=x^{\ast}Ax.
\]
Comparing these two equalities, we see that $\left\langle Ax,x\right\rangle
=\left\langle x,Ax\right\rangle $. Comparing this with $\left\langle
Ax,x\right\rangle =\overline{\left\langle x,Ax\right\rangle }$, we obtain
$\left\langle x,Ax\right\rangle =\overline{\left\langle x,Ax\right\rangle }$.
This entails $\left\langle x,Ax\right\rangle \in\mathbb{R}$ (since the only
complex numbers $z\in\mathbb{C}$ that satisfy $z=\overline{z}$ are the real
numbers). Therefore, $\left\langle Ax,x\right\rangle =\left\langle
x,Ax\right\rangle \in\mathbb{R}$. Thus, the statement $\mathcal{D}$ is proved.

\item \textit{Proof of }$\mathcal{D}\Longrightarrow\mathcal{A}$\textit{:}
Assume that $\mathcal{D}$ holds. Thus, $\left\langle Ax,x\right\rangle
\in\mathbb{R}$ for each $x\in\mathbb{C}^{n}$. Again, we can see that each
$x\in\mathbb{C}^{n}$ satisfies%
\[
\left\langle Ax,x\right\rangle =x^{\ast}Ax\ \ \ \ \ \ \ \ \ \ \text{and}%
\ \ \ \ \ \ \ \ \ \ \left\langle x,Ax\right\rangle =\underbrace{\left(
Ax\right)  ^{\ast}}_{=x^{\ast}A^{\ast}}x=x^{\ast}A^{\ast}x.
\]
Thus, each $x\in\mathbb{C}^{n}$ satisfies%
\begin{align*}
x^{\ast}Ax  & =\left\langle Ax,x\right\rangle =\overline{\left\langle
Ax,x\right\rangle }\ \ \ \ \ \ \ \ \ \ \left(  \text{since }\left\langle
Ax,x\right\rangle \in\mathbb{R}\right)  \\
& =\left\langle x,Ax\right\rangle =x^{\ast}A^{\ast}x,
\end{align*}
so that
\[
x^{\ast}Ax-x^{\ast}A^{\ast}x=0,
\]
so that%
\[
x^{\ast}\left(  A^{\ast}-A\right)  x=0.
\]
Applying our Lemma to $M=A^{\ast}-A$, we thus conclude that $A^{\ast}-A=0$. In
other words, $A^{\ast}=A$. This proves $\mathcal{A}$.

\item \textit{Proof of }$\mathcal{A}\Longrightarrow\mathcal{E}$\textit{:} If
$A$ is Hermitian, then $A^{\ast}=A$, so that%
\[
\left(  S^{\ast}AS\right)  ^{\ast}=S^{\ast}\underbrace{A^{\ast}}%
_{=A}\underbrace{\left(  S^{\ast}\right)  ^{\ast}}_{=S}=S^{\ast}AS,
\]
and therefore $S^{\ast}AS$ is again Hermitian. This proves $\mathcal{A}%
\Longrightarrow\mathcal{E}$.

\item \textit{Proof of }$\mathcal{E}\Longrightarrow\mathcal{A}$\textit{:} If
statement $\mathcal{E}$ holds, then we can apply it to $S=I_{n}$ (and $k=n$),
and conclude that $I_{n}^{\ast}AI_{n}$ is Hermitian; but this is simply saying
that $A$ is Hermitian. So $\mathcal{E}\Longrightarrow\mathcal{A}$ follows.
\end{itemize}

The theorem is proved.
\end{proof}

As a reminder: Sums of Hermitian matrices are Hermitian, but products are not
(in general).

\subsection{Definiteness}

\begin{definition}
Let $A\in\mathbb{C}^{n\times n}$ be a Hermitian matrix.

\textbf{(a)} We say that $A$ is \textbf{positive semidefinite} if it satisfies%
\[
\left\langle Ax,x\right\rangle \geq0\ \ \ \ \ \ \ \ \ \ \text{for all }%
x\in\mathbb{C}^{n}.
\]


\textbf{(b)} We say that $A$ is \textbf{positive definite} if it satisfies%
\[
\left\langle Ax,x\right\rangle >0\ \ \ \ \ \ \ \ \ \ \text{for all nonzero
}x\in\mathbb{C}^{n}.
\]


\textbf{(c)} We say that $A$ is \textbf{negative semidefinite} if it satisfies%
\[
\left\langle Ax,x\right\rangle \leq0\ \ \ \ \ \ \ \ \ \ \text{for all }%
x\in\mathbb{C}^{n}.
\]


\textbf{(d)} We say that $A$ is \textbf{negative definite} if it satisfies%
\[
\left\langle Ax,x\right\rangle <0\ \ \ \ \ \ \ \ \ \ \text{for all nonzero
}x\in\mathbb{C}^{n}.
\]


\textbf{(e)} We say that $A$ is \textbf{indefinite} if it is neither positive
semidefinite nor negative semidefinite, i.e., if there exist vectors
$x,y\in\mathbb{C}^{n}$ such that%
\[
\left\langle Ax,x\right\rangle <0<\left\langle Ay,y\right\rangle .
\]

\end{definition}

Here are some examples of definiteness:

\begin{example}
Let $n\in\mathbb{N}$. Let $J=\left(
\begin{array}
[c]{cccc}%
1 & 1 & \cdots & 1\\
1 & 1 & \cdots & 1\\
\vdots & \vdots & \ddots & \vdots\\
1 & 1 & \cdots & 1
\end{array}
\right)  $. This matrix $J$ is real symmetric, thus Hermitian. Is it positive
definite? Positive semidefinite?

Let $x=\left(  x_{1},x_{2},\ldots,x_{n}\right)  ^{T}\in\mathbb{C}^{n}$. Then,%
\begin{align*}
\left\langle Jx,x\right\rangle  & =\sum_{i=1}^{n}\ \ \sum_{j=1}^{n}%
\overline{x_{i}}x_{j}=\left(  \sum_{i=1}^{n}\overline{x_{i}}\right)  \left(
\sum_{j=1}^{n}x_{j}\right)  =\left(  \overline{\sum_{i=1}^{n}x_{i}}\right)
\left(  \sum_{j=1}^{n}x_{j}\right)  \\
& =\left(  \overline{\sum_{i=1}^{n}x_{i}}\right)  \left(  \sum_{i=1}^{n}%
x_{i}\right)  =\left\vert \sum_{i=1}^{n}x_{i}\right\vert ^{2}\geq0.
\end{align*}
So $J$ is positive semidefinite.

Is $J$ positive definite? To have $\left\langle Jx,x\right\rangle =0$ is
equivalent to having $\sum_{i=1}^{n}x_{i}=0$. When $n=1$ (or $n=0$), this is
equivalent to having $x=0$, so $J$ is positive definite in this case. However,
if $n>1$, then this is not equivalent to having $x=0$, and in fact the vector
$e_{1}-e_{2}$ is an example of a nonzero vector $x\in\mathbb{C}^{n}$ such that
$\left\langle Jx,x\right\rangle =0$. So $J$ is not positive definite unless
$n\leq1$.
\end{example}

\begin{example}
Consider a diagonal matrix
\[
D:=\operatorname*{diag}\left(  \lambda_{1},\lambda_{2},\ldots,\lambda
_{n}\right)  =\left(
\begin{array}
[c]{cccc}%
\lambda_{1} & 0 & \cdots & 0\\
0 & \lambda_{2} & \cdots & 0\\
\vdots & \vdots & \ddots & \vdots\\
0 & 0 & \cdots & \lambda_{n}%
\end{array}
\right)
\]
with $\lambda_{1},\lambda_{2},\ldots,\lambda_{n}\in\mathbb{R}$. When is $D$
positive semidefinite?

We want $\left\langle Dx,x\right\rangle \geq0$ for all $x\in\mathbb{C}^{n}$.
Let $x=\left(  x_{1},x_{2},\ldots,x_{n}\right)  ^{T}\in\mathbb{C}^{n}$. Then,%
\[
\left\langle Dx,x\right\rangle =\sum_{i=1}^{n}\lambda_{i}\overline{x_{i}}%
x_{i}=\sum_{i=1}^{n}\lambda_{i}\left\vert x_{i}\right\vert ^{2}.
\]
If $\lambda_{1},\lambda_{2},\ldots,\lambda_{n}\geq0$, then we therefore
conclude that $\left\langle Dx,x\right\rangle \geq0$, so that $D$ is positive
semidefinite. Otherwise, $D$ is not positive semidefinite, since we can pick
an $x=e_{j}$ where $j$ satisfies $\lambda_{i}<0$. So $D$ is positive
semidefinite if and only if $\lambda_{1},\lambda_{2},\ldots,\lambda_{n}\geq0$.
A similar argument shows that $D$ is positive definite if and only if
$\lambda_{1},\lambda_{2},\ldots,\lambda_{n}>0$.
\end{example}

\begin{example}
The Hilbert matrix%
\[
\left(
\begin{array}
[c]{cccc}%
\dfrac{1}{1} & \dfrac{1}{2} & \cdots & \dfrac{1}{n}\\
\dfrac{1}{2} & \dfrac{1}{3} & \cdots & \dfrac{1}{n+1}\\
\vdots & \vdots & \ddots & \vdots\\
\dfrac{1}{n} & \dfrac{1}{n+1} & \cdots & \dfrac{1}{2n}%
\end{array}
\right)
\]
(i.e., the $n\times n$-matrix whose $\left(  i,j\right)  $-th entry is
$\dfrac{1}{i+j-1}$) is positive definite. In other words, for any $x=\left(
x_{1},x_{2},\ldots,x_{n}\right)  ^{T}\in\mathbb{C}^{n}$, we have%
\[
\sum_{i=1}^{n}\ \ \sum_{j=1}^{n}\dfrac{\overline{x_{i}}x_{j}}{i+j-1}\geq0.
\]
This is not obvious at all, and will be a HW exercise (with hints). More
generally, if $a_{1},a_{2},\ldots,a_{n}$ are positive reals, then the $n\times
n$-matrix whose $\left(  i,j\right)  $-th entry is $\dfrac{1}{a_{i}+a_{j}}$ is
positive definite.
\end{example}

As an application of positive semidefiniteness, the Schoenberg theorem
generalizes the triangle inequality. Recall that the triangle inequality says
that three nonnegative real numbers $x,y,z$ are the mutual distances of $3$
points in the plane if and only if $x\leq y+z$ and $y\leq z+x$ and $z\leq
x+y$. In higher dimensions, the analogous criterion is the following:

\begin{theorem}
[Schoenberg's theorem]Let $n\in\mathbb{N}$ and $r\in\mathbb{N}$. Let $d_{i,j}$
be a nonnegative real for each $i,j\in\left[  n\right]  $. Assume that
$d_{i,i}=0$ for all $i\in\left[  n\right]  $, and furthermore $d_{i,j}%
=d_{j,i}$ for all $i,j\in\left[  n\right]  $. Then, there exist $n$ points
$P_{1},P_{2},\ldots,P_{n}\in\mathbb{R}^{r}$ satisfying%
\[
\left\vert P_{i}-P_{j}\right\vert =d_{i,j}\ \ \ \ \ \ \ \ \ \ \text{for all
}i,j\in\left[  n\right]
\]
if and only if the $\left(  n-1\right)  \times\left(  n-1\right)  $-matrix
whose $\left(  i,j\right)  $-th entry is
\[
d_{i,n}^{2}+d_{j,n}^{2}-d_{i,j}^{2}\ \ \ \ \ \ \ \ \ \ \text{for all }%
i,j\in\left[  n-1\right]
\]
is positive semidefinite and has rank $\leq r$.
\end{theorem}

We will not prove this here.

\begin{remark}
If $A\in\mathbb{R}^{n\times n}$ and $\left\langle Ax,x\right\rangle \geq0$ for
all $x\in\mathbb{R}^{n}$, then we \textbf{cannot} conclude that $A$ is
positive semidefinite. The reason is that it does not follow that $A$ is
symmetric. For example, $A=\left(
\begin{array}
[c]{cc}%
2 & 1\\
0 & 2
\end{array}
\right)  $ satisfies
\[
\left\langle Ax,x\right\rangle =2x_{1}^{2}+x_{1}x_{2}+2x_{2}^{2}=\dfrac{1}%
{2}\left(  x_{1}+x_{2}\right)  ^{2}+\dfrac{3}{2}\left(  x_{1}^{2}+x_{2}%
^{2}\right)  \geq0
\]
for each $x=\left(
\begin{array}
[c]{c}%
x_{1}\\
x_{2}%
\end{array}
\right)  \in\mathbb{R}^{2}$, but it is not symmetric.
\end{remark}

\begin{theorem}
Let $A\in\mathbb{C}^{n\times n}$ be a Hermitian matrix. Then:

\textbf{(a)} The matrix $A$ is positive semidefinite if and only if all
eigenvalues of $A$ are nonnegative.

\textbf{(b)} The matrix $A$ is positive definite if and only if all
eigenvalues of $A$ are positive.

(Recall that the eigenvalues of $A$ are reals by the spectral theorem.)
\end{theorem}

\begin{proof}
\textbf{(a)} Assume that $A$ is positive semidefinite. Let $\lambda$ be an
eigenvalue of $A$. Let $x\neq0$ be a corresponding eigenvector. Then,
$Ax=\lambda x$. However, $\left\langle Ax,x\right\rangle \geq0$ since $A$ is
positive semidefinite. So $\left\langle \lambda x,x\right\rangle \geq0$.
However, $\left\langle \lambda x,x\right\rangle =\lambda\left\langle
x,x\right\rangle $, so that $\lambda\left\langle x,x\right\rangle \geq0$. We
can cancel $\left\langle x,x\right\rangle $ (since $\left\langle
x,x\right\rangle >0$). Thus, we get $\lambda\geq0$. Therefore, all eigenvalues
of $A$ are $\geq0$.

Conversely, assume that all eigenvalues of $A$ are $\geq0$. By the spectral
theorem, we can write $A$ as
\[
A=UDU^{\ast},\ \ \ \ \ \ \ \ \ \ \text{where }D=\operatorname*{diag}\left(
\lambda_{1},\lambda_{2},\ldots,\lambda_{n}\right)  ,
\]
where $\lambda_{1},\lambda_{2},\ldots,\lambda_{n}$ are the eigenvalues of $A$.
However,
\[
D=\operatorname*{diag}\left(  \lambda_{1},\lambda_{2},\ldots,\lambda
_{n}\right)  =\left(  \operatorname*{diag}\left(  \sqrt{\lambda_{1}}%
,\sqrt{\lambda_{2}},\ldots,\sqrt{\lambda_{n}}\right)  \right)  ^{2}%
\]
(the square roots here are well-defined, since the $\lambda_{i}$ are
nonnegative by assumption). ............ [To be finished]
\end{proof}


\end{document}