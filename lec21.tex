\documentclass[numbers=enddot,12pt,final,onecolumn,notitlepage]{scrartcl}%
\usepackage[headsepline,footsepline,manualmark]{scrlayer-scrpage}
\usepackage[all,cmtip]{xy}
\usepackage{amssymb}
\usepackage{amsmath}
\usepackage{amsthm}
\usepackage{framed}
\usepackage{comment}
\usepackage{color}
\usepackage{hyperref}
\usepackage[sc]{mathpazo}
\usepackage[T1]{fontenc}
\usepackage{tikz}
\usepackage{needspace}
\usepackage{tabls}
\usepackage{wasysym}
\usepackage{easytable}
\usepackage{pythonhighlight}
%TCIDATA{OutputFilter=latex2.dll}
%TCIDATA{Version=5.50.0.2960}
%TCIDATA{LastRevised=Friday, November 19, 2021 11:49:24}
%TCIDATA{SuppressPackageManagement}
%TCIDATA{<META NAME="GraphicsSave" CONTENT="32">}
%TCIDATA{<META NAME="SaveForMode" CONTENT="1">}
%TCIDATA{BibliographyScheme=Manual}
%TCIDATA{Language=American English}
%BeginMSIPreambleData
\providecommand{\U}[1]{\protect\rule{.1in}{.1in}}
%EndMSIPreambleData
\usetikzlibrary{arrows.meta}
\usetikzlibrary{chains}
\newcounter{exer}
\newcounter{exera}
\numberwithin{exer}{subsection}
\theoremstyle{definition}
\newtheorem{theo}{Theorem}[subsection]
\newenvironment{theorem}[1][]
{\begin{theo}[#1]\begin{leftbar}}
{\end{leftbar}\end{theo}}
\newtheorem{lem}[theo]{Lemma}
\newenvironment{lemma}[1][]
{\begin{lem}[#1]\begin{leftbar}}
{\end{leftbar}\end{lem}}
\newtheorem{prop}[theo]{Proposition}
\newenvironment{proposition}[1][]
{\begin{prop}[#1]\begin{leftbar}}
{\end{leftbar}\end{prop}}
\newtheorem{defi}[theo]{Definition}
\newenvironment{definition}[1][]
{\begin{defi}[#1]\begin{leftbar}}
{\end{leftbar}\end{defi}}
\newtheorem{remk}[theo]{Remark}
\newenvironment{remark}[1][]
{\begin{remk}[#1]\begin{leftbar}}
{\end{leftbar}\end{remk}}
\newtheorem{coro}[theo]{Corollary}
\newenvironment{corollary}[1][]
{\begin{coro}[#1]\begin{leftbar}}
{\end{leftbar}\end{coro}}
\newtheorem{conv}[theo]{Convention}
\newenvironment{convention}[1][]
{\begin{conv}[#1]\begin{leftbar}}
{\end{leftbar}\end{conv}}
\newtheorem{quest}[theo]{Question}
\newenvironment{question}[1][]
{\begin{quest}[#1]\begin{leftbar}}
{\end{leftbar}\end{quest}}
\newtheorem{warn}[theo]{Warning}
\newenvironment{warning}[1][]
{\begin{warn}[#1]\begin{leftbar}}
{\end{leftbar}\end{warn}}
\newtheorem{conj}[theo]{Conjecture}
\newenvironment{conjecture}[1][]
{\begin{conj}[#1]\begin{leftbar}}
{\end{leftbar}\end{conj}}
\newtheorem{exam}[theo]{Example}
\newenvironment{example}[1][]
{\begin{exam}[#1]\begin{leftbar}}
{\end{leftbar}\end{exam}}
\newtheorem{exmp}[exer]{Exercise}
\newenvironment{exercise}[1][]
{\begin{exmp}[#1]\begin{leftbar}}
{\end{leftbar}\end{exmp}}
\newenvironment{statement}{\begin{quote}}{\end{quote}}
\newenvironment{fineprint}{\medskip \begin{small}}{\end{small} \medskip}
\iffalse
\newenvironment{proof}[1][Proof]{\noindent\textbf{#1.} }{\ \rule{0.5em}{0.5em}}
\newenvironment{question}[1][Question]{\noindent\textbf{#1.} }{\ \rule{0.5em}{0.5em}}
\newenvironment{warning}[1][Warning]{\noindent\textbf{#1.} }{\ \rule{0.5em}{0.5em}}
\newenvironment{teachingnote}[1][Teaching note]{\noindent\textbf{#1.} }{\ \rule{0.5em}{0.5em}}
\fi
\let\sumnonlimits\sum
\let\prodnonlimits\prod
\let\cupnonlimits\bigcup
\let\capnonlimits\bigcap
\renewcommand{\sum}{\sumnonlimits\limits}
\renewcommand{\prod}{\prodnonlimits\limits}
\renewcommand{\bigcup}{\cupnonlimits\limits}
\renewcommand{\bigcap}{\capnonlimits\limits}
\setlength\tablinesep{3pt}
\setlength\arraylinesep{3pt}
\setlength\extrarulesep{3pt}
\voffset=0cm
\hoffset=-0.7cm
\setlength\textheight{22.5cm}
\setlength\textwidth{15.5cm}
\newcommand\arxiv[1]{\href{http://www.arxiv.org/abs/#1}{\texttt{arXiv:#1}}}
\newenvironment{verlong}{}{}
\newenvironment{vershort}{}{}
\newenvironment{noncompile}{}{}
\newenvironment{teachingnote}{}{}
\excludecomment{verlong}
\includecomment{vershort}
\excludecomment{noncompile}
\excludecomment{teachingnote}
\newcommand{\CC}{\mathbb{C}}
\newcommand{\RR}{\mathbb{R}}
\newcommand{\QQ}{\mathbb{Q}}
\newcommand{\NN}{\mathbb{N}}
\newcommand{\ZZ}{\mathbb{Z}}
\newcommand{\KK}{\mathbb{K}}
\newcommand{\id}{\operatorname{id}}
\newcommand{\lcm}{\operatorname{lcm}}
\newcommand{\rev}{\operatorname{rev}}
\newcommand{\powset}[2][]{\ifthenelse{\equal{#2}{}}{\mathcal{P}\left(#1\right)}{\mathcal{P}_{#1}\left(#2\right)}}
\newcommand{\set}[1]{\left\{ #1 \right\}}
\newcommand{\abs}[1]{\left| #1 \right|}
\newcommand{\tup}[1]{\left( #1 \right)}
\newcommand{\ive}[1]{\left[ #1 \right]}
\newcommand{\floor}[1]{\left\lfloor #1 \right\rfloor}
\newcommand{\lf}[2]{#1^{\underline{#2}}}
\newcommand{\underbrack}[2]{\underbrace{#1}_{\substack{#2}}}
\newcommand{\horrule}[1]{\rule{\linewidth}{#1}}
\newcommand{\are}{\ar@{-}}
\newcommand{\nnn}{\nonumber\\}
\newcommand{\sslash}{\mathbin{/\mkern-6mu/}}
\newcommand{\numboxed}[2]{\underbrace{\boxed{#1}}_{\text{box } #2}}
\newcommand{\ig}[2]{\includegraphics[scale=#1]{#2.png}}
\newcommand{\UNFINISHED}{\begin{center} \Huge{\textbf{Unfinished material begins here.}} \end{center} }
\iffalse
\NOEXPAND{\today}{\today}
\NOEXPAND{\sslash}{\sslash}
\NOEXPAND{\numboxed}[2]{\numboxed}
\NOEXPAND{\UNFINISHED}{\UNFINISHED}
\fi
\ihead{Math 504 notes}
\ohead{page \thepage}
\cfoot{\today}
\begin{document}

\title{Math 504: Advanced Linear Algebra}
\author{Hugo Woerdeman, with edits by Darij Grinberg\thanks{Drexel University, Korman
Center, 15 S 33rd Street, Philadelphia PA, 19104, USA}}
\date{\today\ (unfinished!)}
\maketitle
\tableofcontents

\section*{Math 504 Lecture 21}

\section{Hermitian matrices (cont'd)}

\subsection{Introduction to majorization theory (cont'd)}

Recall:

\begin{itemize}
\item If $x\in\mathbb{R}^{n}$ is a column vector and $i\in\left[  n\right]  $,
then $x_{i}$ denotes the $i$-th coordinate (= entry) of $x$.

\item If $x\in\mathbb{R}^{n}$ is a column vector, then $x^{\downarrow}$
denotes the weakly decreasing permutation of $x$ (that is, the vector with the
same entries as $x$ but in weakly decreasing order).

\item For two vectors $x,y\in\mathbb{R}^{n}$, we say that $x$
\textbf{majorizes} $y$ (and we write $x\succcurlyeq y$) if and only if
\[
\sum_{i=1}^{m}x_{i}^{\downarrow}\geq\sum_{i=1}^{m}y_{i}^{\downarrow
}\ \ \ \ \ \ \ \ \ \ \text{for each }m\in\left[  n\right]
\]
and%
\[
x_{1}+x_{2}+\cdots+x_{n}=y_{1}+y_{2}+\cdots+y_{n}.
\]


\item A \textbf{Robin Hood move} (short \textbf{RH move}) transforms a vector
$x\in\mathbb{R}^{n}$ into a vector $y\in\mathbb{R}^{n}$ that is constructed as
follows: Pick two distinct $i,j\in\left[  n\right]  $ satisfying $x_{i}\leq
x_{j}$, and pick two numbers $u,v\in\left[  x_{i},x_{j}\right]  $ satisfy
$u+v=x_{i}+x_{j}$, and replace the $i$-th and the $j$-th entries of $x$ by $u$
and $v$.

This RH move is said to be an \textbf{order-preserving RH move} (short
\textbf{OPRH move}) if both $x$ and $y$ are weakly decreasing.
\end{itemize}

Last time we proved:

\begin{theorem}
[RH criterion for majorization]Let $x,y\in\mathbb{R}^{n}$ be two weakly
decreasing column vectors. Then, $x\succcurlyeq y$ if and only if $y$ can be
obtained from $x$ by a (finite) sequence of OPRH moves.
\end{theorem}

\bigskip

Now, what can we do with majorization? Probably the most important property of
majorizing pairs of vectors is the following:

\begin{theorem}
[Karamata's inequality]Let $I\subseteq\mathbb{R}$ be an interval. Let
$f:I\rightarrow\mathbb{R}$ be a convex function. Let $x\in I^{n}$ and $y\in
I^{n}$ be two vectors such that $x\succcurlyeq y$. Then,%
\[
f\left(  x_{1}\right)  +f\left(  x_{2}\right)  +\cdots+f\left(  x_{n}\right)
\geq f\left(  y_{1}\right)  +f\left(  y_{2}\right)  +\cdots+f\left(
y_{n}\right)  .
\]

\end{theorem}

Before we prove this, let us state a simple corollary:

\begin{corollary}
[Jensen's inequality]Let $I\subseteq\mathbb{R}$ be an interval. Let
$f:I\rightarrow\mathbb{R}$ be a convex function. Let $x_{1},x_{2},\ldots
,x_{n}\in I$. Let $m=\dfrac{x_{1}+x_{2}+\cdots+x_{n}}{n}$. Then,%
\[
f\left(  x_{1}\right)  +f\left(  x_{2}\right)  +\cdots+f\left(  x_{n}\right)
\geq nf\left(  m\right)  .
\]

\end{corollary}

\begin{proof}
This follows from Karamata's inequality, since it is easy to see (exercise)
that $\left(  x_{1},x_{2},\ldots,x_{n}\right)  ^{T}\succcurlyeq\left(
m,m,\ldots,m\right)  ^{T}$.
\end{proof}

Let us now prove Karamata's inequality:

\begin{proof}
[Proof of Karamata's inequality.] It is enough to prove the claim in the case
when $x$ and $y$ are weakly decreasing (because permuting the entries of any
of $x$ and $y$ does not change anything).

Furthermore, it is enough to prove the claim in the case when
$x\overset{\text{OPRH}}{\longrightarrow}y$ (this means that $y$ is obtained
from $x$ by a single OPRH move). In fact, if we can show this, then it will
mean that the sum $f\left(  x_{1}\right)  +f\left(  x_{2}\right)
+\cdots+f\left(  x_{n}\right)  $ decreases (weakly) every time we apply an
OPRH move. Therefore, if $y$ is obtained from $x$ by a sequence of OPRH moves,
then $f\left(  x_{1}\right)  +f\left(  x_{2}\right)  +\cdots+f\left(
x_{n}\right)  \geq f\left(  y_{1}\right)  +f\left(  y_{2}\right)
+\cdots+f\left(  y_{n}\right)  $. But the theorem we proved last time shows
that this is always satisfied when $x\succcurlyeq y$.

So let us assume that $x\overset{\text{OPRH}}{\longrightarrow}y$. Thus, $y$ is
obtained from $x$ by picking two entries $x_{i}$ and $x_{j}$ with $x_{i}\leq
x_{j}$ and replacing them by $u$ and $v$, where $u,v\in\left[  x_{i}%
,x_{j}\right]  $ with $u+v=x_{i}+x_{j}$. Consider these $x_{i},x_{j},u,v$. It
clearly suffices to show that%
\[
f\left(  x_{i}\right)  +f\left(  x_{j}\right)  \geq f\left(  u\right)
+f\left(  v\right)  .
\]


How do we do this? From $u\in\left[  x_{i},x_{j}\right]  $, we obtain
\[
u=\lambda x_{i}+\left(  1-\lambda\right)  x_{j}\ \ \ \ \ \ \ \ \ \ \text{for
some }\lambda\in\left[  0,1\right]
\]
(namely, $\lambda=\dfrac{1}{x_{i}-x_{j}}\left(  u-x_{j}\right)  $). Consider
this $\lambda$. Then,%
\[
v=\left(  1-\lambda\right)  x_{i}+\lambda x_{j}\ \ \ \ \ \ \ \ \ \ \left(
\text{since }u+v=x_{i}+x_{j}\right)  .
\]


However, $f$ is convex. From $u=\lambda x_{i}+\left(  1-\lambda\right)  x_{j}%
$, we obtain%
\[
f\left(  u\right)  =f\left(  \lambda x_{i}+\left(  1-\lambda\right)
x_{j}\right)  \leq\lambda f\left(  x_{i}\right)  +\left(  1-\lambda\right)
f\left(  x_{j}\right)  \ \ \ \ \ \ \ \ \ \ \left(  \text{since }f\text{ is
convex}\right)
\]
and%
\[
f\left(  v\right)  =f\left(  \left(  1-\lambda\right)  x_{i}+\lambda
x_{j}\right)  \leq\left(  1-\lambda\right)  f\left(  x_{i}\right)  +\lambda
f\left(  x_{j}\right)  .
\]
Adding together these two inequalities, we obtain%
\begin{align*}
f\left(  u\right)  +f\left(  v\right)    & \leq\left(  \lambda f\left(
x_{i}\right)  +\left(  1-\lambda\right)  f\left(  x_{j}\right)  \right)
+\left(  \left(  1-\lambda\right)  f\left(  x_{i}\right)  +\lambda f\left(
x_{j}\right)  \right)  \\
& =f\left(  x_{i}\right)  +f\left(  x_{j}\right)
,\ \ \ \ \ \ \ \ \ \ \text{qed.}%
\end{align*}
So Karamata's inequality is proved.
\end{proof}

Karamata's inequality has lots of applications, since there are many convex
functions around. For instance:

\begin{itemize}
\item $f\left(  t\right)  =t^{n}$ defines a convex function on $\mathbb{R}$
whenever $n\in\mathbb{N}$ is even.

\item $f\left(  t\right)  =t^{n}$ defines a convex function on $\mathbb{R}%
_{+}$ whenever $n\in\mathbb{R}\setminus\left(  0,1\right)  $. Otherwise, it
defines a concave function (so that $-f$ is a convex function).

\item $f\left(  t\right)  =\sin t$ defines a concave function on $\left[
0,\pi\right]  $ and a convex function on $\left[  \pi,2\pi\right]  $.
\end{itemize}

\bigskip

Karamata's inequality has a converse: If $x,y\in\mathbb{R}^{n}$ are two
vectors such that%
\[
f\left(  x_{1}\right)  +f\left(  x_{2}\right)  +\cdots+f\left(  x_{n}\right)
\geq f\left(  y_{1}\right)  +f\left(  y_{2}\right)  +\cdots+f\left(
y_{n}\right)
\]
for every convex function $f:\mathbb{R}\rightarrow\mathbb{R}$, then
$x\succcurlyeq y$. Even better:

\begin{theorem}
[absolute-value criterion for majorization]Let $x\in\mathbb{R}^{n}$ and
$y\in\mathbb{R}^{n}$ be two vectors. Then, $x\succcurlyeq y$ if and only if
all $t\in\mathbb{R}$ satisfy%
\[
\left\vert x_{1}-t\right\vert +\left\vert x_{2}-t\right\vert +\cdots
+\left\vert x_{n}-t\right\vert \geq\left\vert y_{1}-t\right\vert +\left\vert
y_{2}-t\right\vert +\cdots+\left\vert y_{n}-t\right\vert .
\]

\end{theorem}

\begin{proof}
$\Longrightarrow:$ Assume that $x\succcurlyeq y$. Let $t\in\mathbb{R}$.
Consider the function%
\begin{align*}
f_{t}:\mathbb{R}  & \rightarrow\mathbb{R},\\
z  & \mapsto\left\vert z-t\right\vert .
\end{align*}
This function $f_{t}$ is convex (this follows easily from the triangle
inequality). Hence, Karamata's inequality yields%
\[
f_{t}\left(  x_{1}\right)  +f_{t}\left(  x_{2}\right)  +\cdots+f_{t}\left(
x_{n}\right)  \geq f_{t}\left(  y_{1}\right)  +f_{t}\left(  y_{2}\right)
+\cdots+f_{t}\left(  y_{n}\right)  .
\]
By the definition of $f_{t}$, this means
\[
\left\vert x_{1}-t\right\vert +\left\vert x_{2}-t\right\vert +\cdots
+\left\vert x_{n}-t\right\vert \geq\left\vert y_{1}-t\right\vert +\left\vert
y_{2}-t\right\vert +\cdots+\left\vert y_{n}-t\right\vert .
\]
So we have proved the \textquotedblleft$\Longrightarrow$\textquotedblright%
\ direction of the theorem.

$\Longleftarrow:$ We shall refer to the inequality%
\[
\left\vert x_{1}-t\right\vert +\left\vert x_{2}-t\right\vert +\cdots
+\left\vert x_{n}-t\right\vert \geq\left\vert y_{1}-t\right\vert +\left\vert
y_{2}-t\right\vert +\cdots+\left\vert y_{n}-t\right\vert
\]
as the \textbf{absolute value inequality}. So we assume that the absolute
value inequality holds for all $t\in\mathbb{R}$. (Actually, it will suffice to
assume that it holds for all $t\in\left\{  x_{1},x_{2},\ldots,x_{n}%
,y_{1},y_{2},\ldots,y_{n}\right\}  $.)

We must prove that $x\succcurlyeq y$.

WLOG assume that $x$ and $y$ are weakly decreasing (since permuting the
entries changes neither the absolute value inequality nor the claim
$x\succcurlyeq y$).

Let $t=\max\left\{  x_{1},y_{1}\right\}  $. Then, for all $i\in\left[
n\right]  $, we have $t=\max\left\{  x_{1},y_{1}\right\}  \geq x_{1}\geq
x_{i}$ (since $x$ is weakly decreasing) and therefore $\left\vert
x_{i}-t\right\vert =t-x_{i}$ and similarly $\left\vert y_{i}-t\right\vert
=t-y_{i}$. Thus, the absolute value inequality rewrites as%
\[
\left(  t-x_{1}\right)  +\left(  t-x_{2}\right)  +\cdots+\left(
t-x_{n}\right)  \geq\left(  t-y_{1}\right)  +\left(  t-y_{2}\right)
+\cdots+\left(  t-y_{n}\right)  .
\]
In other words,%
\[
nt-\left(  x_{1}+x_{2}+\cdots+x_{n}\right)  \geq nt-\left(  y_{1}+y_{2}%
+\cdots+y_{n}\right)  .
\]
In other words,%
\[
x_{1}+x_{2}+\cdots+x_{n}\leq y_{1}+y_{2}+\cdots+y_{n}.
\]


Similarly, by taking $t=\min\left\{  x_{n},y_{n}\right\}  $, we obtain%
\[
x_{1}+x_{2}+\cdots+x_{n}\geq y_{1}+y_{2}+\cdots+y_{n}.
\]


Combining these two inequalities, we obtain%
\[
x_{1}+x_{2}+\cdots+x_{n}=y_{1}+y_{2}+\cdots+y_{n}.
\]


Now, let $k\in\left[  n\right]  $. Set $t=x_{k}$. Then, since $x$ is weakly
decreasing, we have%
\[
x_{1}\geq x_{2}\geq\cdots\geq x_{k}=t\geq x_{k+1}\geq x_{k+2}\geq\cdots\geq
x_{n}.
\]
Thus,%
\begin{align*}
& \left\vert x_{1}-t\right\vert +\left\vert x_{2}-t\right\vert +\cdots
+\left\vert x_{n}-t\right\vert \\
& =\left(  \underbrace{\left\vert x_{1}-t\right\vert }_{=x_{1}-t}%
+\underbrace{\left\vert x_{2}-t\right\vert }_{=x_{2}-t}+\cdots
+\underbrace{\left\vert x_{k}-t\right\vert }_{=x_{k}-t}\right)  +\left(
\underbrace{\left\vert x_{k+1}-t\right\vert }_{=t-x_{k+1}}%
+\underbrace{\left\vert x_{k+2}-t\right\vert }_{=t-x_{k+2}}+\cdots
+\underbrace{\left\vert x_{n}-t\right\vert }_{=t-x_{n}}\right)  \\
& =\left(  \left(  x_{1}-t\right)  +\left(  x_{2}-t\right)  +\cdots+\left(
x_{k}-t\right)  \right)  +\left(  \left(  t-x_{k+1}\right)  +\left(
t-x_{k+2}\right)  +\cdots+\left(  t-x_{n}\right)  \right)  \\
& =\left(  x_{1}+x_{2}+\cdots+x_{k}\right)  -kt+\left(  n-k\right)  t-\left(
x_{k+1}+x_{k+2}+\cdots+x_{n}\right)  \\
& =2\left(  x_{1}+x_{2}+\cdots+x_{k}\right)  +\left(  n-2k\right)  t-\left(
x_{1}+x_{2}+\cdots+x_{n}\right)
\end{align*}
(here, I have added $x_{1}+x_{2}+\cdots+x_{k}$ to the first parenthesis and
subtracted it back from the last) and%
\begin{align*}
& \left\vert y_{1}-t\right\vert +\left\vert y_{2}-t\right\vert +\cdots
+\left\vert y_{n}-t\right\vert \\
& =\left(  \underbrace{\left\vert y_{1}-t\right\vert }_{\geq y_{1}%
-t}+\underbrace{\left\vert y_{2}-t\right\vert }_{\geq y_{2}-t}+\cdots
+\underbrace{\left\vert y_{k}-t\right\vert }_{\geq y_{k}-t}\right)  +\left(
\underbrace{\left\vert y_{k+1}-t\right\vert }_{\geq t-y_{k+1}}%
+\underbrace{\left\vert y_{k+2}-t\right\vert }_{\geq t-y_{k+2}}+\cdots
+\underbrace{\left\vert y_{n}-t\right\vert }_{\geq t-y_{n}}\right)  \\
& \geq\left(  \left(  y_{1}-t\right)  +\left(  y_{2}-t\right)  +\cdots+\left(
y_{k}-t\right)  \right)  +\left(  \left(  t-y_{k+1}\right)  +\left(
t-y_{k+2}\right)  +\cdots+\left(  t-y_{n}\right)  \right)  \\
& =\left(  y_{1}+y_{2}+\cdots+y_{k}\right)  -kt+\left(  n-k\right)  t-\left(
y_{k+1}+y_{k+2}+\cdots+y_{n}\right)  \\
& =2\left(  y_{1}+y_{2}+\cdots+y_{k}\right)  +\left(  n-2k\right)  t-\left(
y_{1}+y_{2}+\cdots+y_{n}\right)  .
\end{align*}
Now, the absolute value inequality
\[
\left\vert x_{1}-t\right\vert +\left\vert x_{2}-t\right\vert +\cdots
+\left\vert x_{n}-t\right\vert \geq\left\vert y_{1}-t\right\vert +\left\vert
y_{2}-t\right\vert +\cdots+\left\vert y_{n}-t\right\vert
\]
becomes%
\begin{align*}
& 2\left(  x_{1}+x_{2}+\cdots+x_{k}\right)  +\left(  n-2k\right)  t-\left(
x_{1}+x_{2}+\cdots+x_{n}\right)  \\
& \geq\left\vert y_{1}-t\right\vert +\left\vert y_{2}-t\right\vert
+\cdots+\left\vert y_{n}-t\right\vert \\
& \geq2\left(  y_{1}+y_{2}+\cdots+y_{k}\right)  +\left(  n-2k\right)
t-\left(  y_{1}+y_{2}+\cdots+y_{n}\right)  .
\end{align*}
Subtracting $\left(  n-2k\right)  t$ from both sides, we obtain%
\begin{align*}
& 2\left(  x_{1}+x_{2}+\cdots+x_{k}\right)  -\left(  x_{1}+x_{2}+\cdots
+x_{n}\right)  \\
& \geq2\left(  y_{1}+y_{2}+\cdots+y_{k}\right)  -\left(  y_{1}+y_{2}%
+\cdots+y_{n}\right)  .
\end{align*}
Adding the equality%
\[
x_{1}+x_{2}+\cdots+x_{n}=y_{1}+y_{2}+\cdots+y_{n}%
\]
to this inequality, we obtain%
\[
2\left(  x_{1}+x_{2}+\cdots+x_{k}\right)  \geq2\left(  y_{1}+y_{2}%
+\cdots+y_{k}\right)  .
\]
In other words,%
\[
x_{1}+x_{2}+\cdots+x_{k}\geq y_{1}+y_{2}+\cdots+y_{k}.
\]
Since $x$ and $y$ are weakly decreasing, this shows that $x\succcurlyeq y$
(since $x_{1}+x_{2}+\cdots+x_{n}=y_{1}+y_{2}+\cdots+y_{n}$). This proves the
\textquotedblleft$\Longleftarrow$\textquotedblright\ direction of the theorem.
\end{proof}

\bigskip

Majorizing pairs of vectors are closely related to \textbf{doubly stochastic
matrices}:

\begin{definition}
A matrix $S\in\mathbb{R}^{n\times n}$ is said to be \textbf{doubly stochastic}
if its entries $S_{i,j}$ satisfy the following three conditions:

\begin{enumerate}
\item We have $S_{i,j}\geq0$ for all $i,j$.

\item We have $\sum_{j=1}^{n}S_{i,j}=1$ for each $i\in\left[  n\right]  $.

\item We have $\sum_{i=1}^{n}S_{i,j}=1$ for each $j\in\left[  n\right]  $.
\end{enumerate}
\end{definition}

In other words, a doubly stochastic matrix is an $n\times n$-matrix whose
entries are nonnegative reals and whose rows and columns have sum $1$ each.

\begin{exercise}
Show that even if we allow $S$ to be rectangular, the conditions 2 and 3 still
force $S$ to be a square matrix.
\end{exercise}

\begin{example}
\textbf{(a)} The matrix $\left(
\begin{array}
[c]{ccc}%
\dfrac{1}{2} & \dfrac{1}{3} & \dfrac{1}{6}\\
\dfrac{1}{2} & \dfrac{1}{4} & \dfrac{1}{4}\\
0 & \dfrac{5}{12} & \dfrac{7}{12}%
\end{array}
\right)  $ is doubly stochastic. \medskip

\textbf{(b)} Each doubly stochastic $2\times2$-matrix has the form%
\[
\left(
\begin{array}
[c]{cc}%
\lambda & 1-\lambda\\
1-\lambda & \lambda
\end{array}
\right)  \ \ \ \ \ \ \ \ \ \ \text{for some }\lambda\in\left[  0,1\right]  .
\]


\textbf{(c)} Any permutation matrix is doubly stochastic.
\end{example}

\begin{proposition}
Let $S\in\mathbb{R}^{n\times n}$ be a matrix whose entries are nonnegative
reals. Let $e=\left(  1,1,\ldots,1\right)  ^{T}\in\mathbb{R}^{n}$. Then, $S$
is doubly stochastic if and only if $Se=e$ and $e^{T}S=e^{T}$.
\end{proposition}

\begin{corollary}
Any product of doubly stochastic matrices is again doubly stochastic.
\end{corollary}

\begin{exercise}
Prove this proposition and this corolllary.
\end{exercise}

Now, the connection with majorization:

\begin{theorem}
Let $x,y\in\mathbb{R}^{n}$ be two vectors. Then, $x\succcurlyeq y$ if and only
if there exists a doubly stochastic matrix $S$ such that $y=Sx$.
\end{theorem}

\begin{proof}
$\Longrightarrow:$ Assume that $x\succcurlyeq y$. We must prove that there
exists a doubly stochastic matrix $S$ such that $y=Sx$.

By example \textbf{(c)} (and the corollary), it suffices to show this in the
case when $x$ and $y$ are weakly decreasing.

By the corollary, it suffices to show this in the case when
$x\overset{\text{OPRH}}{\longrightarrow}y$ (since in the general case, $y$ is
obtained from $x$ by a sequence of such moves).

So let us assume that $x\overset{\text{OPRH}}{\longrightarrow}y$. Thus, $y$ is
obtained from $x$ by picking two entries $x_{i}$ and $x_{j}$ with $x_{i}\leq
x_{j}$ and replacing them by $u$ and $v$, where $u,v\in\left[  x_{i}%
,x_{j}\right]  $ with $u+v=x_{i}+x_{j}$. Consider these $x_{i},x_{j},u,v$. It
clearly suffices to show that%
\[
f\left(  x_{i}\right)  +f\left(  x_{j}\right)  \geq f\left(  u\right)
+f\left(  v\right)  .
\]


How do we do this? From $u\in\left[  x_{i},x_{j}\right]  $, we obtain
\[
u=\lambda x_{i}+\left(  1-\lambda\right)  x_{j}\ \ \ \ \ \ \ \ \ \ \text{for
some }\lambda\in\left[  0,1\right]
\]
(namely, $\lambda=\dfrac{1}{x_{i}-x_{j}}\left(  u-x_{j}\right)  $). Consider
this $\lambda$. Then,%
\[
v=\left(  1-\lambda\right)  x_{i}+\lambda x_{j}\ \ \ \ \ \ \ \ \ \ \left(
\text{since }u+v=x_{i}+x_{j}\right)  .
\]


This entails that $y=Sx$, where $S$ is a matrix given by%
\begin{align*}
S_{i,i}  & =\lambda,\ \ \ \ \ \ \ \ \ \ S_{i,j}=1-\lambda
,\ \ \ \ \ \ \ \ \ \ S_{j,i}=1-\lambda,\ \ \ \ \ \ \ \ \ \ S_{j,j}=\lambda,\\
S_{k,k}  & =1\ \ \ \ \ \ \ \ \ \ \text{for each }k\notin\left\{  i,j\right\}
,\\
S_{k,\ell}  & =0\ \ \ \ \ \ \ \ \ \ \text{for all remaining }k,\ell.
\end{align*}
For example, if $i=2$ and $j=4$, then
\[
S=\left(
\begin{array}
[c]{cccc}%
1 &  &  & \\
& \lambda &  & 1-\lambda\\
&  & 1 & \\
& 1-\lambda &  & \lambda
\end{array}
\right)
\]
(where all empty cells are filled with zeroes). In this case,%
\[
Sx=S\left(
\begin{array}
[c]{c}%
x_{1}\\
x_{2}\\
x_{3}\\
x_{4}%
\end{array}
\right)  =\left(
\begin{array}
[c]{c}%
x_{1}\\
\lambda x_{2}+\left(  1-\lambda\right)  x_{4}\\
x_{3}\\
\left(  1-\lambda\right)  x_{2}+\lambda x_{4}%
\end{array}
\right)  =\left(
\begin{array}
[c]{c}%
x_{1}\\
u\\
x_{3}\\
v
\end{array}
\right)  =y.
\]
So we are done proving the \textquotedblleft$\Longrightarrow$%
\textquotedblright\ direction.

$\Longleftarrow:$ Assume that $y=Sx$ for some doubly stochastic $S$. Then,%
\[
y_{i}=\left(  Sx\right)  _{i}=\sum_{j=1}^{n}S_{i,j}x_{j}%
\ \ \ \ \ \ \ \ \ \ \text{for every }i\in\left[  n\right]  .
\]
Thus, for every convex function $f:\mathbb{R}\rightarrow\mathbb{R}$, we have%
\begin{align*}
& f\left(  y_{1}\right)  +f\left(  y_{2}\right)  +\cdots+f\left(
y_{n}\right)  \\
& =\underbrace{f\left(  \sum_{j=1}^{n}S_{1,j}x_{j}\right)  }_{\substack{\leq
\sum_{j=1}^{n}S_{1,j}f\left(  x_{j}\right)  \\\text{(by the weighted Jensen
inequality,}\\\text{since the }S_{1,j}\text{'s are nonnegative reals}%
\\\text{with sum }=1\text{)}}}+\underbrace{f\left(  \sum_{j=1}^{n}S_{2,j}%
x_{j}\right)  }_{\substack{\leq\sum_{j=1}^{n}S_{2,j}f\left(  x_{j}\right)
\\\text{(by the weighted Jensen inequality,}\\\text{since the }S_{2,j}\text{'s
are nonnegative reals}\\\text{with sum }=1\text{)}}}+\cdots
+\underbrace{f\left(  \sum_{j=1}^{n}S_{n,j}x_{j}\right)  }_{\substack{\leq
\sum_{j=1}^{n}S_{n,j}f\left(  x_{j}\right)  \\\text{(by the weighted Jensen
inequality,}\\\text{since the }S_{n,j}\text{'s are nonnegative reals}%
\\\text{with sum }=1\text{)}}}\\
& \leq\sum_{j=1}^{n}S_{1,j}f\left(  x_{j}\right)  +\sum_{j=1}^{n}%
S_{2,j}f\left(  x_{j}\right)  +\cdots+\sum_{j=1}^{n}S_{n,j}f\left(
x_{j}\right)  \\
& =\sum_{j=1}^{n}\underbrace{\left(  S_{1,j}+S_{2,j}+\cdots+S_{n,j}\right)
}_{\substack{=1\\\text{(since }S\text{ is doubly stochastic)}}}f\left(
x_{j}\right)  \\
& =\sum_{j=1}^{n}f\left(  x_{j}\right)  =f\left(  x_{1}\right)  +f\left(
x_{2}\right)  +\cdots+f\left(  x_{n}\right)  .
\end{align*}
In particular, we can apply this to the convex function $f_{t}:\mathbb{R}%
\rightarrow\mathbb{R},\ z\mapsto\left\vert z-t\right\vert $ for any
$t\in\mathbb{R}$, and we obtain%
\[
\left\vert y_{1}-t\right\vert +\left\vert y_{2}-t\right\vert +\cdots
+\left\vert y_{n}-t\right\vert \leq\left\vert x_{1}-t\right\vert +\left\vert
x_{2}-t\right\vert +\cdots+\left\vert x_{n}-t\right\vert .
\]
In other words,%
\[
\left\vert x_{1}-t\right\vert +\left\vert x_{2}-t\right\vert +\cdots
+\left\vert x_{n}-t\right\vert \geq\left\vert y_{1}-t\right\vert +\left\vert
y_{2}-t\right\vert +\cdots+\left\vert y_{n}-t\right\vert .
\]
Therefore, by the absolute value criterion, we have $x\succcurlyeq y$. This
proves the \textquotedblleft$\Longleftarrow$\textquotedblright\ direction.


\end{proof}


\end{document}